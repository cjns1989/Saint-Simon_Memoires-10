\PassOptionsToPackage{unicode=true}{hyperref} % options for packages loaded elsewhere
\PassOptionsToPackage{hyphens}{url}
%
\documentclass[oneside,14pt,french,]{extbook} % cjns1989 - 27112019 - added the oneside option: so that the text jumps left & right when reading on a tablet/ereader
\usepackage{lmodern}
\usepackage{amssymb,amsmath}
\usepackage{ifxetex,ifluatex}
\usepackage{fixltx2e} % provides \textsubscript
\ifnum 0\ifxetex 1\fi\ifluatex 1\fi=0 % if pdftex
  \usepackage[T1]{fontenc}
  \usepackage[utf8]{inputenc}
  \usepackage{textcomp} % provides euro and other symbols
\else % if luatex or xelatex
  \usepackage{unicode-math}
  \defaultfontfeatures{Ligatures=TeX,Scale=MatchLowercase}
%   \setmainfont[]{EBGaramond-Regular}
    \setmainfont[Numbers={OldStyle,Proportional}]{EBGaramond-Regular}      % cjns1989 - 20191129 - old style numbers 
\fi
% use upquote if available, for straight quotes in verbatim environments
\IfFileExists{upquote.sty}{\usepackage{upquote}}{}
% use microtype if available
\IfFileExists{microtype.sty}{%
\usepackage[]{microtype}
\UseMicrotypeSet[protrusion]{basicmath} % disable protrusion for tt fonts
}{}
\usepackage{hyperref}
\hypersetup{
            pdftitle={SAINT-SIMON},
            pdfauthor={Mémoires\_X},
            pdfborder={0 0 0},
            breaklinks=true}
\urlstyle{same}  % don't use monospace font for urls
\usepackage[papersize={4.80 in, 6.40  in},left=.5 in,right=.5 in]{geometry}
\setlength{\emergencystretch}{3em}  % prevent overfull lines
\providecommand{\tightlist}{%
  \setlength{\itemsep}{0pt}\setlength{\parskip}{0pt}}
\setcounter{secnumdepth}{0}

% set default figure placement to htbp
\makeatletter
\def\fps@figure{htbp}
\makeatother

\usepackage{ragged2e}
\usepackage{epigraph}
\renewcommand{\textflush}{flushepinormal}

\usepackage{indentfirst}
\usepackage{relsize}

\usepackage{fancyhdr}
\pagestyle{fancy}
\fancyhf{}
\fancyhead[R]{\thepage}
\renewcommand{\headrulewidth}{0pt}
\usepackage{quoting}
\usepackage{ragged2e}

\newlength\mylen
\settowidth\mylen{...................}

\usepackage{stackengine}
\usepackage{graphicx}
\def\asterism{\par\vspace{1em}{\centering\scalebox{.9}{%
  \stackon[-0.6pt]{\bfseries*~*}{\bfseries*}}\par}\vspace{.8em}\par}

\usepackage{titlesec}
\titleformat{\chapter}[display]
  {\normalfont\bfseries\filcenter}{}{0pt}{\Large}
\titleformat{\section}[display]
  {\normalfont\bfseries\filcenter}{}{0pt}{\Large}
\titleformat{\subsection}[display]
  {\normalfont\bfseries\filcenter}{}{0pt}{\Large}

\setcounter{secnumdepth}{1}
\ifnum 0\ifxetex 1\fi\ifluatex 1\fi=0 % if pdftex
  \usepackage[shorthands=off,main=french]{babel}
\else
  % load polyglossia as late as possible as it *could* call bidi if RTL lang (e.g. Hebrew or Arabic)
%   \usepackage{polyglossia}
%   \setmainlanguage[]{french}
%   \usepackage[french]{babel} % cjns1989 - 1.43 version of polyglossia on this system does not allow disabling the autospacing feature
\fi

\title{SAINT-SIMON}
\author{Mémoires\_X}
\date{}

\begin{document}
\maketitle

\hypertarget{chapitre-premier.}{%
\chapter{CHAPITRE PREMIER.}\label{chapitre-premier.}}

1711

~

{\textsc{Défaite entière du czar en personne sur le Pruth, qui se sauve
avec ce qui lui reste par un traité et par l'avarice du grand vizir, qui
lui coûte la tête.}} {\textsc{- Chalais\,; quel\,; va trouver la
princesse des Ursins en Espagne.}} {\textsc{- Princesse des Ursins forme
et avance le projet d'une souveraineté pour elle, et de l'usage qu'elle
en fera\,; se fait bâtir, sans paraître, une superbe demeure en
Touraine.}} {\textsc{- Sort de cette demeure et du projet de
souveraineté.}} {\textsc{- Campagne d'Espagne oisive.}} {\textsc{- Mort
de Castel dos Rios, vice-roi du Pérou.}} {\textsc{- Prince de
Santo-Buono lui succède.}} {\textsc{- Don Domingo Guerra rappelé en
Espagne\,; son caractère\,; ses emplois.}} {\textsc{- Arpajon fait
chevalier de la Toison d'or.}} {\textsc{- Retour de Fontainebleau.}}
{\textsc{- Cardinal de Noailles interdit plusieurs jésuites\,; voit le
roi et le Dauphin à leur retour.}} {\textsc{- Intrigues pour allonger
l'affaire, sous prétexte de la finir.}} {\textsc{- Lettres au roi de
quantité d'évêques.}} {\textsc{- Le Dauphin logé a Versailles dans
l'appartement de Monseigneur.}} {\textsc{- Retour du duc de Noailles par
ordre du roi, qu'il salue, et est mal reçu.}} {\textsc{- Biens de France
du prince de Carignan confisqués\,; douze mille livres de pension dessus
au prince d'Espinoy.}} {\textsc{- Chimères de M. de Chevreuse mettent en
péril l'érection nouvelle de Chaulnes pour son second fils.}} {\textsc{-
Vidame d'Amiens fait duc et pair de Chaulnes.}} {\textsc{- Cris de la
cour.}} {\textsc{- Le Dauphin désapprouve cette grâce.}} {\textsc{- Rare
réception du duc de Chaulnes au parlement.}} {\textsc{-
Plénipotentiaires nommés pour la paix.}} {\textsc{- Utrecht choisi pour
le lieu de la traiter.}} {\textsc{- Retour des généraux, de Tallard de
sa prison en Angleterre, et du roi Jacques de ses voyages par le
royaume.}} {\textsc{- Comte de Toulouse fort heureusement taillé par
Maréchal\,; la galerie et le grand appartement fermés jusqu'à sa
parfaite guérison.}} {\textsc{- Mort et caractère de
M\textsuperscript{lle} de La Rochefoucauld.}} {\textsc{- Mort et
caractère de Sebville.}} {\textsc{- Mort, état, et caractère de
M\textsuperscript{me} de Grancey.}} {\textsc{- Mort et singuliers
mariages de la maréchale de L'Hôpital.}} {\textsc{- Abbé de Pomponne
conseiller d'État d'Église.}} {\textsc{- Tremblement de terre peu
perceptible.}} {\textsc{- Nouvelle tontine.}} {\textsc{- Grand prieur à
Lyon.}}

~

On apprit en ce même temps le malheur du czar contre le grand vizir, sur
la rivière du Pruth. Ce prince, piqué de la protection que la Porte
avait accordée au roi de Suède retiré à Bender, en voulut avoir raison
par les armes, et tomba dans la même faute qui avait perdu le roi de
Suède contre lui. Les Turcs l'attirèrent sur le Pruth à travers des
déserts, où, manquant de tout, il fallut périr ou hasarder tout par un
combat fort inégal. Il était à la tête de soixante mille hommes\,; il en
perdit plus de trente mille sur la place, le reste mourant de faim et de
misère\,; et lui sans aucune ressource, sans pouvoir éviter d'être
prisonnier des Turcs avec tout ce qu'il avait avec lui. Dans une
extrémité si pressante, une femme de rien, qu'il avait ôtée à son mari,
tambour dans ses troupes, et qu'il avait publiquement épousée après
avoir répudié et confiné la sienne dans un couvent, lui proposa de
tenter le grand vizir pour le laisser retourner libre dans ses États
avec tout ce qui était resté de la défaite. Le czar approuva la
proposition, sans en espérer de succès. Il envoya sur-le-champ au grand
vizir, avec ordre de lui parler en secret. Il fut ébloui de l'or et des
pierreries, et de plusieurs choses précieuses qui lui furent offertes\,;
il les accepta, les reçut, et signa avec le czar un traité de paix par
lequel il lui était permis de se retirer en ses États par le plus court
chemin, avec tout ce qui l'accompagnait, les Turcs lui fournissant des
vivres dont il manquait entièrement\,; et le czar s'engageait à rendre
Azof dès qu'il serait arrivé chez lui\,; de raser tous les forts et de
brûler tous les vaisseaux qu'il avait sur la mer Noire, de laisser
retourner le roi de Suède par la Poméranie, et de payer aux Turcs et à
ce prince tous les frais de la guerre.

Le grand vizir trouva une telle opposition au divan à passer ce traité,
et une telle hardiesse dans le ministre du roi de Suède, qui
l'accompagnait, à exciter contre lui tous les principaux de son armée,
que peu s'en fallut qu'il ne fût rompu, et que le czar avec tout ce qui
lui restait ne subît le sort d'être fait prisonnier\,: il n'était pas en
état de la moindre résistance. Le grand vizir n'avait qu'à le vouloir
pour l'exécuter sur-le-champ. Outre la gloire de mener à Constantinople
le czar, sa cour et ses troupes, on peut juger de ce qu'il en eût coûté
à ce prince\,; mais ses riches dépouilles auraient été pour le Grand
Seigneur, et le grand vizir les aima mieux pour soi. Il paya donc
d'autorité et de menaces, et se hâta de faire partir le czar et de
s'éloigner en même temps. Le ministre de Suède, chargé des protestations
des principaux chefs des Turcs, courut à Constantinople, où le grand
vizir fut étranglé en arrivant. Le czar n'oublia jamais ce service de sa
femme, dont le courage et la présence d'esprit l'avait sauvé. L'estime
qu'il en conçut, jointe à l'amitié, l'engagea à la faire couronner
czarine, à lui faire part de toutes ses affaires et de tous ses
desseins. Échappé au danger, il fut longtemps sans rendre Azof, et à
démolir ses forts de la mer Noire. Pour ses vaisseaux, il les conserva
presque tous, et ne voulut pas laisser retourner le roi de Suède en
Allemagne, ce qui pensa rallumer la guerre avec le Turc.

Chalais prit congé à Fontainebleau pour s'en aller en Espagne, prendre
un bâton d'exempt des gardes du corps, dans la compagnie wallone, dont
M. de Bournonville était capitaine. M\textsuperscript{me} des Ursins
avait toujours conservé un grand attachement pour son premier mari, pour
son nom, pour ses proches. Celui-ci était fils unique de son frère aîné
qui n'était jamais sorti de sa province, et ce fils n'avait paru ni à la
cour ni dans le service. Le père était fort mal aisé, et le fils, qui
n'avait rien, fut trop heureux de cette ressource\,; on le retrouvera
dans la suite plus d'une fois. Outre cette affection,
M\textsuperscript{me} des Ursins fut bien aise d'avoir quelqu'un
entièrement à elle, qui ne tînt qu'à elle, qui ne pût espérer rien que
d'elle, et qui ne fût connu de personne en France ni en Espagne.

Non contente d'y régner en toute autorité et puissance, elle osa songer
à avoir elle-même de quoi régner. Elle saisit la conjoncture du don que
le roi d'Espagne fit à l'électeur de Bavière, de ce qui était demeuré
dans son obéissance aux Pays-Bas, pour y faire stipuler que l'électeur y
donnerait des terres jusqu'à cent mille livres de rente à elle pour en
jouir sa vie durant en toute souveraineté. Bientôt après il fut convenu
avec l'électeur que le chef-lieu de ces terres, qui doivent être
contiguës et n'en former qu'une seule, serait la Roche en Ardennes, et
que la souveraineté en porterait le nom. On verra dans la suite cette
souveraineté prendre diverses formes, changer de lieu, et se dissiper
enfin en fumée, et cela dura longtemps. M\textsuperscript{me} des Ursins
s'en tint si assurée, qu'elle bâtit là-dessus un beau projet\,: ce fut
d'échanger avec le roi la souveraineté qui lui serait assignée sur sa
frontière, et pour celle-là, d'avoir en souveraineté la Touraine et le
pays d'Amboise sa vie durant, réversible après à la couronne, de quitter
l'Espagne, et de venir en jouir le reste de ses jours.

Dans ce dessein qu'elle crut immanquable, elle envoya en France
d'Aubigny, cet écuyer si favori dont il a été parlé ici plus d'une fois,
avec ordre de lui préparer une belle demeure pour la trouver toute prête
à la recevoir. Il acheta un champ près de Tours, et plus encore
d'Amboise, sans terres ni seigneurie, parce qu'étant souveraine de la
province, elle n'en avait pas besoin. Il se mit aussitôt à y bâtir
très-promptement, mais solidement, un vaste et superbe château,
d'immenses basses-cours, et des communs prodigieux, avec tous les
accompagnements des plus grands et des plus beaux jardins, à la
magnificence desquels les meubles répondirent en tous genres. La
province, les pays voisins, Paris, la cour même en furent dans
l'étonnement. Personne ne pouvait comprendre une dépense si prodigieuse
pour une simple guinguette, puisque une maison au milieu d'un champ,
sans terres, sans revenus, sans seigneurie, ne peut avoir d'autre nom,
et moins encore une cage si vaste et si superbe pour l'oiseau qui la
construisait. Ce fut longtemps une égnime, et cette folie de
M\textsuperscript{me} des Ursins fut, comme on le verra, la première
cause de sa perte. On n'en dira pas davantage sur le succès de cette
chimère qui ne laissa pas d'accrocher la paix par l'opiniâtreté du roi
d'Espagne, qui ne céda enfin qu'à l'autorité du roi qui le força de se
désister de cet article, dont les alliés se moquèrent toujours avec
mépris jusqu'à n'avoir jamais voulu en entendre parler dans les formes.
{[}Je n'en parlerai pas davantage{]}, parce que ce point est fort bien
expliqué dans les Pièces\,; mais, pour n'y plus revenir, il faut voir ce
que devint cet admirable palais, si complétement achevé en tout, et
meublé entièrement avant que M\textsuperscript{me} des Ursins eût perdu
l'espérance d'y jouer la souveraine.

On ne pouvait imaginer qu'un aussi petit compagnon que l'était
d'Aubigny, quelques richesses qu'il eût amassées, pût ni osât faire un
pareil bâtiment pour soi. Ce ne fut que peu à peu que l'obscurité fut
percée. On soupçonna que M\textsuperscript{me} des Ursins le faisait
agir, et se couvrait de son nom. On pensait qu'elle pouvait lasser, ou
se lasser enfin de l'Espagne, et voulait venir achever sa vie dans son
pays sans y traîner à la cour ni dans Paris, après avoir si
despotiquement régné ailleurs. Mais un palais, qui pourtant n'était
qu'une guinguette, ne s'entendait pas pour sa retraite\,; ce ne fut que
l'éclat que sa prétendue souveraineté fit par toute l'Europe qui
commença à ouvrir les yeux sur Chanteloup\,; c'est le nom de ce palais,
dont à la fin on sut la destination. La chute entière de cette
ambitieuse femme, qui se verra ici dans son temps, ne lui permit pas
d'habiter cette belle demeure. Elle demeura en propre à d'Aubigny, qui y
reçut très-bien les voisins et les curieux, ou les passants de
considération, à qui il ne cacha plus que ce n'était ni pour soi, ni de
son bien, qu'il l'avait bâtie et meublée. Il s'y établit, il s'y fit
aimer et estimer. Il y perdit sa femme qui ne lui laissa qu'une fille
unique fort jeune\,; ainsi il s'était marié du vivant de
M\textsuperscript{me} des Ursins, ou aussitôt après sa mort, et cette
fille très-riche a épousé le marquis d'Armentières, qui sert
actuellement d'officier général, et qui en a plusieurs enfants. Orry,
dès lors contrôleur général, en fit le mariage. Peu auparavant Aubigny
était mort, et avait chargé Orry du soin de sa fille et de ses biens,
comme étant le fils de son meilleur ami, de ce môme Orry qui avait été
plus d'une fois en Espagne, et dont plus d'une fois il a été parlé ici.

La campagne n'avait été rien en Espagne\,; il n'y eut que des
bagatelles. L'archiduc, trop affaibli pour rien entreprendre de bonne
heure, ne songea plus qu'au départ, dès que l'empereur son frère fut
mort, et n'eut plus d'argent que pour la dépense du voyage. M. de
Vendôme en manquait aussi, et ne laissa pas de faire accroire longtemps
aux deux cours qu'il ferait le siège de Barcelone, pour lequel il amassa
des préparatifs. Le roi et la reine d'Espagne passèrent l'hiver à
Saragosse, et l'été fort inutilement à Corella. Le duc de Noailles,
destiné avec ses troupes, qui n'avaient rien à faire en Catalogne, à
servir sous M. de Vendôme, était allé, dès le mois de mars, à la cour
d'Espagne, où M. de Vendôme ne fut que de rares instants, sous prétexte
des préparatifs de la campagne. La contrainte ne l'accommodait pas, il
aimait mieux régner et paresser librement dans ses quartiers. L'été et
l'automne s'écoulèrent de la sorte, et tout à la fin la cour d'Espagne
retourna à Madrid. Elle donna la vice-royauté du Pérou au prince
Caraccioli de Santo-Buono, grand d'Espagne, qui avait perdu tous ses
biens de Naples.

Cette vice-royauté vaquait par la mort du marquis de Castel dos Rios,
qui était ambassadeur d'Espagne en France à l'avènement de Philippe V à
la couronne, et rappela en Espagne don Domingo Guerra, qui avait été
chancelier de Milan, place extrêmement principale qu'il avait perdue
depuis l'occupation des Impériaux, et était à Paris depuis longtemps. Il
eut les premières places d'affaires en Espagne, et à la fin les perdit.
C'était une très-bonne tête, fort instruit, fort expérimenté, grand
travailleur, fort espagnol et assez peu français. Bientôt après Arpajon,
qui servait de lieutenant général en Espagne, et qui y avait été heureux
en deux petites expéditions qui ne roulèrent que sur lui, fut honoré de
l'ordre de la Toison d'or.

Le lundi 14 septembre, le roi revint de Fontainebleau par Petit-Bourg,
et arriva le lendemain de bonne heure. Le cardinal de Noailles, qui
avait eu ordre de s'y trouver ce même jour, parut à la descente du
carrosse. Il eut aussitôt après une assez longue audience du roi, puis
du Dauphin encore plus longue. Ce prince avait fort travaillé à cette
affaire à Fontainebleau, et j'en avais appris des nouvelles à mesure par
l'archevêque de Bordeaux. Elle avait alors deux points\,: le personnel
entre le cardinal de Noailles et les évêques de la Rochelle et de Luçon,
où celui de Gap s'était fourré depuis comme diable en miracles\,; et le
livre du P. Quesnel, c'est-à-dire la doctrine, dont le personnel n'avait
été que le chausse-pied. Ils sentaient bien l'odieux du chausse-pied qui
ne pourrait se soutenir, et qui entraînerait à la fin celui de la
doctrine, si elle n'était soutenue que par ces trois agresseurs. Le P.
Tellier qui gouvernait l'évêque de Meaux, et qui par lui allongeait
l'affaire auprès du Dauphin, se servit de cet entre-temps pour faire
écrire au roi, par tous les évêques qu'il put gagner, des lettres
d'effroi sur la doctrine, et de condamnation du livre du P. Quesnel. Les
créatures des jésuites, les faibles qui n'osèrent se brouiller avec
l'entreprenant confesseur, les avares et les ambitieux firent un nombre
qui imposa. Le cardinal de Noailles eut le vent de ces pratiques, qui se
dirigeaient toutes aux jésuites de la rue Saint-Antoine. Les PP.
Lallemant, Doucin et Tournemine en étaient les principaux artisans. Il
leur échappa quelques menaces fort indiscrètes et fort insolentes,
d'autres gros bonnets en furent les échos. Le cardinal de Noailles ôta à
ceux-là les pouvoirs de confesser et de prêcher, et cela fit un nouveau
vacarme.

Les choses en étaient là au retour de Fontainebleau, et les lettres des
évêques au roi prêtes à pleuvoir, parce qu'il fallut du temps à
Saint-Louis pour composer le même thème en tant de façons différentes,
envoyer dans les diocèses, et obtenir la signature et l'envoi. M. de
Meaux avait eu beau fournir des embarras, le procédé était insoutenable,
et M. le Dauphin le voulut finir, avec d'autant plus d'empressement que
l'interdiction de ce petit nombre de jésuites allait apporter de
nouvelles aigreurs. Le roi néanmoins, quelque prévenu qu'il fût par le
P. Tellier, écouta, assez bien les raisons du cardinal de Noailles, sur
cette interdiction, quoiqu'elle lui déplût, et ne voulut pas qu'elle fît
obstacle à ce que le Dauphin avait réglé. Il l'expliqua ce même jour au
cardinal de Noailles, qui s'y soumit de bonne grâce. Voysin avait en
poche le consentement des trois évêques, qui, dans l'espérance que le
cardinal ferait quelque difficulté dont ils feraient retomber la
mauvaise satisfaction sur lui, n'avait eu garde de s'en vanter, et ne
l'apporta au Dauphin que cinq jours après.

Le jugement fut\,: que les trois évêques feraient en commun un nouveau
mandement en réparation des précédentes\,; qu'avant de le publier il
serait envoyé à Paris pour y être examiné par personnes nommées par le
Dauphin, communiqué après au cardinal, et, s'il en était content,
publié. Ensuite le roi lui devait envoyer une lettre des trois évêques
que Sa Majesté avait déjà reçue, pour réparer de plus en plus ce qu'ils
avaient écrit contre lui\,; et dans l'une et l'autre pièce, pas un mot
du livre du P. Quesnel. Le Dauphin, fort ignorant des profondeurs des
jésuites et de l'ambition de l'évêque de Meaux, crut avoir tout fini, et
que le bruit qui s'était fait sur ce livre tomberait avec la querelle
personnelle dont il était venu au secours, ou que, s'il y avait en effet
de la réalité dans les plaintes si nouvelles d'un livre si anciennement
approuvé et estimé sans contradiction de personne, les choses se
passeraient en douceur et en honnêteté entre des évêques raccommodés. Il
n'était pourtant pas difficile de voir l'artifice. Un mandement à faire,
puis à mettre à l'examen était de quoi tirer de longue, et faire naître
toutes les difficultés qu'on voudrait\,; et le silence spécieux sur le
livre laissait toute liberté là-dessus, après la réconciliation même
faite, sous le beau prétexte de la pureté de la doctrine. Mais le
Dauphin aurait fait scrupule de penser si mal de son prochain. Combien
était-il éloigné d'imaginer ce nombre de lettres qui se fabriquaient
alors, et la surprenante aventure qui en mit au jour sous les yeux du
public le scélérat mystère, et qui l'a transmis à la postérité\,! Le
Dauphin en arrivant de Fontainebleau prit l'appartement de Monseigneur.

Le lendemain de l'arrivée de Fontainebleau le duc de Noailles revint
d'Espagne, et salua le roi chez M\textsuperscript{me} de Maintenon. Il
en avait reçu l'ordre. Je différerai d'en expliquer les raisons jusque
tout à la fin de cette année, pour n'y être pas interrompu par le récit
d'autres événements.

Le roi, ayant su que le prince de Carignan, fils du célèbre muet, avait
servi dans l'armée de M. de Savoie, confisqua tous ses biens en France,
et donna dessus douze mille livres de rente au prince d'Espinoy, qui
avait aussi des biens confisqués en Flandre. C'est ce même prince de
Carignan qui, longtemps depuis, épousa la bâtarde de M. de Savoie et de
M\textsuperscript{me} de Verue, avec qui il vint après vivre et mourir à
Paris d'une manière honteuse\,; et qui, par les manéges encore plus
honteux de sa femme, y obtint tant de millions.

M. de Chevreuse, à qui j'avais fortement reproché ses absences qui lui
avaient coûté à Marly le dangereux délai de son affaire de Chaulnes,
lors de l'édit et de l'érection de d'Antin, avait fort travaillé à la
remettre à flot pendant tout Fontainebleau. On disait quelquefois de lui
qu'il était malade de raisonnement\,: et la vérité est qu'il le fut
tellement en cette occasion, qu'il eut souvent besoin de mon secours
pour l'empêcher d'en mourir, c'est-à-dire son affaire de manquer.
Chaulnes avait été érigé en duché-pairie pour le maréchal de Chaulnes,
frère du connétable de Luynes. Il est vrai que ce fut à l'occasion et en
faveur de son mariage avec l'héritière de Picquigny qui le savait bien
dire, à laquelle appartenait aussi le comté de Chaulnes\,; mais
l'érection n'en fut pas moins masculine, et bornée, comme toutes les
autres qui n'ont pas de clauses extraordinaires et expresses, aux hoirs
masculins issus de ce mariage de mâle en mâle. Les deux fils de ce
mariage, ducs l'un après l'autre, n'en avaient point eu\,; le
duché-pairie était donc éteint, ou il n'y en aura jamais, et depuis la
mort du dernier duc de Chaulnes, si connu par ses ambassades, il n'en
avait pas été question. M. de Chevreuse, grand artisan de quintessences,
et qu'on a vu, à l'occasion du procès de M. de Luxembourg, n'avoir point
voulu être des nôtres par la chimère de l'ancienne érection de
Chevreuse, s'en était bâtie une à part lui sur Chaulnes. Je crois avoir
remarqué ici quelque part que, lorsqu'il se maria, M. de Chaulnes,
cousin germain de son père, lui assura tout son bien au cas qu'il mourût
sans enfants, avec substitution au second fils qui naîtrait de son
mariage. Le cas était arrivé, il était exécuté.

M. de Chevreuse, depuis la mort de M. de Chaulnes, se qualifiait duc de
Luynes, de Chaulnes et de Chevreuse. Comme je vivais dans la plus libre
familiarité avec lui, je lui voyais souvent sur son bureau des
certificats pour des chevau-légers, etc., où ces titres étaient\,; et
toujours je lui disais : «\,Seigneur du duché de Chaulnes\,; mais duc
non.\,» II riochait, ne répondait qu'à demi, et disait qu'il le pouvait
prétendre. Lorsqu'il fut question de l'édit, il fallut discuter ensemble
plus sérieusement une prétention dont, à l'imitation de d'Antin, il
voulait faire le chausse-pied de son second fils. Il prétendit donc que
M. de Chaulnes, par la donation et la substitution de ses biens, et en
particulier de Chaulnes, les avait donnés et substitués comme il les
possédait, et par conséquent la dignité de laquelle il jouissait.

Je serais infini, et très-inutilement, si je m'amusais à réfuter ici un
paradoxe aussi absurde et aussi nouveau\,; mais il fallut en discuter
avec lui la nouveauté et l'absurdité, et se livrer à l'ennuyeuse
complaisance de laisser couler ses longs raisonnements. Il me mit après
en avant des coutumes particulières des lieux, qui pouvaient bien régler
les transmissions des biens, mais jamais en aucun cas celle des
dignités. Enfin il se retrancha sur une compensation, en abandonnant la
prétention de la première érection de Chevreuse. C'était étayer une
chimère par une autre. Chevreuse avait été érigé en duché-pairie pour M.
de Chevreuse, dernier fils du duc de Guise, tué aux derniers états de
Blois. Il avait épousé la veuve du connétable de Luynes, mère du duc de
Luynes, père du duc de Chevreuse, à qui je parlais. Sa grand'mère avait
eu pour ses reprises le duché de Chevreuse à la mort de ce second mari,
lequel duché, c'est-à-dire la terre, était passé d'elle à son fils, puis
à son petit-fils avec ses autres biens. Chevreuse, duché-pairie alors
éteint, avait été érigé de nouveau, mais sans pairie, et vérifié au
parlement pour M. de Chevreuse par la faveur de M. Colbert, dont il
venait d'épouser la fille aînée, et jamais M. de Chevreuse n'avait osé
rien prétendre au delà.

Je pris donc la liberté de me moquer de cette seconde chimère, comme
j'avais fait de la première\,; et je lui conseillai fort de n'appuyer
point sur des fondements si ruineux, ou pour mieux dire si parfaitement
nuls, mais de se fonder uniquement sur l'amitié et les services de M. le
chancelier, et sur la bonté distinguée que le roi avait pour lui, qui
l'avait empêché de rejeter la proposition, que le chancelier avait eu
l'adresse de lui faire, d'une érection nouvelle en faveur du vidame
d'Amiens, laquelle, entre deux amis et pour lui en dire le vrai, n'était
en aucun sens faisable ni recevable, et de n'aller pas gâter son affaire
par des idées chimériques qui impatienteraient le chancelier et le
rebuteraient, qui était pourtant l'instrument unique duquel il pût
espérer une si prodigieuse fortune pour son fils. Mais je parlais à un
homme qui se trompait lui-même de la meilleure foi du monde, et qui, à
force de métaphysique et de géométrie, se croyait rendre sensibles, et
aux autres ensuite, les raisonnements les plus faux, qu'il soutenait de
beaucoup d'esprit et d'un bien-dire naturel. Il ne put se déprendre de
ses chimères, ni s'empêcher d'en vouloir persuader le chancelier.

Celui-ci qui était vif, net, conséquent avec justesse, dont les
principes étaient certains et les conséquences naturelles, petillait,
interrompait, faisait des négatives sèches\,; et après se plaignait à
moi d'un homme qui n'était pas content qu'on fît son second fils duc et
pair sans raison quelconque autre que l'amitié, et qui voulait que ce
fût à des titres fous, chimériques, nuls, qui ne se lassait jamais en
raisonnements absurdes, et qui ne finissait point. J'avertis plus d'une
fois M. de Chevreuse qu'il raisonnerait tant qu'il échouerait. Je n'y
gagnai rien. C'était un homme froid, tranquille, qui se possédait,
puissant en dialectique dont il abusait presque toujours, qui s'y
confiait, qui espérait toujours, et qui ne se rebutait jamais, qui de
plus, lorsqu'il s'était bien persuadé une chose, écoutait tout ce qu'on
lui opposait avec le dernier mépris effectif, quoique voilé de toute la
douceur et la politesse possible. Avec cette conduite il poussa si bien
le chancelier à bout qu'il me déclara plusieurs fois qu'il n'y pouvait
plus tenir, et à deux différentes qu'il n'en voulait plus ouïr parler.
J'eus bien de la peine deux jours durant à l'apaiser et à renouer
l'affaire. Mais la seconde fut si forte qu'il déclara à M. de Chevreuse
qu'il pouvait faire son fils duc et pair, du roi à lui, s'il voulait, et
l'embâter de tous ses beaux raisonnements (car le chancelier poussé
laissa échapper ce terme)\,; mais que pour lui, il était las de perdre
son temps à ouïr répéter les mêmes absurdités en cent façons qui ne les
rendaient pas plus supportables, à quelques sauces qu'il les mît, et que
de ce duché-là, il n'en voulait plus ouïr parler, ni se charger d'en
reparler au roi.

M. de Chevreuse, fort effrayé malgré tout son sang-froid, vint aussitôt
me conter sa déconvenue, et me prier instamment de la raccommoder.
J'avoue que, pour un homme de mon âge, je ne me retins pas avec lui,
piqué de lui voir perdre et gâter une si inespérable affaire par cette
inflexibilité d'attachement à son sens, et encore si évidemment absurde.
Il essuya ma bordée. Je lui en valus une autre de M. de Beauvilliers,
qui ne le trouvait pas en duchés moins chimérique que je le trouvais
moi-même. Avec ce secours, mais qui jusque-là n'avait agi que
faiblement, je tirai parole qu'il ne parlerait plus au chancelier, sinon
pour le prier d'agir auprès du roi en conséquence de ce qu'il avait déjà
fait, et qu'en aucun temps il n'entrerait en aucun autre détail, surtout
sur ses idées de prétentions, et, après un édit fait {[}par{]} le
chancelier pour les anéantir toutes. Avec cette sûreté, je parlai au
chancelier, que j'eus grand'peine à vaincre\,; il fallut plusieurs
jours. Enfin il me promit de parler au roi, à condition qu'il ne verrait
seulement pas M. de Chevreuse. Ce fut donc moi qui agis seul auprès du
chancelier, dans la fin du voyage de Fontainebleau et au commencement du
retour à Versailles. L'affaire enfin fut accordée immédiatement avant
d'aller à Marly\,; et le lendemain que le roi y fut, qui était un jeudi
8 octobre, il déclara qu'il faisait le vidame d'Amiens duc et pair de
Chaulnes par une nouvelle érection. La joie extrême de la famille ne fut
pas pure\,; la cour parut consternée, et ne se contraignit pas. Un
troisième duché dans la maison d'Albert, érigé pour un cadet de l'âge du
vidame, excita des propos mortifiants\,; et ce qui les dut toucher
davantage, et qui causa une surprise générale, le Dauphin s'en expliqua
tout haut avec mesure, mais en desapprouvant nettement la grâce et ne
blâmant pas la licence qu'elle rencontrait, ce qui lui fît beaucoup
d'honneur dans le monde, et montra que ceux avec qui il vivait dans la
plus grande habitude d'estime et de confiance ne seraient pas en état
d'emporter des choses qu'il ne croirait ni justes ni raisonnables.

Qu'il me soit permis de donner ici quelques moments au futile et au
délassement, pour la singularité de la chose, d'autant qu'elle ne touche
à rien d'essentiel à qui a toujours été intimement de mes amis, et qui
d'ailleurs fut parfaitement publique. Je la raconterai ici tout de
suite, parce qu'elle ne mériterait pas la peine d'y revenir. Tout étant
consommé pour cette érection, et prêt pour la réception du nouveau duc
de Chaulnes, le parlement s'assembla à l'heure accoutumée, et les
princes du sang et les autres pairs y prirent leurs places. M. de
Chaulnes, qui devait se tenir à la porte de la grand'chambre en dedans
pour les voir arriver et les saluer, comme c'est l'ordre, n'était point
arrivé. On causait en place les uns avec les autres, et à la fin on
s'impatientait. Au bout d'une heure on soupçonna quelque accident\,; et
pour ne passer pas toute la matinée de la sorte on voulut enfin en être
éclairci. Le premier président envoya un huissier s'en informer à
l'hôtel de Luynes. Il trouva le duc de Chaulnes à qui on faisait la
barbe, qui dit qu'il s'allait dépêcher, et qui ne parut nullement
embarrassé de l'auguste séance qui l'attendait depuis si longtemps. On
peut juger du succès du rapport de l'huissier. La parure du candidat fut
encore fort longue\,; enfin il arriva d'un air riant et tranquille. Tout
était rapporté, il n'eut qu'à prêter serment, et à prendre place.

La coutume est que le premier président fait un compliment au pair
d'érection nouvelle aussitôt qu'il est assis en place, et qu'il n'en
fait point aux pairs reçus par le titre de pairie successive. Voilà donc
le premier président qui ôte son bonnet, se tourne vers la place où
était le nouveau pair, lui dit deux mots, se couvre, continue, et se
découvre et s'incline en finissant. Aussitôt M. de Chaulnes ôte son
chapeau, y glisse un papier qu'il tenait en sa main et l'y déploie, et
se met à vouloir y lire. Le pair, son voisin, le pousse et l'avertit de
mettre son chapeau\,; le Chaulnes le regarde, et sur l'avis redoublé se
couvre, et manifeste son papier en entier. Cela le déconcerte, toutefois
il se met à vouloir lire. Il répète\,: «\,Monsieur,\,» il ânonne\,; bref
il se démonte au point qu'il ne peut lire et qu'il demeure absolument
court. La compagnie ne peut s'empêcher de rire. Il la regarde tout
autour, il prend enfin son parti, il ôte son chapeau sans mot dire,
s'incline au premier président comme pour finir ce qu'il n'avait pas
commencé, regarde après encore la compagnie, et se met à rire aussi avec
elle. Voilà quelle fut la réception du duc de Chaulnes qui n'a jamais
été oubliée, parce qu'elle n'eut jamais sa pareille. Il fut le premier
après à en rire avec tout le monde.

Ménager, gros négociant, qui, par son esprit et sa capacité dans le
commerce, devint négociateur, arriva le 19 octobre de Londres à
Versailles, chez Torcy, qui le mena aussitôt trouver le roi chez
M\textsuperscript{me} de Maintenon. On sut par lui que la reine Anne
avait nommé ses trois plénipotentiaires pour la paix. Le maréchal
d'Huxelles et l'abbé de Polignac, qui depuis longtemps étaient avertis,
furent déclarés ceux du roi, et Ménager avec eux, en troisième, en égal
caractère, ce qui sembla assez étrange. Ceux d'Espagne le furent aussi,
et Bergheyek pour le second. Je ne fais que coter ces dates parce que
toute la négociation, depuis son principe jusqu'à sa fin, se trouve
parfaitement racontée dans les Pièces. Utrecht fut le lieu de
l'assemblée, et les plénipotentiaires du roi partirent bientôt après.

Nos généraux d'armée arrivèrent et furent bien reçus\,; et tôt après eux
Tallard, qui le fut aussi très-bien. Il était prisonnier en Angleterre
depuis sept ans qu'il avait été pris à la bataille d'Hochstedt, relégué
et très-observé à Nottingham, sans en pouvoir découcher, et sans avoir
pu aller à Londres ni revenir ici sur sa parole. Ce retour sans échange,
sans rançon et sans queue fut les prémices publiques de la bonne volonté
de la reine Anne. Le roi Jacques revint aussi à Saint-Germain, après
avoir employé tout l'été à voir les principales provinces du royaume,
quelques-unes de nos armées et plusieurs de nos ports.

Le samedi 7 novembre, au matin, le comte de Toulouse fut taillé fort
heureusement par Maréchal. La pierre était fort grosse et pointue, et
l'opération fut parfaite\,; elle ne fut suivie d'aucun accident, et la
guérison fut entière. Maréchal en eut dix mille écus qu'il fit
difficulté d'accepter, et que le roi lui ordonna de prendre à la fin de
la cure. Il en avait refusé deux mille de Pagon qu'il avait autrefois
taillé et parfaitement guéri, que le roi lui fit payer du sien. Le roi
était à Marly du 2 novembre\,; il avait visité souvent le comte de
Toulouse auparavant, dont il prit de grands soins. M\textsuperscript{me}
la duchesse d'Orléans et M\textsuperscript{me} la Duchesse demeurèrent
tout ce voyage à Versailles auprès de lui. Le roi, qui retourna le 15 à
Versailles, interdit le passage de la galerie et du grand appartement,
même aux princes du sang, parce que le comte de Toulouse en aurait eu du
bruit, et cela dura jusqu'à sa parfaite convalescence. Ce fut une grande
incommodité pour le commerce d'une aile à l'autre, qui ne put plus se
faire que par les cours. Le comte de Toulouse s'était préparé avec
sagesse, piété et tranquillité, et montra une fermeté très-simple. Il ne
lui en resta aucune suite, et il courut depuis le cerf comme auparavant.

M. de La Rochefoucauld perdit l'aînée de ses trois sœurs qui n'avait que
deux ans moins que lui, qui avait de l'esprit et beaucoup de mérite, de
vertu et de maintien. C'était celle qui était la plus comptée dans sa
famille et dans le monde. J'ai parlé ailleurs de ces trois filles, et de
leur vie commune dans un coin à part de l'hôtel de La Rochefoucauld, à
l'occasion de la mort de Gourville.

Sebville mourut aussi en même temps\,; il était officier général et
vieux. Il avait été envoyé du roi à Vienne et ailleurs. C'était un fort
honnête homme, et qui n'était pas sans mérite et sans talents.

En même temps mourut encore M\textsuperscript{me} de Grancey, fille du
maréchal de Grancey, qui n'avait jamais été mariée, et qui était l'aînée
de M\textsuperscript{me} de Maré, dont j'ai parlé plus d'une fois. Elle
avait été belle\,; et à son âge elle se la croyait encore, moyennant
force rouge et blanc et les parures de la jeunesse. Elle avait été
extrêmement du grand monde, fort galante, et avait longtemps gouverné le
Palais-Royal sous le stérile personnage de maîtresse de Monsieur, qui
avait d'autres goûts qu'il crut un temps masquer par là, et en effet par
le pouvoir entier qu'elle eut toujours sur le chevalier de Lorraine.
Elle ne paraissait guère à la cour qui n'était pas son terrain.
Monsieur, pour la faire appeler Madame, l'avait faite dame d'atours de
la reine d'Espagne, sa fille, qu'elle accompagna en cette qualité
jusqu'à la frontière.

La maréchale de L'Hôpital mourut aussi, célèbre par ses trois mariages
et fort vieille, retirée depuis longtemps aux petites Carmélites. Elle
s'appelait Françoise Mignot. Je ne sais si elle était fille de ce
cuisinier que Boileau a rendu célèbre pour gâter tout un repas. Elle
épousa d'abord Pierre de Portes, trésorier et receveur général du
Dauphiné. Elle avait de la beauté, de l'esprit, du manége, et des écus
qui la firent, en 1653, seconde femme du maréchal de L'Hôpital, si connu
pour avoir tué le maréchal d'Ancre, contre les défenses expresses
réitérées de Louis XIII, qui ne voulait que s'assurer de sa personne. Il
mourut dans une grande fortune en 1660. La maréchale sa veuve, qui
n'avait point d'enfants, fit si bien qu'elle épousa en troisièmes noces,
le 14 décembre 1672, en sa maison de Paris, rue des Fossés-Montmartre,
paroisse de Saint-Eustache, Jean-Casimir, successivement prince de
Pologne, jésuite, cardinal, roi de Pologne, qui avait abdiqué, s'était
retiré en France où il avait force grands bénéfices, et entre autres
l'abbaye de Saint-Germain des Prés où il logeait, et où il est enterré.
Le mariage fut su et très-connu, mais jamais déclaré\,; elle demeura
M\textsuperscript{me} la maréchale, et lui garda ses bénéfices.

L'abbé de Pomponne, revenu de son ambassade de Venise et de ses
négociations en Italie, vieillissait tristement dans le second ordre,
aumônier du roi. Cela était fâcheux à un fils et à un beau-frère de
ministres, qui n'y étaient pas accoutumés, et qui croyaient, par les
mauvais exemples récents, les premières places de l'Église faites pour
eux. Torcy, tout timide qu'il était, ne le put digérer plus longtemps.
Il n'y avait rien à reprendre aux mœurs ni à la conduite de son
beau-frère\,; mais le roi ne lui avait pas caché son invincible
répugnance à placer le nom d'Arnauld dans un siège épiscopal. Torcy se
réduisit donc à la ressource que le chancelier avait procurée à l'abbé
Bignon son neveu, que la dépravation de ses mœurs avait exclu de
l'épiscopat. La place de conseiller d'État d'Église, qu'avait le feu
archevêque de Reims, n'était pas remplie. Torcy fit encore parler le roi
sur son beau-frère, qui s'expliqua comme il avait déjà fait, lorsque
cette exclusion engagea Torcy d'employer l'abbé de Pomponne en Italie\,;
mais en même temps le roi en dit du bien et témoigna être fâché de
l'empêchement dirimant. Là-dessus Torcy tourna court sur la place de
conseiller d'État, et l'obtint sur-le-champ. L'abbé de Pomponne s'y
donna tout entier, faute de mieux, et en prit l'occasion de quitter sa
place d'aumônier du roi.

On sentit sur les huit heures du soir du 6 novembre, à Paris et à
Versailles, un tremblement de terre si léger, qu'assez peu de gens s'en
aperçurent. Il fut très-sensible vers la Touraine et le Poitou en
quelques endroits, le même jour et à la même heure, en Saxe et dans
quelques villes d'Allemagne voisines. En ce même temps on établit à
Paris une nouvelle tontine\footnote{On appelait tontine une association
  financière composée de personnes qui mettaient chacune un capital en
  commun pour en retirer une rente viagère placée sur leur tête ou sur
  celle d'autrui\,; avec la condition que l'intérêt serait réversible à
  chaque décès sur les survivants. Le nom de \emph{tontine }venait du
  Napolitain Laurent Tontin, qui avait obtenu de Louis XIII, en 1635,
  l'autorisation de fonder à Paris un établissement de ce genre.}.

Le grand prieur, qui n'avait pu obtenir la liberté du fils de Massenar,
dont il a été parlé lors de l'enlèvement du grand prieur en représailles
par le père de cet homme qui était dans Pierre-Encise, avait peu à peu
obtenu quelque liberté des Suisses\,: il vint enfin à bout de l'avoir
tout entière, et permission du roi de venir demeurer à Lyon, mais sans
approcher la cour ni Paris de plus près. Il y demeura depuis tant que le
roi vécut.

\hypertarget{chapitre-ii.}{%
\chapter{CHAPITRE II.}\label{chapitre-ii.}}

1711

~

{\textsc{Mariage du czaréwitz avec la sœur de l'impératrice régnante.}}
{\textsc{- Départ de l'archiduc pour l'Italie et l'Allemagne, qui laisse
l'archiduchesse à Barcelone avec Staremberg.}} {\textsc{- Mohnez.
Espagnol, doyen de la Rote, interdit par le pape.}} {\textsc{- Duc
d'Uzeda\,; sa maison\,; sa grandesse\,; ses emplois\,; sa défection\,;
renvoie l'ordre du Saint-Esprit.}} {\textsc{- Sa vie et sa fin
obscure.}} {\textsc{- Catastrophe, à Vienne, de son fils.}} {\textsc{-
Entrevue du duc de Savoie et de l'archiduc dans la chartreuse de
Pavie.}} {\textsc{- L'archiduc, élu empereur, reçoit à Milan les
ambassadeurs et le légat Imperiali.}} {\textsc{- Quel était ce
cardinal.}} {\textsc{- Étiquette prise d'Espagne sur les attelages.}}
{\textsc{- L'empereur à Insprück\,; y reçoit froidement le prince
Eugène.}} {\textsc{- Causes de sa disgrâce et ses suites jusqu'à sa
triste mort.}} {\textsc{- Tortose manqué par les Impériaux.}} {\textsc{-
Mariage de la fille d'Amelot avec Tavannes, qui manque la grandesse par
le roi.}} {\textsc{- Mariage du chevalier de Croissy.}} {\textsc{- Six
mille livres de pension à d'O.}} {\textsc{- Trois cent mille livres de
brevet de retenue au duc de Tresmes, à qui cela en fait cinq cent
mille.}} {\textsc{- Causes du retour du duc de Noailles et de sa secrète
disgrâce.}} {\textsc{- Embarras et fâcheuse situation du duc de Noailles
à la cour.}} {\textsc{- Noailles se jette à Desmarets.}} {\textsc{-
Noailles brouillé avec M. {[}le duc{]} et M\textsuperscript{me} la
duchesse d'Orléans, et pourquoi.}} {\textsc{- Noailles se propose de
lier avec moi.}} {\textsc{- Caractère du duc de Noailles.}} {\textsc{-
Je me laisse entraîner à la liaison du duc de Noailles.}} {\textsc{- Duc
de Noailles, brouillé avec M. {[}le duc{]} et M\textsuperscript{me} la
duchesse d'Orléans, me prie de le raccommoder avec eux.}} {\textsc{- Mes
raisons de le faire\,; j'y réussis.}} {\textsc{- Sa délicate mesure.}}
{\textsc{- Duc de Noailles me confie à sa manière la cause de son retour
d'Espagne et sa situation.}} {\textsc{- Ses vues dans cette
confidence.}} {\textsc{- Son extrême désir de m'engager à le rapprocher
du duc de Beauvilliers, conséquemment du Dauphin.}} {\textsc{- Mes
raisons de le faire\,; j'y réussis.}} {\textsc{- Ma liaison avec le
cardinal de Noailles, qui devient intime jusqu'à sa mort.}} {\textsc{-
Scélératesse du complot des jésuites contre le cardinal de Noailles mise
au net par le paquet de l'abbé de Savoie à son oncle l'évêque de
Clermont, tombé entre les mains du cardinal de Noailles, qui n'en sait
pas profiter.}} {\textsc{- Cris publics.}} {\textsc{- Le Dauphin ne se
cache pas sur son avis de chasser le P. Tellier, et me le dit.}}
{\textsc{- Affaire du cardinal renvoyée en total au Dauphin pour la
finir.}} {\textsc{- Grand mot qu'il me dit en faveur du cardinal.}}
{\textsc{- Il m'ordonne de m'instruire à fond sur les matières des
libertés de l'Église gallicane et sur l'affaire du cardinal de Noailles,
et me dit qu'il la veut finir définitivement avec moi.}}

~

Le czar, à peine sorti d'entre les mains des Turcs, conclut le mariage
du fils unique qu'il avait de sa première femme qu'il avait répudiée, et
qu'il tenait dans un couvent, avec la deuxième petite-fille du vieux duc
Ulric de Wolfenbuttel, sœur de l'archiduchesse qu'on va voir
impératrice. Le czar le conclut à Carlsbad où il prenait les eaux, d'où
il partit pour l'aller voir célébrer à Torgau\,; ce fut un funeste
mariage.

L'archiduc qui, depuis longtemps, n'avait plus de pensées que d'aller
recueillir la vaste succession de l'empereur son frère, se revoir avec
l'impératrice sa mère, dont il avait toujours été le mieux aimé, et se
retrouver chez soi dans Vienne, libre des inquiétudes et des étrangers
parmi lesquels il était comme banni, et régner dans les mêmes lieux où
il n'avait vécu qu'en servitude, eut peine à se tirer des mains des
Catalans. Il leur laissa pour vice-roi le comte de Staremberg, général
de ses troupes, qui lui avait été donné pour conseil et pour
conducteur\,; qu'il avait pris en grande estime et amitié, et qui la
méritait\,; La Corsana, comme ministre castillan, et Perlas, qui était
devenu son favori, comme secrétaire d'État et ministre catalan. Il fit
espérer son retour à la ville de Barcelone et à tout son parti en
Espagne, et mit enfin à la voile, suivi de trois députés catalans,
nommés Corbellone, Pinos et Cardone. Sa flotte était de quarante ou
cinquante bâtiments de toutes sortes, anglais, hollandais et catalans.
Il ne put emmener l'archiduchesse\,; il aurait désespéré les Catalans
qui s'opiniâtrèrent à la garder à Barcelone comme le gage de son retour
et le centre des affaires, à la tête desquelles il la mit pour la forme
en son absence. Leur mariage était, et fut toujours depuis extrêmement
uni, chose si rare parmi les princes, et la séparation leur coûta
beaucoup.

Depuis que les hauteurs du marquis de Prié, ambassadeur du feu empereur
à Rome, du temps que le maréchal de Tessé y était, avaient, comme je
l'ai raconté alors, forcé le pape à reconnaître l'archiduc en qualité de
roi d'Espagne, par les violences qu'il fit exercer par les troupes
impériales dans les États de l'Église, il n'y avait plus de nonce à
Madrid, qui en avait été chassé, ni d'ambassadeur d'Espagne à Rome.
Molinez, doyen de la Rote, qui en était auditeur pour la Castille, était
le seul ministre d'Espagne à Rome, où il était fort considéré. Le bruit
confirmé du prochain départ de l'archiduc de Barcelone pour l'Italie fit
parler à Rome de lui envoyer un légat comme roi d'Espagne, sans attendre
qu'il fût élu empereur. Molinez en parla aux ministres, puis au pape,
qui à la fin lui avoua que la résolution en était prise. Molinez,
très-attaché à Philippe V, ne se rebuta point, et n'oublia aucune des
raisons qui pouvaient détourner ce qu'il appelait un affront fait au roi
son maître\,; à la fin il pressa si vivement le pape, et lui parla si
haut, que le pontife se fâcha\,; et, pour se défaire de ses
remontrances, l'interdit de toutes ses fonctions, et alla même jusqu'à
lui défendre de dire la messe. Cette affaire fit grand bruit dans toute
l'Europe, et même Rome, neutre, ne l'approuva pas. Molinez se tint chez
lui fort visité, par l'estime qu'il avait acquise, et n'en sortit plus
jusqu'à ce qu'il eût reçu des ordres de Madrid. Le roi s'en plaignit
fort à Rome et de la chose et de la cause\,; mais le parti y était pris,
et cette cour n'était pas pour reculer.

Le duc d'Uzeda était ambassadeur d'Espagne à Rome\,: il était de cette
grande et nombreuse maison d'Acuña y Pacheco, de laquelle sont aussi les
marquis de Villena et ducs d'Escalone, comte de San-Estevan de Gormaz,
les ducs d'Ossone, les comtes de Montijo, le marquis de Bedmar
d'aujourd'hui, et ce vieillard illustre le marquis de Mancera, dont j'ai
parlé plus d'une fois, tous grands d'Espagne de première classe, et tous
fort grands seigneurs. Uzeda fut érigé en duché, et donné par Philippe
III au fils aîné du duc de Lerme, son premier ministre, mort cardinal et
disgracié, en faisant ce fils grand d'Espagne\,; cette grandesse tomba
de fille en fille. La dernière qui en hérita était fille du cinquième
duc d'Ossone, qui la porta en mariage à un cadet de sa même maison, qui
s'appelait le comte de Montalvan, et qui prit, en se mariant, le nom et
le rang de duc d'Uzeda. Il fut gentilhomme de la chambre, gouverneur et
capitaine général de Galice, puis vice-roi de Sicile, d'où il passa à
l'ambassade de Rome, où il logea Louville, lorsque Philippe V, étant à
Naples, l'envoya remercier le pape de lui avoir envoyé un légat. Le duc
d'Uzeda fut fait chevalier du Saint-Esprit avec les premiers grands
espagnols, qui le reçurent peu de temps après, et le dut à la bonne
réception qu'il fit à Louville, qu'il persuada fort de son attachement
pour Philippe V, qui était vrai alors. Mais la décadence de ses affaires
en Italie, et la chute du duc de Medina-Celi dans l'alliance et l'intime
confidence duquel il était, le jetèrent secrètement dans le parti
d'Autriche auquel il se lia\,; et sorti de Rome lorsque cette cour
reconnut l'archiduc roi d'Espagne, il s'arrêta en Italie d'abord par la
difficulté du passage pour retourner en Espagne\,; {[}ce{]} qui après
son changement secret lui servit de prétexte à demeurer en Italie, qui
ne fut pas si spécieux qu'il ne donnât beaucoup de soupçon de sa
conduite, et après de sa fidélité par son opiniâtre désobéissance aux
ordres souvent réitérés de se rendre en Espagne, et il fut fort accusé
d'avoir fait manquer une entreprise pour reprendre la Sardaigne, il y
avait deux ans, dont il avait le secret.

Le passage de l'archiduc par l'Italie, fut l'occasion qu'il prit de
lever le masque. Ce prince arriva le 12 octobre à Saint-Pierre d'Arena,
faubourg de Gênes, où cette république le reçut superbement. Le duc
d'Uzeda renvoya au roi l'ordre du Saint-Esprit, alla trouver et
reconnaître publiquement l'archiduc à Gênes, comme roi d'Espagne et
comme son souverain, et reçut de lui, comme tel, l'ordre de la Toison
d'or. Il y perdit ses biens d'Espagne, et n'en fut point récompensé par
la cour de Vienne, qui le laissa languir pauvre et méprisé en Italie.
Lassé au bout de quelques années de ne pouvoir rien obtenir, il s'en
alla avec sa famille à Vienne où il éprouva de plus près le même
abandon. Il y est mort avec le vain titre de président du conseil
d'Espagne, qui n'avait rien à administrer puisque la paix était faite,
et que l'empereur y avait renoncé et reconnu Philippe V. Son fils, duc
d'Uzeda après lui, demeura à Vienne et y a fini enfin
très-malheureusement en prison, sur des soupçons étranges, et sans qu'on
ait oui parler de lui depuis qu'il fut arrêté.

Le duc de Savoie, fort mécontent comme on l'a vu du feu empereur, se
flatta de tirer un meilleur parti de l'archiduc, et voulut le voir à son
passage\,; il en obtint une audience à jour nommé dans la chartreuse de
Pavie par où ce prince, allant à Milan, passa incognito sous le nom de
comte de Tyrol.

Il apprit à Milan qu'il avait été le 12 octobre, élu empereur à
Francfort par toutes les voix, excepté celles de Cologne et de Bavière
qui n'y avaient pas été admises, parce que ces deux électeurs étaient au
ban de l'empire\,; le nouvel empereur en prit aussitôt la qualité. Milan
se surpassa à le magnifiquement recevoir. Il y donna audience au
cardinal Imperiali, légat \emph{a latere}, avec beaucoup de pompe.
C'était un des plus accrédités du sacré collége, qui avait le plus de
poids et de part aux affaires\,; un des plus capables et des plus
\emph{papables}, avec de l'honneur, des lettres et une grande décence\,;
riche, magnifique, mais suspect à la France pour être fils de ce doge de
Gênes qui, après le bombardement, fut obligé de venir, étant toujours
doge, demander pardon au roi, accompagné de quatre sénateurs, et qui
trouva moyen de s'acquitter avec esprit et dignité d'une fonction si
humiliante, et de plaire et se faire estimer de tout le monde. Son fils,
quoique fort sage et mesuré, n'avait pas oublié ce voyage, et on sentait
trop aisément, pour ses espérances au pontificat, qu'il était fort
ennemi de la France et fort autrichien, ce qui lui coûta l'exclusion de
la France et la tiare que le conclave suivant fut d'accord de lui
déférer. Les ambassadeurs de Savoie, Venise et Gênes eurent aussi leur
audience\,; mais ils eurent ordre de n'y venir qu'en des carrosses à
quatre chevaux\,; ce fut apparemment pour soutenir le caractère du roi
d'Espagne qui seul va où il est à six chevaux ou mules, et les
ambassadeurs, cardinaux, grands, n'en peuvent avoir que quatre.
L'audience fut constamment refusée à l'ambassadeur du grand-duc qui, à
son gré, s'était montré trop favorable aux deux couronnes. Tout ce qu'il
y eut d'illustre en Italie s'empressa d'aller faire sa cour à Milan.

L'archiduc alla droit de Milan à Insprück, où il s'arrêta et où le
prince Eugène s'était rendu pour le saluer\,; l'accueil fut médiocre
pour un homme de la naissance, des services et de la réputation de ce
grand et heureux capitaine\,; il était particulièrement aimé et estimé
du feu empereur, dont il avait toute la confiance. Ce prince capricieux
n'avait jamais aimé ni bien traité l'archiduc son frère. Celui-ci avait
sans cesse manqué de tout en Espagne de la part de la cour de Vienne\,;
il s'en prenait au prince Eugène, qui pouvait tout sur ces sortes de
dispositions, et surtout il ne lui avait point pardonné son refus
opiniâtre de venir conduire et pousser la guerre d'Espagne. Staremberg,
qui n'aimait point le prince Eugène par des intrigues de cour et des
suites de partis opposés, souffrait impatiemment les manquements
d'argent et de toutes choses qui l'assujettissaient pour tout aux
Anglais, et qui ôtaient à Staremberg les moyens et les occasions de se
signaler, d'élever sa gloire et sa fortune. Il en était piqué contre le
prince Eugène, et s'en était vengé en aliénant de lui l'archiduc Eugène,
qui sentait sa situation avec ce prince, ne se rassurait ni sur ses
lauriers ni sur le besoin qu'il avait de lui. Il ne craignait pas tant
pour ses emplois que pour l'autorité avec laquelle il s'était accoutumé
à les exercer. Il avait des ennemis puissants à Vienne, car le mérite,
surtout grandement récompensé, est toujours envié. C'est ce qui le hâta
d'aller trouver l'archiduc encore en voyage, avant que ceux de la cour
de Vienne l'eussent joint. Néanmoins ses soumissions, ses protestations,
les éclaircissements où il s'efforça d'entrer ne purent fondre les
glaces qu'il trouva consolidées pour lui dans l'archiduc, et c'est ce
qui lui donna un nouveau degré de chaleur pour la continuation de la
guerre, pour perpétuer le besoin de soi et pour éloigner un temps de
paix où il se verrait exposé à mille dégoûts à Vienne, où il avait régné
jusqu'alors présent et absent, et c'est ce qui le précipita dans ce
déshonorant voyage d'Angleterre, où il fit un si étrange personnage, et
qui se voit si bien dans la description qui s'en trouve dans les Pièces,
à propos des négociations de la paix.

Le peu de satisfaction qu'il eut à Insprück lui annonça à quoi il devait
s'attendre. La paix faite, il vécut à Vienne de dégoûts, sous une
considération apparente, dans les premières places du militaire et du
civil, sous lesquelles enfin, avec les années, son esprit succomba
plutôt que sa santé, et le précipita à chercher et à trouver la fin de
sa vie, ce que j'ai voulu dire ici en deux mots, parce que cet événement
dépasse de beaucoup le terme que je me suis proposé de donner à ces
Mémoires. Le prince Eugène cacha comme il put son chagrin, quitta
Insprück promptement pour retourner en Hollande mettre obstacle de tout
son crédit à la paix, et aller essayer d'étranges choses en Angleterre
pour y remettre à flot Marlborough à la guerre, où il ne recueillit que
de la honte et du mépris. C'est ainsi qu'on voit quelquefois qu'au lieu
de se plaindre que la vie est trop courte, il arrive à de grands hommes
de vivre beaucoup trop longtemps. L'archiduc devait partir d'Insprück
pour arriver à Francfort le 18 et y être couronné empereur le 23.

Pendant ce temps-là Staremberg entreprit de prendre Tortose sur quelque
intelligence qu'il y avait. Il en fit approcher trois mille hommes si
diligemment et si secrètement, qu'ils attaquèrent la place par trois
différents endroits la nuit et en même temps, sans qu'on s'y attendît.
Le gouverneur était à l'armée de M. de Vendôme. Le lieutenant du roi se
défendit si bien, qu'avec une très-médiocre garnison il les rechassa de
leurs trois attaques, reprit le chemin couvert dont ils s'étaient rendus
maîtres, leur tua plus de cinq cents hommes, leur en prit autant, et les
poursuivit quelque temps dans leur retraite.

Amelot maria sa fille à Tavannes, l'aîné de la maison, qui depuis a
commandé longtemps en Bourgogne, et dont le frère est devenu évêque,
comte de Châlons, archevêque de Rouen et grand aumônier de la reine.
Amelot, illustre par le succès de ses ambassades, et adoré en Espagne,
n'avait eu aucune récompense de ses travaux, que la charge de président
à mortier pour son fils après tant de réputation et de si justes
espérances. Il tenta la grandesse dont sa robe l'excluait, pour
Tavannes, en épousant sa fille. Il y trouva toute la facilité à laquelle
il devait s'attendre de la cour d'Espagne, que M\textsuperscript{me} des
Ursins gouvernait si despotiquement. Mais le roi n'y voulut jamais
consentir. Ce n'était plus ici le temps d'Amelot. Son mérite avait trop
effrayé malgré sa sagesse et sa modestie. J'ai expliqué cette anecdote
lors de son retour d'Espagne.

Torcy maria aussi, ou laissa marier son frère à une fille de Brunet,
riche financier, qui de chevalier de Croissy devint comte de Croissy.

D'O, comme devenu menin du Dauphin, eut six mille livres de pension, et
le duc de Tresmes trois cent mille livres de brevet de retenue sur sa
charge de premier gentilhomme de la chambre\,; il en avait déjà un de
deux cent mille livres, tellement qu'il en eut cinq cent mille livres.

Il est temps de revenir au duc de Noailles. On a vu que, n'y ayant plus
rien à faire pour lui en Catalogne, ses troupes avaient passé à l'armée
de M. de Vendôme, et lui, dès le commencement de mars, à Saragosse où
était la cour d'Espagne, destiné lui-même à servir sous les ordres de ce
général. La faiblesse et les manquements de quantité de choses tinrent
toute cette campagne les armées oisives, à quelques légères entreprises
près, qui ne troublèrent point la paresse de Vendôme, qui était dans ses
quartiers avec toute son armée, ni la cour assidue de Noailles qui
demeura toujours auprès du roi d'Espagne à Saragosse et à Corella.
L'ambition de gouverner, facilitée de la considération et des accès que
le neveu de M\textsuperscript{me} de Maintenon trouvait dans une cour
qu'il avait déjà fort pratiquée, jointe à celle que lui donnait son
emploi dans l'armée, dont il en avait commandé une en chef, et ses
liaisons intimes avec M. de Vendôme dont on a vu en son temps l'origine,
engagèrent le duc de Noailles à une folie et à tenter ce qui ne pouvait
que le perdre, au lieu de se contenter des prospérités les plus
flatteuses dont il jouissait avec solidité.

Il trouva à Saragosse le marquis d'Aguilar, duquel j'ai parlé plus d'une
fois, qui avait quitté la charge de colonel du régiment des gardes
espagnoles, pour celle de capitaine de {]}a première compagnie des
gardes du corps espagnole qui l'approchait davantage du roi. Tous deux
s'étaient connus aux voyages précédents que le duc de Noailles avait
faits près du roi d'Espagne. Tous deux s'étaient plu. Ils avaient lié
ensemble une amitié conforme à leur génie, à leur esprit, à leur
caractère qui était parfaitement homogène. Je ne sais lequel des deux
imagina le projet, mais il est certain que tous deux l'embrassèrent,
agirent d'un grand concert, et n'oublièrent rien pour un succès qu'ils
crurent les devoir porter à devenir en Espagne les maîtres de la cour et
de l'État.

La reine était attaquée des écrouelles qui la conduisirent enfin au
tombeau. Son mal l'empêchait de suivre le roi aux chasses continuelles
et aux promenades, la tenait encore dans la retraite de son appartement,
dans d'autres temps qu'elle passait auparavant avec le roi, la rendait
particulière et beaucoup moins accessible au public, et l'obligeait à
une coiffure embéguinée, qui lui cachait la gorge et une partie du
visage. Les deux amis n'ignoraient pas que le roi ne pouvait se passer
d'une femme, et qu'il était accoutumé à s'en laisser gouverner. Ils se
persuadèrent que l'empire dont la princesse des Ursins jouissait n'était
fondé que sur celui que la reine avait pris sur le roi\,; que si elle le
perdait la camarera mayor tomberait avec elle\,; et, jugeant du roi par
eux-mêmes, ils ne doutèrent pas de se servir utilement du mal de la
reine pour en dégoûter le roi. Ce grand pas fait, ils avaient résolu de
lui donner une maîtresse, et se flattèrent que sa dévotion céderait à
ses besoins. Avec une maîtresse de leurs mains qui aurait un continuel
besoin d'eux en conseil et en appuis pour se soutenir elle-même, ils
comptèrent de la substituer à la reine auprès du roi, et de devenir
eux-mêmes dans la cour et dans la monarchie ce qu'y était la princesse
des Ursins.

Ce pot au lait de la bonne femme, et qui en eut aussi le sort, ne fait
pas honneur aux deux têtes qui l'entreprirent, moins encore à un
étranger si grandement, si agréablement et si prématurément établi dans
son pays. Ils commencèrent aussitôt à travailler à cette entreprise. Ils
profitèrent de tous les moments de s'insinuer de plus en plus dans la
familiarité du roi. Aguilar avait été ministre de la guerre\,; il
s'était aussi mêlé des finances. Noailles, par son commandement et par
son personnel en notre cour, n'avait pas moins d'occasion et de matière
que l'autre d'entrer en des conversations importantes et suivies avec le
roi, secondés qu'ils étaient de la faveur de la reine et de l'appui de
M\textsuperscript{me} des Ursins, auxquelles ils faisaient une cour
d'autant plus assidue et plus souple qu'ils avaient plus d'intérêt de
leur cacher ce qu'ils méditaient contre elles. Cela dura ainsi pendant
tout le séjour de Saragosse, où ils ne songèrent qu'à s'établir
puissamment dans la confiance du roi. Le voyage de Corella, qui fit une
légère séparation de lieu du roi et de la reine, leur parut propre à
entamer leur dessein. Ils prirent le roi par le faible qu'ils lui
connaissoient sur sa santé, et lui firent peur, sous le masque
d'affection et de l'importance dont sa santé et sa vie étaient à l'État,
de gagner le mal de la reine, en continuant de coucher avec elle, et
poussèrent jusqu'à l'inquiéter d'y manger. Ce soin pour sa conservation
fut assez bien reçu pour leur donner espérance\,; ils continuèrent, elle
augmenta\,; ils poussèrent leur pointe\,; ils plaignirent le roi sur ses
besoins\,; ils battirent la campagne sur la force et les raisons de
nécessité\,; en un mot, ils lui proposèrent une maîtresse. Tout allait
bien jusque-là, mais ce mot de maîtresse effaroucha la piété du roi, et
les perdit. Il les écarta doucement, ne les écouta plus que sur d'autres
matières, ne leur parla plus avec ouverture. Sa contrainte et sa réserve
avec eux leur fut un présage funeste qu'ils ne purent détourner.

Dès que le roi se retrouva entre la reine et M\textsuperscript{me} des
Ursins, il leur raconta la belle et spécieuse proposition qui lui avait
été faite par deux hommes, qu'elles lui vantaient incessamment, et
qu'elles se croyaient si attachés. On peut juger de l'effet du récit.
Toutefois il n'y parut pas au dehors\,; elles voulurent s'assurer de
leur vengeance. La reine en écrivit à la Dauphine avec la dernière
amertume, et la princesse des Ursins à M\textsuperscript{me} de
Maintenon, avec tout l'art dans lequel elle était si grande maîtresse.
Quelque intérieurement irrités que le roi et M\textsuperscript{me} de
Maintenon fussent de la souveraineté que M\textsuperscript{me} des
Ursins entreprenait de se faire, colère dont il n'est pas encore temps
de parler qu'en passant, ils se sentirent piqués jusqu'au vif.

Le roi blessé du côté de la religion, de l'ambition, de la hardiesse\,;
M\textsuperscript{me} de Maintenon de celui de la toute-puissance
qu'elle croyait exercer en Espagne par la princesse des Ursins qui était
son endroit le plus sensible\,; tous deux de l'ingratitude, et de ce
qu'ils appelèrent avec la Dauphine la perfidie d'un homme comblé en un
tel âge, et à un tel excès, de biens, de charges et de dignités, de
grands emplois, de distinctions, de toutes les sortes de faveur et de
leur confiance, duquel ils se croyaient les plus assurés, et qui en
abusait avec une telle audace. L'amitié, l'amusement, la confiance
entière que M\textsuperscript{me} de Maintenon avait surtout prise en ce
neveu qu'elle regardait comme son fils, comme son ami, quelquefois comme
son conseil, et comme ne faisant qu'un avec elle, et ne pouvant avoir
d'autres intérêts que les siens, fit dans son cœur une blessure profonde
qui, à force de temps et de changements de choses, parut guérie à
l'extérieur\,; mais ne le fut jamais dans le fond ni pour l'amitié, ni
pour l'estime, ni pour la confiance, et laissa jusqu'à la fin de sa vie
un fâcheux malaise entre eux. La Dauphine, toujours investie par les
Noailles, qui avait goûté l'esprit de badinage, et quelquefois de
sérieux, du duc de Noailles, et à qui, pour plaire à
M\textsuperscript{me} de Maintenon, elle avait laissé prendre un accès
auprès d'elle, et une familiarité publique qui n'avait jamais été
permise qu'à lui, et qui le regardait comme un ami, n'en fut que plus
blessée contre lui, pour la reine sa sœur, qu'elle aimait beaucoup et
avec qui elle était dans un continuel commerce. Elle sut un gré infini à
M\textsuperscript{me} de Maintenon de prendre l'affaire si amèrement
contre un homme si proche à qui elle était si accoutumée\,; et
M\textsuperscript{me} de Maintenon à elle de lui voir porter l'intérêt
de sa sœur avec tant de vivacité. Ce groupe secret, intime, suprême, ne
fit donc que s'échauffer et s'irriter mutuellement, et le Dauphin y
entra en quart, au point où il était avec eux, dans l'horreur d'une
action pour ce monde si folle, et pour la religion si criminelle. Les
réponses en Espagne ne tardèrent pas, dont la force fut pleinement au
gré de la reine d'Espagne et de la princesse des Ursins.

Le duc de Noailles eut par la même voie un ordre sec et précis de
revenir sur-le-champ à la réception de ces lettres. L'extérieur,
parfaitement gardé jusque-là, n'eut plus de ménagement. Aguilar reçut
ordre de donner sur l'heure la démission de sa charge, qui fut à
l'instant donnée au comte de San-Estevan de Gormaz, grand d'Espagne par
sa femme, et fils du marquis de Villena, desquels j'ai parlé ailleurs,
et en même temps de partir sur-le-champ pour sa commanderie, où il fut
relégué quelque temps. Le duc de Noailles, dans le très-peu de jours
qu'il mit à arranger son voyage, ne trouva plus que des portes fermées
et des visages qui le furent encore plus. Il arriva, comme je l'ai dit,
à Versailles le surlendemain du retour de Fontainebleau, et salua le roi
chez M\textsuperscript{me} de Maintenon, qui, pour le public, l'y
voulurent voir comme ils l'y avaient toujours vu à ses retours. Mais la
réception y fut étrangement courte et différente. On ne tarda pas à
s'apercevoir au sec du roi pour lui, à sa retenue et à son embarras avec
le roi, avec le Dauphin, et surtout avec la Dauphine, qu'il y avait
quelque chose de grave et de fort extraordinaire sur son compte, car on
n'avait pas encore pénétré qu'il eût eu ordre de revenir, ni la cause
encore moins. Les dames de l'intérieur remarquèrent qu'elles le
rencontrèrent bien plus rarement chez M\textsuperscript{me} de
Maintenon, et que dans ce peu qu'elles l'y voyaient la contrainte et
l'embarras du neveu, le sec et le bref de la tante, sautaient aux yeux,
et faisaient un contraste entier avec les manières que jusqu'alors elles
leur avaient toujours vues ensemble. Ces choses toujours continuées
percèrent peu à peu. Elles excitèrent toute la curiosité, et bientôt
après on sut, mais parmi les plus instruits seulement, la cause de la
disgrâce que j'appris des premiers par ces dames du palais, à qui la
Dauphine s'ouvrait volontiers.

Le duc de Noailles, également occupé à cacher une situation si fâcheuse,
et à y chercher des ressources, s'y trouva étrangement embarrassé\,; les
siennes naturelles et qui l'avaient si rapidement mené, lui devenaient
inutiles\,: M\textsuperscript{me} de Maintenon, blessée au cœur par son
plus cher intérêt\,; le roi par la chose même, et par le dépit de s'être
si lourdement mépris à prodiguer ses grâces les plus signalées\,; la
Dauphine offensée pour la reine sa sœur, pour elle-même, et qui se
piquait encore de l'être\,; le Dauphin, dans l'extrême piété dont il
était, contre tous les principes duquel il se trouvait surpris. Sa
famille si brillante, si établie, si nombreuse, outrée contre lui de
s'être perdu ainsi, comme de gaieté de cœur, ne pouvait rien en sa
faveur. Sa mère d'excellent conseil n'avait jamais eu qu'un manége qui
avait toujours tenu le roi et M\textsuperscript{me} de Maintenon en
garde contre elle, même assez peu décemment. Sa femme, une folle qui,
toute nièce unique qu'elle était de M\textsuperscript{me} de Maintenon,
lui était devenue pesante à l'excès, et qui, loin d'oser lui ouvrir la
bouche, ne la voyait que par mesure, et presque toujours pour en être
grondée, sans liaison en aucun temps avec la Dauphine, sans
considération dans le monde, qu'on ne lui avait jamais laissé voir que
par le trou d'une bouteille. Son oncle perdu avec M\textsuperscript{me}
de Maintenon, et fort avancé de l'être près du roi. Ses trois sœurs,
dames du palais, et fort bien avec la Dauphine, mais la Dauphine hors de
mesure d'écouter rien. Nul seigneur en charge à qui il pût ou voulût
avoir recours, et pour les ministres, son cas n'était pas graciable
auprès de gens à principes et de la haute piété des ducs de Chevreuse et
de Beauvilliers, et fils et neveu de gens dont le premier ne pouvait lui
attirer leur grâce, et l'autre, quoi qu'il eût fait pour conserver au
duc de Beauvilliers ses places aux dépens de son propre frère, n'en
était pas moins pour eux l'ennemi fatal de l'archevêque de Cambrai.

L'évêque de Meaux n'était pas assez simple pour s'ingérer de raccommoder
avec M\textsuperscript{me} de Maintenon le neveu de celui qui le voulait
perdre. Il en était de même de La Chétardie, son directeur, et du P.
Tellier auprès du roi. Voysin, vil esclave de M\textsuperscript{me} de
Maintenon, ne se serait pas hasardé à lui déplaire. Pontchartrain
malfaisant et sans crédit ni volonté\,; le chancelier se sentait les
reins trop rompus\,; Torcy était la timidité même. Desmarets parut au
duc de Noailles le seul dont il pût espérer secours. Desmarets était un
sanglier tellement enfoncé dans sa bauge, qu'il ignorait presque tout ce
qui se passait hors de sa sphère. Il ne comptait et ne croyait qu'en
M\textsuperscript{me} de Maintenon\,: il ne se douta seulement pas de la
situation du duc de Noailles. Il se trouva donc flatté de le voir se
jeter à lui\,; et s'il la sut bien longtemps depuis, il se trouva
tellement lié qu'il ne put s'en défaire ou qu'il ne l'osa. C'était donc
tenir à quelqu'un que cette liaison si prompte que saisit le duc de
Noailles. Il la cultiva d'assiduité, de flatteries, et de souplesses\,;
un contrôleur général, ministre et accrédité était toujours bon à avoir
pour qui surtout n'avait personne, en attendant qu'il vît jour à se
servir de lui pour le raccommoder, ce qui néanmoins ne se trouva pas.

M. de Noailles, qui avait été fort bien avec M. {[}le duc{]} et
M\textsuperscript{me} la duchesse d'Orléans, était brouillé avec eux
pour l'affaire de Renaut, qu'il lui avait donné, et qu'il avait eu
auparavant à lui, et pour des tracasseries avec M\textsuperscript{me} la
duchesse. Dans son état florissant, il s'en serait, je crois, peu
soucié, mais dans celui où il se trouvait, les miettes mêmes lui
semblaient aiguës, il aurait voulu au moins les ramasser. Ma liaison
intime avec eux était publique\,; je passais pour l'ami de cœur et de
confiance la plus totale du duc de Beauvilliers, et même du duc de
Chevreuse\,: on n'ignorait pas que j'étais au même point avec le
chancelier. Ce qui se passait de secret et d'intime entre le Dauphin et
moi ne se savait pas, mais on était en grand soupçon sur moi de ce
côté-là par le chausse-pied du duc de Beauvilliers, par l'air et les
manières qui échappaient pour moi au Dauphin, quand je paraissais devant
lui en public, par les entretiens tête à tête qu'il avait souvent dans
le salon de Marly avec M\textsuperscript{me} de Saint-Simon, et dans
leurs parties où elle se trouvait presque toujours\,; ni lui ni la
Dauphine ne se contraignaient plus sur le désir de la voir succéder à la
duchesse du Lude, et d'une manière encore que celle-ci, qui le savait et
en parlait, ne pouvait en être peinée. Le roi et le monde la traitaient
avec une distinction marquée de tout temps, et qui augmentait
toujours\,; je l'étais bien du roi, et le monde avait les yeux fort
ouverts sur moi. Tout cela apparemment persuada au duc de Noailles que,
pour un temps ou pour un autre, j'étais un homme qu'il fallait gagner,
et il ne fut pas quinze jours de retour qu'il commença à dresser vers
moi ses batteries.

Le duc de Noailles maintenant arrivé au bâton, au commandement des
premières armées et au ministère, va désormais figurer tant, et en tant
de manières, qu'il serait difficile d'aller plus loin avec netteté sans
le faire connaître, encore qu'il soit plein de vie et de santé, et qu'il
ait trois ans moins que moi. C'est un homme né pour faire la plus grande
fortune quand il ne l'aurait pas trouvée toute faite chez lui. Sa taille
assez grande mais épaisse, sa démarche lourde et forte, son vêtement uni
ou tout au plus d'officier, voudraient montrer la simplicité la plus
naturelle\,; il la soutient avec le gros de ce que, faute de meilleure
expression, on entend par une apparence de sans façon et de camarade. On
a rarement plus d'esprit et plus de toutes sortes d'esprit, plus d'art
et de souplesse à accommoder le sien à celui des autres, et à leur
persuader, quand cela lui est bon, qu'il est pressé des mêmes désirs et
des mêmes affections dont ils le sont eux-mêmes, et pour le moins aussi
fortement qu'eux, et qu'il en est supérieurement occupé. Doux quand il
lui plaît, gracieux, affable, jamais importuné quand même il l'est le
plus\,; gaillard, amusant\,: plaisant de la bonne et fine plaisanterie,
mais d'une plaisanterie qui ne peut offenser\,; fécond en saillies
charmantes\,; bon convive, musicien\,; prompt à revêtir comme sien tous
les goûts des autres, sans jamais la moindre humeur\,; avec le talent de
dire tout ce qu'il veut, comme il veut, et de parler toute une journée
sans toutefois qu'il s'en puisse recueillir quoi que ce soit, et cela
même au milieu du salon de Marly, et dans les moments de sa vie les plus
inquiets, les plus chagrins, les plus embarrassants. Je parle pour
l'avoir vu bien des fois sachant ce qu'il m'en avait dit lui-même, et
lui demandant après, dans mon étonnement, comment il pouvait faire.

Aisé, accueillant, propre à toute conversation, sachant de tout, parlant
de tout, l'esprit orné, mais d'écorce\,; en sorte que sur toute espèce
de savoir force superficie, mais on rencontre le tuf pour peu qu'on
approfondisse, et alors vous le voyez maître passé en galimatias de
propos délibéré. Tous les petits soins, toutes les recherches, tous les
avisements les moins prévus coulent de source chez lui pour qui il veut
capter, et se multiplient, et se diversifient avec grâce et gentillesse,
et ne tarissent point, et ne sont point sujets à dégoûter. Tout à tous
avec une aisance surprenante, et n'oublie pas dans les maisons à plaire
à certains anciens valets. L'élocution nette, harmonieuse, toutefois
naturelle et agréable\,; assez d'élégance, beaucoup d'éloquence, mais
qui sent l'art, comme avec beaucoup de politesse et de grâce dans ses
manières, elles ne laissent pas de sentir quelque sorte de grossièreté
naturelle\,; et toutefois des récits charmants, le don de créer des
choses de riens pour l'amusement, et de dérider et d'égayer même les
affaires les plus sérieuses et les plus épineuses, sans que tout cela
paroisse lui coûter rien.

Voilà sans doute bien de l'agréable et de grands talents de cour\,;
heureux s'il n'en avait point d'autres. Mais les voici\,: tant d'appas,
d'esprit de société, de commerce\,; tant de piéges d'amitié, d'estime,
de confiance, cachent presque tous les monstres que les poëtes ont
feints dans le Tartare\,; une profondeur d'abîme, une fausseté à toute
épreuve, une perfidie aisée et naturelle accoutumée à se jouer de
tout\,: une noirceur d'âme qui fait douter s'il en a une, et qui assure
qu'il ne croit rien\,; un mépris de toute vertu de la plus constante
pratique\,; et tour à tour, selon le besoin et les temps, la débauche
publique abandonnée, et l'hypocrisie la plus ouverte et la plus suivie.
En tous ces genres de crimes, un homme qui s'étend à tout, qui
entreprend tout, qui, pris sur le fait, ne rougit de rien, et n'en
pousse que plus fortement sa pointe\,; maître en inventions et en
calomnies, qui ne tarit jamais, et qui demeure bien rarement court\,;
qui se trouvant à découvert et dans l'impuissance, se reploie prestement
comme les serpents, dont il conserve le venin parmi toutes les bassesses
les plus abjectes dont il ne se lasse point, et dont il ne cesse
d'essayer de vous regagner dans le dessein bien arrêté de vous
étrangler\,; et tout cela sans humeur, sans haine, sans colère, tout
cela à des amis de la plus grande confiance, dont il avoue n'avoir
jamais eu aucun lieu de se plaindre, et auxquels il ne nie pas des
obligations du premier ordre. Le grand ressort d'une perversité si
extrêmement rare est l'ambition la plus démesurée, qui lui fait tramer
ce qu'il y a de plus noir, de plus profond, de plus incroyable, pour
ruiner tout ce qu'il y craint d'obstacles, et tout ce qui peut, même
sans le vouloir, rendre son chemin moins sûr et moins uni. Avec cela une
imagination également vaste, fertile, déréglée, qui embrasse tout, qui
s'égare partout, qui s'embarrasse et qui sans cesse se croise elle-même,
qui devient aisément son bourreau, et qui est également poussée par une
audace effrénée, et contrainte par une timidité encore plus forte, sous
le contraste desquelles il gémit, il se roule, il s'enferme\,; il ne
sait que faire, que devenir, et {[}sa timidité{]} protége néanmoins
rarement contre ses crimes.

En même temps, avec tout son esprit, ses talents, ses connaissances,
l'homme le plus radicalement incapable de travail et d'affaires. L'excès
de son imagination, la foule de vues, l'obliquité de tous les desseins
qu'il bâtit en nombre tous à la fois, les croisières qu'ils se font les
uns aux autres, l'impatience de les suivre et de les démêler mettent une
confusion dans sa tête, de laquelle il ne peut sortir. C'est, à la
guerre, la source de tant de mouvements inutiles dont il harasse ses
troupes, sans aucun fruit, et si souvent à contre-temps, en général par
des marches et des contre-marches que personne ne comprend, en détail
par des détachements qui vont et qui reviennent sans objet, en tout par
des contre-ordres, six, huit, dix tous de suite, quelquefois en une
heure aux mêmes troupes, souvent à toute l'armée pour marcher et ne
marcher pas, qui en font le désespoir, le mépris et la ruine. En
affaires, il saisit un projet, il le suit huit jours, quelquefois
jusqu'à quinze ou vingt. Tout y cède, tout y est employé, toute autre
chose languit dans l'abandon, il ne respire que pour ce projet. Un autre
naît et se grossit dans sa tête, fait disparaître le premier, en prend
la place avec la même ardeur, est éteint par un troisième, et toujours
ainsi. C'est un homme de grippe, de fantaisie, d'impétuosité successive,
qui n'a aucune suite dans l'esprit que pour les trames, les brigues, les
piéges, les mines qu'il creuse et qu'il fait jouer sous les pieds. C'est
où il a beaucoup de suite et où il épuise toute la sienne pour les
affaires.

On verra en son temps les preuves de fait de ce qui se lit ici\,; et on
les verra les unes avec horreur, les autres avec toute la surprise que
peuvent donner les propositions les plus étranges et les plus insensées.
Enfin ce qui trouvera à peine croyance d'un homme d'autant d'esprit et
employé de si bonne heure, on le verra incapable de faire un mémoire
raisonné sur quoi que ce soit, et incapable d'écrire une lettre
d'affaires\footnote{Il ne serait pas inutile, pour contrôler ce passage
  des \emph{Mémoires de Saint-Simon}, d'étudier les papiers du maréchal
  de Noailles, d'où l'abbé Millot a tiré les \emph{Mémoires de Noailles,
  }qui font partie de toutes les collections de Mémoires relatifs à
  l'histoire de France. Cette étude prouverait, je crois, que le
  jugement de Saint-Simon est d'une sévérité excessive.}. À force de
raisonner, de parler, de dicter, de reprendre, de corriger, de raturer,
de changer, de refondre, tout s'évapore, il ne demeure rien\,; les jours
et les mois s'écoulent, la tête tourne aux secrétaires, il ne sort rien,
mais rien, quoi que ce soit. De dépit, quand c'est chose qu'il faut
pourtant qui existe et montrer, il se résout enfin de la faire faire par
un inconnu qu'il a déniché et qu'il a mis sous clef dans un grenier, à
qui souvent encore il fait faire et défaire dix fois, et avec la plus
tranquille effronterie il produit cet ouvrage comme sien. Un homme en
apparence si ouvert, si aimable, si fait exprès pour jeter de la poudre
aux yeux des plus réservés, pour montrer si naturellement tout ce qui
peut engager de tous les côtés possibles, et pour en donner jusqu'en
capacité de toutes les sortes les plus avantageuses impressions, qui en
même temps ne pense que pour soi, ne fait aucun pas, quelque futile ou
indifférent qu'il paroisse, qui n'ait rapport à son objet, qui pense
toujours sombrement, profondément, à qui nul moyen ne coûte, qui avale
la trahison et l'iniquité comme l'eau, qui sait imaginer, ourdir de
loin, et suivre les plus infernales trames, est un de ces hommes que la
miséricorde de Dieu a rendus si rares, qui, avec la noirceur des plus
grands criminels, n'a pas même ce que, faute d'expression, on appelle la
vertu qu'il faut pour exécuter de grands crimes, mais rassemble en soi
pour les autres les plus grands dangers, et ne leur plaît que pour les
perdre, comme les sirènes des poëtes. Pour sa valeur, au moins plus
qu'obscurcie par l'étrange timidité de général, j'en abandonne le
jugement à ceux qui l'ont vu en besogne. Il en a essuyé quelquefois de
bons mots le long des lignes. Ses incertitudes continuelles, et ses
occupations qui l'ont tenu si fort sous clef à l'armée et à la cour ne
l'y ont pas fait aimer.

Mon caractère droit, franc, libre, naturel, et beaucoup trop simple,
était fait exprès pour être pris dans ses piéges. Comme je l'ai dit, il
tourna court à moi. Je n'en vis que la partie aimable\,; j'y pris
aisément les écorces estimables pour les choses mêmes, il n'était pas
encore démasqué\,; au moins j'ignorais le masque, et je n'étais pas
encore instruit de la cause de son retour. J'imaginai bien que ce
n'était pas, comme l'on dit, à mes beaux yeux que je devais les avances
et les recherches empressées d'un homme avec qui je n'avais jamais vécu,
et que les ailes de la faveur avaient si continuellement porté dans des
routes brillantes tandis que je rampais. Je crus bien qu'il voyait
derrière moi M. le duc d'Orléans, M. de Beauvilliers, peut-être le
Dauphin dans le lointain, et qu'à tout hasard il avait envie de me
ramasser par le chemin. Je compris que c'était un conseil de sa mère,
dont je parlerai ailleurs, qui avait toujours eu de l'amitié pour moi,
quoique sans liaison bien étroite, et qui chercha toujours tant qu'elle
put, mais par des voies honnêtes, à avoir tout pour soi et rien contre.
Je fus séduit par qui avait tout pour séduire\,: l'esprit, les grâces,
le raisonnement, et pour le dehors les plus grands et les plus brillants
établissements en tout genre.

Je répondis à ses avances, peu à peu à ses ouvertures où je ne mis rien
du mien, et où il me paraissait qu'il mettait fort du sien. Ses
campagnes, les choses d'Espagne servirent d'introduction\,; quelqu'une
d'un intérieur de cour qui me passait souvent, parce que la scène en
était chez M\textsuperscript{me} de Maintenon, conduisit la confiance\,;
et quand elle fut un peu établie par les raisonnements sur la position
présente et future, ce raffiné musicien me pinça mélodieusement deux
cordes qui lui rendirent tout le son qu'il s'en était promis\,: l'un
regardait notre dignité si abattue\,; l'autre, l'état de son oncle
auquel je reviendrai à part. Il me savait, comme bien d'autres, fort
touché de notre rang, il m'était arrivé là-dessus des choses que j'ai
racontées et qui n'étaient pas ignorées\,; et son onde qui, comme toute
sa famille, avait mis en lui toutes ses complaisances, lui avait déjà
appris que je m'intéressais en lui. Je me voyais donc parfaitement
homogène à lui sur ces deux points si importants\,; et il fallait,
surtout en l'écoutant, être pour ainsi dire en son âme, pour imaginer
qu'il pût n'être pas un en tout et partout avec le cardinal de Noailles,
et par les plus communs et les plus pressants intérêts, et que sur
l'autre point il ne fût pas sensible à ce qui constituait et qui
comblait le plus la grandeur solide et radicale de sa fortune et de son
état autant qu'il me le disait, avec un air de naïveté et de vivacité
qui avivaient ses raisonnements là-dessus. Ces deux pivots de notre
amitié dans la suite, et qui de là devinrent la base de la confiance que
peu à peu je pris en lui, il ne les amena qu'après leur avoir aplani les
voies par d'autres choses, et bientôt après il sut bien s'en servir pour
ce qu'il se proposait, et pour augmenter en même temps ma confiance par
ses confidences.

La première, et qui ne tarda pas, fut celle de l'état où il se trouvait
avec M. {[}le duc{]} et M\textsuperscript{me} la duchesse d'Orléans. Il
ne m'apprenait rien, et il pouvait bien le juger ainsi. Je ne le lui
cachai pas. Il m'avoua que cela l'embarrassait, se plaignit d'eux, se
disculpa à moi sur l'un et sur l'autre, ne me dissimula point qu'il me
serait obligé de les sonder et de le remettre bien avec eux, moins parce
qu'il y avait à gagner avec des gens qui ne pouvaient quoi que ce soit,
que pour n'être pas brouillé après une amitié liée, et pour une aventure
où il avait aussi peu de part qu'était celle de Renaut, mais dont
l'obscurité était aussi désagréable. J'entrai dans ses raisons, et je
lui promis de parler à M. {[}le duc{]} et à M\textsuperscript{me} la
duchesse d'Orléans, d'autant plus volontiers qu'ignorant encore la
triste situation du duc de Noailles pour le fond, quoique j'en aperçusse
déjà l'écorce, je ne doutais pas qu'il ne se relevât promptement par le
secours de sa tante, et que je trouvais qu'en ce raccommodement il y
avait plus à gagner pour M. {[}le duc{]} et M\textsuperscript{me} la
duchesse d'Orléans que pour lui qui, dans un intérieur de privance tel
que je le croyais avec sa tante, pouvait si aisément leur devenir utile,
quand ce ne serait qu'en avertissant et en découvrant. Je le représentai
ainsi à l'un et à l'autre. M\textsuperscript{me} la duchesse d'Orléans y
entra assez\,; M. le duc d'Orléans, qui n'était jamais bien revenu de
son affaire d'Espagne, et qui l'avait fort sur le cœur, se montra plus
difficile. Ce siège dura quelques jours, à la fin j'en vins à bout. Je
le dis au duc de Noailles. Il me remercia fort, puis me proposa un autre
embarras du côté de sa tante si elle le voyait relié avec M. le duc
d'Orléans, et les mesures infinies qu'il avait à garder avec une femme
si délicate, si aisée a blesser, et dont la jalousie de tout autre
ménagement s'effarouchait à son égard aussi facilement qu'à celui des
autres. C'est qu'il me cachait la situation où il se trouvait avec elle,
et qu'il craignait de l'empirer si elle soupçonnait qu'ainsi mal avec
elle, il se jetât d'un côté, qu'elle haïssait autant, et sans sa
participation qu'il n'était pas en état de sonder.

Moi, qui ignorais ce fond, j'attribuai cette mesure craintive à une
connaissance encore plus grande qu'il avait de l'éloignement du roi, et
surtout de sa tante pour M. le duc d'Orléans, que celle que nous
n'ignorions pas\,; et cette pensée me fut une raison de plus de désirer
et de presser le renouement, que j'espérais dans la suite pouvoir
contribuer à émousser M\textsuperscript{me} de Maintenon, et la rendre
moins ennemie de M. le duc d'Orléans, en lui mettant le duc de Noailles
pour contre-poids à M. du Maine. J'en parlai en ces termes-là à M. le
duc d'Orléans, et plus mesurément à M\textsuperscript{me} la duchesse
d'Orléans. Ils y entrèrent l'un et l'autre, et ils voulurent bien que le
duc de Noailles allât chez eux en un temps d'obscurité et de solitude,
sans explication, et comme le passé non avenu, en un mot sur le pied
précédent\,; que le duc de Noailles ne les vît pas plus souvent que
lui-même croirait le pouvoir faire, et qu'en public il ne se marquât
rien de ce changement entre eux. Cela fut exécuté de la sorte. La visite
se passa très-bien à ce qu'il m'en revint des deux côtés\,; les
suivantes furent très-rares. Le bâton, que le duc de Noailles prit au
1\^{}er janvier, y servit de nouvelle excuse qu'il me pria souvent de
réitérer.

Content de ce premier succès, qui nourrissait et augmentait notre
confiance, il craignit apparemment que le temps ne me découvrît ce qu'il
m'avait caché, et que le temps aussi m'avait appris, mais dont je ne
crus pas sage de lui ouvrir le propos\,; plus que cela encore, il espéra
que je ne serais pas plus difficile ni moins heureux auprès du duc de
Beauvilliers que je l'avais été pour lui auprès de M. {[}le duc{]} et de
M\textsuperscript{me} la duchesse d'Orléans. Sa situation avec le
Dauphin et la Dauphine le tenait à la gorge, et il n'était pas en une
meilleure avec le duc de Beauvilliers, par qui seul néanmoins, car il ne
voyait pas d'autre route, il pût rapprocher le Dauphin et par lui la
Dauphine, et se frayer après, par ses sœurs à qui cela rouvrirait la
bouche, une protection par la Dauphine, pour fondre peu à peu les glaces
de M\textsuperscript{me} de Maintenon pour lui. C'est au moins ce que je
pus comprendre de ses propos couverts, coupés, entortillés, qui
suivirent la confidence qu'il me fit des mauvais offices qu'on lui avait
rendus en Espagne, où, pour perdre Aguilar, on l'avait perdu ici sans
qu'il l'eût mérité, ni qu'il sût même ce qu'il s'était passé d'Aguilar
au roi d'Espagne, parce que ce dernier avait été si promptement chassé
qu'il était parti pour sa commanderie sans qu'il eût pu le voir, ni
personne non plus que lui. Il ne convint jamais du dessein de donner une
maîtresse, au moins pour lui, ni qu'il en eût jamais ouï parler à son
ami Aguilar\,; et toujours sur les plaintes de ce que lui coûtait cette
amitié par la jalousie du mérite des emplois et de la faveur d'un
seigneur de la cour d'Espagne qu'on avait cru perdre plus sûrement en ne
les séparant pas, et dont le malheur retombait à plomb sur lui dans la
nôtre, sans qu'on eût voulu l'écouter en celle d'Espagne, dont il
portait très-innocemment toute la colère ici.

Je vis un homme fâché lorsque je lui appris que son aventure ne m'était
plus nouvelle\,; que j'avais cru de ma discrétion de ne lui pas montrer
que j'en étais instruit\,; et que je n'en étais pas moins touché de sa
confidence. Je pris pour bon tout ce qu'il m'ajusta sur le projet de
donner une maîtresse au roi d'Espagne et de ses suites sur lesquelles il
s'étendit fort, et sur la folie, établi comme il l'était ici, de ce
qu'il aurait pu espérer en Espagne. Tous vilains cas sont reniables. Il
ne me persuada point contre ce que je savais, et dont la colère de
l'intérieur, et surtout de sa tante, faisait foi, auparavant si aveuglée
pour lui\,; mais je crus sage de ne pas presser une telle apostume. Je
regardai ce trait d'ambition comme une verdeur de jeunesse gâtée par
tout ce qui peut flatter le plus à tout âge, et ce coup de fouet comme
une leçon qui le mûrirait et l'instruirait avec tout l'esprit qu'il
avait.

Ces plaintes qu'il me fit se prolongèrent quelques jours avant d'en
venir au point que je sentis après qui l'avait pressé de me les faire,
et ce fut lorsqu'il y vint où l'ambage de ses discours me fit entrevoir
ce qu'il se proposait par le duc de Beauvilliers. Il s'étendit sur son
mérite, sur l'impression que sa vertu avait toujours faite sur lui\,; il
savait trop à qui il parlait pour ne pas dire merveille sur ce chapitre,
qu'il conclut par ses désirs de pouvoir se rapprocher de lui, et tout ce
qui se suit de là. Il me sonda délicatement comme pour ne me rien
proposer d'embarrassant\,; et, comme il aime à parler et à s'étendre, je
le laissai volontiers se satisfaire, rêvant cependant à ce que moi-même
je ferais. Ce qui me détermina fut la persuasion que l'unique neveu de
M\textsuperscript{me} de Maintenon, qui avait jusqu'alors marqué pour
lui un goût si abandonné, rentrerait à la fin dans ses bonnes grâces, et
par elles dans celles du roi et de la Dauphine encore, légère comme elle
était, et incapable d'une forte amitié et plus encore d'une longue
haine, investie des Noailles au point et par les endroits où elle
l'était\,; pour l'avenir, qu'un homme d'autant d'esprit, de talents,
d'emplois, frère de ces mêmes dames du palais, et premier capitaine des
gardes, approcherait toujours le Dauphin devenu roi de fort près\,;
qu'il n'était pas possible qu'il ne lui plût à la longue\,; et que pour
le présent et le futur, il valait mieux l'avoir à soi, qu'à compter un
jour avec lui après avoir refusé et méprisé ses avances. Ce raisonnement
qui me saisit m'emporta tellement, que je me rendis facile à travailler
à une réunion. Lorsqu'il m'en pria et qu'il m'en pressa tout de suite,
je ne laissai pas de le vouloir sonder à mon tour. Sa mère, en femme
sage et habile, avait su profiter de la douceur et de l'équanimité du
duc de Chevreuse, pour relier avec lui aussitôt que ce grand orage du
quiétisme fut passé. Il avait été à diverses reprises ou choisi par MM.
de Bouillon et de Noailles, ou suggéré par le roi pour accommoder leurs
vifs démêlés d'affaires et de procédés qui regardaient la vicomté de
Turenne\,; et les terres de M. de Noailles dont les devoirs et la
mouvance même étaient réciproquement prétendus et niés, ce qui les avait
souvent extrêmement commis. Ces affaires n'étaient point finies, et
souvent M. de Chevreuse s'en mêlait encore. Je demandai donc au duc de
Noailles pourquoi il ne s'adressait pas à un canal si naturel et si
puissant sur M. de Beauvilliers. Il me répondit assez naturellement qu'à
la nature de ce qui lui était imputé en Espagne, à la piété pleine de
maximes de M. de Chevreuse, et à la froideur dont il l'avait retrouvé,
il croyait n'avoir guère moins besoin de secours auprès de lui qu'à
l'égard de M de Beauvilliers, et que je l'obligerais doublement si je
voulais bien parler de lui à tous les deux. Parler à l'un c'était parler
à l'autre\,; en affaires moins encore qu'en société, cela ne pouvait se
séparer\,; et jamais l'un n'aurait pris un parti sur le duc de Noailles
sans l'autre. J'étais trop avant avec eux et depuis trop longtemps pour
l'ignorer, mais je voulus être instruit de la façon d'être d'alors du
duc de Noailles avec M. de Chevreuse, et je le fus. Déterminé que
j'étais de parler à l'un, c'était l'être aussi de parler à l'autre, et
je m'en chargeai.

Je n'eus pas peine à remarquer, aux remercîments que j'en reçus, la
différence entière que faisait le duc de Noailles de se raccommoder avec
eux ou avec M. {[}le duc{]} et M\textsuperscript{me} la duchesse
d'Orléans. Son bien-dire ici me parut tout autrement aiguisé, et son
empressement aussi, jusqu'à ce que j'eusse une réponse à lui faire.
Néanmoins je sentais tout l'éloignement de cour et de religion qu'avait
le duc de Beauvilliers pour le fils du feu maréchal de Noailles, et pour
le neveu du cardinal de Noailles et de M\textsuperscript{me} de
Maintenon. M. de Chevreuse qui par la raison que j'ai rapportée en était
moins éloigné, fut celui à qui je m'adressai d'abord. Son accortise
naturelle le ploya assez aisément au raisonnement qui m'avait déterminé,
et le disposa ensuite à le faire valoir à M. de Beauvilliers, que
j'attaquai après. Je trouvai que je ne m'étais pas trompé. La
proposition fut mal reçue. J'insistai pour être entendu jusqu'au bout\,;
je déployai mes raisons, les louanges de ce que je trouvais dans M. de
Noailles, les avantages qui se pouvaient rencontrer avec lui, les
inconvénients de le rejeter, tandis qu'il n'y en avait aucun à le
recevoir. Je m'étendis sur ce qu'il ne s'agissait de rien en
particulier, sinon en général d'être avec lui sur un pied honnête de
bienveillance générale, de le voir et de lui parler en général
quelquefois, avec toute liberté d'étendre et de resserrer ce léger
commerce, selon qu'il se trouverait convenir aux temps et aux occasions,
et cependant s'assurer de l'avoir en laisse. Le duc de Beauvilliers
voulut prendre quelques jours pour y penser. Je m'étais assuré du duc de
Chevreuse, que je comptais qui achèverait de le déterminer dans
l'ébranlement où je l'avais mis, et la chose succéda comme je l'avais
prévue.

M. de Beauvilliers me permit donc de répondre au duc de Noailles de sa
part avec quelque chose de plus que de la politesse, mais il me chargea
en même temps de lui bien faire entendre combien il était important
d'éviter de faire une nouvelle, d'exciter la curiosité et l'inquiétude,
et de laisser apercevoir un changement de conduite l'un avec l'autre par
se parler souvent, et plus qu'en passant, quand ils se trouveraient
devant le monde aux lieux et aux heures publiques, ou par des visites
moins que rares et sans précautions pour n'y trouver point de témoins.
M. de Chevreuse, dont les suites des affaires de Turenne rendaient la
taille plus aisée, se prêta aussi un peu plus. Je m'acquittai de ce que
l'un et l'autre m'avaient chargé {[}de lui dire{]} avec la précision la
plus exacte, et je comblai le duc de Noailles d'une joie que ces mesures
étroites ne purent diminuer. Jamais son commerce avec M. de Chevreuse
n'avait pu lui en ouvrir aucun avec M. de Beauvilliers\,; et M. de
Beauvilliers, auquel il avait toujours inutilement buté par rapport à
son jeune prince, dans les temps où il ne pouvait rien, était en son
absence devenu tout à coup l'étoile du matin, et le Dauphin la brillante
aurore qui donnait les couleurs à tout.

Rien de si vif, de si expressif que les remercîments que je reçus du duc
de Noailles de lui avoir ramené ces deux seigneurs, avec lesquels il
fallait maintenant compter, et plus encore à l'avenir, Beauvilliers
surtout qui pénétrait la cour de ses rayons. Ils se virent donc, ils
furent contents les uns des autres jusque-là que les deux ducs me surent
gré de l'entremise, et me le témoignèrent, et le Noailles ne sut comment
m'exprimer l'excès de son contentement et de sa reconnaissance. Il
s'échafaudait par-dessus ses espérances, et se flattait d'arriver
bientôt par ce chemin jusqu'au Dauphin. Son impatience là-dessus ne put
souffrir de délai. Il s'expliqua là-dessus avec moi, il ne ménagea pas
même l'ouverture comme la première fois. Il me dit que l'obligation
serait trop grande pour oser s'en flatter sitôt, après avoir été reçu
par le duc de Beauvilliers, mais qu'il me laissait faire, et que les
preuves d'amitié qu'il recevait de moi si importantes coup sur coup lui
donnaient la confiance d'en tout espérer. Je sondai le terrain, je
sentis que le duc de Noailles avait été goûté\,; j'en profitai. Je fis
sentir au duc de Beauvilliers tout ce qu'un service prompt et qu'on
n'ose demander ajoute à la grandeur du service\,; cette considération
entra, elle fit effet. Incontinent après, c'est-à-dire au bout de sept
ou huit jours, les manières silencieuses et sèches du Dauphin changèrent
peu à peu pour le duc de Noailles, qui dans son transport me le vint
dire avec tous les remercîments pour moi, et les expressions pour le duc
de Beauvilliers qu'un succès si prompt et si peu espéré mit à la bouche
d'un homme qui y avait si fort buté comme au salut présent de sa
fortune, et à l'ouverture de toutes ses espérances pour l'avenir.
Malheureusement pour tout, ce n'est pas la peine de s'y étendre
davantage. Revenons maintenant pour un moment au cardinal de Noailles.

C'était un homme avec qui mon âge et mon état ne m'avaient fourni aucune
sorte de liaison ni commerce. Sa déplorable faiblesse pour la ruine
radicale de Port-Royal des Champs, et l'exil du Charmel dont j'ai parlé
en son temps, m'avaient même donné de l'éloignement pour lui. Mais le
guet-apens qui lui avait été dressé par ces deux évêques, l'insolence
hypocrite dont il était soutenu, l'innocence évidente opprimée dans
leurs filets par une injustice qui sautait aux yeux, et cette innocence
que bridait la patience, la charité, la confiance en la bonté et la
simplicité de sa cause, et une funeste lenteur naturelle, m'avait piqué
contre l'iniquité et le complot qui était palpable, dont les progrès
croissaient toujours. J'étais ami intime de plusieurs de ses amis et
amies qui m'en parlaient souvent\,; et le P. Tellier qui me tâtait
là-dessus avec ses ruses, n'en avait pas assez pour me cacher de
grossières friponneries. Il avait eu le crédit de faire défendre au
cardinal de Noailles d'aller à la cour. Cela m'avait révolté tellement
que j'allai à l'archevêché, un matin que son audience finissait, lui
témoigner la part que je prenais aux peines qu'on lui faisait. Il fut
extrêmement touché de ma visite, et beaucoup aussi du peu de ménagements
que j'y apportais en me montrant chez lui en une heure si publique. Il
me témoigna combien il sentait l'un et l'autre. Il entra fort avant en
matière avec moi, et de ce moment naquit une liaison entre nous, qui
s'est toujours étrécie, et qui n'a fini qu'avec lui. Bientôt après, il
eut permission de voir le roi, et ce ne fut qu'assez longtemps après que
son affaire fut renvoyée au Dauphin.

À peine fut-on de retour de Fontainebleau à Versailles que la mine, si
artistement chargée, joua avec tout l'effet que les mineurs s'en étaient
promis. Le roi fut accablé de lettres d'évêques hypocritement tremblants
pour la foi, et qui, dans le péril extrême où ils trouvaient que le
cardinal de Noailles la mettait, se sentaient forcés par leur
conscience, et pour la conservation du précieux dépôt qui leur était
confié, et dont le père de famille leur redemanderait un rigoureux
compte, de se jeter aux pieds du fils aîné de l'Église, du destructeur
de l'hérésie, du Constantin, du Théodose de nos jours, pour lui demander
la protection qu'il n'avait jamais refusée à la bonne et sainte
doctrine. Ce pathétique, tourné en diverses façons, fut soutenu de la
frayeur mensongère dont étaient saisis de pauvres évêques inconnus, qui
se trouvaient avoir à combattre l'archevêque de la capitale, orné de la
pourpre romaine, puissant en famille, en amis, en faveur, en crédit. Le
fracas fut grand\,; et le roi, à qui ces lettres étaient à tous moments
présentées à pleines mains par le P. Tellier, et par lui bien
commentées, entra dans un effroi comme si la religion eût été perdue.
M\textsuperscript{me} de Maintenon reçut aussi quelques lettres
semblables, que l'évêque de Meaux lui faisait d'autant mieux valoir
qu'il était dans la bouteille, et M\textsuperscript{me} de Maintenon
animait le roi de plus en plus. Mais au plus fort de ce triomphe, il
arriva un malheur qui eût fait avorter une affaire si fortement
conduite, si le cardinal de Noailles eût bien voulu prendre la peine
d'en profiter.

Je répète ici que je ne prétends pas grossir ces Mémoires du récit d'une
affaire qui remplit des in-folio, mais en coter seulement les endroits
qui m'ont passé par les mains. Je renvoie donc à ces livres le comment
de ceci avec tout le reste\,; mais il arriva que la lettre originale du
P. Tellier à l'évêque de Clermont, qui le pressait d'écrire au roi, et
l'instruisait pour l'y résoudre de la pareille démarche à lui promise
par beaucoup d'évêques\,; le modèle tout fait de sa lettre au roi qu'il
n'avait qu'à faire copier, la signer, et la lui adresser\,; ce qu'il lui
devait écrire à lui en accompagnement\,; et la lettre originale que lui
écrivait son neveu, l'abbé Bochard de Saron, trésorier de la
Sainte-Chapelle de Vincennes, en lui envoyant celles que je viens de
marquer de la part du P. Tellier qui les lui avait remises, tombèrent
entre les mains du cardinal de Noailles. Cela montrait la trame si
manifestement qu'il n'y avait ni manteau ni couverture à y mettre. Le
cardinal n'avait qu'à s'en aller trouver le roi à l'instant\,; et sans
se dessaisir de ces importantes pièces, les lui faire lire, lui en
commenter courtement toute l'horreur, et lui montrer les suites de ce
qui se brassait si ténébreusement contre lui, aux dépens du repos du roi
et de l'Église, lui demander justice en général, et en particulier de
chasser le P. Tellier si loin, qu'on n'en pût plus entendre parler, en
aller user de même avec M\textsuperscript{me} de Maintenon, puis faire
tout le fracas que méritait une si profonde scélératesse. Le P. Tellier
était perdu sans ressource, les évêques écrivains convaincus, l'affaire
en poudre, et le cardinal plus en crédit et plus assuré que jamais.

Au lieu d'un parti si aisé et si sage, le cardinal, plein de confiance
en la proie qu'il tenait, en parla, la montra, attendit le jour de son
audience. La chose transpira, le P. Tellier fut averti, l'excès du
danger lui donna des ailes et des forces\,; il prévint le roi comme il
put\,; il réussit, tant ce prince lui était abandonné. Le cardinal
trouva les devants pris. Son étonnement et l'indignation de voir le roi
froid sur une imposture aussi énorme et aussi claire l'étourdirent. Il
ne s'aperçut pas assez que le roi ne laissait pas d'être incertain,
ébranlé\,: c'était où il fallait de la force pour l'emporter, et ne lui
laisser pas l'intervalle de huit jours jusqu'à sa prochaine audience
pour se rassurer et se laisser prendre aux nouveaux piéges de son
confesseur. Il n'y mit que de la douceur et de la misère, et il échoua
ainsi au port. Le P. Tellier, qui, malgré son audace, ses mensonges et
ses ruses, tremblait de l'effet qu'aurait cette audience du cardinal, se
rassura quand il n'en vit aucun. Il en profita en scélérat habile et qui
sent à qui il a affaire. Il en fut quitte pour la plus terrible peur que
lui et les siens eussent eue de leur vie. Ils travaillèrent sans relâche
auprès du roi et de M\textsuperscript{me} de Maintenon, ils furent
quelque temps sans oser pousser le cardinal de Noailles, dans la crainte
du public qui jeta les hauts cris, ils se donnèrent le temps de les
laisser amortir, et à eux de reprendre haleine\,; et de là continuèrent
hardiment ce qu'ils avaient entrepris.

Le Dauphin ne put être pris comme le roi. Lui et la Dauphine en
parlèrent fort librement\,; et ce prince me dit et le dit encore à
d'autres, qu'il fallait avoir chassé le P. Tellier. Dès la fin de
Fontainebleau, le roi avait remis au Dauphin la totalité de l'affaire du
cardinal de Noailles. Il y travailla trop théologiquement, et je crus
avoir aperçu qu'il était entré en grande défiance des jésuites sur cette
affaire, ce qui est clair par ce que je viens de rapporter de lui sur le
P. Tellier, mais encore de l'évêque de Meaux. Ce qui m'en a persuadé,
c'est que la dernière fois que je travaillai avec lui, qui fut deux
jours avant le retour de Marly à Versailles, et cinq ou six jours avant
la maladie qui emporta la Dauphine, après une séance de plus de deux
heures où il n'avait point été question de l'affaire du cardinal de
Noailles, il m'en parla comme nous serrions nos papiers, et cette
conversation fut assez longue. Il m'y dit un mot bien remarquable.
Louant la piété, la candeur, la douceur du cardinal de Noailles\,:
«\,Jamais, ajouta-il, on ne me persuadera qu'il soit janséniste,\,» et
s'étendit en preuves de son opinion.

Cette conversation finit par m'ordonner de m'instruire à fond de ce qui
regarde les matières des libertés de l'Église gallicane, et à fond de
l'affaire du cardinal de Noailles, que le roi lui avait totalement
renvoyée pour la finir, et à laquelle il travaillait beaucoup, qu'il la
voulait finir avec moi, et me recommanda à deux ou trois reprises de me
mettre bien au fait de ces deux points, d'aller à Paris consulter qui je
croirais de meilleur, et de prendre les livres les plus instructifs sur
Rome et nos libertés, parce qu'il voulait travailler foncièrement sur
ces deux points avec moi, et finir ainsi l'affaire du cardinal, qui
allait trop loin et trop lentement, et la finir sans retour avec moi.
Jamais ce prince ne m'avait laissé rien entrevoir de ce dessein,
quoiqu'il m'eût parlé quelquefois de cette affaire\,; et j'ai toujours
cru qu'il ne le conçut que par le dégoût et les soupçons que lui donna
la manifestation de toute l'horreur de cette intrigue par la découverte
de ce paquet de l'abbé de Saron. Il me fit promettre de m'appliquer sans
délai à l'exécution de ses ordres, et de ne pas perdre un instant à me
mettre en état d'y travailler avec lui. J'allais en effet passer pour
cela quelques jours à Pans, quand je fus arrêté par la maladie de la
Dauphine, et, peu de jours après, tout à fait, par le coup le plus
funeste que la France pût recevoir.

\hypertarget{chapitre-iii.}{%
\chapter{CHAPITRE III.}\label{chapitre-iii.}}

1712

~

{\textsc{Pelletier se démet de la place de premier président.}}
{\textsc{- M. du Maine la fait donner au président de Mesmes.}}
{\textsc{- Extraction et fortune des Mesmes.}} {\textsc{- Caractère de
Mesmes, premier président.}} {\textsc{- Nos plénipotentiaires vont à
Utrecht.}} {\textsc{- Cardone manqué par nos troupes.}} {\textsc{-
L'empereur couronné à Francfort.}} {\textsc{- Marlborough dépouillé veut
sortir d'Angleterre.}} {\textsc{- Duc d'Ormond général en sa place.}}
{\textsc{- Troupes anglaises rappelées de Catalogne.}} {\textsc{-
Garde-robe de la Dauphine ôtée, puis mal rendue à la comtesse de
Mailly.}} {\textsc{- Éclat entre M\textsuperscript{me} la duchesse de
Berry et M\textsuperscript{me} la duchesse d'Orléans pour des perles et
pour la de Vienne, femme de chambre confidente, chassée.}} {\textsc{-
Pierreries de Monseigneur.}} {\textsc{- Judicieux présent du Dauphin.}}
{\textsc{- Dîners particuliers du roi\,; musique, etc., chez
M\textsuperscript{me} de Maintenon.}} {\textsc{- Tailleurs au pharaon
chassés de Paris.}} {\textsc{- Voyage de Marly.}} {\textsc{- Avis de
poison au Dauphin et à la Dauphine venus par Boudin et par le roi
d'Espagne.}} {\textsc{- Mariage de la princesse d'Auvergne avec Mézy par
l'infamie du cardinal de Bouillon.}} {\textsc{- Mort de
M\textsuperscript{me} de Pomponne.}} {\textsc{- Mort de
M\textsuperscript{me} de Mortagne.}} {\textsc{- Mort et caractère de
Tressan, évêque du Mans\,; ses neveux.}} {\textsc{- Mort de l'abbé de
Saint-Jacques.}} {\textsc{- Extraction et fortune des Aligre.}}
{\textsc{- Éloge de l'abbé de Saint-Jacques.}} {\textsc{- Mort de
Gondrin.}} {\textsc{- Plaisant contraste de La Vallière.}} {\textsc{-
Mort de Razilly et sa dépouille.}} {\textsc{- Conduite étrange de
M\textsuperscript{me} la duchesse de Berry là-dessus.}} {\textsc{- Éloge
et mort du maréchal Catinat.}} {\textsc{- Mort de Magnac.}} {\textsc{-
Mort de Lussan, chevalier de l'ordre.}}

~

Cette année commença par le changement de premier président du parlement
de Paris. Pelletier, médiocre président à mortier, pour tenir comme
l'ancien les audiences des après-dînées, avait succédé dans la première
place à Harlay, par le crédit de son père, pour qui le roi avait
conservé beaucoup d'amitié et de considération, depuis même qu'il se fut
retiré du ministère. Les qualités nécessaires à une place aussi
laborieuse et aussi importante manquaient au nouveau premier président.
Il sentait un poids difficile à soutenir, et qui lui devint
insupportable depuis l'accident, rapporté en son lieu, du plancher qui
fondit sous lui comme il était à table, dont néanmoins personne ne fut
blessé, mais la frayeur qu'il eut, et la commotion qui se fit peut-être
dans sa tête, l'affaiblit de sorte qu'il ne put plus souffrir le
travail. Il traîna depuis sa charge plus qu'il ne la fit, dans laquelle
son père le retenait. Il était très-riche. Sa charge de président à
mortier avait passé à son fils, qui longues années depuis fut aussi
premier président, ne valut pas son père, et s'en démit comme lui.
Pelletier n'avait rien à gagner à demeurer en place. Il le sentait, elle
l'accablait, mais son père l'y retenait. Dès qu'il l'eut perdu, il ne
songea plus qu'à se délivrer, et il envoya sa démission au roi le
dernier jour de l'année qui vient de finir. Cinq jours après, M. du
Maine la fit donner au président de Mesmes, et le roi voulut que ce fût
ce cher fils qui le lui apprît, à qui il était si principal d'avoir un
premier président totalement à lui. Ce magistrat paraîtra si souvent
dans la suite qu'il est nécessaire de le connaître, et de reprendre les
choses de plus haut.

Ces Mesrnes sont des paysans du Mont-de-Marsan, où il en est demeuré
dans ce premier état qui payent encore aujourd'hui la taille, nonobstant
la généalogie que les Mesmes qui ont fait fortune, se sont fait
fabriquer, imprimer et insérer partout où ils ont pu, et d'abuser le
monde, quoiqu'il n'ait pas été possible de changer les alliances, ni de
dissimuler tout à fait les petits emplois de plume et de robe à travers
l'enflure et la parure des articles\footnote{Le mot \emph{articles} est
  surchargé dans le manuscrit, et les précédents éditeurs ont lu
  \emph{artistes}.}\emph{. }Le premier au net qui se trouve avoir quitté
les sabots fut un professeur en droit dans l'université de Toulouse, que
la reine de Navarre, sœur de François I\^{}er, employa dans ses
affaires, et le porta à la charge de lieutenant civil à Paris. Son fils
professa aussi le droit à Toulouse, puis fut successivement conseiller à
la cour des aides, au grand conseil, et maître des requêtes. Il sera
mieux connu par le nom qu'il porta de sieur de Malassise, d'où la courte
paix qu'il négocia avec les huguenots, comme second du premier maréchal
de Biron, en 1570, qui n'était pas lors maréchal de France, mais qui
était déjà boiteux d'une blessure, fut appelé la \emph{paix boiteuse et
mal assise. }Il fut père du sieur de Roissy, successivement conseiller
au parlement, maître des requêtes, qui eut un brevet de conseiller
d'Etat et d'intendant des finances, et qui fut père de trois fils qui
établirent puissamment cette famille, et de deux filles, dont l'aînée
épousa le sieur Lambert d'Herbigny, maître des requêtes, l'autre
Maximilien de Bellefourière, qui fut mère du marquis de Soyecourt, si à
la mode et fort en faveur, grand maître de la garde-robe, en 1653,
chevalier du Saint-Esprit en 1661, et qui acheta en 1669 la charge de
grand veneur du chevalier de Rohan, Il était gendre du président de
Maisons, surintendant des finances, et mourut à Paris, en 1679. Ses deux
fils furent tués tous deux à la bataille de Fleurus, sans alliance, en
1690\,; et leur sœur mariée pour rien à Seiglière Bois-Franc porta à ses
enfants tous les biens de Bellefourière, de Soyecourt, sa grand'mère,
héritière, et des Longueil-Maisons qu'elle a vu éteindre. Ces riches
aventures arrivent toujours à des filles de qualité dont on veut se
défaire pour rien, et qui épousent des vilains.

Les trois frères de ces deux sœurs, enfants du sieur de Roissy, et
petits enfants du sieur de Malassise, furent le sieur de Mesmes, le
sieur d'Avaux, et le sieur d'Irval.

Le sieur de Mesmes fut lieutenant civil à Paris, en 1613, et député du
tiers état aux derniers états généraux tenus à Paris, en 1614. Il mourut
président à mortier, en 1650, et il avait épousé\footnote{Le président
  de Mesmes s'était marié en premières noces avec Jeanne de Montluc,
  morte en 1639\,; ce fut en secondes noces qu'il épousa Marie des
  Fossés.} la fille unique de Gabriel des Fossés, dit La Talée, marquis
d'Everly, gouverneur de Montpellier et de Lorraine, chevalier du
Saint-Esprit, en 1633. Cette héritière avait épousé en premières noces
Gilles de Saint-Gelais dit Lezignen\footnote{Ce nom paraît le même que
  celui de Lésignan ou Lusignan.}, dont elle avait eu une fille unique,
qui épousa le duc de Créqui, et qui fut dame d'honneur de la reine\,; et
de son second mariage la maréchale-duchesse de Vivonne, et une naine
pleine d'esprit, religieuse de la Visitation Sainte-Marie à Chaillot.
Ainsi les duchesses de Créqui et de Vivonne étaient sœurs de mère.

Le sieur d'Avaux est le célèbre d'Avaux qui se comtisa dans ses
ambassades. Il négocia à Rome, à Venise, à Mantoue, à Turin, à Florence,
chez la plupart des princes d'Allemagne\,; ambassadeur en Danemark, en
Suède, en Pologne, et plénipotentiaire à Hambourg, à Munster, à
Osnabrück, où il eut tant de démêlés avec Servien, son collègue, qui eut
plus de crédit que lui à la cour. Il fut greffier de l'ordre, ministre
d'État, et surintendant des finances, mais un peu en peinture, comme il
l'avoue par quelques-unes de ses lettres. Servien, son fléau, qui
l'était avec lui\footnote{Le président de Mesmes s'était marié en
  premières noces avec Jeanne de Montluc, morte en 1639\,; ce fut en
  secondes noces qu'il épousa Marie des Fossés.}, en avait toute
l'autorité. D'avaux ne se maria point, et mourut comme son frère aîné,
en 1650, quelques mois après lui.

Le sieur d'Irval prit le nom de Mesmes à la mort de son frère aîné, dont
il eut la charge de président à mortier. Il laissa deux fils, l'aîné qui
succéda à son nom et à sa charge, et qui épousa la fille de Bertran,
sieur de La Bazinière, trésorier de l'épargne et prévôt grand maître des
cérémonies de l'ordre du Saint-Esprit, qui avait épousé pour rien
M\textsuperscript{lle} de Barbezières-Chemerault, fille d'honneur de la
reine. La Bazinière tomba en déroute, en recherches\footnote{La
  Bazinière fut un des financiers poursuivis, en 1601, à l'époque de
  l'arrestation et du procès de Fouquet.}, fut mis à la Bastille, privé
de ses charges et du cordon bleu qui ne lui fut point rendu. C'était un
riche, délicieux et fastueux financier, qui jouait gros jeu, qui était
souvent de celui de la reine, et qui la quittait familièrement à moitié
partie, et la faisait attendre pour achever qu'il eût fait sa collation
qu'il faisait apporter dans l'antichambre, et dont il régalait les
dames. Il était si bon homme et si obligeant qu'on lui passait toutes
ces impertinences\,: fort galant, libéral, magnifique, homme de grande
chère, et si aimé que tout le monde s'intéressa pour lui. Il parut
constant qu'il n'y avait nulle friponnerie en son fait, mais un grand
désordre, faute de travail et d'avoir su régler sa dépense. Il sortit
enfin d'affaires\,; et quoique dépouillé et réduit au petit pied, il fut
le reste de sa vie, qui fut encore longue, bien reçu partout et
accueilli de la meilleure compagnie. Je l'ai vu chez mon père, avec un
joli équipage, et, tout vieux qu'il était, l'homme le plus propre et le
plus recherché. Il mourut en 1688, tout à la fin, quinze on seize ans
après être sorti d'affaires. Son gendre eut sa charge de l'ordre, qui
mourut neuf ou dix mois avant lui. Son frère qui ne se maria point, et
qui, tout conseiller d'État de robe qu'il était, se faisait appeler le
comte d'Avaux, fut survivancier, puis titulaire de sa charge de l'ordre,
ambassadeur à Venise, en Hollande, près du roi Jacques en Irlande, en
Suède, et encore en Hollande, et mourut d'une seconde taille, en 1709.
J'en ai parlé ailleurs.

Son aîné, le président de Mesmes, gendre de La Bazinière, eut trois fils
et deux filles\,; l'aîné, qui fut premier président cette année\,; un
abbé de Mesmes fort débordé\,; un chevalier de Malte qui ne le fut guère
moins, et que le crédit de son frère chargea de bénéfices et de
commanderies, et qu'il fit ambassadeur de Malte\,; M\textsuperscript{me}
de Fontenilles, dont j'aurai lieu de parler dans la suite, et une
ursuline. Après ce détail nécessaire, venons au nouveau premier
président.

Il porta le nom de sieur de Neuchâtel du vivant de son père. C'était un
grand et gros homme, de figure colossale, trop marqué de petite vérole,
mais dont toute la figure, jusqu'au visage, avait beaucoup de grâces
comme ses manières, et avec l'âge quelque chose de majestueux. Toute son
étude fut celle du grand monde à qui il plut, et fut mêlé dans les
meilleures compagnies de la cour et dans les plus gaillardes. D'ailleurs
il n'apprit rien et fut extrêmement débauché, tellement que son père le
prit en telle aversion qu'il osait à peine paraître devant lui. Il ne
lui épargnait pas les coups de bâton, et lui jetait quelquefois des
assiettes à la tête, ayant bonne compagnie à sa table, qui se mettait
entre-deux et tâchait de les raccommoder souvent\,; mais le fils était
incorrigible, et ne songeait qu'à se divertir et à dépenser. Cette vie
libertine le lia avec la jeunesse la plus distinguée qu'il recherchait
avec soin, et ne voyait que le moins qu'il pouvait de palais et de gens
de robe. Devenu président à mortier par la mort de son père, il ne
changea guère de vie, mais il se persuada qu'il était un seigneur, et
vécut à la grande.

Les gens distingués qui fréquentaient la maison de son père, les
alliances proches de M. de La Trémoille, de M. d'Elbœuf, et des enfants
de M\textsuperscript{me} de Vivonne qui vivait et qui les liait, le
tentaient de se croire de la même espèce, gâté qu'il était par la même
sorte de gens avec qui il avait toujours vécu. Il n'oublia pas de lier
avec les courtisans qu'il put atteindre. D'Antin fut de ce nombre par
ses cousines\,; et par ces degrés, il parvint jusqu'à M. et
M\textsuperscript{me} du Maine, qui, dans leurs projets, avaient besoin
de créatures principales dans le parlement, et qui ne négligèrent pas de
s'attacher un président à mortier. Celui-ci, ravi de s'en voir si bien
reçu, songea à se faire une protection puissante du fils, favori du
roi\,; et se dévoua jusqu'à la dernière indécence à toutes les
fantaisies de M\textsuperscript{me} du Maine. Il y introduisit son frère
le chevalier\,; ils furent de toutes les fêtes de Sceaux, de toutes les
nuits blanches\footnote{Voy. t. V, p.~2, une note sur l'origine de ce
  nom donné aux fêtes de Sceaux.}. Le chevalier n'eut pas honte de jouer
aux comédies, ni le président d'y faire le baladin, à huis clos entre
une vingtaine de personnes. Il en devint l'esclave à n'oser ne pas tout
quitter pour s'y rendre, et à se laisser peindre travesti, dans un
tableau historique, de ces gentillesses, avec des valets de Sceaux, à
côté du suisse en livrée. Ce ridicule lui en donna beaucoup dans le
monde, et déplut fort au parlement. Il le sentit, mais il était aux
fers, et il importait à ses vues de fortune de ne les pas rompre.
Avançant en ancienneté parmi les présidents à mortier, il comprit qu'il
était temps de fréquenter le palais un peu davantage, et la magistrature
à qui sa négligence à la voir avait marqué trop de mépris. Il ne crut
pas même indifférent de s'abaisser à changer un peu de manières pour les
avocats, procureurs, greffiers un peu distingués\,; et néanmoins n'en
refroidit pas son commerce avec les gens de la cour et du grand monde,
dont il avait pris tout à fait le ton et les manières.

Il chercha aussi à suppléer à son ignorance en apprenant bien ce qu'on
appelle le trantran du palais, et à connaître le faible de chacun de
Messieurs qui avaient du crédit et de la considération dans leurs
chambres\,; beaucoup d'esprit, grande présence d'esprit, élocution
facile, naturelle, agréable\,; pénétration, reparties promptes et
justes\,; hardiesse jusqu'à l'effronterie\,; ni âme, ni honneur, ni
pudeur\,; petit-maître en mœurs, en religion, en pratique\,; habile à
donner le change, à tromper, à s'en moquer, à tendre des pièges, à se
jouer de paroles et d'amis, ou à leur être fidèle, selon qu'il convenait
à ses intérêts\,; d'ailleurs d'excellente compagnie, charmant convive,
un goût exquis en meubles, en bijoux, en fêtes, en festins, et en tout
ce qu'aime le monde\,; grand brocanteur et panier percé sans
s'embarrasser jamais de ses profusions, avec les mains toujours
ouvertes, mais pour le gros, et l'imagination fertile à s'en procurer\,;
poli, affable, accueillant avec distinction, et suprêmement glorieux,
quoique avec un air de respect pour la véritable seigneurie, et les plus
bas ménagements pour les ministres et pour tout ce qui tenait à la cour.

Rien n'a mieux dépeint son principal ridicule qu'un de ce grand nombre
de noëls qu'on s'avisa de faire une année pour caractériser beaucoup de
gens de la cour et de la ville, qu'on introduisit à la crèche les uns
après les autres. Je ne me souviens plus du couplet, sinon qu'il
débutait\,: \emph{Je suis \textbf{M. de Mesmes\emph{, et qu'il
finissait\,: }qui vient prier le poupon à sou}per en carême. }Il avait
eu la charge de l'ordre de son oncle, et un logement, non à Versailles
mais à Fontainebleau, qu'avait eu son père, et que son père avait
conservé en se défaisant d'une charge de lecteur du roi qu'il avait eue
assez longtemps. C'en est assez, maintenant sur ce magistrat, qui à
toute force voulait être un homme de qualité et de cour, et qui se
faisait souvent moquer de lui par ceux qui l'étaient en effet, et avec
qui il vivait tant qu'il pouvait.

Les passe-ports arrivèrent le premier jour de cette année pour nos
plénipotentiaires. Ils eurent incontinent après leur audience du roi,
chacun séparément, et partirent l'un après l'autre pour Utrecht, dans
les huit premiers jours de cette année. En même temps M. de Vendôme fit
tenter par Muret, lieutenant général, le siége de Cardone, qu'il fallut
lever assez promptement avec quelques pertes. L'archiduc avait fait
passer cinq ou six mille hommes de ses troupes en Catalogne, où il
soupçonnait que ce qu'il y avait laissé d'Anglais ne demeureraient pas
longtemps. Ce prince avait reçu la couronne impériale à Francfort, et
s'en était allé à Vienne, après avoir écrit aux états généraux une
lettre violente et pressante pour les détourner de la paix, à laquelle
il voyait que tout tendait en Angleterre, où le duc de Marlborough ne se
crut plus en sûreté, et obtint de la reine la permission de passer la
mer avec la duchesse sa femme, dès qu'ils se virent dépossédés de toutes
leurs charges de cour et de guerre, le duc d'Ormond nommé en sa place
pour commander les troupes de la reine en Flandre\,; et peu après, le
duc d'Argyle, général des troupes d'Angleterre en Catalogne, eut ordre
de leur faire repasser la mer et les ramena en Angleterre.

Il arriva dans tous les premiers jours de cette année un fâcheux dégoût
à M\textsuperscript{me} de Mailly, dame d'atours de
M\textsuperscript{me} la Dauphine. La dépense de sa garde-robe passait
de loin le double de celle de la feue reine\,; et avec cela la princesse
manquait tellement de tout ce qui fait la commodité, la nouveauté et
l'agrément des parures, que le cri en fut public, et que les dames
prêtaient journellement à la Dauphine des palatines, des manchons et
toutes sortes de colifichets. L'indolence de M\textsuperscript{me} de
Mailly laissait tout faire à une de ses femmes de chambre, qui se
croyait nièce de M\textsuperscript{me} de Maintenon, parce que sa
maîtresse l'était. Desmarets, de plus en plus ancré, avait des prises
continuelles avec la dame d'atours sur sa grande dépense, et sur les
payements qu'elle pressait avec hauteur. Il s'en lassa, il en parla à
M\textsuperscript{me} de Maintenon et au roi, qui consultèrent la
Dauphine. Sa patience et sa douceur s'était lassée aussi après des
années de silence et de tolérance, tellement que l'administration de la
garde-robe lui fut ôtée et donnée à M\textsuperscript{me} Cantin,
première femme de chambre, et celle de M\textsuperscript{me} de Mailly
fut chassée pour s'être trouvée avoir bien fait ses affaires aux dépens
de la garde-robe et des marchands. M\textsuperscript{me} de Mailly cria,
pleura, dit qu'on la déshonorait\,; et tempêta tant auprès de
M\textsuperscript{me} de Maintenon qu'au bout d'une quinzaine on lui
rendit quelques sauve-l'honneur, mais le réel et l'autorité sur la
garde-robe elle ne put les rattraper. Elle ne fut plainte de personne\,;
l'excès de la gloire dont elle était lui avait aliéné tout le monde,
scandalisé d'ailleurs de voir la Dauphine si mal servie.

Ces premiers jours de l'année eurent un autre orage intérieur.
M\textsuperscript{me} la duchesse de Berry qui gouvernait père et mari,
donnait toutes sortes de dégoûts à M\textsuperscript{me} sa mère, et se
laissait conduire elle-même par une de ses femmes de chambre, de
beaucoup mais d'un très-mauvais esprit, qui s'appelait de Vienne, fille
de la nourrice de M. le duc d'Orléans, qui la considérait aussi pour
l'avoir auparavant trouvée fort à son gré. Feu Monsieur avait eu de la
reine mère un collier de perles dont la beauté et la rareté p assoient
pour être uniques. M\textsuperscript{me} la duchesse d'Orléans l'aimait
fort et s'en parait souvent. C'en fut assez pour que
M\textsuperscript{me} la duchesse de Berry le voulût avoir pour l'ôter à
M\textsuperscript{me} sa mère\,; et pour la piquer davantage elle le lui
demanda, sûre d'en être refusée\,; lui dit qu'elle l'aurait bien sans
elle, puisqu'il ne lui appartenait pas mais à M. le duc d'Orléans, de
qui en effet elle l'obtint. La scène fut forte entre elles.
M\textsuperscript{me} la duchesse de Berry affecta de porter ce collier
et de le montrer à tout le monde. Les choses furent poussées si loin que
Madame en fut parler au roi dans son cabinet. Elle ne se borna pas
apparemment au procédé du collier de perles. L'embarras et la
brouillerie de la mère et de la fille parurent en public\,; la fille ne
put soutenir la colère du roi et se tint au lit, où la Dauphine vint
l'exhorter plusieurs fois.

M. le duc de Berry était trop amoureux pour n'être pas aussi affligé
qu'elle, et M. le duc d'Orléans ne savait que devenir entre eux. Il
était question de bien pis que des perles. Le roi voulut que la femme de
chambre fût chassée, et malmena M. le duc de Berry, qui se hasarda de
lui en parler. Cet ordre mit M\textsuperscript{me} la duchesse de Berry
hors de toute mesure. Il lui parut un affront que son orgueil ne pouvait
supporter, indépendamment de toutes les privations qu'elle trouvait dans
cette perte\,; mais elle eut beau pleurer, crier, hurler, invectiver
père et mari de la sacrifier à leur faiblesse, il fallut obéir, chasser
la femme de chambre, aller demander pardon à M\textsuperscript{me} sa
mère, à qui elle ne pardonna jamais, et lui rapporter le collier de
perles. M\textsuperscript{me} la duchesse d'Orléans, satisfaite sur le
principal, lui fit inutilement des merveilles, lui promit de la
raccommoder avec le roi, et la mena dans son cabinet après le souper
deux jours après, parce que le roi voulut lui faire sentir sa disgrâce.
Il lui parla en père, mais en roi et en maître, en sorte qu'il ne manqua
rien à son humiliation que de pouvoir être intérieurement humiliée. Elle
reparut après quelques jours au souper du roi et en public, à son
ordinaire, cachant à grand'peine la rage qui la dévorait.

M\textsuperscript{me} de Saint-Simon, qui se tenait à quartier tant
qu'elle pouvait d'un intérieur où il n'y avait qu'à perdre et qui ne se
pouvait régler, ne prit aucune part en toute cette aventure, sinon
d'être témoin le moins qu'elle put des larmes et des fureurs. J'en usai
de même à l'égard de M. {[}le duc{]} et de M\textsuperscript{me} la
duchesse d'Orléans. Depuis ce que j'ai rapporté que M. le duc d'Orléans
avait dit à M\textsuperscript{me} sa fille, qu'elle avait si étrangement
pris sur moi, je ne mettais presque plus le pied chez elle, et jamais je
ne parlais d'elle à M. son père, qui aussi n'osait m'en parler\,; mais
je ne vis jamais homme si mal à son aise. Il donna une pension à la
femme de chambre, et la maria en province quelque temps après. On ferait
des volumes de tout ce qui se passait chez M\textsuperscript{me} la
duchesse de Berry. Le récit en suprendrait assurément, mais au fond il
ne vaudrait guère la peine d'être fait, et je n'en prétends raconter que
ce qui a éclaté, ou qui a été plus singulièrement marqué.

Ce fut pendant la fin de cet orage domestique que du Mont apporta une
après-dînée les pierreries de Monseigneur, dont les trois lots étaient
faits relativement à ce qui en avait été réglé au total et au genre de
partage de toute la succession. La Dauphine était descendue chez le
Dauphin pour les voir. Ce prince prit sur sa part deux belles bagues,
dont une de grand prix que Monseigneur portait fort souvent, et la donna
pour cela même à du Mont d'une manière fort obligeante\,; l'autre il
l'envoya à La Croix, cet ami intime de M\textsuperscript{lle} Choin dont
j'ai parlé, qui avait prêté de l'argent à Monseigneur sans vouloir
prendre d'intérêts.

Au commencement de cette année, le roi se mit à faire porter son dîner,
une fois ou deux la semaine, chez M\textsuperscript{me} de Maintenon, ce
qui ne s'était point encore vu, et ce qu'il continua le reste de sa
vie\,; mais dans la belle saison, ces dîners se faisaient souvent à
Trianon et à Marly, sans y coucher. La compagnie était fort courte, et
toujours la même\,: la Dauphine, qui malheureusement n'en vit que les
premiers\,; M\textsuperscript{me} de Maintenon\,; M\textsuperscript{me}s
de Dangeau, de Lévi, d'O et de Caylus, la seule qui ne fût pas dame du
palais. Qui que ce soit n'y entrait, non pas même le maître d'hôtel en
quartier. Les gens du roi portaient le couvert et les plats à la porte à
ceux de M\textsuperscript{me} de Maintenon qui servaient. La table se
prolongeait quelquefois une demi-heure plus qu'un dîner ordinaire. Le
roi y demeurait peu après le dîner, et revenait le soir à l'ordinaire.
Quelque temps après il jouait là quelquefois après dîner, quand il
faisait fort mauvais temps, avec les mêmes dames, au brelan ou au
reversi, fort petit jeu\,; et dans la suite, quelquefois les soirs des
vendredis qu'il n'avait point de ministres. Cela fit fort considérer ces
dames choisies\,; mais cela ne leur procura rien, non pas même la
liberté d'oser parler au roi, en ces heures-là, d'aucunes choses qui pût
les regarder ni leur famille. Ces dîners furent quelquefois suivis d'une
musique, où le roi revenait après avoir passé une demi-heure chez lui,
et qui durait jusque sur les six heures. C'était les jours de mauvais
temps, et {[}cela{]} s'introduisit dès le second dîner. Quelquefois
elles étaient les soirs au lieu de l'après-dînée, et personne n'y
entrait non plus qu'à ces dîners. On chassa en même temps de Paris
plusieurs hommes et femmes qui taillaient au pharaon\footnote{C'est-à-dire
  qui tenaient la banque à ce jeu de hasard et jouaient seuls contre
  plusieurs personnes.}, qui était un jeu avec raison fort défendu, et
que cette exécution fit entièrement cesser.

Le lundi 18 janvier, le roi alla à Marly. Je marque exprès ce voyage. À
peine y fut-on établi que Boudin, premier médecin de la Dauphine qui
l'amusait fort, qui l'avait été de Monseigneur, et duquel j'ai parlé
ailleurs, l'avertit de prendre garde à elle, et qu'il avait des avis
sûrs qu'on la voulait empoisonner et le Dauphin aussi, à qui il en parla
de même\,; il ne s'en contenta pas, il le débita en plein salon, d'un
air effarouché, et il épouvanta tout le monde. Le roi voulut lui parler
en particulier. Il assura toujours que l'avis était bon, sans qu'il sût
pourtant d'où il lui venait, et demeura ferme dans cette contradiction,
car s'il ignorait d'où lui venait l'avis, comment pouvait-il le juger et
l'assurer bon\,? Ce fut une première bouffée que ses amis arrêtèrent\,;
mais le propos public avait été lâché et réitéré. Ce qu'il y eut de fort
singulier, c'est qu'à vingt-quatre heures près de cet avis donné par
Boudin, le Dauphin en reçut un pareil du roi d'Espagne qui le lui
donnait vaguement, et sans citer personne, mais comme étant bien averti.
En celui-ci, il ne fut mention que du Dauphin nettement, et
implicitement et obscurément de la Dauphine. Au moins ce fut ainsi que
le Dauphin s'en expliqua, et je n'ai point su qu'il en ait dit davantage
à personne. On eut l'air de mépriser des choses en l'air, dont on ne
connaissoit point l'origine\,; mais l'intérieur ne laissa pas d'en être
frappé, et il se répandit un sérieux de silence et de consternation dans
la cour à travers des occupations et des amusements ordinaires.

Le cardinal de Bouillon, reçu chez les ennemis avec tant d'honneur et
d'éclat, y était peu à peu tombé dans le mépris. Il avait perdu son
neveu, sur la désertion, l'établissement et la fortune duquel il avait
bâti les plus folles espérances. Ce neveu n'ayait laissé qu'une fille
qui avait lors trois ou quatre ans, et qui était héritière de
Berg-op-Zoom et d'autres biens du côté de sa mère, fille du feu duc
d'Aremberg et d'Arschot, grand d'Espagne, de la maison de Ligne, et de
la fille du feu marquis de Grana-Garetto, gouverneur des Pays-Bas. La
longue minorité de cette enfant unique laissait sa mère maîtresse de sa
tutelle, de ses revenus, et de lui choisir un mari lorsqu'elle serait en
âge. Elle demeurait à Bruxelles avec sa mère la duchesse d'Aremberg à
qui son rang, ses richesses, sa vertu et sa conduite, attiraient la
première considération, et avec le duc d'Aremberg son frère qui n'en
avait pas moins de son côté, qui épousa depuis une Pignatelli, sœur du
comte d'Egmont, qui devint le favori du prince Eugène, et qui est
aujourd'hui chevalier de la Toison d'or du dernier empereur,
feld-maréchal de ses armées, grand bailli et gouverneur de Mons et du
Hainaut, mestre de camp général des Pays-Bas autrichiens, et général de
l'armée de la reine de Hongrie, dans un âge encore peu avancé. C'était
là une mère et un frère d'un appui, pour la princesse d'Auvergne, à
n'avoir pas à compter avec MM. de Bouillon pour la gestion des biens, ni
pour l'établissement de sa fille. Le cardinal de Bouillon qu'ils avaient
logé chez eux à Bruxelles voyait cela à regret\,; il était tombé dans
l'indigence par la saisie de ses bénéfices et la confiscation de ses
biens, ceux de sa petite-nièce lui faisaient grande envie.

Un fort mince gentilhomme qu'on appelait Mésy, qui avait été page chez
MM. de Bouillon, était devenu écuyer de la princesse d'Auvergne qui,
depuis quelque temps, le regardait de bon œil. Le cardinal s'en aperçut,
suivit ses soupçons, les trouva très-bien fondés. La gloire du prétendu
descendant des anciens ducs de Guyenne, et celle du premier homme de
l'Église après le pape, comme il se le disait, devait être extrêmement
blessée d'une pareille découverte, et encore plus alarmée des suites.
Mais la vanité céda aux besoins\,; il imagina qu'en favorisant ces
amours jusqu'à les porter à l'union conjugale, et venant après à
éclater, il déshonorerait si parfaitement la princesse d'Auvergne par la
honte de la mésalliance, qu'il la ferait déchoir de la tutelle, et que
cette tutelle lui tomberait au préjudice de la duchesse d'Aremberg,
parce que Berg-op-Zoom et d'autres biens encore venaient à l'enfant du
côté de son père et emporteraient même les maternels.

Dans cet infâme dessein il parla à Mésy, et comme par amitié et par
intérêt pour sa fortune, l'encouragea à pousser sa pointe et à la
tourner du côté du mariage, en quoi il lui promit toute protection.
Instruit après par Mésy de ses progrès, il parla à sa nièce dont
l'embarras ne se peut exprimer\,; il en profita pour la rassurer et en
tirer l'aveu de sa faiblesse, la plaignit, et la combla de trouver un
consolateur et un confident dans celui qu'elle avait le plus à redouter.
De là peu à peu il fit l'homme de bien avec elle, et l'évêque, pour
mettre sa conscience en sûreté en flattant sa passion. Il fit accroire à
la princesse d'Auvergne et à Mésy que leur mariage demeurerait secret,
et ne serait par conséquent sujet à aucune suite fâcheuse du côté des
Bouillon ni du côté des Aremberg\,; il leur offrit de les marier
lui-même\,; il les y résolut, et il les maria dans l'hôtel d'Aremberg.

Quelques mois se passèrent dans les transports de l'amour, de la
reconnaissance, de la confidence. Le cardinal s'applaudissait en secret
de son crime, et se moquait de leur simplicité en attendant son temps.
L'amante se crut grosse\,; ce fut celui d'en profiter. Le mariage se
divulgua\,; le duc et la duchesse d'Aremberg furent outrés de rage et de
dépit, et d'étonnement de trouver le cardinal de Bouillon moins emporté
qu'il ne l'était. À la fin la chose éclata tout à fait. L'écuyer et sa
dame furent chassés de la maison, sans savoir où se réfugier. Le
cardinal, très-court d'argent, les assista peu en cachette, et leur fit
entendre qu'il ne pouvait à l'extérieur se séparer de sentiment du duc
et de la duchesse d'Aremberg. Tant qu'il en demeura en ces termes, ils
eurent patience dans l'espérance d'en être secourus\,; mais bientôt il
fut question d'ôter la tutelle de la petite-fille, que la duchesse
d'Aremberg, sa grand'mère, prétendit. À l'instant le cardinal la lui
disputa\,; et pour rendre sa prétention meilleure, se hasarda à déclamer
contre l'indignité d'un pareil mariage, qui faisait un tel affront à sa
maison, conduit et consommé dans la maison maternelle.

Le jugement manqua ici au cardinal de Bouillon comme dans toutes les
occasions de sa vie. Pour ravir le bien il attaquait la vigilance de la
duchesse d'Aremberg, et la voulait rendre responsable de l'égarement de
sa fille et sa nièce\footnote{Phrase elliptique, comme il y en a souvent
  dans Saint-Simon. La fille de la duchesse d'Aremberg était nièce du
  cardinal de Bouillon.}, et l'en châtier en lui ôtant la tutelle de
l'enfant. C'est ce qui le perdit, je ne dirai pas d'honneur, ce ne fut
qu'un en-sus de ce qu'il n'avait plus il y avait longtemps, et de
{[}ce{]} que même il n'eut jamais, mais l'en-sus fut violent, et
retentit cruellement partout où les Aremberg et les Bouillon étaient
connus. Mésy expliqua toute l'affaire, sa femme la raconta à qui voulut
l'entendre\,; la duchesse d'Aremberg les fit interroger juridiquement\,;
il tint à peu que le cardinal ne le fût lui-même. Ce fut un prodigieux
fracas que cette révélation de son crime dont sa conduite pour la
tutelle ne laissait plus la vue obscure. Prêt à succomber, il aima mieux
se désister, et la tutelle entière fut donnée à la duchesse d'Aremberg,
sans que le cardinal de Bouillon fût compté pour rien. L'ignominie dont
cette affaire le couvrit dans l'asile où il avait cru régner le jeta
dans un nouveau désespoir que son peu de moyens et le mépris public qui
ne lui fut pas ménagé, rendit extrêmes.

Sa famille en France {[}fut{]} enragée contre lui, et tout ce qui tenait
aux Aremberg dans les Pays-Bas, hors de toute mesure avec un allié si
proche, qui payait leur assistance et leur hospitalité d'une perfidie si
signalée et d'un si infâme intérêt. Ce nouvel accident le rendit errant
de ville en ville et de lieu en lieu sans savoir où s'arrêter, jusqu'à
ce qu'enfin il se fixa auprès d'Utrecht, où il ne vit presque personne.
Les deux amants errèrent de leur côté. L'indigence éteignit leur amour.
Mésy oublia son premier état et fit le mari fâcheux jusqu'à maltraiter
sa femme, qu'il quitta dans la suite, et ils allèrent où ils purent,
chacun de son côté. La petite mineure fut élevée par la duchesse
d'Aremberg, sa grand'mère, qui la maria à un palatin, cadet de la
branche de Sultzbach, dont les aînés moururent sans mâles. Eux-mêmes ne
vécurent pas longtemps, mais ils laissèrent postérité dont l'aîné est
aujourd'hui électeur palatin.

Deux femmes très-différentes moururent fort vieilles au commencement de
cette année\,: M\textsuperscript{me} de Pomponne, veuve du ministre
d'État, belle-mère de Torcy et sœur de Lavocat, duquel j'ai parlé (t.
II, p.~373)\,; c'était une femme pieuse, retirée, qui aimait ses écus,
et qui n'avait jamais fait grande figure dans les ambassades ni pendant
le ministère de son mari, quoique dans une grande union ensemble.
L'autre fut M\textsuperscript{me} de Mortagne, fort décrépite, dont la
maison et la considération était usée depuis longtemps. Il y aurait
beaucoup à dire de cette manière de fée si je n'en avais suffisamment
parlé.

Deux hommes d'Église moururent aussi en même temps, tout aussi
différents l'un de l'autre. Tressan, évêque du Mans, qui avait eu la
charge de premier aumônier de Monsieur, après le fameux évêque de
Valence Cosnac, mort archevêque d'Aix avec le cordon bleu. Tressan était
un drôle de beaucoup d'esprit, tout tourné à l'intrigue et à la fortune,
qui eut beaucoup de crédit sur Monsieur et qui figura fort chez lui sans
s'y faire estimer. Il y attrapa force bénéfices, et vécut fort dans le
grand monde. À la fin il se hasarda trop à mesurer son crédit. Le
chevalier de Lorraine et le marquis d'Effiat ne voulurent pas compter
avec lui, ni lui avec eux\,; ils furent les plus forts. Les dégoûts et
bientôt les mépris plurent sur l'évêque\,; il lutta, puis chancela
longtemps\,; à la fin il fallut quitter prise de peur d'être chassé en
plein. Il vendit à l'abbé de Grancey, et de dépit se fixa au Mans, d'où
il gouverna tout ce qu'il put encore, et dans la province faute de
mieux. Il y fît enfin le béat, et amassa force écus. Il n'oublia rien
auprès des jésuites pour avoir son neveu pour coadjuteur, qu'il farcit
de tout ce qu'il put donner de chapelles et de rogatons de bénéfices,
dont il amassa plus de trente titres à la fois, qu'il accumula les uns
après les autres. Une meilleure fortune l'attendait, mais l'évêque ne la
vit ni n'eut lieu de l'espérer, et il laissa cet abbé en habit rapiécé,
et son autre neveu dans le ruisseau. Il avait servi dans la gendarmerie.
Le goût italien et fort à découvert l'avait banni de la société des
honnêtes gens. Il avait beaucoup d'esprit, mais tourné au mauvais. Il
lui échappa des vers qui mirent le roi en colère et le firent chasser du
service. Tombé depuis dans une grande misère, elle lui a servi de
prédicateur. Il s'est retiré au noviciat des jésuites. Il sort à pied
sans valet, fort mal vêtu et plus mal coiffé, en sorte qu'avec sa vue
basse, on le prend pour un pauvre honteux. La fortune de son frère,
archevêque de Rouen, n'a rien changé à la sienne, mais a poussé son fils
dans les gardes du corps, qui a hérité de la même veine poétique, et qui
aurait eu aussi le même sort de son père si le duc d'Ayen, son capitaine
avec qui il avait partagé le crime, eût pu être séparé de lui. Tous deux
eurent la peur entière. C'était encore beaucoup pour le temps où cela
arriva.

L'autre ecclésiastique fut l'abbé de Saint-Jacques, fils et petit-fils
des deux chanceliers Aligre. Je reviendrai à lui après un mot de
curiosité sur la singularité unique de deux chanceliers père et fils.
Les histoires et les Mémoires particuliers du règne de Louis XIII
expliquent si bien la disgrâce du chancelier de Sillery qui avait si
grandement figuré dans les affaires sous Henri IV, qui le fit garde des
sceaux, puis chancelier, en décembre 1606 et en janvier 1607, du
commandeur de Sillery, son frère, qui avait été ambassadeur à Rome et en
Espagne et qui mourut prêtre, et de Puysieux, secrétaire d'État, fils du
chancelier, que je ne fais que le remarquer ici. Cet office dont le
poids avait embarrassé le maréchal d'Ancre qui gouvernait Marie de
Médicis, régente pendant la minorité de Louis XIII, avait attiré des
disgrâces à ceux qui en étaient revêtus en divers temps, dont le mérite
de Sillery ne fut pas à couvert. Les sceaux passèrent en différentes
mains, et quelquefois les mêmes les tinrent plus d'une fois. Du Vair,
Mangot, le connétable de Luynes, les cinq derniers mois de sa vie, de
Vie, Caumartin les eurent peu chacun. Louis XIII, encore plein des
impressions de cette pratique de sa minorité, et qui l'avait suivie
depuis qu'il se fut affranchi du pesant joug de la reine mère, résolut
pourtant de remplir la charge de chancelier à la mort de Sillery,
arrivée le 1\^{}er octobre 1624\,; mais il ne voulut d'aucun sujet dont
le mérite pût figurer et faire compter avec soi. À la mort de Caumartin
il avait donné les sceaux en janvier 1624 à un des anciens du conseil
faute de mieux\,; il se trouvait tel que Louis XIII le voulait pour en
faire un chancelier, et il le fit succéder à Sillery au mois d'octobre
de la même année.

Aligre était cet ancien. Il était de Chartres, petits-fils d'un
apothicaire et fils d'un homme qui, pour son petit état, s'était enrichi
dans son négoce sans sortir de chez lui. Il mit son fils dans la maison
du comte de Soissons, à la mort duquel il fut tuteur onéraire de son
fils\footnote{Le tuteur onéraire était celui qui administrait les biens
  d'un mineur et en avait la responsabilité. Le tuteur honoraire, au
  contraire, n'était chargé que de surveiller l'éducation du mineur.}.
Cette protection le fit conseiller au grand conseil, et le premier de sa
race qui ait porté robe, il parvint après à devenir conseiller d'État,
et monta de là à la première charge de la robe, par les raisons qui
viennent d'être rapportées. Il ne put s'y maintenir longtemps. La reine
mère, réconciliée avec le roi son fils, voulut établir ses créatures.
Les sceaux furent donnés à Marillac le 1\^{}er juin 1626, et Aligre
envoyé chez lui à la Rivière, petite maison qu'il avait sous le château
de Pont-gouin, terre et maison de campagne des évêques de Chartres.
Aligre mourut en décembre 1635 à la Rivière, sans en être sorti
nonobstant les révolutions des sceaux, et cette maison de la Rivière est
devenue un beau château et une petite terre entre les mains de sa
postérité.

Il faut remarquer qu'il avait épousé Élisabeth Chapellier, sœur de M.
Chapellier, femme de Jacques Turpin, père et mère d'Elisabeth Turpin,
femme de Michel Le Tellier, chancelier de France\,; ainsi, ce chancelier
était cousin germain du second chancelier Aligre, fils du premier
chancelier de ce nom. Ce second chancelier Aligre fut conseiller au
grand conseil, intendant à Caen, intendant des finances et adjoint un
moment avec Morangis, sous le nom de directeur des finances. Il avait eu
une commission à Venise étant fort jeune, et une autre depuis pour être
un des commissaires du roi aux états de Languedoc, enfin conseiller
d'État et doyen du conseil, et comme tel premier des commissaires nommés
pour assister aux sceaux lorsque le roi les voulut tenir lui-même, à la
mort du chancelier Séguier, arrivée à Saint-Germain en Laye, 28 janvier
1672\footnote{Le tuteur onéraire était celui qui administrait les biens
  d'un mineur et en avait la responsabilité. Le tuteur honoraire, au
  contraire, n'était chargé que de surveiller l'éducation du mineur.},
et ne remplir point la charge de chancelier. Le Tellier, secrétaire
d'État de la guerre dès 1643 et devenu bientôt après ministre d'État
fort puissant, avait porté de tout son crédit son cousin Aligre aux
emplois par où il avait passé, quoique ce fût un homme sans aucune sorte
de mérite ni de lumière, et ce qu'on appelle vulgairement un très-pauvre
homme. Le Tellier eut grande envie de succéder à Séguier. Louvois, son
trop célèbre fils, était secrétaire d'État en survivance\,; il était
lors âgé de trente-deux ans\,; il était de son chef ministre d'État
comme son père, et avait eu la charge de chancelier de l'ordre à la mort
de M. de Péréfixe, archevêque de Paris. Il avait eu grande part sous son
père à la guerre de 1667 et aux conquêtes que le roi avait faites\,; il
en eut une plus entière dans les suivantes\,; et lors de cette vacance
de l'office de chancelier, lui et son père digéraient et préparaient
tout pour cette fameuse guerre qui fut déclarée en avril 1672, et qui
fut suivie de tant de rapides conquêtes en Hollande.

Cette position parut favorable au père et au fils qui étaient d'un grand
secours l'un à l'autre. Néanmoins, soit que le roi ne voulût pas se
priver du père dans les importantes fonctions de sa charge à l'ouverture
d'une si grande guerre, ou que, accoutumé à des chanceliers
octogénaires, il trouvât Le Tellier trop jeune, qui n'avait pas encore
soixante et dix ans, ils ne purent l'emporter. Pressés en même temps par
le départ du roi qui s'allait mettre à la tête de ses armées, et qui,
pendant qu'il les commanderait, ne pouvait continuer à tenir les sceaux,
ils firent en sorte que le roi, deux jours avant son départ, donnât les
sceaux à Aligre sans faire de chancelier, comme étant le plus ancien des
conseillers d'État, et le premier commissaire à l'assistance aux sceaux
tenus par le roi\,; ainsi ils se réservèrent la vacance et l'espérance
de la remplir par le mépris du concurrent, qui, leur devant tout et les
sceaux mêmes, ne pourrait et n'oserait s'en fâcher, ou s'ils n'y
pouvaient atteindre, tourner court sur le garde des sceaux tout fait,
lui procurer aisément par ce chausse-pied la place vacante, et avoir
ainsi un chancelier de paille, qui, par ce qu'il leur était et devait,
et par son imbécillité, ne les pourrait jamais embarrasser. Ils le
tinrent ainsi au filet vingt mois durant. À la fin l'indécence d'une si
longue vacance et la difficulté qu'ils trouvèrent dans le roi pour Le
Tellier, les fit tourner court à ce dernier parti, et Aligre fut fait
chancelier en janvier 1674. Il le fut et toujours en place jusqu'au 25
octobre 1677 qu'il mourut à Versailles, à plus de quatre-vingt-cinq ans.
Le Tellier eut alors sa revanche et lui succéda quatre jours après. Il
jouit huit ans de cette grande place, en faveur et en pleine santé de
corps et d'esprit, et mourut au milieu de sa brillante famille en sa
petite maison de Chaville près Versailles, le 30 octobre 1685, à
quatre-vingt-trois ans.

Ce second chancelier Aligre, qui peu à peu lui et ses enfants ont cru
s'ennoblir en changeant l'H en D et s'appelant d'Aligre\footnote{Saint-Simon
  écrit tantôt \emph{Haligre}, tantôt \emph{Aligre}\,; nous avons suivi
  l'orthographe qui est généralement adoptée.}, avait un deuxième fils
qui fit profession de bonne heure parmi les chanoines réguliers, et qui
eut en 1643 l'abbaye de Saint-Jacques, près de Provins. C'était un homme
d'esprit et de savoir, plus éminent encore en vertu, et qui se confina
dans son abbaye. On ne fut pas longtemps à s'apercevoir de l'étrange
incapacité de son père dans la place de chancelier, à qui ses
secrétaires faisaient faire tout ce qu'ils voulaient, et tant de choses
pour de l'argent que la famille en fut alarmée et vit la nécessité d'un
tuteur. Un étranger était à craindre\,; le fils aîné, plus imbécile que
le père, ne put aller plus loin qu'être maître des requêtes et intendant
de Caen\,; il fallut avoir recours au second, et au nom du roi
qu'employa Le Tellier pour tirer l'abbé de Saint-Jacques de son cloître,
qui résista tant qu'il put\,; il le mit auprès du chancelier, l'autorisa
à être présent à tout le travail particulier de son père, qui ne signa
plus rien et ne décida plus qu'en sa présence, et dont les secrétaires
eurent défense du roi très-expresse d'expédier quoi que ce fût sans
l'ordre de l'abbé sur chaque expédition. De cette manière c'était lui
qui était chancelier et garde des sceaux d'effet, et qui le fut
excellent en exactitude, en probité, en capacité, et qui, par son
esprit, sa douceur, sa modestie et la facilité de son accès, satisfit
également tout ce qui eut affaire à son père et à lui.

Il ne mit pas le pied hors de chez le chancelier pendant plusieurs
années qu'il y fut, y était présent à tout pour décider et diriger tout,
et, le peu de temps qu'il pouvait ménager, il le donnait à Dieu, retiré
dans sa chambre, sans avoir l'air moins libre et moins agréable avec la
compagnie dans les heures qu'il était obligé d'y être. Aussitôt que son
père fut mort, il porta les sceaux au roi, dont les louanges et les
désirs ne purent le retenir, comme ils n'avaient pu l'engager d'accepter
ni charges ni bénéfices, encore moins d'évêchés. Il demeura quelques
jours pour rendre compte de plusieurs choses à sa famille, et à M. Le
Tellier, devenu chancelier, et s'en retourna à Saint-Jacques, d'où rien
ne put plus le faire sortir. Il y entretint toute la régularité de la
règle, sans rien exiger de plus que cette exactitude, mais pour lui,
sans se séparer de ses religieux pour les exercices communs. Il ne
s'épargna aucune sorte d'austérité, et il parvint enfin à celle des
anciens anachorètes. Ses aumônes surprenaient tous les ans par leur
abondance à proportion de ses moyens, et il vécut ainsi croissant
toujours en mérite, adoré dans sa maison, et en vénération singulière
partout, sans se relâcher jamais jusqu'à sa mort, âgé de
quatre-vingt-seize ans, avec sa tête tout entière. Cette longueur d'une
vie si prodigieuse en austérités de toute espèce, de douceur de
gouvernement, d'agrément de conversation, lorsqu'il était forcé de
parler, de sagesse de conduite et d'instruction, fut un autre miracle
qui ne s'était point vu depuis les anciens Pères des déserts, quoique au
milieu d'une communauté simplement régulière.

D'Antin perdit Gondrin, son fils aîné, qui laissa des enfants d'une sœur
du duc de Noailles, qui, longtemps après, se remaria au comte de
Toulouse. Elle fut si affligée qu'elle en tomba malade au point qu'on
lui apporta les sacrements. Toute sa famille y était présente, et la
maréchale de Noailles sa mère, qui l'aimait passionnément, était fondue
en larmes au pied de son lit, qui priait Dieu à genoux, tout haut et de
tout son cœur, et qui, dans l'excès de sa douleur, s'offrait elle-même à
lui et tous ses enfants s'il les voulait prendre. La Vallière, qui était
là aussi à quelque distance et qui l'entendit, se leva doucement, alla à
elle et lui dit tout haut d'un air fort pitoyable\,: «\, Madame, les
gendres en sont-ils aussi\,?» Personne de ce qui y était ne put résister
à l'éclat de rire qui les prit tous, et la maréchale aussi, avec un
scandale fort ridicule, et qui courut aussitôt par toute la cour\,; la
malade se porta bientôt mieux, et on n'en rit que de plus belle.

Razilly mourut assez brusquement à Marly. Je l'ai suffisamment fait
connaître, lorsque j'ai parlé de la charge qu'il eut de premier écuyer
de M. le duc de Berry et de l'injuste dépit qu'en eut
M\textsuperscript{me} la duchesse de Berry. Les grandes commodités de
l'emploi le firent rechercher par des gens de la première qualité. Le
chevalier de Roye, le marquis de Lévi, mort duc et pair, s'y
présentèrent entre autres\,; tous deux en eurent parole positive de la
bouche de M\textsuperscript{me} la duchesse de Berry, qu'on savait bien
qui déciderait M. le duc de Berry\,; tous deux, à l'insu l'un de
l'autre, nous en firent confidence. M\textsuperscript{me} de Lévi, qui
avait eu tant de part au mariage de M\textsuperscript{me} la duchesse de
Berry, appuyée du duc de Chevreuse son père et du duc de Beauvilliers,
elle-même de tous les particuliers du roi chez M\textsuperscript{me} de
Maintenon, n'imaginait pas que cela pût balancer\,; le comte et la
comtesse de Roucy de même, avec le reste de crédit de M. de La
Rochefoucauld, et les places des Pontchartrain. Pendant qu'ils s'en
flattaient, d'Antin s'avisa de parler à M. {[}le duc{]} et à
M\textsuperscript{me} la duchesse de Berry pour Sainte-Maure, son
cousin, demeuré malade à Versailles, et l'emporta. Les deux prétendants,
si sûrs de leur fait par la parole qu'ils avaient eue, furent
étrangement surpris et si piqués, qu'ils la publièrent, et que, non
contents du bruit peu mesuré qu'ils en firent, ne se contraignirent pas
d'en dire leur avis à M\textsuperscript{me} la duchesse de Berry, dont
l'embarras et le dépit fut extrême, surtout contre la comtesse de Roucy
et M\textsuperscript{me} de Lévi qui lui parlèrent avec la dernière
hauteur, jusqu'à lui dire qu'après ce trait elles n'auraient plus qu'à
lui faire la révérence en lieux publics et jamais ailleurs, parce qu'ils
n'auraient jamais ni besoin ni dépendance d'elle. Elle se plaignit à son
tour du manque de respect\,; mais elle n'était ni aimée, ni estimée, ni
comptée\,; on savait à quoi elle en était avec le roi,
M\textsuperscript{me} de Maintenon, et au fond avec la Dauphine. Le roi
ne s'en mêla point, et le monde trouva qu'elle n'avait que ce qu'elle
méritait. Elle ne laissa pas de craindre les particuliers de
M\textsuperscript{me} de Lévi, et quelque temps après voulut elle-même
la rapprocher, puis lui faire parler. Ses avances furent méprisées\,;
elle ne le lui pardonna jamais. M\textsuperscript{me} de Lévi s'en
moqua, et garda trop peu de mesures en propos, et même en contenance,
lorsqu'elle se trouvait dans les mêmes lieux. Sainte-Maure eut quarante
mille écus à donner aux enfants de Razilly, tous bien faits, honnêtes
gens et dans le service, dont l'aîné eut la lieutenance générale de
Touraine qu'avait son père.

J'ai si souvent parlé ici du maréchal Catinat, de sa vertu, de sa
sagesse, de sa modestie, de son désintéressement, de la supériorité si
rare de ses sentiments, de ses grandes parties de capitaine, qu'il ne me
reste plus à dire que sa mort dans un âge très-avancé, sans avoir été
marié, ni avoir acquis aucunes richesses, dans sa petite maison de
Saint-Gatien, près Saint-Denis, où il s'était retiré, d'où il ne sortait
plus depuis quelques années, et où il ne voulait presque plus recevoir
personne. Il y rappela, par sa simplicité, par sa frugalité, par le
mépris du monde, par la paix de son âme, et l'uniformité de sa conduite,
le souvenir de ces grands hommes qui, après les triomphes les mieux
mérités, retournaient tranquillement à leur charrue, toujours amoureux
de leur patrie, et peu sensibles à l'ingratitude de Rome qu'ils avaient
si bien servie. Catinat mit sa philosophie à profit par une grande
piété. Il avait de l'esprit, un grand sens, une réflexion mûre, il
n'oublia jamais le peu qu'il était. Ses habits, ses équipages, ses
meubles, sa maison, tout était de la dernière simplicité\,; son air
l'était aussi et tout son maintien. Il était grand, brun, maigre, un air
pensif et assez lent, assez bas, de beaux yeux et fort spirituels. Il
déplorait les fautes signalées qu'il voyait se succéder sans cesse,
l'extinction suivie de toute émulation, le luxe, le vide, l'ignorance,
la confusion des états, l'inquisition mise à la place de la police\,; il
voyait tous les signes de destruction, et il disait qu'il n'y avait
qu'un comble très-dangereux de désordre qui pût enfin rappeler l'ordre
dans ce royaume.

Magnac, lieutenant général, inspecteur de cavalerie et gouverneur du
Mont-Dauphin, mourut en même temps dans une grande vieillesse. J'en ai
parlé plus d'une fois, surtout à l'occasion de la bataille de
Friedlingen que Villars croyait perdue, désespéré sous un arbre fort
loin, à qui il apprit qu'il l'avait gagnée, en sorte que je n'ai rien à
ajouter.

Lussan, qui était à M. le Prince, qui le fit faire chevalier de l'ordre
par grâce, en 1688, et duquel j'ai aussi parlé ailleurs, mourut aussi en
ce même temps à quatre-vingt-quatre ou quatre-vingt-cinq ans.

\hypertarget{chapitre-iv.}{%
\chapter{CHAPITRE IV.}\label{chapitre-iv.}}

1712

~

{\textsc{La Dauphine à Marly pour la dernière fois.}} {\textsc{- M. le
Duc éborgné.}} {\textsc{- Retour à Versailles.}} {\textsc{- Tabatière
très-singulièrement perdue.}} {\textsc{- La Dauphine malade.}}
{\textsc{- La Dauphine change de confesseur et reçoit les sacrements,}}
{\textsc{- Mort de la Dauphine.}} {\textsc{- Éloge, fraits et caractère
de la Dauphine.}} {\textsc{- Le roi à Marly.}} {\textsc{- Le Dauphin à
Versailles, puis à Marly.}} {\textsc{- État du Dauphin, que je vois pour
la dernière fois.}} {\textsc{- Le Dauphin malade.}} {\textsc{- Le
Dauphin croit Boudin bien averti.}} {\textsc{- Bouldue\,; quel\,; juge
Boudin bien averti.}} {\textsc{- Mort du Dauphin.}} {\textsc{- Je veux
tout quitter et me retirer de la cour et du monde\,;
M\textsuperscript{me} de Saint-Simon m'en empêche sagement.}} {\textsc{-
Éloge, traits et caractère du Dauphin.}}

~

Le roi, comme je l'ai dit, était allé à Marly le lundi 18 janvier. La
Dauphine s'y rendit de bonne heure avec une grande fluxion sur le
visage, et se mit au lit en arrivant. Elle se leva à sept heures, parce
que le roi voulut qu'elle tînt le salon. Elle y joua en déshabillé, tout
embéguinée, vit le roi chez M\textsuperscript{me} de Maintenon peu avant
son souper, et de là vint se mettre au lit, où elle soupa. Elle ne se
leva le lendemain 19 que pour jouer dans le salon et voir le roi, d'où
elle revint se mettre au lit et y souper. Le 20, sa fluxion diminua, et
elle fut mieux\,; elle y était assez sujette par le désordre de ses
dents. Elle vécut les jours suivants à son ordinaire.

Le samedi 30, le Dauphin et M. le duc de Berry allèrent avec M. le Duc
faire des battues. Il gelait assez fort\,; le hasard fit que M. le duc
de Berry se trouva au bord d'une mare d'eau fort grande et longue, et M.
le Duc de l'autre côté fort loin, vis-à-vis de lui. M. le duc de Berry
tira\,; un grain de plomb, qui glissa et rejaillit sur la glace, porta
jusqu'à M. le Duc à qui il creva un œil. Le roi apprit cet accident dans
ses jardins. Le lendemain dimanche, M. le duc de Berry alla se jeter aux
genoux de M\textsuperscript{me} la Duchesse. Il n'avait osé y aller la
veille, ni voir depuis M. le Duc qui prit ce malheur avec beaucoup de
patience. Le roi le fut voir le dimanche, le Dauphin aussi et la
Dauphine qui y avait été déjà la veille. Ils y retournèrent le lendemain
lundi 1\^{}er février. Le roi fut aussi chez M\textsuperscript{me} la
Duchesse, et s'en retourna à Versailles. M\textsuperscript{me} la
Princesse, toute sa famille, et plusieurs dames familières de
M\textsuperscript{me} la Duchesse, vinrent s'établir à Marly. M. le duc
de Berry fut cruellement affligé. M. le Duc fut assez mal et assez
longtemps, puis eut la rougeole tout de suite à Marly, et après quelque
intervalle de guérison, la petite vérole à Saint-Maur.

Le vendredi 5 février, le duc de Noailles donna une fort belle boîte
pleine d'excellent tabac d'Espagne à la Dauphine, qui en prit et le
trouva fort bon. Ce fut vers la fin de la matinée\,; en entrant dans son
cabinet où personne n'entrait, elle mit cette boîte sur la table et l'y
laissa. Sur le soir la fièvre lui prit par frissons. Elle se mit au lit
et ne put se lever, même pour aller dans le cabinet du roi, après le
souper. Le samedi 6 la Dauphine, qui avait eu la fièvre toute la nuit,
ne laissa pas de se lever à son heure ordinaire et de passer la journée
à l'ordinaire, mais le soir la fièvre la reprit. Elle continua
médiocrement toute la nuit, et le dimanche 7 encore moins\,; mais sur
les six heures du soir, il lui prit tout à coup une douleur au-dessous
de la tempe, qui ne s'étendait pas tant qu'une pièce de six sous, mais
si violente qu'elle fit prier le roi qui la venait voir de ne point
entrer. Cette sorte de rage de douleur dura sans relâche jusqu'au lundi
8, et résista au tabac en fumée et à mâcher, à quantité d'opium et à
deux saignées du bras. La fièvre se montra davantage lorsque les
douleurs furent un peu calmées\,; elle dit qu'elle avait plus souffert
qu'en accouchant.

Un état si violent mit la chambre en rumeur sur la boîte que le duc de
Noailles lui avait donnée. En se mettant au lit le jour qu'elle l'avait
reçue et que la fièvre lui prit, qui était le vendredi 5, elle en parla
à ses dames, louant fort la boîte et le tabac, puis dit à
M\textsuperscript{me} de Lévi de la lui aller chercher dans son cabinet
où elle la trouverait sur la table. M\textsuperscript{me} de Lévi y fut,
ne la trouva point\,; et pour le faire court, toute espèce de
perquisition faite, jamais on ne la revit depuis que la Dauphine l'eut
laissée dans son cabinet sur cette table. Cette disparition avait paru
fort extraordinaire dès le moment qu'on s'en aperçut, mais les
recherches inutiles qui continuèrent à s'en faire, suivies d'accidents
si étranges et si prompts, jetèrent les plus sombres soupçons. Ils
n'allèrent pas jusqu'à celui qui avait donné la boîte, ou ils furent
contenus avec une exactitude si générale qu'ils ne l'atteignirent point.
La rumeur s'en restreignit même dans un cercle peu étendu. On espérait
toujours beaucoup d'une princesse adorée, et à la vie de laquelle tenait
la fortune diverse suivant les divers états de ce qui composait ce petit
cercle. Elle prenait du tabac à l'insu du roi, avec confiance, parce que
M\textsuperscript{me} de Maintenon ne l'ignorait pas\,; mais cela lui
aurait fait une vraie affaire auprès de lui s'il l'avait découvert\,; et
c'est ce qu'on craignait en divulguant la singularité de la perte de
cette boîte.

La nuit du lundi au mardi 9 février, l'assoupissement fut grand toute
cette journée, pendant laquelle le roi s'approcha du lit bien des fois,
la fièvre forte, les réveils courts avec la tête engagée, et quelques
marques sur la peau qui firent espérer que ce serait la rougeole, parce
qu'il en courait beaucoup, et que quantité de personnes connues en
étaient en ce même temps attaquées à Versailles et à Paris. La nuit du
mardi au mercredi 10 se passa d'autant plus mal que l'espérance de
rougeole était déjà évanouie. Le roi vint dès le matin chez
M\textsuperscript{me} la Dauphine, à qui on avait donné l'émétique.
L'opération en fut telle qu'on la pouvait désirer, mais sans produire
aucun soulagement. On força le Dauphin qui ne bougeait de sa ruelle de
descendre dans les jardins pour prendre l'air, dont il avait grand
besoin, mais son inquiétude le ramena incontinent dans la chambre. Le
mal augmenta sur le soir, et à onze heures il y eut un redoublement de
fièvre considérable. La nuit fut très-mauvaise. Le jeudi, 11 février, le
roi entra à neuf heures du matin chez la Dauphine, d'où
M\textsuperscript{me} de Maintenon ne sortait presque point, excepté les
temps où le roi était chez elle. La princesse était si mal qu'on résolut
de lui parler de recevoir ses sacrements. Quelque accablée qu'elle fut,
elle s'en trouva surprise\,; elle fit des questions sur son état, on lui
fit les réponses les moins effrayantes qu'on put, mais sans se départir
de la proposition, et peu à peu des raisons de ne pas différer. Elle
remercia de la sincérité de l'avis, et dit qu'elle allait se disposer.

Au bout de peu de temps on craignit les accidents. Le P. La Rue,
jésuite, son confesseur et qu'elle avait toujours paru aimer, s'approcha
d'elle pour l'exhorter à ne différer pas sa confession. Elle le regarda,
répondit qu'elle l'entendait bien et en demeura là. La Rue lui proposa
de le faire à l'heure même et n'en tira aucune réponse. En homme
d'esprit il sentit ce que c'était, et en homme de bien il tourna court à
l'instant. Il lui dit qu'elle avait peut-être quelque répugnance de se
confesser à lui, qu'il la conjurait de ne s'en pas contraindre, surtout
de ne pas craindre quoi que ce soit là-dessus\,; qu'il lui répondait de
prendre tout sur lui\,; qu'il la priait seulement de lui dire qui elle
voulait, et que lui-même l'irait chercher et le lui amènerait. Alors
elle lui témoigna qu'elle serait bien aise de se confesser à M. Bailly,
prêtre de la mission de la paroisse de Versailles. C'était un homme
estimé, qui confessait ce qui était de plus régulier à la cour, et qui,
au langage du temps, n'était pas net du soupçon de jansénisme, quoique
fort rare parmi ces barbichets. Il confessait M\textsuperscript{me}s du
Châtelet et de Nogaret, dames du palais, à qui quelquefois la Dauphine
en avait entendu parler. Bailly se trouva être allé à Paris. La
princesse en parut peinée et avoir envie de l'attendre\,; mais, sur ce
que lui remontra le P. de La Rue qu'il était bon de ne pas perdre un
temps précieux qui, après qu'elle aurait reçu les sacrements, serait
utilement employé par les médecins, elle demanda un récollet qui
s'appelait le P. Noël, que le P. La Rue fut chercher lui-même à
l'instant, et le lui amena.

On peut imaginer l'éclat que fit ce changement de confesseur en un
moment si critique et si redoutable, et tout ce qu'il fit penser. J'y
reviendrai après. Il ne faut pas interrompre un récit si intéressant et
si funestement curieux. Le Dauphin avait succombé. Il avait caché son
mal tant qu'il avait pu pour ne pas quitter le chevet du lit de la
Dauphine. La fièvre trop forte pour être plus longtemps dissimulée
l'arrêtait, et les médecins, qui lui voulaient épargner d'être témoin
des horreurs qu'ils prévoyaient, n'oublièrent rien et par eux-mêmes et
par le roi pour le retenir chez lui, et l'y soutenir de moment en moment
par les nouvelles factices de l'état de son épouse.

La confession fut longue. L'extrême-onction fut administrée incontinent
après, et le saint-viatique tout de suite, que le roi fut recevoir au
pied du grand escalier. Une heure après, la Dauphine demanda qu'on fît
les prières des agonisants. On lui dit qu'elle n'était point en cet
état-là, et avec des paroles de consolation on l'exhorta à essayer de se
rendormir. La reine d'Angleterre vint de bonne heure l'après-dînée\,;
elle fut conduite par la galerie dans le salon qui la sépare de la
chambre où était la Dauphine. Le roi et M\textsuperscript{me} de
Maintenon étaient dans ce salon, où on fit entrer les médecins pour
consulter en leur présence\,; ils étaient sept de la cour ou mandés de
Paris. Tous d'une voix opinèrent à la saignée du pied avant le
redoublement\,; et, au cas qu'elle n'eût pas le succès qu'ils en
désiraient, à donner l'émétique dans la fin de la nuit. La saignée du
pied fut exécutée à sept heures du soir. Le redoublement vint, ils le
trouvèrent moins violent que le précédent. La nuit fut cruelle. Le roi
vint de fort bonne heure chez la Dauphine. L'émétique qu'elle prit sur
les neuf heures fit peu d'effet. La journée se passa en symptômes plus
fâcheux les uns que les autres\,; une connaissance par rares
intervalles. Tout à fait sur le soir la tête tourna dans la chambre où
on laissa entrer beaucoup de gens, quoique le roi y fût, qui peu avant
qu'elle expirât en sortit, et monta en carrosse au pied du grand
escalier avec M\textsuperscript{me} de Maintenon et
M\textsuperscript{me} de Caylus, et s'en alla à Marly. Ils étaient l'un
et l'autre dans la plus amère douleur, et n'eurent pas la force d'entrer
chez le Dauphin.

Jamais princesse arrivée si jeune ne vint si bien instruite, et ne sut
mieux profiter des instructions qu'elle avait reçues. Son habile père,
qui connaissoit à fond notre cour, la lui avait peinte, et lui avait
appris la manière unique de s'y rendre heureuse. Beaucoup d'esprit
naturel et facile l'y seconda, et beaucoup de qualités aimables lui
attachèrent les cœurs, tandis que sa situation personnelle avec son
époux, avec le roi, avec M\textsuperscript{me} de Maintenon lui attira
les hommages de l'ambition. Elle avait su travailler à s'y mettre dès
les premiers moments de son arrivée\,; elle ne cessa tant qu'elle vécut
de continuer un travail si utile, et dont elle recueillit sans cesse
tous les fruits. Douce, timide, mais adroite, bonne jusqu'à craindre de
faire la moindre peine à personne, et, toute légère et vive qu'elle
était, très-capable de vues et de suite de la plus longue haleine, la
contrainte jusqu'à la gêne, dont elle sentait tout le poids, semblait ne
lui rien coûter. La complaisance lui était naturelle, coulait de
source\,; elle en avait jusque pour sa cour.

Régulièrement laide, les joues pendantes, le front trop avancé, un nez
qui ne disait rien, de grosses lèvres mordantes, des cheveux et des
sourcils châtain brun fort bien plantés, des yeux les plus parlants et
les plus beaux du monde, peu de dents et toutes pourries dont elle
parlait et se moquait la première, le plus beau teint et la plus belle
peau, peu de gorge mais admirable, le cou long avec un soupçon de goître
qui ne lui seyait point mal, un port de tête galant, gracieux,
majestueux et le regard de même, le sourire le plus expressif, une
taille longue, ronde, menue\,; aisée, parfaitement coupée, une marche de
déesse sur les nuées\,; elle plaisait au dernier point. Les grâces
naissaient d'elles-mêmes de tous ses pas, de toutes ses manières et de
ses discours les plus communs. Un air simple et naturel toujours, naïf
assez souvent, mais assaisonné d'esprit, charmait, avec cette aisance
qui était en elle, jusqu'à la communiquer à tout ce qui l'approchait.

Elle voulait plaire même aux personnes les plus inutiles et les plus
médiocres, sans qu'elle parût le rechercher. On était tenté de la croire
toute et uniquement à celles avec qui elle se trouvait. Sa gaieté jeune,
vive, active, animait tout, et sa légèreté de nymphe la portait partout
comme un tourbillon qui remplit plusieurs lieux à la fois, et qui y
donne le mouvement et la vie. Elle ornait tous les spectacles, était
l'âme des fêtes, des plaisirs, des bals, et y ravissait par les grâces,
la justesse et la perfection de sa danse. Elle aimait le jeu, s'amusait
au petit jeu, car tout l'amusait\,; elle préférait le gros, y était
nette, exacte, la plus belle joueuse du monde, et en un instant faisait
le jeu de chacun\,; également gaie et amusée à faire, les après-dînées,
des lectures sérieuses, à converser dessus, et à travailler avec ses
dames sérieuses\,; on appelait ainsi ses dames du palais les plus âgées.
Elle n'épargna rien jusqu'à sa santé, elle n'oublia pas jusqu'aux plus
petites choses, et sans cesse, pour gagner M\textsuperscript{me} de
Maintenon, et le roi par elle. Sa souplesse à leur égard était sans
pareille et ne se démentit jamais d'un moment. Elle l'accompagnait de
toute la discrétion que lui donnait la connaissance d'eux, que l'étude
et l'expérience lui avaient acquise, pour les degrés d'enjouement ou de
mesure qui étaient à propos. Son plaisir, ses agréments, je le répète,
sa santé même, tout leur fut immolé. Par cette voie elle s'acquit une
familiarité avec eux, dont aucun des enfants du roi, non pas même les
bâtards, n'avait pu approcher.

En public, sérieuse, mesurée, respectueuse avec le roi, et en timide
bienséance avec M\textsuperscript{me} de Maintenon, qu'elle n'appelait
jamais que \emph{ma tante}, pour confondre joliment le rang et l'amitié.
En particulier, causante, sautante, voltigeante autour d'eux, tantôt
perchée sur le bras du fauteuil, de l'un ou de l'autre, tantôt se jouant
sur leurs genoux, elle leur sautait au cou, les embrassait, les baisait,
les caressait, les chiffonnait, leur tirait le dessous du menton, les
tourmentait, fouillait leurs tables, leurs papiers, leurs lettres, les
décachetait, les lisait quelquefois malgré eux, selon qu'elle les voyait
en humeur d'en rire, et parlant quelquefois dessus. Admise à tout, à la
réception des courriers qui apportaient les nouvelles les plus
importantes, entrant chez le roi à toute heure, même des moments pendant
le conseil, utile et fatale aux ministres mêmes, mais toujours portée à
obliger, à servir, à excuser, à bien faire, à moins qu'elle ne fût
violemment poussée contre quelqu'un, comme elle fut contre
Pontchartrain, qu'elle nommait quelquefois au roi \emph{votre vilain
borgne}, ou par quelque cause majeure, comme elle le fut contre
Chamillart. Si libre, qu'entendant un soir le roi et
M\textsuperscript{me} de Maintenon parler avec affection de la cour
d'Angleterre dans les commencements qu'on espéra la paix par la reine
Anne\,: «\,Ma tante, se mit-elle à dire, il faut convenir qu'en
Angleterre les reines gouvernent mieux que les rois, et savez-vous bien
pourquoi, ma tante\,?» et toujours courant et gambadant, «\, c'est que
sous les rois ce sont les femmes qui gouvernent, et ce sont les hommes
sous les reines.\,» L'admirable est qu'ils en rirent tous deux et qu'ils
trouvèrent qu'elle avait raison.

Je n'oserais jamais écrire dans des Mémoires sérieux le trait que je
vais rapporter, s'il ne servait plus qu'aucun à montrer jusqu'à quel
point elle était parvenue d'oser tout dire et tout faire avec eux. J'ai
décrit ailleurs la position ordinaire où le roi et M\textsuperscript{me}
de Maintenon étaient chez elle. Un soir qu'il y avait comédie à
Versailles, la princesse, après avoir bien parlé toutes sortes de
langages, vit entrer Nanon, cette ancienne femme de chambre de
M\textsuperscript{me} de Maintenon, dont j'ai fait mention plus d'une
fois, et aussitôt s'alla mettre, tout en grand habit comme elle était et
parée, le dos à la cheminée, debout, appuyée sur le petit paravent entre
les deux tables. Nanon, qui avait une main comme dans sa poche, passa
derrière elle, et se mit comme à genoux. Le roi, qui en était le plus
proche, s'en aperçut et leur demanda ce qu'elles faisaient là. La
princesse se mit à rire, et répondit qu'elle faisait ce qu'il lui
arrivait souvent de faire les jours de comédie. Le roi insista. «\,
Voulez-vous le savoir, reprit-elle, puisque vous ne l'avez pas encore
remarqué\,? C'est que je prends un lavement d'eau. --- Comment, s'écria
le roi mourant de rire, actuellement là vous prenez un lavement\,? ---
Hé vraiment oui, dit-elle. --- Et comment faites-vous cela\,?» Et les
voilà tous quatre à rire de tout leur cœur. Nanon apportait la seringue
toute prête sous ses jupes, troussait celles de la princesse qui les
tenait comme se chauffant, et Nanon lui glissait le clystère. Les jupes
retombaient, et Nanon remportait sa seringue sous les siennes\,; il n'y
paraissait pas. Ils n'y avaient pas pris garde, ou avaient cru que Nanon
rajustait quelque chose à l'habillement. La surprise fut extrême, et
tous deux trouvèrent cela fort plaisant. Le rare est qu'elle allait avec
ce lavement à la comédie sans être pressée de le rendre, quelquefois
même elle ne le rendait qu'après le souper du roi et le cabinet\,; elle
disait que cela la rafraîchissait, et empêchait que la
touffeur\footnote{La chaleur.} du lieu de la comédie ne lui fît mal à la
tête. Depuis la découverte elle ne s'en contraignit pas plus
qu'auparavant. Elle les connaissoit en perfection, et ne laissait pas de
voir et de sentir ce que c'était que M\textsuperscript{me} de Maintenon
et M\textsuperscript{lle} Choin.

Un soir qu'allant se mettre au lit, où Mgr le duc de Bourgogne
l'attendait, et qu'elle causait sur sa chaise percée avec
M\textsuperscript{me}s de Nogaret et du Châtelet, qui me le contèrent le
lendemain, et c'était là où elle s'ouvrait le plus volontiers, elle leur
parla avec admiration de la fortune de ces deux fées, puis ajouta en
riant\,: «\,Je voudrais mourir avant M. le duc de Bourgogne, mais voir
pourtant ici ce qui s'y passerait\,; je suis sûre qu'il épouserait une
sœur grise ou une tourière des Filles de Sainte-Marie.\,» Aussi
attentive à plaire à Mgr le duc Bourgogne qu'au roi même, quoique
souvent trop hasardeuse, et se fiant trop à sa passion pour elle et au
silence de tout ce qui pouvait l'approcher, elle prenait l'intérêt le
plus vif en sa grandeur personnelle et en sa gloire. On a vu à quel
point elle fut touchée des événements de la campagne de Lille et de ses
suites, tout ce qu'elle fit pour le relever, et combien elle lui fut
utile, en tant de choses si principales dont, comme on l'a expliqué il
n'y a pas longtemps, il lui fut entièrement redevable. Le roi ne se
pouvait passer d'elle. Tout lui manquait dans l'intérieur lorsque des
parties de plaisir, que la tendresse et la considération du roi pour
elle voulait souvent qu'elle fît pour la divertir, l'empêchaient d'être
avec lui\,; et jusqu'à son souper public, quand rarement elle y
manquait, il y paraissait par un nuage de plus de sérieux et de silence
sur toute la personne du roi. Aussi, quelque goût qu'elle eût pour ces
sortes de parties, elle y était fort sobre, et se les faisait toujours
commander. Elle avait grand soin de voir le roi en partant et en
arrivant\,; et, si quelque bal en hiver, ou quelque partie en été lui
faisait percer la nuit, elle ajustait si bien les choses qu'elle allait
embrasser le roi dès qu'il était éveillé, et l'amuser du récit de la
fête.

Je me suis tant étendu ailleurs sur la contrainte où elle était du côté
de Monseigneur, et de toute sa cour particulière, que je n'en répéterai
rien ici, sinon qu'au gros de la cour il n'y paraissait rien, tant elle
avait soin de le cacher par un air d'aisance avec lui, de familiarité
avec ce qui lui était le plus opposé dans cette cour, et de liberté à
Meudon parmi eux, mais avec une souplesse et une mesure infinie. Aussi
le sentait-elle bien, et depuis la mort de Monseigneur se
promettait-elle bien de le leur rendre. Un soir qu'à Fontainebleau, où
toutes les dames des princesses étaient dans le même cabinet qu'elle et
le roi après le souper, elle avait baragouiné toutes sortes de langues,
et fait cent enfances pour amuser le roi qui s'y plaisait, elle remarqua
M\textsuperscript{me} la Duchesse et M\textsuperscript{me} la princesse
de Conti qui se regardaient, se faisaient signe et haussaient les
épaules avec un air de mépris et de dédain. Le roi levé et passé à
l'ordinaire dans un arrière-cabinet pour donner à manger à ses chiens,
et venir après donner le bonsoir aux princesses, la Dauphine prit
M\textsuperscript{me} de Saint-Simon d'une main et M\textsuperscript{me}
de Lévi de l'autre, et leur montrant M\textsuperscript{me} la Duchesse
et M\textsuperscript{me} la princesse de Conti qui n'étaient qu'à
quelques pas de distance\,: «\,Avez-vous vu, avez-vous vu\,?» leur
dit-elle\,; «\, je sais comme elles qu'à tout ce que j'ai dit et fait il
n'y a pas le sens commun, et que cela est misérable, mais il lui faut du
bruit, et ces choses-là le divertissent\,;» et tout de suite s'appuyant
sur leurs bras, elle se mit à sauter et à chantonner\,: \emph{«\,}Hé je
m'en ris\,! hé je me moque d'elles\,! et je serai leur reine, et je n'ai
que faire d'elles ni à cette heure ni jamais, et elles auront à compter
avec moi, et je serai leur reine\,;» sautant et s'élançant et
s'éjouissant de toute sa force. Ces dames lui criaient tout bas de se
taire, que ces princesses l'entendaient, et que tout ce qui était là la
voyait faire, et jusqu'à lui dire qu'elle était folle, car d'elles elle
trouvait tout bon\,; elle de sauter plus fort et de chantonner plus
haut\,: «\,Hé je me moque d'elles\,! je n'ai que faire d'elles, et je
serai leur reine,\,» et ne finit que lorsque le roi rentra. Hélas\,!
elle le croyait, la charmante princesse, et qui ne l'eût cru avec
elle\,? Il plut à Dieu pour nos malheurs d'en disposer autrement bientôt
après. Elle était si éloignée de le penser que le jour de la Chandeleur,
étant presque seule avec M\textsuperscript{me} de Saint-Simon dans sa
chambre presque toutes les dames étant allées devant à la chapelle, et
M\textsuperscript{me} de Saint-Simon demeurée pour l'y suivre au sermon,
parce que la duchesse du Lude avait la goutte, et que la comtesse de
Mailly n'y était pas, auxquelles elle suppléait toujours, la Dauphine se
mit à parler de la quantité de personnes de la cour qu'elle avait
connues et qui étaient mortes, puis de ce qu'elle ferait quand elle
serait vieille, de la vie qu'elle mènerait, qu'il n'y aurait plus guère
que M\textsuperscript{me} de Saint-Simon et M\textsuperscript{me} de
Lauzun de son jeune temps, qu'elles s'entretiendraient ensemble de ce
qu'elles auraient vu et fait, et elle poussa ainsi la conversation
jusqu'à ce qu'elle allât au sermon.

Elle aimait véritablement M. le duc de Berry, et elle avait aimé
M\textsuperscript{me} la duchesse de Berry, et compté d'en faire comme
de sa fille. Elle avait de grands égards pour Madame, et avait
tendrement aimé Monsieur, qui l'aimait de même, et lui avait sans cesse
procuré tous les amusements et tous les plaisirs qu'il avait pu, et tout
cela retomba sur M. le duc d'Orléans, en qui elle prenait un véritable
intérêt, indépendamment de la liaison qui se forma depuis entre elle et
M\textsuperscript{me} la duchesse d'Orléans\,; ils savaient et
s'aidaient de mille choses par elle sur le roi et M\textsuperscript{me}
de Maintenon. Elle avait conservé un grand attachement pour M. et
M\textsuperscript{me} de Savoie, qui étincelait, et pour son pays même,
quelquefois malgré elle. Sa force et sa prudence parurent singulièrement
dans tout ce qui se passa lors et depuis la rupture. Le roi avait
l'égard d'éviter devant elle tout discours qui pût regarder la Savoie,
elle tout l'art d'un silence éloquent, qui par des traits rarement
échappés faisaient sentir qu'elle était toute française, quoiqu'elle
laissât sentir en même temps qu'elle ne pouvait bannir de son cœur son
père et son pays. On a vu combien elle était unie à la reine sa sœur,
d'amitié, d'intérêt et de commerce.

Avec tant de grandes, de singulières et de si aimables parties, elle en
eut et de princesse et de femme, non pour la fidélité et la sûreté du
secret, elle en fut un puits, ni pour la circonspection sur les intérêts
des autres, mais pour des ombres de tableau plus humaines. Son amitié
suivait son commerce, son amusement, son habitude, son besoin\,; je n'en
ai guère vu que M\textsuperscript{me} de Saint-Simon d'exceptée\,;
elle-même l'avouait avec une grâce et une naïveté qui rendait cet
étrange défaut presque supportable en elle. Elle voulait, comme on l'a
dit, plaire à tout le monde\,; mais elle ne se put défendre que
quelques-uns ne lui plussent aussi. À son arrivée et longtemps, elle
avait été tenue dans une grande séparation, mais dès lors approchée par
de vieilles prétendues repenties, dont l'esprit romanesque était demeuré
pour le moins galant, si la caducité de l'âge en avait banni les
plaisirs\,; peu à peu dans la suite plus livrée au monde, les choix de
ce qui l'environna de son âge se firent pour la plupart moins pour la
vertu que par la faveur. La facilité naturelle de la princesse se
laissait conformer aux personnes qui lui étaient les plus familières, et
ce dont on ne sut pas profiter, elle se plaisait autant, et se trouvait
aussi à son aise et aussi amusée d'après-dînées raisonnables, mêlées de
lectures et de conversations utiles, c'est-à-dire pieuses ou
historiques, avec les dames âgées qui étaient auprès d'elle, que des
discours plus libres et dérobés des autres qui l'entraînaient plutôt
qu'elle ne s'y livrait, retenue par sa timidité naturelle et par un
reste de délicatesse. Il est pourtant vrai que l'entraînement alla bien
loin, et qu'une princesse moins aimable et moins universellement aimée,
pour ne pas dire adorée, se serait trouvée dans de cruels inconvénients.
Sa mort indiqua bien ces sortes de mystères, et manifesta toute la
cruauté de la tyrannie que le roi ne cessa point d'exercer sur les âmes
de sa famille. Quelle fut sa surprise, quelle fut celle de la cour,
lorsque, dans ces moments si terribles où on ne redoute plus que ce qui
les suit, et où tout le présent disparaît, elle voulut changer de
confesseur, dont elle répudia même tout l'ordre, pour recevoir les
derniers sacrements\,! On a vu ailleurs qu'il n'y avait que son époux et
le roi qui fussent dans l'ignorance, que M\textsuperscript{me} de
Maintenon n'y était pas, et qu'elle était extrêmement occupée qu'ils y
demeurassent profondément l'un et l'autre tandis qu'elle lui faisait
peur d'eux\,; mais elle aimait ou plutôt elle adorait la princesse, dont
les manières et les charmes lui avaient gagné le cœur\,; elle en amusait
le roi fort utilement pour elle\,; elle-même s'en amusait et, ce qui est
très-véritable quoique surprenant, elle s'en appuyait et quelquefois se
conseillait à elle. Avec toute cette galanterie, jamais femme ne parut
se soucier moins de sa figure, ni y prendre moins de précaution et de
soin\,; sa toilette était faite en un moment, le peu même qu'elle durait
n'était que pour la cour\,; elle ne se souciait de parure que pour les
bals et les fêtes, et ce qu'elle en prenait en tout autre temps, et le
moins encore qu'il lui était possible, n'était que par complaisance pour
le roi.

Avec elle s'éclipsèrent joie, plaisirs, amusements même, et toutes
espèces de grâces\,; les ténèbres couvrirent toute la surface de la
cour\,; elle l'animait tout entière, elle en remplissait tous les lieux
à la fois, elle y occupait tout, elle en pénétrait tout l'intérieur. Si
la cour subsista après elle, ce ne fut plus que pour languir. Jamais
princesse si regrettée, jamais il n'en fut si digne de l'être, aussi les
regrets n'en ont-ils pu passer, et l'amertume involontaire et secrète en
est constamment demeurée, avec un vide affreux qui n'a pu être diminué.

Le roi et M\textsuperscript{me} de Maintenon, pénétrés de la plus vive
douleur, qui fut la seule véritable qu'il ait jamais eue en sa vie,
entrèrent d'abord chez M\textsuperscript{me} de Maintenon en arrivant à
Marly\,; il soupa seul chez lui dans sa chambre, fut peu dans son
cabinet avec M. le duc d'Orléans et ses enfants naturels. M. le duc de
Berry tout occupé de son affliction, qui fut véritable et grande, et
plus encore de celle de Mgr son frère, qui fut extrême, était demeuré à
Versailles avec M\textsuperscript{me} la duchesse de Berry, qui,
transportée de joie de se voir délivrée d'une plus grande et plus aimée
qu'elle, et à qui elle devait tout, suppléa tant qu'elle put au cœur par
l'esprit, et tint une assez bonne contenance. Ils allèrent le lendemain
matin à Marly pour se trouver au réveil du roi. Mgr le Dauphin, malade
et navré de la plus intime et de la plus amère douleur, ne sortit point
de son appartement où il ne voulut voir que M. son frère, son
confesseur, et le duc de Beauvilliers qui, malade depuis sept ou huit
jours dans sa maison de la ville, fît un effort pour sortir de son lit,
pour aller admirer dans son pupille tout ce que Dieu y avait mis de
grand, qui ne parut jamais tant qu'en cette affreuse journée, et en
celles qui suivirent jusqu'à sa mort. Ce fut, sans s'en douter, la
dernière fois qu'ils se virent en ce monde. Cheverny, d'O et Gamaches
passèrent la nuit dans son appartement, mais sans le voir que des
instants. Le samedi matin 13 février, ils le pressèrent de s'en aller à
Marly, pour lui épargner l'horreur du bruit qu'il pouvait entendre sur
sa tête, où la Dauphine était morte. Il sortit à sept heures du matin,
par une porte de derrière de son appartement, où il se jeta dans une
chaise bleue qui le porta à son carrosse. Il trouva en entrant dans
l'une et dans l'autre quelques courtisans plus indiscrets encore
qu'éveillés, qui lui firent leur révérence, et qu'il reçut avec un air
de politesse. Ses trois menins vinrent dans son carrosse avec lui. Il
descendit à la chapelle, entendit la messe, d'où il se fit porter en
chaise à une fenêtre de son appartement par où il entra.
M\textsuperscript{me} de Maintenon y vint aussitôt\,; on peut juger
quelle fut l'angoisse de cette entrevue\,; elle ne put y tenir longtemps
et s'en retourna. Il lui fallut essuyer princes et princesses qui, par
discrétion, n'y furent que des moments, même M\textsuperscript{me} la
duchesse de Berry et M\textsuperscript{me} de Saint-Simon avec elle,
vers qui le Dauphin se tourna avec un air expressif de leur commune
douleur. Il demeura quelque temps seul avec M. le duc de Berry. Le
réveil du roi approchant, ses trois menins entrèrent, et je hasardai
d'entrer avec eux. Il me montra qu'il s'en apercevait avec un air de
douceur et d'affection qui me pénétra. Mais je fus épouvanté de son
regard, également contraint, fixe, avec quelque chose de farouche, du
changement de son visage, et des marques plus livides que rougeâtres,
que j'y remarquai en assez grand nombre et assez larges, et dont ce qui
était dans la chambre s'aperçut comme moi. Il était debout, et peu
d'instants après on le vint avertir que le roi était éveillé\,; les
larmes qu'il retenait lui roulaient dans les yeux. À cette nouvelle il
se tourna sans rien dire, et demeura. Il n'y avait que ses trois menins
et moi, et du Chesne\,; les menins lui proposèrent une fois ou deux
d'aller chez le roi, il ne remua ni ne répondit. Je m'approchai et je
lui fis signe d'aller, puis je le lui proposai à voix basse. Voyant
qu'il demeurait et se taisait, j'osai lui prendre le bras, lui
représenter que tôt ou tard il fallait bien qu'il vît le roi\,; qu'il
l'attendait, et sûrement avec désir de le voir et de l'embrasser\,;
qu'il y avait plus de grâce à ne pas différer\,; et en le pressant de la
sorte, je pris la liberté de le pousser doucement. Il me jeta un regard
à percer l'âme, et partit. Je le suivis quelques pas, et m'ôtai de là
pour prendre haleine. Je ne l'ai pas vu depuis. Plaise à la miséricorde
de Dieu que je le voie éternellement où sa bonté sans doute l'a mis\,!
Tout ce qui était dans Marly pour lors en très-petit nombre était dans
le grand salon. Princes, princesses, grandes entrées étaient dans le
petit, entre l'appartement du roi et celui de M\textsuperscript{me} de
Maintenon\,; elle, dans sa chambre, qui, avertie du réveil du roi, entra
seule chez lui à travers ce petit salon, et tout ce qui y était, qui
entra fort peu après. Le Dauphin, qui entra par les cabinets, trouva
tout ce monde dans la chambre du roi qui, dès qu'il le vit, l'appela
pour l'embrasser tendrement, longuement et à reprises. Ces premiers
moments si touchants ne se passèrent qu'en paroles fort entrecoupées de
larmes et de sanglots.

Le roi, un peu après, regardant le Dauphin, fut effrayé des mêmes choses
dont nous l'avions été dans sa chambre. Tout ce qui était dans celle du
roi le fut, les médecins plus que les autres. Le roi leur ordonna de lui
tâter le pouls, qu'ils trouèrent mauvais, à ce qu'ils dirent après\,;
pour lors ils se contentèrent de dire qu'il n'était pas net, et qu'il
serait fort à propos qu'il allât se mettre au lit. Le roi l'embrassa
encore, lui recommanda fort tendrement de se conserver, et lui ordonna
de s'aller coucher\,; il obéit, et ne se releva plus. Il était assez
tard dans la matinée\,; le roi avait passé une cruelle nuit, et avait
fort mal à la tête\,; il vit à son dîner le peu de courtisans
considérables qui s'y présentèrent. L'après-dînée il alla voir le
Dauphin dont la fièvre était augmentée et le pouls encore plus mauvais,
passa chez M\textsuperscript{me} de Maintenon soupa seul chez lui, et
fut peu dans son cabinet après, avec ce qui avait accoutumé d'y entrer.
Le Dauphin ne vit que ses menins, et des instants, les médecins, peu de
suite, M. son frère, assez son confesseur, un peu M. de Chevreuse, et
passa sa journée en prières, et à se faire faire de saintes lectures. La
liste pour Marly se fit, et les admis advertis comme il s'était pratiqué
à la mort de Monseigneur, qui arrivèrent successivement.

Le lendemain dimanche le roi vécut comme il avait fait la veille.
L'inquiétude augmenta sur le Dauphin. Lui-même ne cacha pas à Boudin, en
présence de du Chesne et de M. de Cheverny, qu'il ne croyait pas en
relever, et qu'à ce qu'il sentait, il ne doutait pas que l'avis que
Boudin avait eu ne fût exécuté. Il s'en expliqua plus d'une fois de
même, et toujours avec un détachement, un mépris du monde, et de tout ce
qu'il a de grand, une soumission et un amour de Dieu incomparables. On
ne peut exprimer la consternation générale. Le lundi 15 le roi fut
saigné, et le Dauphin ne fut pas mieux que la veille. Le roi et
M\textsuperscript{me} de Maintenon le voyaient séparément plus d'une
fois le jour. Du reste personne que M. son frère des moments, ses menins
comme point, M. de Chevreuse quelque peu, toujours en lectures et en
prières. Le mardi 16 il se trouva plus mal, il se sentait dévorer par un
feu consumant auquel la fièvre ne répondait pas à l'extérieur\,; mais le
pouls, enfoncé et fort extraordinaire, était très-menaçant. Le mardi fut
encore plus mauvais, mais il fut trompeur\,; ces marques de son visage
s'étendirent sur tout le corps. On les prit pour des marques de
rougeole. On se flatta là-dessus, mais les médecins et les plus avisés
de la cour n'avaient pu oublier sitôt que ces mêmes marques s'étaient
montrées sur le corps de la Dauphine, ce qu'on ne sut hors de sa chambre
qu'après sa mort.

Le mercredi 17, le mal augmenta considérablement. J'en savais à tout
moment des nouvelles par Cheverny, et quand Boulduc pouvait sortir des
instants de la chambre il me venait parler. C'était un excellent
apothicaire du roi, qui après son père avait toujours été et était
encore le nôtre avec un grand attachement, et qui en savait pour le
moins autant que les meilleurs médecins, comme nous l'avons expérimenté,
et avec cela beaucoup d'esprit et d'honneur, de discrétion et de
sagesse. Il ne nous cachait rien à M\textsuperscript{me} de Saint-Simon
et à moi. Il nous avait fait entendre plus que clairement ce qu'il
croyait de la Dauphine\,; il m'avait parlé aussi net dès le second jour
sur le Dauphin. Je n'espérais donc plus, mais il se trouve pourtant
qu'on espère jusqu'au bout contre toute espérance.

Le mercredi les douleurs augmentèrent comme d'un feu dévorant plus
violent encore\,; le soir, fort tard, le Dauphin envoya demander au roi
la permission de communier le lendemain de grand matin, sans cérémonie
et sans assistants à la messe qui se disait dans sa chambre\,; mais
personne n'en sut rien ce soir-là, et on ne l'apprit que le lendemain
dans la matinée. Ce même soir du mercredi j'allai assez tard chez le duc
et la duchesse de Chevreuse, qui logeaient au premier pavillon, et nous
au second, tous deux du côté du village de Marly. J'étais dans une
désolation extrême\,; à peine voyais-je le roi une fois le jour. Je ne
faisais qu'aller plusieurs fois le jour aux nouvelles, et uniquement
chez M. et M\textsuperscript{me} de Chevreuse, pour ne voir que des gens
aussi touchés que moi, et avec qui je fusse tout à fait libre.
M\textsuperscript{me} de Chevreuse non plus que moi n'avait aucune
espérance\,; M. de Chevreuse, toujours équanime, toujours espérant,
toujours voyant tout en blanc, essaya de nous prouver, par ses
raisonnements de physique et de médecine, qu'il y avait plus à espérer
qu'à craindre, avec une tranquillité qui m'excéda et qui me fit fondre
sur lui avec assez d'indécence, mais au soulagement de
M\textsuperscript{me} de Chevreuse et de ce peu qui était avec eux. Je
m'en revins passer une cruelle nuit. Le jeudi matin, 18 février,
j'appris dès le grand matin que le Dauphin, qui avait attendu minuit
avec impatience, avait ouï la messe bientôt après, y avait communié,
avait passé deux heures après dans une grande communication avec Dieu,
que la tête s'était après embarrassée\,; et M\textsuperscript{me} de
Saint-Simon me dit ensuite qu'il avait reçu l'extrême-onction\,; enfin,
qu'il était mort à huit heures et demie. Ces Mémoires ne sont pas faits
pour y rendre compte de mes sentiments. En les lisant on ne les sentira
que trop, si jamais longtemps après moi ils paroissent, et dans quel
état je pus être et M\textsuperscript{me} de Saint-Simon aussi. Je me
contenterai de dire qu'à peine parûmes-nous les premiers jours un
instant chacun, que je voulus tout quitter et me retirer de la cour et
du monde, et que ce fut tout l'ouvrage de la sagesse, de la conduite, du
pouvoir de M\textsuperscript{me} de Saint-Simon sur moi que de m'en
empêcher avec bien de la peine. Ce prince, héritier nécessaire puis
présomptif de la couronne, naquit terrible, et sa première jeunesse fit
trembler\,; dur et colère jusqu'aux derniers emportements, et jusque
contre les choses inanimées\,; impétueux avec fureur, incapable de
souffrir la moindre résistance, même des heures et des éléments, sans
entrer en des fougues à faire craindre que tout ne se rompît dans son
corps\,; opiniâtre à l'excès\,; passionné pour toute espèce de volupté,
et des femmes, et, ce qui est rare à la fois, avec un autre penchant
tout aussi fort. Il n'aimait pas moins le vin, la bonne chère, la chasse
avec fureur, la musique avec une sorte de ravissement, et le jeu encore,
où il ne pouvait supporter d'être vaincu, et où le danger avec lui était
extrême\,; enfin, livré à toutes les passions et transporté de tous les
plaisirs\,; souvent farouche, naturellement porté à la cruauté\,;
barbare en railleries et à produire les ridicules avec une justesse qui
assommait. De la hauteur des cieux il ne regardait les hommes que comme
des atomes avec qui il n'avait aucune ressemblance quels qu'ils fussent.
À peine MM. ses frères lui paraissaient-ils intermédiaires entre lui et
le genre humain, quoiqu'on {[}eût{]} toujours affecté de les élever tous
trois ensemble dans une égalité parfaite. L'esprit, la pénétration
brillaient en lui de toutes parts. Jusque dans ses furies ses réponses
étonnaient. Ses raisonnements tendaient toujours au juste et au profond,
même dans ses emportements. Il se jouait des connaissances les plus
abstraites. L'étendue et la vivacité de son esprit étaient prodigieuses,
et l'empêchaient de s'appliquer à une seule chose à la fois jusqu'à l'en
rendre incapable. La nécessité de le laisser dessiner en étudiant, à
quoi il avait beaucoup de goût et d'adresse, et sans quoi son étude
était infructueuse, a peut-être beaucoup nui à sa taille.

Il était plutôt petit que grand, le visage long et brun, le haut parfait
avec les plus beaux yeux du monde, un regard vif, touchant, frappant,
admirable, assez ordinairement doux, toujours perçant, et une
physionomie agréable, haute, fine, spirituelle jusqu'à inspirer de
l'esprit. Le bas du visage assez pointu, et le nez long, élevé, mais
point beau, n'allait pas si bien\,; des cheveux châtains si crépus et en
telle quantité qu'ils bouffaient à l'excès\,: les lèvres et la bouche
agréables quand il ne parlait point, mais quoique ses dents ne fussent
pas vilaines, le râtelier supérieur s'avançait trop, et emboîtait
presque celui de dessous, ce qui, en parlant et en riant, faisait un
effet désagréable. Il avait les plus belles jambes et les plus beaux
pieds qu'après le roi j'aie jamais vus à personne, mais trop longues,
aussi bien que ses cuisses, pour la proportion de son corps. Il sortit
droit d'entre les mains des femmes. On s'aperçut de bonne heure que sa
taille commençait à tourner. On employa aussitôt et longtemps le collier
et la croix de fer, qu'il portait tant qu'il était dans son appartement,
même devant le monde, et on n'oublia aucun des jeux et des exercices
propres à le redresser. La nature demeura la plus forte. Il devint
bossu, mais si particulièrement d'une épaule, qu'il en fut enfin
boiteux, non qu'il n'eût les cuisses et les jambes parfaitement égales,
mais parce que, à mesure que cette épaule grossit, il n'y eut plus, des
deux hanches jusqu'aux deux pieds, la même distance, et au lieu d'être à
plomb il pencha de côté. Il n'en marchait ni moins aisément, ni moins
longtemps, ni moins vite, ni moins volontiers, et il n'en aima pas moins
la promenade à pied, et à monter à cheval, quoiqu'il y fût très-mal. Ce
qui doit surprendre, c'est qu'avec des yeux, tant d'esprit si élevé, et
parvenu à la vertu la plus extraordinaire et à la plus éminente et la
plus solide piété, ce prince ne se vit jamais tel qu'il était pour sa
taille, ou ne s'y accoutuma jamais. C'était une faiblesse qui mettait en
garde contre les distractions et les indiscrétions, et qui donnait de la
peine à ceux de ses gens qui dans son habillement et dans l'arrangement
de ses cheveux masquaient ce défaut naturel le plus qu'il leur était
possible, mais bien en garde de lui laisser sentir qu'ils aperçussent ce
qui était si visible. Il en faut conclure qu'il n'est pas donné à
l'homme d'être ici-bas exactement parfait.

Tant d'esprit, et une telle sorte d'esprit, joint à une telle vivacité,
à une telle sensibilité, à de telles passions, et toutes si ardentes,
n'était pas d'une éducation facile. Le duc de Beauvilliers, qui en
sentait également les difficultés et les conséquences, s'y surpassa
lui-même par son application, sa patience, la variété des remèdes. Peu
aidé par les sous-gouverneurs, il se secourut de tout ce qu'il trouva
sous sa main. Fénelon, Fleury, sous-précepteur, qui a donné une si belle
\emph{Histoire de l'Église}, quelques gentilshommes de la manche,
Moreau, premier valet de chambre, fort au-dessus de son état sans se
méconnaître, quelques rares valets de l'intérieur, le duc de Chevreuse
seul du dehors, tous mis en œuvre et tous en même esprit, travaillèrent
chacun sous la direction du gouverneur, dont l'art, déployé dans un
récit, ferait un juste ouvrage également curieux et instructif. Mais
Dieu, qui est le maître des cœurs, et dont le divin esprit souffle où il
veut, fit de ce prince un ouvrage de sa droite, et entre dix-huit et
vingt ans il accomplit son œuvre. De cet abîme sortit un prince affable,
doux, humain, modéré, patient, modeste, pénitent, et, autant et
quelquefois au delà de ce que son état pouvait comporter, humble et
austère pour soi. Tout appliqué à ses devoirs et les comprenant
immenses, il ne pensa plus qu'à allier les devoirs de fils et de sujet
avec ceux auxquels il se voyait destiné. La brièveté des jours faisait
toute sa douleur. Il mit toute sa force et sa consolation dans la
prière, et ses préparatifs en de pieuses lectures. Son goût pour les
sciences abstraites, sa facilité à les pénétrer lui déroba d'abord un
temps qu'il reconnut bientôt devoir à l'instruction des choses de son
état, et à la bienséance d'un rang destiné à régner, et à tenir en
attendant une cour.

L'apprentissage de la dévotion et l'appréhension de sa faiblesse pour
les plaisirs le rendirent d'abord sauvage. La vigilance sur lui-même, à
qui il ne passait rien et à qui il croyait devoir ne rien passer, le
renferma dans son cabinet comme dans un asile impénétrable aux
occasions. Que le monde est étrange\,! il l'eût abhorré dans son premier
état, et il fut tenté de mépriser le second. Le prince le sentit, et le
supporta\,; il attacha avec joie cette sorte d'opprobre à la croix de
son Sauveur, pour se confondre soi-même dans l'amer souvenir de son
orgueil passé. Ce qui lui fut de plus pénible, il le trouva dans les
traits appesantis de sa plus intime famille. Le roi, avec sa dévotion et
sa régularité d'écorce, vit bientôt avec un secret dépit un prince de
cet âge censurer, sans le vouloir, sa vie par la sienne, se refuser un
bureau neuf pour donner aux pauvres le prix qui y était destiné, et le
remercier modestement d'une dorure nouvelle dont on voulait rajeunir son
petit appartement. On a vu combien il fut piqué de son refus trop
obstiné de se trouver à un bal de Marly le jour des Rois. Véritablement
ce fut la faute d'un novice. Il devait ce respect, tranchons le mot,
cette charitable condescendance, au roi son grand-père, de ne l'irriter
pas par cet étrange contraste\,; mais au fond et en soi action bien
grande qui l'exposait à toutes les suites du dégoût de soi qu'il donnait
au roi, et aux propos d'une cour dont ce roi était l'idole, et qui
tournait en ridicule une telle singularité.

Monseigneur ne lui était pas une épine moins aiguë, tout livré à la
matière et à autrui, dont la politique, je dis longtemps avant les
complots de Flandres, redoutait déjà ce jeune prince, n'en apercevait
que l'écorce et sa rudesse, et s'en aliénait comme d'un censeur.
M\textsuperscript{me} la duchesse de Bourgogne, alarmée d'un époux si
austère, n'oubliait rien pour lui adoucir les moeurs. Ses charmes, dont
il était pénétré, la politique et les importunités effrénées de jeunes
dames de sa suite, déguisées en cent formes diverses, l'appât des
plaisirs et des parties auxquels il n'était rien moins qu'insensible,
tout était déployé chaque jour. Suivaient dans l'intérieur des cabinets
les remontrances de la dévote fée et les traits piquants du roi,
l'aliénation de Monseigneur grossièrement marquée, les préférences
malignes de sa cour intérieure, et les siennes trop naturelles pour M.
le duc de Berry, que son aîné, traité là en étranger qui pèse, voyait
chéri et attiré avec applaudissement. Il faut une âme bien forte pour
soutenir de telles épreuves, et tous les jours, sans en être ébranlé\,;
il faut être puissamment soutenu de la main invisible quand tout appui
se refuse au dehors, et qu'un prince de ce rang se voit livré aux
dégoûts des siens devant qui tout fléchit, et presque au mépris d'une
cour qui n'était plus retenue, et qui avait une secrète frayeur de se
trouver un jour sous ses lois. Cependant, rentré de plus en plus en
lui-même par le scrupule de déplaire au roi, de rebuter Monseigneur, de
donner aux autres de l'éloignement de la vertu, l'écorce rude et dure
peu à peu s'adoucit, mais sans intéresser la solidité du tronc. Il
comprit enfin ce que c'est que quitter Dieu pour Dieu, et que la
pratique fidèle des devoirs propres de l'état où Dieu a mis est la piété
solide qui lui est la plus agréable. Il se mit donc à s'appliquer
presque uniquement aux choses qui pouvaient l'instruire au
gouvernement\,; il se prêta plus au monde, il le fit même avec tant de
grâce et un air si naturel, qu'on sentit bientôt sa raison de s'y être
refusé, et sa peine à ne faire que s'y prêter, et le monde qui se plaît
tant à être aimé commença à devenir réconciliable.

Il réussit fort au gré des troupes en sa première campagne en Flandre
avec le maréchal de Boufflers. Il ne plut pas moins à la seconde, où il
prit Brisach avec le maréchal de Tallard\,; il s'y montra partout fort
librement, et fort au delà de ce que voulait Marsin, qui lui avait été
donné pour son mentor. Il fallut lui cacher le projet de Landau pour le
faire revenir à la cour, qui n'éclata qu'ensuite. Les tristes
conjonctures des années suivantes ne permirent pas de le renvoyer à la
tête des armées. À la fin on y crut sa présence nécessaire pour les
ranimer, et y rétablir la discipline perdue. Ce fut en 1708. On a vu
l'horoscope que la connaissance des intérêts et des intrigues m'en fit
faire au duc de Beauvilliers dans les jardins de Marly, avant que la
déclaration fût publique, et on a vu l'incroyable succès, et par quels
rapides degrés de mensonges, d'art, de hardiesse démesurée d'une
impudence à trahir le roi, l'État, la vérité jusqu'alors inouïe, une
infernale cabale, la mieux organisée qui fût jamais, effaça ce prince
dans le royaume dont il devait porter la couronne, et dans sa maison
paternelle, jusqu'à rendre odieux et dangereux d'y dire un mot en sa
faveur. Cette monstrueuse anecdote a été si bien expliquée en son lieu
que je ne fais que la rappeler ici. Une épreuve si étrangement nouvelle
et cruelle était bien dure à un prince qui voyait tout réuni contre lui,
et qui n'avait pour soi que la vérité suffoquée par tous les prestiges
des magiciens de Pharaon\,; il la sentit dans tout son poids, dans toute
son étendue, dans toutes ses pointes. Il la soutint aussi avec toute la
patience, la fermeté, et surtout avec toute la charité d'un élu qui ne
voit que Dieu en tout, qui s'humilie sous sa main, qui se purifie dans
le creuset que cette divine main lui présente, qui lui rend grâces de
tout, qui porte la magnanimité jusqu'à ne vouloir dire ou faire que
très-précisément ce qu'il se doit, à l'État, à la vérité, et qui est
tellement en garde contre l'humanité qu'il demeure bien en deçà des
bornes les plus justes et les plus saintes.

Tant de vertu trouva enfin sa récompense dès ce monde, et avec d'autant
plus de pureté, que le prince, bien loin d'y contribuer, se tint encore
fort en arrière. J'ai assez expliqué tout ce qui regarde cette précieuse
révolution, {[}pour{]} que je me contente ici de la montrer, et que les
ministres et la cour aux pieds de ce prince devenu le dépositaire du
cœur du roi, de son autorité dans les affaires et dans les grâces, et de
ses soins pour le détail du gouvernement. Ce fut alors qu'il redoubla
plus que jamais d'application aux choses du gouvernement, et à
s'instruire de tout ce qui pouvait l'en rendre plus capable. Il bannit
tout amusement de sciences pour partager son cabinet entre la prière
qu'il abrégea, et l'instruction qu'il multiplia\,; et le dehors entre
son assiduité auprès du roi, ses soins pour M\textsuperscript{me} de
Maintenon, la bienséance et son goût pour son épouse, et l'attention à
tenir une cour, et à s'y rendre accessible et aimable. Plus le roi
l'éleva, plus il affecta de se tenir soumis en sa main, plus il lui
montra de considération et de confiance, plus il y sut répondre par le
sentiment, la sagesse, les connaissances, surtout par une modération
éloignée de tout désir et de toute complaisance en soi-même, beaucoup
moins de la plus légère présomption. Son secret et celui des autres fut
toujours impénétrable chez lui.

Sa confiance en son confesseur n'allait pas jusqu'aux affaires\,; j'en
ai rapporté deux exemples mémorables sur deux très-importantes aux
jésuites qu'ils attirèrent devant le roi, contre lesquels il fut de
toutes ses forces. On ne sait si celle qu'il aurait prise en M. de
Cambrai aurait été plus étendue\,; on n'en peut juger que par celle
qu'il avait en M. de Chevreuse, et plus en M. de Beauvilliers qu'en qui
que ce fût. On peut dire de ces deux beaux-frères qu'ils n'étaient qu'un
coeur et qu'une âme, et que M. de Cambrai en était la vie et le
mouvement\,; leur abandon pour lui était sans bornes, leur commerce
secret était continuel. Il était sans cesse consulté sur grandes et sur
petites choses, publiques, politiques, domestiques\,; leur conscience de
plus était entre ses mains\,; le prince ne l'ignorait pas\,; et je me
suis toujours persuadé, sans néanmoins aucune notion autre que
présomption, que le prince même le consultait par eux, et que c'était
par eux que s'entretenait cette amitié, cette estime, cette confiance
pour lui si haute et si connue. Il pouvait donc compter, et il comptait
sûrement aussi parler et entendre tous les trois, quand il parlait ou
écoutait l'un d'eux. Sa confiance néanmoins avait des degrés entre les
deux beaux-frères\,; s'il l'avait avec abandon pour quelqu'un, c'était
certainement pour le duc de Beauvilliers. Toutefois il y avait des
choses où ce duc n'entamait pas son sentiment, par exemple beaucoup de
celles de la cour de Rome, d'autres qui regardaient le cardinal de
Noailles, quelques autres de goût et d'affections\,; c'est ce que j'ai
vu de mes yeux et ouï de mes oreilles.

Je ne tenais à lui que par M. de Beauvilliers, et je ne crois pas faire
un acte d'humilité de dire qu'en tous sens et en tous genres, j'étais
sans aucune proportion avec lui. Néanmoins il a souvent concerté avec
moi pour faire ou sonder, ou parler, ou inspirer, approcher, écarter de
ce prince par moi, pris ses mesures sur ce que je lui disais\,; et plus
d'une fois, lui rendant compte de mes tête-à-tête avec le prince, il m'a
fait répéter de surprise des choses qu'il m'avouait sur lesquelles il ne
s'était jamais tant ouvert avec lui, et d'autres qu'il ne lui avait
jamais dites. Il est vrai que celles-là ont été rares, mais elles ont
été, et elles ont été plus d'une fois. Ce n'est pas assurément que ce
prince eût en moi plus de confiance. J'en serais si honteux, et pour lui
et pour moi, que, s'il avait été capable d'une si lourde faute, je me
garderais bien de la laisser sentir\,; mais je m'étends sur ce détail
qui n'a pu être aperçu que de moi, pour rendre témoignage à cette
vérité\,: que la confiance la plus entière de ce prince, et la plus
fondée sur tout ce qui la peut établir et la rendre toujours durable,
n'alla jamais jusqu'à l'abandon, et à une transformation qui devient
trop souvent le plus grand malheur des rois, des cours, des peuples et
des États même.

Le discernement de ce prince n'était donc point asservi, mais comme
l'abeille il recueillait la plus parfaite substance des plus belles et
des meilleures fleurs. Il tâchait à connaître les hommes, à tirer d'eux
les instructions et les lumières qu'il en pouvait espérer. Il conférait
quelquefois, mais rarement avec quelques-uns, mais à la passade, sur des
matières particulières\,; plus rarement en secret sur des
éclaircissements qu'il jugeait nécessaires, mais sans retour et sans
habitude. Je n'ai point su, et cela ne m'aurait pas échappé, qu'il
travaillât habituellement avec personne qu'avec les ministres, et le duc
de Chevreuse l'était, et avec les prélats dont j'ai parlé sur l'affaire
du cardinal de Noailles. Hors ce nombre, j'étais le seul qui eusse ses
derrières libres et fréquents, soit de sa part ou de la mienne. Là, il
découvrait son âme et pour le présent et pour l'avenir avec confiance,
et toutefois avec sagesse, avec retenue, avec discrétion. Il se laissait
aller sur les plans qu'il croyait nécessaires, il se livrait sur les
choses générales, il se retenait sur les particulières, et plus encore
sur les particuliers\,; mais, comme il voulait sur cela même tirer de
moi tout ce qui pouvait lui servir, je lui donnais adroitement lieu à
des échappées, et souvent avec succès, par la confiance qu'il avait
prise en moi de plus en plus, et que je devais toute au duc de
Beauvilliers, et en sous-ordre au duc de Chevreuse, à qui je ne rendais
pas le même compte qu'à son beau-frère, mais à qui je ne laissais pas de
m'ouvrir fort souvent comme lui à moi.

Un volume ne décrirait pas suffisamment ces divers tête-à-tête entre ce
prince et moi. Quel amour du bien\,! quel dépouillement de soi-même\,!
quelles recherches\,! quels fruits\,! quelle pureté d'objet, oserai-je
le dire, quel reflet de la Divinité dans cette âme candide, simple,
forte, qui, autant qu'il leur est donné ici-bas, en avait conservé
l'image\,! On y sentait briller les traits d'une éducation également
laborieuse et industrieuse, également savante, sage, chrétienne, et les
réflexions d'un disciple lumineux, qui était né pour le commandement.
Là, s'éclipsaient les scrupules qui le dominaient en public. Il voulait
savoir à qui il avait et à qui il aurait affaire\,; il mettait au jeu le
premier pour profiter d'un tête-à-tête sans fard et sans intérêt. Mais
que le tête-à-tête avait de vaste, et que les charmes qui s'y trouvaient
étaient agités par la variété où le prince s'espaçait et par art, et par
entraînement de curiosité, et par la soif de savoir\,! De l'un à l'autre
il promenait son homme sur tant de matières, sur tant de choses, de gens
et de faits, que qui n'aurait pas eu à la main de quoi le satisfaire en
serait sorti bien mal content de soi, et ne l'aurait pas laissé
satisfait. La préparation était également imprévue et impossible.
C'était dans ces impromptus que le prince cherchait à puiser des vérités
qui ne pouvaient ainsi rien emprunter d'ailleurs, et à éprouver, sur des
connaissances ainsi variées, quel fond il pouvait faire en ce genre sur
le choix qu'il avait fait.

De cette façon, son homme, qui avait compté ordinairement sur une
matière à traiter avec lui, et en avoir pour un quart d'heure, pour une
demi-heure, y passait deux heures et plus, suivant que le temps en
laissait plus ou moins de liberté au prince. Il se ramenait toujours à
la matière qu'il avait destinée de traiter en principal\,; mais à
travers les parenthèses qu'il présentait, et qu'il maniait en maître, et
dont quelques-unes étaient assez souvent son principal objet. Là, nul
verbiage, nul compliment, nulles louanges, nulles chevilles, aucune
préface, aucun conte, pas la plus légère plaisanterie\,; tout objet,
tout dessein, tout serré, substantiel, au fait, au but, rien sans
raison, sans cause, rien par amusement et par plaisir\,; c'était là que
la charité générale l'emportait sur la charité particulière, et que ce
qui était sur le compte de chacun se discutait exactement\,; c'était là
que les plans, les arrangements, les changements, les choix se
formaient, se mûrissaient, se découvraient, souvent tout mâchés, sans le
paraître, avec le duc de Beauvilliers, quelquefois avec lui et le duc de
Chevreuse, qui néanmoins étaient tous deux ensemble très-rarement avec
lui. Quelquefois encore il y avait de la réserve pour tous les deux ou
pour l'un ou l'autre, quoique rare pour M. de Beauvilliers\,; mais en
tout et partout un inviolable secret dans toute sa profondeur.

Avec tant et de si grandes parties, ce prince si admirable ne laissait
pas de laisser voir un recoin d'homme, c'est-à-dire quelques défauts, et
quelquefois même peu décents\,; et c'est ce que, avec tant de solide et
de grand, on avait peine à comprendre, parce qu'on ne voulait pas se
souvenir qu'il n'avait été que vice et que défaut, ni réfléchir sur le
prodigieux changement, et ce qu'il avait dû coûter, qui en avait fait un
prince déjà si proche de toute perfection qu'on s'étonnait, en le voyant
de près, qu'il ne l'eût pas encore atteinte jusqu'à son comble. J'ai
touché ailleurs quelques-uns de ces légers défauts, qui, malgré son âge,
étaient encore des enfances, qui se corrigeaient assez tous les jours
pour faire sainement augurer que bientôt elles disparaîtroient toutes.
Un plus important, et que la réflexion et l'expérience auraient sûrement
guéri, c'est qu'il était quelquefois des personnes, mais rarement, pour
qui l'estime et l'amitié de goût, même assez familière, ne marchaient
pas de compagnie. Ses scrupules, ses malaises, ses petitesses de
dévotion diminuaient tous les jours, et tous les jours il croissait en
quelque chose\,; surtout il était bien guéri de l'opinion de préférer
pour les choix la piété à tout autre talent, c'est-à-dire de faire un
ministre, un ambassadeur, un général plus par rapport à sa piété qu'à sa
capacité et à son expérience\,; il l'était encore sur le crédit à donner
à la piété, persuadé qu'il était enfin que de fort honnêtes gens, et
propres à beaucoup de choses, le peuvent être sans dévotion, et doivent
cependant être mis en œuvre, et du danger encore de faire des
hypocrites.

Comme il avait le sentiment fort vif, il le passait aux autres, et ne
les en aimait et n'estimait pas moins. Jamais homme si amoureux de
l'ordre ni qui le connût mieux, ni si désireux de le rétablir en tout,
d'ôter la confusion, et de mettre gens et choses en leurs places.
Instruit au dernier point de tout ce qui doit régler cet ordre par
maximes, par justice et par raison, et attentif, avant qu'il fût le
maître, de rendre à l'âge, au mérite, à la naissance, au rang, la
distinction propre à chacune de ces choses, et de la marquer en toutes
occasions. Ses desseins allongeraient trop ces Mémoires. Les expliquer
serait un ouvrage à part, mais un ouvrage à faire mourir de regrets.
Sans entrer dans mille détails sur le comment, sur les personnes, je ne
puis toutefois m'en refuser ici quelque chose en gros. L'anéantissement
de la noblesse lui était odieux, et son égalité entre elle
insupportable. Cette dernière nouveauté qui ne cédait qu'aux dignités,
et qui confondait le noble avec le gentilhomme, et ceux-ci avec les
seigneurs, lui paraissait de la dernière injustice, et ce défaut de
gradation une cause prochaine {[}de ruine{]} et destructive d'un royaume
tout militaire. Il se souvenait qu'il n'avait dû son salut dans ses plus
grands périls sous Philippe de Valois, sous Charles V, sous Charles VII,
sous Louis XII, sous François I\^{}er, sous ses petits-fils, sous Henri
IV, qu'à cette noblesse, qui se connaissoit et se tenait dans les bornes
de ses différences réciproques, qui avait la volonté et le moyen de
marcher au secours de l'État, par bandes et par provinces, sans embarras
et sans confusion, parce qu'aucun n'était sorti de son état, et ne
faisait difficulté d'obéir à plus grand que soi. Il voyait au contraire
ce secours éteint par les contraires\,; pas un qui n'en soit venu à
prétendre l'égalité à tout autre, par conséquent plus rien d'organisé,
plus de commandement et plus d'obéissance.

Quant aux moyens, il était touché, jusqu'au plus profond du cœur, de la
ruine de la noblesse, des voies prises et toujours continuées pour l'y
réduire et l'y tenir, de l'abâtardissement que la misère et le mélange
du sang par les continuelles mésalliances nécessaires pour avoir du
pain, avaient établi dans les courages et pour valeur, et pour vertu, et
pour sentiments. Il était indigné de voir cette noblesse française si
célèbre, si illustre, devenue un peuple presque de la même sorte que le
peuple même, et seulement distinguée de lui en ce que le peuple a la
liberté de tout travail, de tout négoce, des armes même, au lieu que la
noblesse est devenue un autre peuple qui n'a d'autre choix qu'une
mortelle et ruineuse oisiveté, qui par son inutilité à tout la rend à
charge et méprisée, ou d'aller à la guerre se faire tuer, à travers les
insultes des commis des secrétaires d'État, et des secrétaires des
intendants, sans que les plus grands de toute cette noblesse par leur
naissance, et par les dignités qui, sans les sortir de son ordre, les
met au-dessus d'elle, puissent éviter ce même sort d'inutilité, ni les
dégoûts des maîtres de la plume lorsqu'ils servent dans les armées.
Surtout il ne pouvait se contenir contre l'injure faite aux armes, par
lesquelles cette monarchie s'est fondée et maintenue, qu'un officier
vétéran, souvent couvert de blessures, même lieutenant général des
armées, retiré chez soi avec estime, réputation, pensions même, y soit
réellement mis à la taille avec tous les autres paysans de sa paroisse,
s'il n'est pas noble, par eux et comme eux, et comme je l'ai vu arriver
à d'anciens capitaines chevaliers de Saint-Louis et à pension, sans
remède pour les en exempter, tandis que les exemptions sont sans nombre
pour les plus vils emplois de la petite robe et de la finance, même
après les avoir vendus, et quelquefois héréditaires.

Ce prince ne pouvait s'accoutumer qu'on ne pût parvenir à gouverner
l'État en tout ou en partie, si on n'avait été maître des requêtes, et
que ce fût entre les mains de la jeunesse de cette magistrature que
toutes les provinces fussent remises pour les gouverner en tout genre,
et seuls, chacun la sienne à sa pleine et entière discrétion, avec un
pouvoir infiniment plus grand, et une autorité plus libre et plus
entière, sans nulle comparaison, que les gouverneurs de ces provinces en
avaient jamais eue, qu'on avait pourtant voulu si bien abattre qu'il ne
leur en était resté que le nom et les appointements uniques, et il ne
trouvait pas moins scandaleux que le commandement de quelques provinces
fût joint et quelquefois attaché à la place du chef du parlement de la
même province, en absence du gouverneur et du lieutenant général en
titre, laquelle était nécessairement continuelle, avec le même pouvoir
sur les troupes qu'eux. Je ne répéterai point ce qu'il pensait sur le
pouvoir et sur l'élévation des secrétaires d'État, des autres ministres,
et la forme de leur gouvernement. On l'a vu il n'y a pas longtemps,
comme sur le dixième on a vu ce qu'il pensait et sentait sur la finance
et les financiers. Le nombre immense de gens employés a lever et à
percevoir les impositions ordinaires et extraordinaires, et la manière
de les lever\,; la multitude énorme d'offices et d'officiers de justice
de toute espèce\,; celle des procès, des chicanes, des frais\,;
l'iniquité de la prolongation des affaires, les ruines et les cruautés
qui s'y commettent, étaient des objets d'une impatience qui lui
inspirait presque celle d'être en pouvoir d'y remédier.

La comparaison qu'il faisait des pays d'états\footnote{On appelait pays
  d'états dans l'ancienne monarchie, ceux qui jouissaient du privilège
  d'avoir des assemblées provinciales, comme le Languedoc, la Bretagne,
  la Bourgogne, la Provence, l'Artois, le Hainaut, le Cambrésis (pays de
  Cambrai), le comté de Pau ou de Béarn, le Bigorre, le comté de Foix,
  le pays de Gex, la Bresse, le Bugey, le Valromey, le Marsan, le
  Nebouzan, les Quatre-Vallées (dans l'Armagnac), le pays de Labour,
  etc. Les états de Dauphiné, supprimés sous Louis XIII, ne furent
  rétablis que peu de temps avant la Révolution. Les pays d'états
  votaient l'impôt qu'ils devaient payer et en faisaient eux-mêmes la
  répartition.} avec les autres lui avait donné la pensée de partager le
royaume en parties, autant qu'il se pourrait, égales pour la richesse,
de faire administrer chacune par ses états, de les simplifier tous
extrêmement pour en bannir la cohue et le désordre, et d'un extrait
aussi fort simplifié de tous ces états des provinces en former
quelquefois des états généraux du royaume. Je n'ose achever un grand
mot, un mot d'un prince pénétré\,: «\,qu'un roi est fait pour les
sujets, et non les sujets pour lui,\,» comme il ne se contraignait pas
de le dire en public, et jusque dans le salon de Marly, un mot enfin de
père de la patrie, mais un mot qui hors de son règne, que Dieu n'a pas
permis, serait le plus affreux blasphème. Pour en revenir aux états
généraux, ce n'était pas qu'il leur crût aucune sorte de pouvoir. Il
était trop instruit pour ignorer que ce corps, tout auguste que sa
représentation le rende, n'est qu'un corps de plaignants, de
remontrants, et quand il plaît au roi de le lui permettre, un corps de
proposants. Mais ce prince, qui se serait plu dans le sein de sa nation
rassemblée, croyait trouver des avantages infinis d'y être informé des
maux et des remèdes par des députés qui connaîtroient les premiers par
expérience, et de consulter les derniers avec ceux sur qui ils devaient
porter. Mais dans ces états il n'en voulait connaître que trois, et
laissait fermement dans le troisième celui qui si nouvellement a paru
vouloir s'en tirer.

À l'égard des rangs, des dignités et des charges, on a vu que les rangs
étrangers, ou prétendus tels, n'étaient pas dans son goût et dans ses
maximes, et ce qui en était pour la règle des rangs. Il n'était pas plus
favorable aux dignités étrangères. Son dessein aussi n'était pas de
multiplier les premières dignités du royaume. Il voulait néanmoins
favoriser la première noblesse par des distinctions. Il sentait combien
elles étaient impossibles et irritantes par naissance entre les vrais
seigneurs, et il était choqué qu'il n'y eût ni distinction ni récompense
à leur donner, que les premières et le comble de toutes. Il pensait
donc, à l'exemple, mais non sur le modèle de l'Angleterre, à des
dignités moindres en tout que celles de ducs\,: les unes héréditaires et
de divers degrés, avec leurs rangs et leurs distinctions propres\,; les
autres à vie sur le modèle, en leur manière, des ducs non vérifiés ou à
brevet. Le militaire en aurait eu aussi, dans le même dessein et par la
même raison, au-dessous des maréchaux de France. L'ordre de Saint-Louis
aurait été beaucoup moins commun, et celui de Saint-Michel tiré de la
boue où on l'a jeté, et remis en honneur pour rendre plus réservé celui
de l'ordre du Saint-Esprit. Pour les charges, il ne comprenait pas
comment le roi avait eu pour ses ministres la complaisance de laisser
tomber les premières après les grandes de sa cour dans l'abjection où de
l'une à l'autre toutes sont tombées. Le Dauphin aurait pris plaisir d'y
être servi et environné par de véritables seigneurs, et il aurait
illustré d'autres charges moindres, et ajouté quelques-unes de nouveau
pour des personnes de qualité moins distinguées. Ce tout ensemble, qui
eût décoré sa cour et l'État, lui aurait fourni beaucoup plus de
récompenses. Mais il n'aimait pas les perpétuelles, que la même charge,
le même gouvernement devînt comme patrimoine par l'habitude de passer
toujours de père en fils. Son projet de libérer peu à peu toutes les
charges de cour et de guerre, pour en ôter à toujours la vénalité,
n'était pas favorable aux brevets de retenue ni aux survivances, qui ne
laissaient rien aux jeunes gens à prétendre ni à désirer.

Quant à la guerre, il ne pouvait goûter l'ordre du tableau\footnote{Voy.,
  sur l'ordre du tableau, t. VII, p.~387, note.} que Louvois a introduit
pour son autorité particulière, pour confondre qualité, mérite et néant,
et pour rendre peuple tout ce qui sert. Ce prince regardait cette
invention comme la destruction de l'émulation, par conséquent du désir
de s'appliquer, d'apprendre, et de faire, comme la cause de ces immenses
promotions qui font des officiers généraux sans nombre, qu'on ne peut
pour la plupart employer ni récompenser, et parmi lesquels on en trouve
si peu qui aient de la capacité et du talent, ce qui remonte enfin
jusqu'à ceux qu'il faut bien faire maréchaux de France, et entre ces
derniers jusqu'aux généraux des armées, et dont l'État éprouve les
funestes suites, surtout depuis le commencement de ce siècle, parce que
ceux qui ont précédé cet établissement n'étaient déjà plus ou hors
d'état de servir.

Cette grande et sainte maxime\,: que les rois sont faits pour leurs
peuples et non les peuples pour les rois ni aux rois, était si avant
imprimée en son âme qu'elle lui avait rendu le luxe et la guerre
odieuse. C'est ce qui le faisait quelquefois expliquer trop vivement sur
la dernière, emporté par une vérité trop dure pour les oreilles du
monde, qui a fait quelquefois dire sinistrement qu'il n'aimait pas la
guerre. Sa justice était munie de ce bandeau impénétrable qui en fait
toute la sûreté. Il se donnait la peine d'étudier les affaires qui se
présentaient à juger devant le roi aux conseils de finance et des
dépêches\,; et, si elles étaient grandes, il y travaillait avec les gens
du métier, dont il puisait des connaissances, sans se rendre esclave de
leurs opinions. Il communiait au moins tous les quinze jours avec un
recueillement et un abaissement qui frappait, toujours en collier de
l'ordre et en rabat et manteau court. Il voyait son confesseur jésuite
une ou deux fois la semaine, et quelquefois fort longtemps, ce qu'il
abrégea beaucoup dans la suite, quoiqu'il approchât plus souvent de la
communion.

Sa conversation était aimable, tant qu'il pouvait solide, et par goût\,;
toujours mesurée à ceux avec qui il parlait. Il se délassait volontiers
à la promenade\,: c'était là où ces {[}qualités{]} paraissaient le plus.
S'il s'y trouvait quelqu'un avec qui il pût parler de sciences, c'était
son plaisir, mais plaisir modeste, et seulement pour s'amuser et
s'instruire en dissertant quelque peu, et en écoutant davantage. Mais ce
qu'il y cherchait le plus c'était l'utile, des gens à faire parler sur
la guerre et les places, sur la marine et le commerce, sur les pays et
les cours étrangères, quelquefois sur des faits particuliers mais
publics., et sur des points d'histoire ou des guerres passées depuis
longtemps. Ces promenades, qui l'instruisaient beaucoup, lui
conciliaient les esprits, les cœurs, l'admiration, les plus grandes
espérances. Il avait mis à la place des spectacles, qu'il s'était
retranchés depuis fort longtemps, un petit jeu où les plus médiocres
bourses pouvaient atteindre, pour pouvoir varier et partager l'honneur
de jouer avec lui, et se rendre cependant visible à tout le monde. Il
fut toujours sensible au plaisir de la table et de la chasse. Il se
laissait aller à la dernière avec moins de scrupule, mais il craignait
son faible pour l'autre, et il y était d'excellente compagnie quand il
s'y laissait aller.

Il connaissoit le roi parfaitement, il le respectait, et sur la fin il
l'aimait en fils, et lui faisait une cour attentive de sujet, mais qui
sentait quel il était. Il cultivait M\textsuperscript{me} de Maintenon
avec les égards que leur situation demandait. Tant que Monseigneur
vécut, il lui rendait tout ce qu'il devait avec soin. On y sentait la
contrainte, encore plus avec M\textsuperscript{lle} Choin, et le malaise
avec tout cet intérieur de Meudon. On en a tant expliqué les causes
qu'on n'y reviendra pas ici. Le prince admirait, autant pour le moins
que tout le monde, que Monseigneur, qui, tout matériel qu'il était,
avait beaucoup de gloire, n'avait jamais pu s'accoutumer à
M\textsuperscript{me} de Maintenon, ne la voyait que par bienséance, et
le moins encore qu'il pouvait, et toutefois avait aussi en
M\textsuperscript{lle} Choin sa Maintenon autant que le roi avait la
sienne, et ne lui asservissait pas moins ses enfants que le roi les
siens à M\textsuperscript{me} de Maintenon. Il aimait les princes ses
frères avec tendresse, et son épouse avec la plus grande passion. La
douleur de sa perte pénétra ses plus intimes moelles. La piété y
surnagea par les plus prodigieux efforts. Le sacrifice fut entier, mais
il fut sanglant. Dans cette terrible affliction rien de bas, rien de
petit, rien d'indécent. On voyait un homme hors de soi, qui s'extorquait
une surface unie, et qui y succombait. Les jours en furent tôt abrégés.
Il fut le même dans sa maladie. Il ne crut point en relever, il en
raisonnait avec ses médecins\,; dans cette opinion, il ne cacha pas sur
quoi elle était fondée\,; on l'a dit il n'y a pas longtemps, et tout ce
qu'il sentit depuis le premier jour jusqu'au dernier l'y confirma de
plus en plus. Quelle épouvantable conviction de la fin de son épouse et
de la sienne\,! mais, grand Dieu\,! quel spectacle vous donnâtes en lui,
et que n'est-il permis encore d'en révéler des parties également
secrètes, et si sublimes qu'il n'y a que vous qui les puissiez donner et
en connaître tout le prix\,! quelle imitation de Jésus-Christ sur la
croix\,! on ne dit pas seulement à l'égard de la mort et des
souffrances, elle s'éleva bien au-dessus. Quelles tendres, mais
tranquilles vues\,! quel surcroît de détachement\,! quels vifs élans
d'actions de grâces d'être préservé du sceptre et du compte qu'il faut
en rendre\,! quelle soumission, et combien parfaite\,! quel ardent amour
de Dieu\,! quel perçant regard sur son néant et ses péchés\,! quelle
magnifique idée de l'infinie miséricorde\,! quelle religieuse et humble
crainte\,! quelle tempérée confiance\,! quelle sage paix\,! quelles
lectures\,! quelles prières continuelles\,! quel ardent désir des
derniers sacrements\,! quel profond recueillement\,! quelle invincible
patience\,! quelle douceur, quelle constante bonté pour tout ce qui
l'approchait\,! quelle charité pure, qui le pressait d'aller à Dieu\,!
La France tomba enfin sous ce dernier châtiment\,; Dieu lui montra un
prince qu'elle ne méritait pas. La terre n'en était pas digne, il était
mûr déjà pour la bienheureuse éternité.

\hypertarget{chapitre-v.}{%
\chapter{CHAPITRE V.}\label{chapitre-v.}}

1712

~

{\textsc{Obsèques pontificales à Rome pour le Dauphin.}} {\textsc{-
Époque et date de leur cessation à Rome et à Paris pour les papes et
pour nos rois.}} {\textsc{- Étrange pensée de l'archevêque de Reims sur
le duc de Noailles.}} {\textsc{- Pourquoi {[}il était{]} mal avec les
Noailles.}} {\textsc{- Embarras du P. La Rue qui surprend étrangement le
roi du changement de confesseur.}} {\textsc{- Appareil funèbre chez la
Dauphine.}} {\textsc{- Prétentions des évêques refusées.}} {\textsc{-
Règles de ces choses.}} {\textsc{- Carreau et goupillon, à qui donnés et
par qui présentés.}} {\textsc{- Annonce à haute voix\,; pour qui.}}
{\textsc{- Garde par les dames, et quelle.}} {\textsc{- Première
garde\,; comment réglée par le roi entre les duchesses et la maison de
Lorraine.}} {\textsc{- Eau bénite de peu du sang royal et du comte de
Toulouse, et point d'autres.}} {\textsc{- Le corps du Dauphin porté sans
cérémonie près de celui de la Dauphine.}} {\textsc{- Transport en
cérémonie des deux cœurs au Val-de-Grâce.}} {\textsc{- Mgr le duc de
Bretagne Dauphin.}} {\textsc{- Madame entre les soirs dans le cabinet du
roi après le souper.}} {\textsc{- M. le duc d'Orléans, seul de tous les
princes, donne en cérémonie l'eau bénite au Dauphin.}} {\textsc{- Convoi
des deux corps à Saint-Denis en cérémonie.}} {\textsc{- Retour du roi à
Versailles, où il voit en passant la foule des mantes et des manteaux,
qui vont après chez tout le sang royal sans ordre et pour la première
fois.}} {\textsc{- Privance de la duchesse du Lude.}} {\textsc{- Le roi
voit à la fois tous les ministres étrangers en manteaux\,; reçoit les
harangues des autres.}} {\textsc{- Extrémité des deux jeunes fils de
France, qui sont nommés sans cérémonie.}} {\textsc{- Mort du petit
Dauphin.}} {\textsc{- Le roi d'aujourd'hui comment sauvé.}} {\textsc{-
Le corps et le cœur du petit Dauphin portés sans cérémonie près de ceux
de M. {[}le Dauphin{]} et de M\textsuperscript{me} la Dauphine.}}
{\textsc{- M. le duc d'Anjou, aujourd'hui roi, succède au titre et au
rang de Dauphin.}} {\textsc{- Douleur de M. le duc de Berry, et en
Espagne.}} {\textsc{- Singularité des obsèques jusqu'à Saint-Denis.}}
{\textsc{- Deuil aussi singulier que ces obsèques.}} {\textsc{- État du
duc de Beauvilliers et le mien.}} {\textsc{- Cassette du Dauphin qui me
met en grand péril, dont l'adresse du duc de Beauvilliers me sauve.}}

~

La consternation fut vraie et générale. Elle pénétra les terres et les
cours étrangères. Tandis que les peuples pleuraient celui qui ne pensait
qu'à leur soulagement, et toute la France un prince qui ne voulait
régner que pour la rendre heureuse et florissante, les souverains de
l'Europe pleurèrent publiquement celui qu'ils regardaient déjà comme
leur exemple, et que ses vertus allaient rendre leur arbitre, et le
modérateur paisible et révéré des nations. Le pape en fut si touché
qu'il résolut de lui-même, et sans aucune sorte d'office, de passer
par-dessus toutes les règles et les formalités de sa cour, et il en fut
unanimement applaudi. Il tint exprès un consistoire, il y déplora la
perte infinie que faisait l'Église et toute la chrétienté\,; il fit un
éloge complet du prince qui causait leurs justes regrets et ceux de
toute l'Europe. Il y déclara enfin que, passant, en faveur de ses
extraordinaires vertus et de la douleur publique\,; par-dessus toute
coutume, il en ferait lui-même dans sa chapelle les obsèques publiques
et solennelles. Il en indiqua tout de suite le jour\,; le sacré collége
et toute la cour romaine y assista, et tous applaudirent à un honneur si
insolite. Il avait toujours été rendu réciproquement aux papes en France
et à nos rois à Rome, mais non à leurs enfants, jusqu'à la mort d'Henri
III.

Sixte V, qui avait ouvert les yeux au célèbre duc de Nevers qui l'était
allé consulter sur la Ligue, et qui lui-même ne l'avait favorisée que le
moins qu'il avait pu, qui loua publiquement Henri III de s'être défait
du duc de Guise, devint furieux deux jours après, lorsqu'il apprit que
le cardinal de Guise avait eu le même sort. Il excommunia Henri III, et
quoi que ce prince pût faire dans le peu de temps que les Guise le
laissèrent vivre depuis, il demeura excommunié même après sa mort,
quoique, dans le court espace qu'il vécut après avoir été frappé, il eût
fait tout ce qui lui fut possible pour mourir en bon chrétien, qu'il eût
été réconcilié à l'Église, et qu'il eût reçu tous les sacrements. Tout
ce que la reine sa veuve fit de démarches à Rome par le célèbre d'Ossat
depuis cardinal, toute l'adresse, l'éloquence, la force des raisons et
des offices qu'il y employa, toute la considération personnelle que ce
grand'homme s'y était acquise, furent inutiles pour obtenir les obsèques
accoutumées pour nos rois. En revanche, on cessa en France de les faire
pour les papes, et réciproquement il n'y en a pas eu depuis. C'est ce
qui ajouta beaucoup à celles que Clément XI, et de lui-même, voulut
faire pour ce sublime Dauphin, et auxquelles tout Rome applaudit contre
ses plus opiniâtres maximes, qui la rendent si politiquement invariable
pour tout ce qui est du cérémonial. De douloureuses choses me ramènent
sur mes pas. La Dauphine mourut comme je l'ai dit, à Versailles, le
vendredi 12 février, entre huit et neuf heures du soir. J'étais retiré
dans ma chambre, pénétré de cette perte\,; l'archevêque de Reims, qui
entrait chez moi à toute heure, y arriva et me trouva seul. Il était
affligé, comme il n'était personne qui pût s'en défendre, il l'était de
plus de la perte de la charge de dame d'atours qu'avait la comtesse de
Mailly, sa belle-sœur, avec laquelle il était intimement de tout temps.
Il savait par elle l'aventure de la tabatière. Le roi ne faisait presque
que de partir, et il s'était trouvé dans la chambre de la pauvre
princesse, tout pendant que le roi y avait demeuré, et il y était
longtemps auparavant. Il me conta d'entrée que le duc de Noailles, qui
était en quartier de capitaine des gardes, y était venu avant le roi,
qu'il lui avait vu un air embarrassé, le regard curieux, une décision
fort nette et trop sereine que cela ne pouvait aller loin, un examen
attentif et quelque chose de fort composé dans toute sa personne\,;
qu'il était demeuré assez longtemps, et s'en était allé pour y revenir
fort peu après avec le roi, où, à travers son embarras qui subsistait,
le contentement perçait\,; enfin il m'en parla comme lui en attribuant
tout le malheur, et me le dit nettement.

Il faut remarquer que tous ces Mailly ne pouvaient souffrir les
Noailles\,; la jalousie les rongeait de la préférence qu'ils avaient sur
eux chez M\textsuperscript{me} de Maintenon, et leur manie était de
trouver fort mauvais que la comtesse de Mailly, fille de son cousin
germain, n'en eût pas été traitée en parfaite égalité de fortune, comme
la fille unique de son propre frère. À cette émulation qui formait leur
haine, l'archevêque en joignait une particulière. Avant son épiscopat,
il avait été député du second ordre à une assemblée du clergé. Il
voulait parvenir, et il s'était livré aux jésuites. Il arriva une
affaire où il s'opposa fièrement au cardinal de Noailles, qui présidait
à l'assemblée, et qui était alors dans sa grande faveur. Surpris de se
voir résister en face par un abbé, il voulut s'expliquer, et lui faire
honnêtement entendre raison. L'abbé n'en poussa que plus vertement sa
pointe, et même avec peu de mesure. Alors le cardinal piqué le malmena
de façon que l'autre ne le lui pardonna jamais. Lui-même autrefois
m'avait conté la querelle, et souvent depuis témoigné qu'il ne
l'oublierait jamais. Je l'en fis souvenir alors pour le rendre suspect à
lui-même\,; mais, voyant qu'il s'animait de plus en plus à me vouloir
persuader, je lui dis que personne ne le pouvait jamais être que le duc
de Noailles pût être capable d'une horreur aussi abominable\,; aussi peu
qu'il eût aucun intérêt en la mort de la Dauphine, lui qui toute sa vie
en avait été si bien traité\,; qui avait trois sœurs, dames du palais,
ses favorites\,; qui avait tant d'intérêt en la vie de
M\textsuperscript{me} de Maintenon qui, à son âge, soutiendrait
difficilement cette perte\,; enfin, outre ces raisons démonstratives,
toutes celles dont je pus m'aviser. Je n'y gagnai rien\,; la cause du
rappel du duc de Noailles commençait à percer. Il me soutint qu'il
voulait gouverner le Dauphin sans partage, à qui il ne pouvait proposer
une maîtresse, comme si en {[}ce{]} genre d'affaires, et de confiance
les ducs de Beauvilliers et de Chevreuse n'eussent pas été des obstacles
plus fâcheux que la Dauphine. J'eus beau dire, l'archevêque demeura
ferme sur la tabatière, dont l'événement est en effet demeuré
inintelligible. Je l'exhortai du moins à condamner au plus profond
silence, et le plus sans réserve, une si horrible pensée\,; et en effet
il l'y contint. Mais il est mort plusieurs années depuis dans sa
persuasion, qui ne put me faire aucune impression. Ceux qui surent à la
fin l'histoire de la boîte, en assez grand nombre, ne furent pas plus
susceptibles que moi de ce soupçon, et personne ne s'avisa de jeter rien
sur le duc de Noailles. Pour moi, je le crus si peu que notre liaison
demeura la même. Quelque intime qu'elle ait été jusqu'à la mort du roi,
je ne sais comment il est arrivé que nous ne nous sommes jamais parlé de
cette fatale tabatière.

Dans le moment que le P. La Rue sortit de chez la Dauphine, instruit de
son intention, il fut au cabinet du roi, à qui il fit dire qu'il avait à
lui parler au moment même. Le roi le fit entrer. II vainquit son
embarras comme il put, et apprit au roi ce qui l'amenait. On ne peut
jamais être plus frappé que le roi le fut. Mille idées fâcheuses lui
entrèrent dans la tête. J'ignore si les scrupules y trouvèrent leur
place\,; ils devaient être grands. L'extrémité retint l'indignation,
mais laissa cours au dépit. La Rue se servit avantageusement de ce qu'il
n'y avait pas un moment à perdre pour abréger une si fâcheuse
conversation.

Le samedi 13, le corps de la Dauphine fut laissé dans son lit à visage
découvert, ouvert le même jour, à onze heures du soir, toute la Faculté
présente, la dame d'honneur et la dame d'atours\,; et le dimanche 14,
mis dans le cercueil sur une estrade de trois marches, porté le
lendemain, lundi 15, dans son grand cabinet de même, où il y avait des
autels où les matins on disait continuellement des messes. Quatre
évêques assis, en rochet et camail, à la ruelle droite, se relevaient
comme les dames, avertis par les agents du clergé. Ils prétendirent des
chaises à dos, le carreau et le goupillon. Ils furent refusés des deux
premiers, ils n'eurent que des siéges ployants et point de carreaux. Ils
mirent tant qu'ils attrapèrent le goupillon.

Pour entendre ce cérémonial que je n'ai pas eu lieu encore d'expliquer,
on ne doit avoir en présence du corps de ces princes que ce qu'on aurait
devant eux vivants. On y est assis à l'église sur des ployants, et cela
décide pour s'asseoir et pour l'espèce du siége\,; de carreaux, personne
n'en a devant eux à l'église que le sang royal, les bâtards, les ducs et
duchesses, et ceux et celles qui ont le rang de prince étranger ou le
tabouret de grâce. Aussi n'y a-t-il que ces personnes-là qui venant
jeter de l'eau bénite en cérémonie, ou chacun à part, sous manteau, les
hérauts, qui sont avec leurs cottes d'armes et leurs caducées au coin du
pied du cercueil, présentent un carreau qu'ils tiennent relevé auprès
d'eux pour faire leur courte prière, après avoir donné l'eau bénite, et
quand on se lève les hérauts ôtent le carreau. Le goupillon est présenté
par les hérauts aux mêmes personnes, à qui ils donnent le carreau, qui
le leur rendent après avoir donné l'eau bénite\,; ils présentent aussi
le goupillon aux officiers de la couronne et à leurs femmes, et pour les
charges uniquement aux premiers gentilshommes de la chambre du roi qui
ne seraient pas ducs, et à leurs femmes, à la dame d'honneur si elle
n'était pas duchesse, à la dame d'atours et au chevalier d'honneur et à
sa femme qui tous se mettent à genoux sans carreau pour faire leur
courte prière. Toutes autres personnes, hommes et femmes, quelles
qu'elles soient, même en mante et en manteau, prennent elles-mêmes le
goupillon dans le bénitier et l'y remettent après avoir jeté de l'eau
bénite, sans que les hérauts fassent le moindre mouvement. Ils sont
avertis de tous ceux et celles qui doivent avoir un carreau par la
proclamation de leurs noms que l'huissier fait de la porte à fort haute
voix, à mesure qu'il en voit entrer, et n'en annonce aucun autre. Au
sang royal, c'est l'aumônier de garde en rochet qui présente le
goupillon et le reprend. Six dames en mante {[}sont{]} assises vis-à-vis
des évêques, qui se relèvent toutes ensemble par six autres tout le
jour, averties chacune de sa garde et de son heure, de la part du roi,
par un billet du grand maître des cérémonies\,; de ces six dames, à
chaque garde deux duchesses ou princesses, alternativement, qui trouvent
deux carreaux devant leurs siéges aux deux premières places (les autres
dames n'en ont point)\,; deux dames du palais non duchesses qui
s'accordent entre elles\,; et deux dames aux deux autres places qui
soient de qualité à avoir mangé avec la princesse, c'est-à-dire avec la
reine, et à avoir entrée dans son carrosse. Les femmes des maréchaux de
France qui ne sont point ducs roulent avec celles-ci, et ont la première
des deux places. S'il y avait d'autres officiers de la couronne non
ducs, il en serait de même de leurs femmes.

Le roi nomma lui-même les deux titrées de la première garde. Il s'était
fait un point de politique d'entretenir les disputes entre les ducs et
les princes étrangers, c'est-à-dire lorrains\,; car, encore qu'il ait
donné le même rang à MM\hspace{0pt}. de Bouillon et de Rohan, il n'a
jamais souffert que ceux-là soient entrés en aucune compétence avec les
ducs, ni avec la maison de Lorraine. Il crut donc faire merveille de
prendre les deux plus anciennes duchesses qui se trouvassent à la cour,
et sous ce prétexte, la duchesse d'Elbœuf, veuve du second duc et pair
et de l'aîné de la maison de Lorraine en France, et la duchesse de
Sully, et de tenir ainsi sa balance égale, donnant aux ducs
M\textsuperscript{me} d'Elbœuf pour duchesse, et si bien pour telle
qu'il la doublait d'une autre duchesse\,; aux Lorrains, que l'aînée de
leur maison avait gardé la première, en conséquence\footnote{Cette
  phrase pourrait paraître obscure\,; Saint-Simon a voulu dire que le
  roi déclara aux Lorrains que l'aînée de leur maison avait gardé la
  première, en conséquence de son titre d'aînée.}. Pourtant elles furent
relevées par deux princesses, M\textsuperscript{me} de Lambesc et sa
tante M\textsuperscript{lle} d'Armagnac, qui ne le trouvèrent pas trop
bon, parce que cela marquait que les duchesses avaient eu la première
garde. Je continuerai les cérémonies de suite jusqu'au départ pour
Saint-Denis, tant pour n'y plus revenir que pour d'autres raisons qui se
verront dans la suite.

Le mercredi 17, Madame, accompagnée de M. le duc d'Orléans, de
M\textsuperscript{me} la princesse de Conti et de ses deux filles, et de
M. le comte de Toulouse, tous en mantes et en grands manteaux, ainsi que
leur suite, alla donner de l'eau bénite. Elle fut reçue par le chevalier
d'honneur à la tête de la maison de M\textsuperscript{me} la Dauphine,
au bout de la dernière pièce tendue de noir, et {[}qui{]} l'y conduisit.
La dame d'honneur ne traversa point dans la même pièce en la recevant et
la conduisant, et s'arrêta à la porte intérieure. Il n'y eut d'eau
bénite en cérémonie que du sang royal, contre tout usage jusqu'alors.

Le vendredi matin 19, le corps de Mgr le Dauphin fut ouvert, un peu plus
de vingt-quatre heures après sa mort, en présence de toute la Faculté,
de quelques menins et du duc d'Aumont, nommé comme duc par le roi. Son
cœur fut porté tout de suite à Versailles auprès de celui de
M\textsuperscript{me} la Dauphine. Ce même jour, entre cinq et six
{[}heures{]}, les deux cœurs furent portés au Val-de-Grâce à Paris.
Chamillart, évêque de Senlis, premier aumônier de M\textsuperscript{me}
la Dauphine, ayant un pouvoir du cardinal de Janson, grand aumônier,
était dans le premier carrosse à la droite au fond, portant les deux
cœurs\,; M\textsuperscript{me} la Princesse au fond à sa gauche\,;
M\textsuperscript{me} de Vendôme, sa fille, et M\textsuperscript{lle} de
Conti au devant\,; la duchesse du Lude à une portière, le duc du Maine à
l'autre. Le duc d'Aumont, comme premier gentilhomme de la chambre,
suivait à la première place du fond d'un carrosse de Mgr le Dauphin,
accompagné de quelques menins. Suivait le carrosse du corps de
M\textsuperscript{me} la Dauphine, rempli de ses dames du palais, dont
deux étaient restées à la garde du corps. Ce cortége arriva après minuit
au Val-de-Grâce, tout y fut fini avant deux heures\,; {[}il{]} revint
après sans cérémonie, et demeura à Paris qui voulut. Dès que ce convoi
fut parti de Versailles, le corps de Mgr le Dauphin, porté de Marly sans
cérémonie, fut placé à la droite de celui de M\textsuperscript{me} la
Dauphine sur la même estrade, qui fut élargie. Le samedi 20, le roi
manda à la duchesse de Ventadour qu'il voulait que désormais Mgr le duc
de Bretagne prît le nom et le rang de Dauphin\,; et ce même soir il fit
entrer Madame dans son cabinet, après son souper, avec les princes et
princesses qui avaient coutume d'y entrer, jusqu'au coucher du roi, et
elle y est depuis entrée tous les soirs. Le lundi 22 février, M. le duc
d'Orléans alla donner l'eau bénite au corps de Mgr le Dauphin. II y fut
reçu et conduit, comme l'avait été Madame, par le duc d'Aumont, comme
premier gentilhomme de la chambre, à la tête des menins, qui tour à tour
gardaient le corps de Mgr le Dauphin.

Le mardi 23 février, les deux corps furent portés de Versailles à
Saint-Denis sur un même chariot. Le roi nomma M. le duc d'Orléans pour
accompagner le corps de Mgr le Dauphin, et quatre princesses pour celui
de M\textsuperscript{me} la Dauphine, qui furent M\textsuperscript{me}
la Duchesse, M\textsuperscript{me} de Vendôme, et
M\textsuperscript{lle}s de Conti et de La Roche-sur-Yon. À la descente
des corps, le duc d'Aumont, comme premier gentilhomme de la chambre,
portait la couronne de Mgr le Dauphin\,; Dangeau, chevalier d'honneur,
celle de M\textsuperscript{me} la Dauphine\,; Souvré, maître de la
garde-robe du roi, le collier de l'ordre du Saint-Esprit. Dans la
marche, qui commença sur les six heures du soir, des aumôniers en rochet
et à cheval soutenaient les coins des poêles\,; deux du roi, deux de
M\textsuperscript{me} la Dauphine\,; de son côté étaient à cheval le
chevalier d'honneur et le premier écuyer\,; trois carrosses précédaient.
Dans le second était au fond M. le duc d'Orléans avec le duc d'Aumont\,;
d'Antin sur le devant avec Souvré, comme maître de la garde-robe\,;
Matignon à une portière, comme menin\,; le capitaine des gardes de M. le
duc d'Orléans à l'autre\,; dans le troisième et le plus proche du
chariot, quatre évêques en rochet et camail, un aumônier du roi en
quartier en rochet, et le curé de Versailles en étole. Trois carrosses
derrière\,: les quatre princesses dans le premier, avec la duchesse du
Lude, qui était un carrosse du roi\,; un de M\textsuperscript{me} la
Dauphine, rempli de ses dames\,; et celui de M\textsuperscript{me} la
Duchesse après, où étaient les dames d'honneur des princesses. Le convoi
commença à entrer à Paris par la porte Saint-Honoré à deux heures après
minuit, sortit de la porte Saint-Denis à quatre heures du matin, et
arriva entre sept et huit heures du matin à Saint-Denis. Il y eut un
grand ordre dans Paris, et aucun embarras. Le samedi 27 février, le roi
revint de Marly à Versailles. Il avait mangé, tout ce voyage, seul dans
sa chambre, matin et soir, à son très-petit couvert. Il ne voulut point
de resperts en forme de sa cour, comme il s'était pratiqué à la mort de
Monseigneur. Il fit dire qu'il verrait tout le monde à la fois tout en
arrivant. Les princes et princesses du sang et bâtards l'attendirent
dans ses cabinets\,; la duchesse du Lude et les dames de
M\textsuperscript{me} la Dauphine, le chevalier d'honneur et les autres
grands officiers à la porte de son cabinet, ensemble\,; les dames dans
sa chambre, les hommes dans son antichambre et dans les pièces
suivantes, jusqu'à la porte de l'appartement de M\textsuperscript{me} de
Maintenon. Tout était en mantes et en manteaux longs. Le roi arriva à
quatre heures, et monta droit dans ses cabinets par son petit degré,
puis traversa lentement jusque chez M\textsuperscript{me} de Maintenon
pour remarquer tout le monde. Il embrassa uniquement la duchesse du
Lude, et lui dit qu'il n'était pas en état de lui parler, mais qu'il la
verrait. Une demi-heure après, M\textsuperscript{me} de Maintenon lui
manda de venir chez elle avec les dames de M\textsuperscript{me} la
Dauphine. Elles y virent le roi sans mante. Il parla obligeamment à
toutes, et retint après la duchesse du Lude, qu'il fit asseoir, et qui
fut longtemps en tiers avec lui et M\textsuperscript{me} de Maintenon.
Il l'a vue beaucoup de fois depuis de la sorte, et comme plus du tout en
public qu'à Marly, quand sa santé lui permettait d'y aller ou d'être des
voyages. Tout ce qui était là en mantes et en manteaux alla comme en
procession chez tous les princes et princesses, commençant par M. {[}le
duc{]} et M\textsuperscript{me} la duchesse de Berry, et finissant par
le comte de Toulouse. Personne n'avait été chez les princes et
princesses du sang à la mort de Monseigneur. On a vu par quel manége M.
du Maine obtint qu'on allât chez les bâtards. En cette occasion, on fut
sans ordre, et comme moutons, chez les princes et princesses du sang. Il
n'y eut que ce seul jour pour les manteaux et les mantes.

Le mardi, 1\^{}er mars, le roi vit dans son cabinet tous les ministres
étrangers avant sa messe, qui étaient tous en manteau long. Le samedi 5
mars, il reçut les harangues du parlement, de la chambre des comptes, de
la cour des aides et de celle des monnaies, la parole portée par chaque
premier président\,; celui de la cour des aides était malade\,;
Graville, second président, parla. Après chaque cour, les gens du roi de
celle qui venait de haranguer s'avancèrent et parlèrent par le premier
avocat général, usage que M. Talon, mort président à mortier, établit du
temps qu'il était avocat général du parlement. La ville harangua la
dernière, et le discours du prévôt des marchands l'emporta sur tous.
C'était le matin après la messe.

Le lendemain dimanche, à pareille heure, le grand conseil vint
haranguer, parce qu'il ne veut point céder au parlement, ni le parlement
encore moins à lui\,; et tout de suite l'Académie française.

Ce même jour, les deux enfants, fils de France, malades depuis quelques
jours, furent très-mal, avec les marques de rougeole qui avaient paru en
M. {[}le Dauphin{]} et M\textsuperscript{me} la Dauphine. Ils avaient
été ondoyés en naissant. Le roi manda à la duchesse de Ventadour de leur
faire suppléer les cérémonies du baptême, de les faire tenir par qui
elle voudrait, et de les faire nommer Louis l'un et l'autre. Elle prit
ce qui se trouva de plus distingué sous sa main. Elle tint le petit
Dauphin avec le comte de La Mothe\,; et le marquis de Prie avec la
duchesse de La Ferté, M. le duc d'Anjou, aujourd'hui roi. Le lendemain
mardi, 8 mars, les médecins de la cour en appelèrent cinq de Paris. Le
roi ne laissa pas de tenir conseil de finances, d'aller tirer après son
dîner, et de travailler le soir avec Voysin chez M\textsuperscript{me}
de Maintenon. Les saignées et les autres remèdes qu'on employa ne purent
sauver le petit Dauphin. Il mourut ce même jour, un peu avant minuit. Il
avait cinq ans et quelques mois, et était bien fait, fort et grand pour
son âge. Il donnait de grandes espérances par l'esprit et la justesse
qu'il montrait en tout\,; il inquiétait aussi par une décision opiniâtre
et par une hauteur extrême.

M. le duc d'Anjou tétait encore. La duchesse de Ventadour, aidée des
femmes de la chambre, s'en empara, ne le laissèrent point saigner ni
prendre aucun remède. La comtesse de Verue, empoisonnée à Turin, et
prête à mourir, avait été sauvée par un contre-poison qu'avait le duc de
Savoie. Elle en avait apporté en revenant. La duchesse de Ventadour lui
en envoya demander, et en donna à M. le duc d'Anjou seulement, parce
qu'il n'avait pas été saigné, et que ce remède ne peut aller avec la
saignée. Il fut bien mal, mais il en réchappa et est roi aujourd'hui. Il
l'a su depuis et a toujours marqué une vraie distinction à
M\textsuperscript{me} de Verue, et pour tout ce qui l'a regardée. Trois
Dauphins moururent donc en moins d'un an, dont un seul enfant, et, en
vingt-quatre jours, le père, la mère et le fils aîné. Le mercredi 9
mars, le corps du petit Dauphin fut ouvert. Dans la nuit, et sans aucune
cérémonie, son cœur fut porté au Val-de-Grâce à Paris, et son corps à
Saint-Denis, et placé sur la même estrade avec ceux de M. {[}le
Dauphin{]} et de M\textsuperscript{me} la Dauphine, ses père et mère. M.
le duc d'Anjou, désormais unique, succéda au titre et au rang de
Dauphin.

J'ai omis ce qui se passa au réveil du roi à la mort de Mgr le Dauphin,
parce que ce ne fut que la répétition parfaite de ce qui s'y passa à la
mort de M\textsuperscript{me} la Dauphine, qui a été raconté. Le roi
embrassa tendrement M. le duc de Berry à plusieurs reprises, lui
disant\,: «\,Je n'ai donc plus que vous.\,» Ce prince était fondu en
larmes\,; on ne peut être plus amèrement ni plus longtemps affligé qu'il
le fut. M\textsuperscript{me} la duchesse de Berry n'osa s'échapper.
Elle tint assez honnête contenance. Au fond sa joie était extrême de se
voir elle et son époux les premiers. L'affliction et l'horreur de ces
coups redoublés furent inconcevables en Espagne.

À la mort de la reine, de la dauphine de Bavière, de Monsieur, en un mot
à toutes ces grandes obsèques, excepté à la mort de Monseigneur, à cause
de la petite vérole qui l'avait emporté, tous les fils de France suivis
de tous les princes du sang et de tous les ducs avaient été en
cérémonie, tous ensemble, donner l'eau bénite\,; et pareillement
ensemble les filles et petites-filles de France, suivies des princesses
du sang et des duchesses. Les cœurs et les corps avaient été accompagnés
de princes du sang et de ducs, et pour les princesses de beaucoup de
princesses, de duchesses et de princesses étrangères, et de dames de
qualité en plusieurs carrosses\,; et les corps avaient été gardés
longtemps avant d'être portés à Saint-Denis. En celles-ci, quoique
doubles, et par conséquent plus nombreuses et plus solennelles,
puisqu'on devait faire autant pour chaque corps que s'il n'y en avait eu
qu'un, et que cela doublait tous les accompagnements, on ne fit qu'une
légère image de ce qui s'était toujours pratiqué pour un seul, tant pour
la durée de la garde avant le transport, que pour l'eau bénite des deux
corps à part, et pour les convois des deux cœurs ensemble, et après des
deux corps ensemble. Le genre de ces étranges morts en fut en gros la
vraie cause, et la hâte de débarrasser le roi à Versailles, et qu'il eut
lui-même de n'avoir plus à ouïr parler de choses si douloureuses, et de
n'entretenir pas l'excitation des propos, fit abréger tout et diminuer
tout, et pour les cérémonies et pour le nombre des personnes qui y
devaient assister. Il n'y parut ni fils de France ni prince du sang,
mais le roi ne laissa pas d'avoir soin, malgré toute sa douleur et ses
poignantes inquiétudes, d'y en faire jouer le personnage à ses deux fils
naturels\,: l'un au convoi des corps, l'autre à l'eau bénite de la
Dauphine, à la suite de Madame et de M. le duc d'Orléans et de trois
princesses du sang seulement.

C'est la première fois que les hommes et les femmes aient été ensemble
donner l'eau bénite en cérémonie. M. le duc d'Orléans unique en retourna
donner en cérémonie au Dauphin\,; l'autre avait été pour la Dauphine
seule avant que le corps du Dauphin fût mis auprès du sien. C'était
séparément à M. le duc et à M\textsuperscript{me} la duchesse de Berry à
conduire les eaux bénites\,; ils devaient être séparément suivis de
Madame et de M. le duc d'Orléans, de M\textsuperscript{me} la duchesse
d'Orléans, de tout le sang royal, des ducs et duchesses, et depuis un
temps de la maison de Lorraine. Jusqu'alors cela s'était passé ainsi à
la reine, à la dauphine de Bavière, à Monsieur\,; je ne doute pas aussi
à sa première épouse. Il est vrai qu'à Monsieur, sous prétexte de cette
compétence des ducs avec la maison de Lorraine que le roi aimait tant,
il ne voulut pas qu'aucun d'eux y allât en cérémonie\,; mais leurs
femmes y furent avec les princesses du sang, à la suite de
M\textsuperscript{me} la duchesse de Bourgogne, où il se passa ce que
j'ai raconté alors. Le cortége des deux cœurs fut mêlé, et tout aussi
court et singulier\,: trois princesses du sang pour l'un, ce devait être
une fille de France avec elles, et des duchesses avec pour l'autre, au
lieu d'un fils de France, de deux princes du sang et de quelques ducs,
M. du Maine unique\,; au convoi des corps, M. le duc d'Orléans seul de
tout le sang royal, avec un mélange de charges pour tout accompagnement
dans le carrosse où il était, et deux ducs, dont l'un était encore
premier gentilhomme de la chambre et en avait servi en ces cérémonies,
l'autre pouvait être regardé comme menin. Pour la Dauphine, quatre
princesses du sang, sans fille ni petite-fille de France, et sans
duchesses ni Lorraines ni dames de qualité, et un seul carrosse après le
leur, pour les dames du palais. Rien ne fut jamais si court, ni si
baroque, jusque-là que la maison même de la Dauphine ni les menins ne
donnèrent point d'eau bénite en cérémonie, c'est-à-dire un premier
gentilhomme de la chambre à la tête des menins, la dame d'honneur à la
tête, des dames de M\textsuperscript{me} la Dauphine, et le chevalier
d'honneur à la tête des officiers premiers et principaux de la maison. À
l'égard de Monseigneur, pour lequel il ne s'observa pas la moindre
cérémonie, la petite vérole dont il mourut en fut la juste raison.

Pour comble de singularité, le roi qui avait voulu, à la mort de
Monseigneur, que les personnes qui drapent lorsqu'il drape, drapassent
quoiqu'il ne portât point ce deuil, ne voulut point que personne drapât
pour M. {[}le Dauphin{]} et M\textsuperscript{me} la Dauphine, excepté
M. le duc et M\textsuperscript{me} la duchesse de Berry. Comme leur
maison drapait à cause d'eux, cela fit une question sur
M\textsuperscript{me} de Saint-Simon, qui prétendait ne point draper, et
eux désiraient qu'elle drapât, et s'appuyaient sur l'exemple des
duchesses de Ventadour et de Brancas, chez Madame. On y répondait que
celles-là étant séparées de corps et de biens d'avec leurs maris,
avaient leurs équipages à elles, au lieu que M\textsuperscript{me} de
Saint-Simon et moi vivions et avions toujours vécu ensemble, qui est le
cas que les équipages de la femme appartiennent au mari. Là-dessus,
grande négociation. Ils prenaient cette draperie à l'honneur. M. {[}le
duc{]} et M\textsuperscript{me} la duchesse de Berry nous la demandèrent
avec tant d'instance, par amitié, comme une chose qui les touchait
sensiblement, qu'il fallut enfin avoir cette complaisance. Tellement que
notre maison fut mi-partie\,: tout ce qui était à moi ou en commun sans
deuil, et en noir tout ce qui était à M\textsuperscript{me} de
Saint-Simon, ce qui était fort ridicule.

M. de Beauvilliers était malade dans son lit à Versailles, et il était à
sa maison de la ville pour être plus en repos au bas de la rue de
l'Orangerie. Il serait difficile de comprendre l'excès de sa douleur, ni
la grandeur de sa piété, de sa résignation, de son courage. Je n'ai rien
vu de si difficile à décrire, de plus impossible à atteindre, de
comparable à admirer. Le jour de la mort de notre Dauphin, je ne sortis
qu'un instant de chez moi, où je m'étais barricadé pour joindre le roi à
sa promenade dans les jardins, qui passa l'après-dînée à portée de mon
pavillon. La curiosité y eut part. Le dépit de le voir presqu'à son
ordinaire ne put soutenir cette promenade qu'un instant. On emportait
alors le corps du Dauphin, j'en aperçus de loin quelque chose. Je me
rejetai chez moi, d'où je ne sortis presque plus du reste du voyage, que
pour aller passer les après-dînées auprès du duc de Beauvilliers,
enfermé chez lui où il ne laissait entrer presque personne. J'avoue que
je faisais le détour entre le canal et les jardins de Versailles, pour
arriver à l'hôtel de Beauvilliers par la porte de l'Orangerie qu'il
joignait, pour me dérober à la vue de ce qui paraissait de funèbre, dont
aucun devoir ne me put faire approcher. Je conviens de la faiblesse. Je
n'étais soutenu ni de la piété supérieure à tout du duc de Beauvilliers,
ni d'une semblable à celle de M\textsuperscript{me} de Saint-Simon, qui
toutefois n'en souffraient pas moins. La vérité est que j'étais au
désespoir. À qui saura où j'en étais arrivé, cet état paraîtra moins
étrange que d'avoir pu supporter un malheur si complet. Je l'essuyais
précisément au même âge où était mon père quand il perdit Louis XIII\,;
au moins en avait-il grandement joui, et moi, \emph{Gustavi paululum
mellis, et ecce morior\,!} Ce n'était pas tout encore.

Il y avait dans la cassette du Dauphin des mémoires qu'il m'avait
demandés. Je les avais faits en toute confiance, lui les avait gardés de
même. J'y étais donc parfaitement reconnaissable. Il y en avait même un
fort long de ma main, qui seul eût suffi pour me perdre sans espérance
de retour auprès du roi. On n'imagine point de pareilles catastrophes.
Le roi connaissoit mon écriture\,; il ne connaissoit pas de même ma
façon de penser, mais il s'en doutait à peu près. J'y avais donné lieu
quelquefois, et de bons amis de cour y avaient suppléé de leur mieux. Ce
péril ne laissait pas de regarder assez directement le duc de
Beauvilliers, un peu plus au lointain le duc de Chevreuse. Le roi qui
par ces mémoires m'aurait aussitôt reconnu, y aurait en même temps
découvert la plus libre et la plus entière confiance entre le Dauphin et
moi, et sur des chapitres les plus importants, et qui lui auraient été
les moins agréables, et il ne se doutait seulement pas que j'approchasse
de son petit-fils plus que tous les autres courtisans. Il n'eût pas pu
croire, intimement lié comme il me savait de tout temps avec le duc de
Beauvilliers, que ce commerce intime et si secret d'affaires se fût
établi sans lui entre le Dauphin et moi\,; et toutefois il fallait que
lui-même portât au roi la cassette de ce prince, à la mort duquel du
Chesne en avait sur-le-champ remis la clef au roi. L'angoisse était donc
cruelle, et il y avait tout à parier que j'en serais perdu et chassé
pour tout le règne du roi. Quel contraste des cieux ouverts que je
voyais sans chimère, et de ces abîmes qui tout à coup s'ouvraient sous
mes pieds\,! Et voilà la cour et le monde\,! J'éprouvai alors le néant
des plus désirables fortunes par un sentiment intime qui toutefois
marque combien on y tient. La frayeur de l'ouverture de cette cassette
n'eut presque point de prise sur moi. Il me fallut des réflexions pour y
revenir de temps en temps. Les regrets de ce qui m'échappait, plus sans
comparaison qu'eux la vue de ce que perdait la France, surtout la
disparition de cet incomparable Dauphin, me perçait le cœur et
suspendait toutes les facultés de mon âme. Je ne voulus longtemps que
m'enfuir et ne revoir jamais la figure trompeuse de ce monde. Même après
que je me fus résolu à y demeurer, la situation naturelle où j'étais
avec M. le duc de Berry et M. le duc d'Orléans, que tant d'autres des
plus grands eussent si chèrement achetée dans la perspective de l'âge du
roi et de celui du petit Dauphin, m'était insipide, je n'oserais dire
pire, par la comparaison de ce qui n'était plus\,; et ma douleur si peu
capable de consolation et de raison qu'elle trahit entièrement tout ce
que j'avais caché jusque-là avec tant de soin et de politique, et
manifesta malgré moi tout ce que j'avais perdu. M\textsuperscript{me} de
Saint-Simon, non moins sensible, non moins touchée, aussi peu capable de
le disssimuler, mais plus sensée, plus forte, et toute à Dieu, recevait
aussi par plus de liberté d'esprit, par plus de mesure en attaches, par
la plus sage prudence, de plus fortes impressions de l'inquiétude de ces
papiers. Les ducs et duchesses de Beauvilliers et de Chevreuse étaient
uniques dans ce secret, et les uniques aussi avec qui en consulter. M.
de Beauvilliers prit le parti de ne confier la cassette à personne,
quoique le roi en eût la clef, et d'attendre que sa santé lui permît de
la porter lui-même, pour essayer, étant avec lui, de dérober ces papiers
à sa vue parmi tous les autres de quelque manière que ce fût. Cette
mécanique était difficile, car il ne savait pas même la position de ces
papiers si dangereux parmi les autres dans la cassette, et cependant
c'était la seule ressource. Une si terrible incertitude dura plus de
quinze jours.

Le lundi, dernier février, le roi vit dans son cabinet sur les cinq
heures le duc de Beauvilliers pour la première fois, qui n'avait pas
{[}été{]} en état de s'y rendre plus tôt. Mon logement était assez près
du sien et de plain-pied, donnant au milieu de la galerie de l'aile
neuve, de plain-pied aussi au grand appartement du roi. Le duc à son
retour entra chez moi, et nous dit, à M\textsuperscript{me} de
Saint-Simon et à moi, que le roi lui avait ordonné de lui porter le
lendemain au soir chez M\textsuperscript{me} de Maintenon la cassette du
Dauphin, et nous répéta que, sans oser ni pouvoir répondre de rien, il
serait bien attentif à éviter, s'il était possible, que le roi vît ce
qui y était de moi\,; et nous promit de revenir le lendemain au retour
de chez M\textsuperscript{me} de Maintenon nous en apprendre des
nouvelles. On peut juger s'il fut attendu, et à portes bien fermées. Il
arriva, et avant de s'asseoir nous fit signe de n'avoir plus
d'inquiétude. Il nous conta que tout le dessus de la cassette, et assez
épaissement, s'était heureusement trouvé rempli d'un fatras de toutes
sortes de mémoires et de projets sur les finances, et de quelques autres
d'intérieurs de province, qu'il en avait lu exprès une quantité au roi
pour le lasser, et qu'il y avait réussi tellement qu'à la fin le roi
s'était contenté d'en entendre les titres, et que fatigué de ne trouver
autre chose, s'était persuadé que le fond n'était pas plus curieux,
avait dit que ce n'était pas la peine d'en voir davantage, et qu'il
n'avait qu'à jeter là tous ces papiers dans le feu. Le duc nous assura
qu'il ne se l'était pas fait dire deux fois, d'autant qu'il avait déjà
avisé au fond un petit bout de mon écriture, qu'il avait promptement
couvert en prenant d'autres papiers pour en lire les titres au roi, et
qu'aussitôt qu'il lui eut lâché la parole, il rejeta confusément dans la
cassette ce qu'il en avait tiré de papiers et mis à mesure sur la table,
et avait été secouer la cassette derrière le feu entre le roi et
M\textsuperscript{me} de Maintenon, pris bien garde en la secouant que
ce mémoire de ma main qui était grand et épais fût couvert d'autres, et
qu'il avait eu grand soin d'empêcher avec les pincettes qu'aucun bout ne
s'écartât, et de voir tout bien brûlé avant de quitter la cheminée. Nous
nous embrassâmes dans le soulagement réciproque, qui fut proportionné
pour ce moment au péril que nous avions couru.

\hypertarget{chapitre-vi.}{%
\chapter{CHAPITRE VI.}\label{chapitre-vi.}}

1712

~

{\textsc{Dauphine empoisonnée.}} {\textsc{- Le maréchal de Villeroy,
raccommodé avec le roi, devient tout d'un coup favori.}} {\textsc{- Le
Dauphin empoisonné.}} {\textsc{- Le duc du Maine et
M\textsuperscript{me} de Maintenon persuadent le roi et le monde que M.
le duc d'Orléans a fait empoisonner le Dauphin et la Dauphine.}}
{\textsc{- Crayon de M. le duc d'Orléans.}} {\textsc{- Éclats populaires
contre M. le duc d'Orléans.}} {\textsc{- Cri général contre M. le duc
d'Orléans.}} {\textsc{- Conduite de la cour à son égard.}} {\textsc{-
Maréchal de Villeroy et autres principaux.}} {\textsc{- Embarras du duc
de Noailles, qui se dit en apoplexie et s'en va à Vichy.}}

~

Les horreurs qui ne se peuvent plus différer d'être racontées glacent ma
main. Je les supprimerais si la vérité si entièrement due à ce qu'on
écrit, si d'autres horreurs qui ont augmenté celles des premières s'il
est possible, si la publicité qui en a retenti dans toute l'Europe, si
les suites les plus importantes auxquelles elles ont donné lieu, ne me
forçaient de les exposer ici comme faisant une partie intégrante et des
plus considérables de ce qui s'est passé sous mes yeux. La maladie de la
Dauphine, subite, singulière, peu connue aux médecins, et très-rapide,
avait dans sa courte durée noirci les imaginations déjà fort ébranlées
par l'avis venu à Boudin si peu auparavant, et confirmé par celui du roi
d'Espagne. La colère du roi du changement de confesseur, qui se serait
durement fait sentir à la princesse si elle eût vécu, céda à la douleur
de sa perte, peut-être mieux à celle de tout son amusement et de tout
son plaisir\,; et la douleur voulut être éclaircie de la cause d'un si
grand malheur pour tâcher de se mettre en état d'en éviter d'autres, ou
de rentrer en repos sur l'inquiétude qui le frappait. La Faculté reçut
donc de sa bouche les ordres les plus précis là-dessus.

Le rapport de l'ouverture du corps n'eut rien de consolant\,: nulle
cause naturelle de mort, mais d'autres vers les parties intérieures de
la tête, voisines de cet endroit fatal où elle avait tant souffert.
Fagon et Boudin ne doutèrent pas du poison, et le dirent nettement au
roi, en présence de M\textsuperscript{me} de Maintenon seule. Boulduc
qui m'assura en être convaincu, et le peu des autres à qui le roi voulut
parler et qui avaient assisté à l'ouverture, le confirmèrent par leur
morne silence. Maréchal fut le seul qui soutint qu'il n'y avait de
marques de poison que si équivoques, qu'il avait ouvert plusieurs corps
où il s'en était trouvé de pareilles, et sur la mort desquels il n'y
avait jamais eu le plus léger soupçon. Il m'en parla de même, à moi à
qui il ne cachait rien, mais il ajouta que néanmoins, à ce qu'il avait
vu, il ne voudrait pas jurer du oui ou du non, mais que c'était
assassiner le roi et le faire mourir à petit feu que de nourrir en lui
une opinion en soi désolante, et qui pour les suites et pour sa propre
vie ne lui laisserait plus aucun repos. En effet, c'est ce qu'opéra ce
rapport, et pour assez longtemps. Le roi outré voulut chercher à savoir
d'où le coup infernal pouvait être parti, sans pouvoir s'apaiser par
tout ce que Maréchal lui put dire, et qui disputa vivement contre Fagon
et Boudin, lesquels maintinrent aussi vivement leurs avis en ce premier
rapport, et n'en démordirent point dans la suite. Boudin, outré d'avoir
perdu sa charge et une princesse pleine de bontés pour lui, même de
confiance, et ses espérances avec elle, répandit comme un forcené qu'on
ne pouvait pas douter qu'elle ne fût empoisonnée. Quelques autres, qui
avaient été à l'ouverture, le dirent à l'oreille à leurs amis\,; en
moins de vingt-quatre heures la cour et Paris en furent remplis.
L'indignation se joignit à la douleur de la perte d'une princesse
adorée, et à l'une et à l'autre la frayeur et la curiosité, qui furent
incontinent augmentées par la maladie du Dauphin.

Il faut interrompre un moment la suite de ces horreurs, pour parler d'un
événement qui devint après considérable. Le maréchal de Villeroy
languissait à Paris, et souvent à Villeroy, dans la plus profonde
disgrâce depuis son dernier retour de Flandre, dont on a vu le détail en
son lieu. Il ne paraissait que de loin à loin à Versailles, toujours
sans y coucher, à Fontainebleau une fois ou deux au plus, où rarement il
couchait une nuit. Il n'était plus question pour lui de Marly. La
sécheresse, le silence du roi, l'air d'être peiné de le voir, était le
même, mais il tenait toujours à M\textsuperscript{me} de Maintenon. Sa
haine pour Chamillart, qui leur était commune, avait réchauffé entre eux
l'ancienne familiarité. La compassion l'engageait à le voir dans sa
maison de la ville toutes les fois qu'il allait à Versailles ou à
Fontainebleau. Ils s'écrivaient souvent\,; et le goût qui effaçait tout
en elle, joint au malaise extrême des affaires, l'engageait même à le
consulter et à en recevoir des mémoires. Ces mystères étaient pour le
gros du monde, mais ils n'échappaient pas aux plus attentifs de la cour.
J'en étais instruit depuis longtemps\,; le roi ne les ignorait pas.
M\textsuperscript{me} de Maintenon n'aurait osé lui cacher une conduite
d'habitude qu'il aurait pu découvrir. Elle espéra trouver par là des
occasions de rapprocher le maréchal, et en effet elle lui montra
quelquefois de ses mémoires qu'elle faisait appuyer par Voysin.
Jusqu'alors néanmoins rien n'avait réussi. La triste conjoncture pressa
M\textsuperscript{me} de Maintenon pour elle-même.

Ces premiers moments du vide extrême que laissait {[}la perte{]} de la
Dauphine, la douleur, les affres dont elle était aiguisée, rendaient le
roi pesant à la sienne. Il était difficile à amuser\,; elle était
elle-même si touchée, si abattue, qu'elle ne trouvait point de ressource
en elle-même. Celle du travail des ministres chez elle y laissait de
grands intervalles par la longueur des soirées de cette saison, et des
journées entières quand il faisait trop mauvais pour sortir, et que le
roi alors passait toujours avant trois heures chez elle, et n'en sortait
qu'à dix pour son souper. D'admettre quelqu'un dans ce particulier avec
eux, n'eût pas été chose aisée avec le roi, ni facile à elle à choisir.
À quelque point qu'elle se vît avec lui, tout lui paraissait dangereux.
Elle songeait bien à multiplier les repas particuliers à Marly et à
Trianon, encore plus que chez elle, pour la commodité de la promenade,
et montrer plus d'objets par le service indispensable, et à y avoir
souvent des musiques\,; mais dans ce service indispensable, elle ne
trouvait rien dans les premiers gentilshommes de la chambre ni dans les
autres grands officiers qui pouvaient suivre, mais qui ne suivaient
guère là, de quoi amuser le roi. Le duc de Noailles indispensable, parce
qu'il était capitaine des gardes en quartier, n'était plus en cette
situation avec le roi ni avec elle depuis son rappel d'Espagne. Le
maréchal de Villeroy lui parut le seul sur qui elle pût jeter les
yeux\,: il avait été élevé auprès du roi\,; il n'avait bougé de la cour
que pour aller aux armées\,; il avait été galant de profession, et le
voulait être encore\,; personne plus que lui du grand monde toute sa
vie\,; il l'avait presque toute passée dans la plus grande familiarité
du roi\,; ils avaient cent contes de leur jeunesse et de leur temps,
dont le roi s'amusait beaucoup\,; le maréchal en avait de toutes les
sortes, il savait ceux de la ville de tous les temps, il en savait des
femmes des frontières\,; il se passionnait de la musique, il parlait
chasses\,; toutes les anciennes intrigues de la cour et du monde lui
étaient présentes\,; c'était une quincaillerie à fournir abondamment.
Plus que tout, elle n'en avait rien à craindre\,; et s'il prenait du
crédit, c'était un homme toujours sûr dans sa main à faire de lui tout
ce qu'elle voudrait. Ces considérations la déterminèrent à faire tous
ses efforts pour le raccommoder. Le roi était demeuré en garde contre
Harcourt depuis ses tentatives pour entrer au conseil\,; d'ailleurs ni
familiarité ancienne, ni fatuité, ni vieux contes. Nul autre de ses
grands officiers ne pouvait être compté pour l'usage qu'elle désirait.
Elle tira donc sur le temps, vanta les serviteurs de jeunesse et de
toute la vie, l'attachement de toute celle du maréchal de Villeroy pour
lui, sa douleur de lui avoir déplu, la longueur de sa pénitence, sa
désolation de ne pouvoir être auprès du roi dans des moments si
calamiteux, la douceur de se retrouver avec ceux avec qui on avait
toujours vécu, et dont on était sûr que le cœur n'avait point de part
aux fautes\,; en un mot, elle sut si bien dire et presser que tout ce
qui était à Marly pensa tomber d'étonnement d'y voir paraître le
maréchal de Villeroy le matin que le Dauphin mourut, et reçu du roi avec
tout l'air d'amitié et de familiarité que la situation de son cœur et de
son esprit lui purent permettre. De ce moment il ne quitta plus la cour,
fut traité du roi mieux que jamais\,; incontinent après admis chez
M\textsuperscript{me} de Maintenon aux musiques quand elles y
recommencèrent, et lui unique, en un mot un favori du roi et de
M\textsuperscript{me} de Maintenon, dont nous verrons les grandes et
trop importantes suites.

L'espèce de de la maladie du Dauphin, ce qu'on sut que lui-même en avait
cru, le soin qu'il eut de faire recommander au roi les précautions pour
la conservation de sa personne, la promptitude et la manière de sa fin,
comblèrent la désolation et les affres, et redoublèrent les ordres du
roi sur l'ouverture de son corps. Elle fut faite dans l'appartement du
Dauphin à Versailles comme elle a été marquée. Elle épouvanta. Ses
parties nobles se trouvèrent en bouillie\,; son cœur, présenté au duc
d'Aumont pour le tenir et le mettre dans le vase, n'avait plus de
consistance, sa substance coula jusqu'à terre entre leurs mains\,; le
sang dissous, l'odeur intolérable dans tout ce vaste appartement. Le roi
et M\textsuperscript{me} de Maintenon en attendaient le rapport avec
impatience. Il leur fut fait le soir même chez elle sans aucun
déguisement.

Fagon, Boudin, quelques autres y déclarèrent le plus violent effet d'un
poison très-subtil et très-violent, qui, comme un feu très-ardent, avait
consumé tout l'intérieur du corps, à la différence de la tête qui
n'avait pas été précisément attaquée, et qui seule l'avait été d'une
manière très-sensible en la Dauphine. Maréchal, qui avait fait
l'ouverture, s'opiniâtra contre Fagon et les autres. Il soutint qu'il
n'y avait aucunes marques précises de poison\,; qu'il avait vu des corps
ouverts à peu près dans le même état, dont on n'avait jamais eu de
soupçon\,; que le poison qui les avait emportés, et tué aussi le
Dauphin, était un venin naturel de la corruption de la masse du sang
enflammé par une fièvre ardente qui paraissait d'autant moins qu'elle
était plus interne\,; que de là était venue la corruption qui avait gâté
toutes les parties, et qu'il ne fallait point chercher d'autres causes
que celles-là, qui étaient celles de la fin très-naturelle qu'il avait
vu arriver à plusieurs personnes, quoique rarement à un degré semblable,
et qui alors n'allait que du plus au moins. Fagon répliqua, Boudin
aussi, avec aigreur tous deux. Maréchal s'échauffa à son tour, et
maintint fortement son avis. Il le conclut par dire au roi et à
M\textsuperscript{me} de Maintenon, devant ces médecins, qu'il ne disait
que la vérité, comme il l'avait vue et comme il la pensait\,; que parler
autrement c'était vouloir deviner, et faire en même temps tout ce qu'il
fallait pour faire mener au roi la vie la plus douloureuse, la plus
méfiante et la plus remplie des plus fâcheux soupçons, les plus noirs et
en même temps les plus inutiles\,; et que c'était effectivement
l'empoisonner. Il se prit après à l'exhorter, pour le repos et la
prolongation de sa vie, à secouer des idées terribles en elles-mêmes\,;
fausses suivant toute son expérience et ses connaissances, et qui
n'enfanteraient que les soucis et les soupçons les plus vagues, les plus
poignants, les plus irrémédiables\,; et se ficha fortement contre ceux
qui s'efforçaient de les lui inspirer.

Il me conta ce détail ensuite, et me dit en même temps que, outre qu'il
croyait que la mort pouvait être naturelle, quoique véritablement il en
doutât à tout ce qu'il avait remarqué d'extraordinaire\,; mais qu'il
avait principalement insisté par la compassion de la situation de cœur
et d'esprit où l'opinion de poison allait jeter le roi, et par
l'indignation d'une cabale qu'il voyait se former dans l'intérieur, dès
la maladie, et surtout depuis la mort de M\textsuperscript{me} la
Dauphine, pour en donner le paquet à M. le duc d'Orléans, et qu'il m'en
avertissait comme son ami et le sien\,; car Maréchal qui était effectif,
et la probité, et la vérité, et la vertu même, était d'ailleurs
grossier, et ne savait ni la force ni la mesure des termes, étant
d'ailleurs tout à fait respectueux et parfaitement éloigné de se
méconnaître.

Je ne fus pas longtemps, malgré ma clôture, à apprendre d'ailleurs ce
qui commençait à percer sur M. le duc d'Orléans. Ce bruit sourd, secret,
à l'oreille, n'en demeura pas longtemps dans ces termes. La rapidité
avec laquelle il remplit la cour, Paris, les provinces, les recoins les
moins fréquentés, le fond des monastères les plus séparés, les solitudes
les plus inutiles au monde et les plus désertes, enfin les pays
étrangers et tous les peuples de l'Europe, me retraça celle avec
laquelle y furent si subitement répandus ces noirs attentats de Flandre,
contre l'honneur de celui que le monde entier pleurait maintenant. La
cabale d'alors, si bien organisée, par qui tout ce qui lui convenait se
trouvait répandu de toutes parts, en un instant, avec un art
inconcevable, cette cabale, dis-je, avait été frappée comme on l'a vu,
et son détestable héros réduit à l'aller faire en Espagne. Mais pour
frappée, quoique hors de mesure et d'espérance par tous les changements
arrivés, elle n'était pas dissipée. M. du Maine et ceux qui restaient de
la cabale et qui continuaient de figurer comme ils pouvaient à la cour,
Vaudemont, sa nièce d'Espinoy, d'autres restes de Meudon, vivaient. Ils
espéraient contre toute espérance\,; ils se roidissoient contre la
fortune si apparemment contraire. Ils en saisirent ce funeste retour,
ils ressuscitèrent\,; et avec M\textsuperscript{me} de Maintenon à leur
tête, que ne se promirent-ils point, et, en effet, jusqu'où
n'allèrent-ils pas\,? On a vu, je ne dis pas les desseins du Dauphin à
l'égard des bâtards, parce qu'ils étaient secrets, mais combien lui et
son épouse avaient désapprouvé leur grandeur, jusque sous les yeux du
roi (t. VII, p.~146 et suiv.). Ni l'un ni l'autre ne leur avaient pas
paru plus favorables depuis. Le duc du Maine en espérait si peu qu'il ne
s'était point approché d'eux\,; et ni par soi ni par
M\textsuperscript{me} de Maintenon même, dont sa grandeur était
l'ouvrage et qui avait été le témoin affligé et embarrassé, au point où
on l'a vu, de leur répugnance, ni par le roi même qui l'avait si
vivement sentie, et si humblement soufferte pour l'émousser, il n'avait
osé depuis rien tenter auprès d'eux. Quoique en médiocre liaison avec
son frère, et sur cela même, mais qui, une fois fait, avait le même
intérêt que lui de s'assurer de ne pas déchoir, et qui, bien avec le
Dauphin et la, Dauphine par le rapport du monde et des parties, était
fort à portée d'eux, rien par là n'avait été essayé là-dessus. La
duchesse du Maine, plus ardente que lui sur les rangs, s'il était
possible, ne bougeait de Sceaux à faire la déesse, et ne daignait pas
approcher de la cour.

M. du Maine, le plus timide des hommes, quoique le plus grand ouvrier
sous terre, vivait en des transes mortelles pour toutes ses grandeurs,
et il avait trop d'esprit encore pour ne pas trembler aussi pour ses
énormes établissements peu sûrs à lui laisser, si on venait à abattre le
trône qu'il s'était bâti. Cependant ses enfants croissaient, le roi
vieillissait\,; il pâlissait d'effroi de la perspective que l'âge du roi
rendait peu éloignée, et que les transes mortelles de tout son être lui
rapprochaient encore plus. Il n'avait qui que ce fût auprès du Dauphin
et de la Dauphine dont il pût tirer secours dans aucun temps\,; il n'y
voyait aucun remède. Leur mort fut donc pour lui la plus parfaite
délivrance, et dans la même mesure qu'elle fut pour toute la France le
malheur le plus comblé. Quelle étoile\,! mais quel coup de baguette\,!
quel subit passage des terreurs du sort d'Encelade à la ferme espérance
de celui de Phaéthon et de le rendre durable\,! Il se vivifia donc des
larmes universelles\,; mais en maître dans les arts les plus ténébreux,
je ne dirai pas les plus noirs, parce que nulle notion ne m'en est
revenue, il crut qu'il lui importait de fixer les soupçons sur
quelqu'un, et c'était pour lui coup double et centuple d'en affubler M.
le duc d'Orléans.

La convalescence de la disgrâce de ce prince auprès du roi encore mal
affermie, et la mort des princes du sang d'âge à représenter et à
parler, lui avaient valu ses immenses et dernières grandeurs. En
accablant ce même prince d'une si affreuse calomnie, et venant à bout de
la persuader au roi et au monde, il comptait bien de le perdre sans
retour de la façon la plus odieuse et la plus ignominieuse\,; et, si la
même baguette qui l'avait si heureusement défait de ce qu'il redoutait
le plus ne lui rendait pas le même service à l'égard de M. le duc de
Berry, il avait lieu de se flatter que ce prince ne résisterait pas à
l'opinion du roi ni à la publique\,; que la douleur de la mort de son
frère lui ferait craindre et haïr celui qu'il en croirait le
meurtrier\,; et cet obstacle rangé, les moyens ne manqueraient pas de
circonvenir ce prince fait, et accessible par tant de côtés, comme il
l'était. Réduisant M. le duc d'Orléans dans une situation aussi cruelle,
sur laquelle il se proposait bien d'entrer avec M\textsuperscript{me} sa
sœur dans ses malheurs et de lui faire valoir par elle son assistance,
c'était un moyen de le tenir de court et de parvenir au mariage du
prince de Dombes avec une de ses filles, sœur de M\textsuperscript{me}
la duchesse de Berry, à quoi tous ses manéges avaient jusqu'alors
échoué, quoique appuyés des plus passionnés désirs de
M\textsuperscript{me} la duchesse d'Orléans ni son adresse à éluder sans
refuser.

Parmi les princes du sang, tous gens d'âge à compter pour rien, le duc
de Chartres, sous l'aile de père et de mère, était d'août 1703 et
n'avait que neuf ans\,; M. le Duc était d'août 1692, il avait vingt
ans\,; le comte de Charolais de juin 1700, il n'avait pas douze ans\,;
le comte de Clermont de juin 1709, il n'avait que trois ans\,; et le
prince de Conti de juin 1704, qui n'avait que huit ans. Il ne pouvait
donc avoir à compter que M. le Duc, dont à vingt ans le roi ne faisait
nul compte, et devant qui ce prince n'eût pas osé souffler, ni
M\textsuperscript{me} la Duchesse non plus. M\textsuperscript{me} la
Princesse, qui n'eut jamais de sens ni d'esprit que pour prier Dieu,
tremblait devant sa fille, la duchesse du Maine\,; elle avait même
remercié le roi en forme de ce qu'il avait fait pour les enfants de M.
du Maine\,; et son autre fille, M\textsuperscript{me} la princesse de
Conti, avait passé sa vie a Paris dans ses affaires domestiques, qui
n'aurait osé approcher du roi. M\textsuperscript{me} de Vendôme
n'existait pas, ni les filles de M\textsuperscript{me} la Duchesse, par
leur âge, à l'égard du roi. C'était donc un champ libre fait exprès pour
M. du Maine. Quel parti n'en sut-il pas tirer\,! M\textsuperscript{me}
de Maintenon n'avait des yeux que pour lui\,; en lui se réunissait toute
sa tendresse par la perte de sa chère Dauphine. Sa haine pour M. le duc
d'Orléans était toujours la même, on en a vu la cause et les fruits. Son
nourrisson si constamment aimé n'eut donc pas peine à lui persuader ce
qui flattait cette haine, ce qui établissait à soi toutes ses
espérances, ou à se porter à n'en douter pas et à le faire accroire au
roi, si eux-mêmes n'en étaient pas persuadés, et à en infatuer le monde.
On ne put se méprendre à l'auteur et à la protectrice de ces horribles
bruits\,; ni l'un ni l'autre ne s'en cachèrent dans l'intérieur.
M\textsuperscript{me} de Maintenon se fâcha contre Maréchal devant le
roi. Il lui échappa qu'on savait bien d'où venait le coup, et de nommer
M. le duc d'Orléans. Le roi y applaudit avec horreur, comme n'en doutant
pas, et tous deux ne parurent pas trouver bon la liberté que prit
Maréchal de se récrier contre cette accusation. M. Fagon, par ses coups
de tête, approuvait cependant cet énorme allégué\,; et Boudin fut assez
forcené pour oser dire qu'il n'y avait pas à douter que ce ne fût ce
prince, et pour hocher la tête impudemment à la sortie que Maréchal eut
le courage de lui faire. Telle fut la scène entière du rapport de
l'ouverture du Dauphin. Le duc du Maine s'en expliqua nombre de fois
dans l'intérieur des cabinets du roi\,; et, quoique ce ne fût pas sans
prendre garde aux valets devant qui il parlait, il y en eut plus d'un,
et à plus d'une reprise, qui le dirent, et par qui d'oreille en oreille
cela se répandit. Bloin, et les autres de l'intérieur qui lui étaient
les plus affidés, ne craignirent point de répandre une accusation si
atroce, comme une chose dont le roi ni M\textsuperscript{me} de
Maintenon ne doutaient point, et de laquelle ils étaient convaincus
eux-mêmes, avec Fagon, qui les autorisa par l'obstination de son
silence, et par des gestes et des airs éloquents lorsqu'on en parlait en
sa présence, et de Boudin qui s'en fît le prédicateur également infâme
et hardi, et qui tinrent le reste de la Faculté de si court, qu'aucun
n'osa dire un seul mot au contraire. Cette même terreur gagna bientôt
toute la cour, dès qu'elle vit tout ce qui approchait le plus
M\textsuperscript{me} de Maintenon déclamer avec d'autant plus de force
que c'était avec un air d'horreur, de crainte, de retenue\,; et tout ce
peu qui tenait au duc et à la duchesse du Maine, et tout Sceaux et
jusqu'à leurs valets, en parler non-seulement à bouche ouverte, mais en
criant vengeance contre M. le duc d'Orléans, et demandant si on ne la
ferait point, avec un air d'indignation et de sécurité la plus effrénée.
De là tout ce qui même {[}était{]} de plus élevé, et de plus à portée de
vouloir et d'espérer plaire, prit à la cour la même hardiesse et le même
ton\,; et ce fut la même opinion et les mêmes propos à la mode qu'en
autre genre on y avait vus si répandus et si dominants pendant la
campagne de Lille contre le prince qu'on regrettait maintenant, et avec
ce même succès d'effroi qui écartait tous contradicteurs et les
réduisait au silence. Maréchal qui sagement ne m'avait d'abord averti
qu'à demi, voyant le commencement de cette tempête, me conta le détail
de ce qui s'était passé chez M\textsuperscript{me} de Maintenon, en
présence du roi, que je viens de rapporter.

M. le duc d'Orléans avait, à l'égard des deux pertes qui faisaient
couler les larmes publiques, l'intérêt le plus directement
contradictoire à celui du duc du Maine\,; et, s'il avait été un monstre
vomi de l'enfer, c'eût été le grand coup pour lui de se défaire du roi,
avec lequel il ne s'était jamais bien remis, et s'était même fort gâté
depuis le mariage de M\textsuperscript{me} la duchesse de Berry, pour
faire régner ceux qu'on regrettait, et se délivrer de la puissance de
M\textsuperscript{me} de Maintenon, son implacable ennemie, qui ne
cessait de lui aliéner le roi, et de lui faire tout le mal qui lui était
possible, jusqu'à lui avoir ôté, même depuis ce mariage, toute
considération à la cour. Nous ne sommes pas encore au temps de faire
connaître ce prince\,; un crayon suffira ici par rapport à son intérêt
et aux horreurs d'une accusation si terriblement inventée, si
cruellement répandue, persuadée et soutenue avec tant d'art, et un art
si peu inférieur au crime qui lui fut imputé, et dont M. du Maine a su
tirer tous les avantages qu'il en avait attendus jusqu'au delà de ses
espérances, et qui eussent mis la confusion dans l'État s'ils eussent
été prodigués à un homme moins failli de cœur et de courage, et d'un
mérite moins universellement décrié de tous points.

Dans tous les temps le Dauphin avait goûté M. le duc d'Orléans. Dès sa
jeunesse le duc de Chevreuse le lui avait fait valoir, parce que le duc
de Montfort, son fils aîné, était intimement avec M. le duc d'Orléans,
et que M. de Chevreuse lui-même le voyait assez souvent, et se plaisait
à s'entretenir avec lui d'histoire, mais surtout de sciences, souvent de
religion, où il voulait le ramener. L'archevêque de Cambrai le voyait
aussi, et se plaisait fort avec lui\,; et réciproquement M. le duc
d'Orléans l'avait pris en amitié, et en telle estime qu'il se déclara
hautement pour lui lors de sa disgrâce, et qu'il ne varia jamais depuis
là-dessus. Cela lui avait attaché tout ce petit troupeau, quoique de
mœurs si différentes\,; et on sait ce que ce petit troupeau pouvait sur
le Dauphin, très-particulièrement l'archevêque de Cambrai, M. de
Chevreuse et le duc de Beauvilliers, qui n'étant qu'un avec eux ne
pouvait être différent d'eux sur M. le duc d'Orléans. Indépendamment de
ces appuis, ces deux princes se rencontraient souvent chez le roi,
très-ordinairement les soirs chez la princesse de Conti, où ils se
mettaient en un coin à parler sciences, et on n'en pouvait parler plus
nettement, plus intelligiblement ni plus agréablement que faisait M. le
duc d'Orléans. C'était donc une liaison de tous les temps entre eux à
être bien aises de se rencontrer, et à leur aise ensemble, autant que
des personnes de cette élévation et de vie aussi différente en pouvaient
former. Le mariage du Dauphin et l'union de ce mariage augmenta encore
la liaison.

La Dauphine était fort attachée à M. et à M\textsuperscript{me} de
Savoie. Elle trouva ici Monsieur, père de M\textsuperscript{me} de
Savoie, et de M. le duc d'Orléans. Elle et Monsieur, comme on l'a vu,
s'aimèrent avec tendresse\,; et cette affection pour mère et pour
grand-père, retomba sur l'oncle, en qui même elle se piqua toujours de
s'intéresser, jusque dans les temps où il fut le plus mal avec le roi et
M\textsuperscript{me} de Maintenon, qui le lui passaient à cause de
l'étroite proximité. À son tour M. le duc d'Orléans, maltraité de
Monseigneur et de toute cette pernicieuse cabale qui le gouvernait,
exactement instruit par moi en Espagne où il était de tous les attentats
de la campagne de Lille, prit hautement à son retour le parti du prince
opprimé, et ce fut un nouveau lien entre eux, et la Dauphine en tiers.
Peu de temps après, l'affaire d'Espagne ayant réduit M. le duc d'Orléans
aux termes les plus dangereux dont Monseigneur se rendit le plus ardent
promoteur, il trouva dans son fils une ferme résistance jusque dans le
conseil, et dans sa belle-fille la plus vive protectrice de son oncle,
quoiqu'elle ne pût ignorer combien elle allait directement en cela
contre ce que voulait et faisait M\textsuperscript{me} de Maintenon.
Dans les suites cette princesse la gagna pour le mariage de
M\textsuperscript{me} la duchesse de Berry, et le roi par elle. Sa
liaison personnelle avec M\textsuperscript{me} la duchesse d'Orléans,
déjà formée, en devint intime, et ne cessa plus, et se resserra de plus
en plus avec M. le duc d'Orléans, et entre son époux et le même prince.

M. de Beauvilliers, si retenu à le voir, ne l'était pas à entretenir une
amitié qu'il croyait si utile dans la maison royale, jusque-là que, sur
les fins, il m'avertit que les propos licencieux auxquels M. le duc
d'Orléans s'abandonnait quelquefois en présence du Dauphin ne pouvaient
que lui nuire et l'éloigner de lui, et de lui dire franchement d'y
prendre garde comme un avis de sa part, à qui le Dauphin s'en était
ouvert. Je le fis, il s'en corrigea, et si bien qu'il me revint par la
même voie que cette retenue réussissait fort bien, que le Dauphin en
avait parlé avec satisfaction au duc de Beauvilliers, qui me chargea de
le dire à M. le duc d'Orléans pour le soutenir et l'encourager dans
cette attention. Il tenait donc immédiatement au Dauphin par un goût de
tous les temps, par l'amusement de la conversation savante\,; par ce qui
tenait le plus intimement au Dauphin, par une conduite sur M. de Cambrai
écrite dans leur cœur à tous, par la proximité et la profession publique
d'intérêt en lui et d'amitié de la Dauphine dans les temps les plus
orageux, et réciproquement par son attachement public pour eux lors des
attentats de Flandre. Il y tenait par l'intimité de leurs épouses, par
les mêmes amis et les mêmes ennemis, par le mariage de
M\textsuperscript{me} la duchesse de Berry qui fut l'ouvrage de la
Dauphine, par la haine commune de M\textsuperscript{me} la Duchesse et
de la cabale de Meudon, qui les voulait tous deux anéantir, en un mot
par tous les liens les plus forts et les plus de toutes les sortes qui
peuvent former et serrer les unions les plus étroites et les plus
intimes\,; sans jamais de contretemps, sans aucune lacune, et sans rien
même qui pût y apporter du changement, puisque la conduite de
M\textsuperscript{me} la duchesse de Berry et celle de M. le duc
d'Orléans à cet égard n'y avait pas produit le plus léger
refroidissement.

Je ne fais que montrer et parcourir toutes ces choses et ces faits pour
les présenter à la fois sous les yeux, parce qu'ils se trouvent tous
racontés épars, en leur temps, en ces Mémoires. Rassemblés ici, on voit
que M. le duc d'Orléans avait pour le moins autant et aussi certainement
tout à gagner à la vie et au règne du Dauphin et de la Dauphine, que le
duc du Maine avait tout à en craindre et à y perdre, et ce contraste est
d'une évidence à sauter aux yeux. Il avait de plus les jésuites qui
faisaient tous une profession ouverte d'attachement pour lui, qui la lui
avait solidement marquée par les services hardis que le P. Tellier lui
avait rendus sur le mariage de M\textsuperscript{me} la duchesse de
Berry, et qui étaient payés pour cela par la protection qu'il leur
donnait, et par la feuille des nombreux bénéfices de son apanage, qui
tous, à l'exception des évêchés, étaient à sa nomination. Que l'on
compare maintenant ensemble l'intérêt de M. le duc d'Orléans, dont le
rang et l'état, au moins de lui et des siens ne pouvait être susceptible
de péricliter en aucun cas possible, et sans charge ni gouvernement à
lui ni à son fils\,; qu'on le compare à l'intérêt du duc du Maine, et
que l'on cherche après l'empoisonneur. Mais ce n'est pas tout. Qu'on se
souvienne qu'il n'avait pas tenu à Monseigneur de faire couper la tête à
M. le duc d'Orléans, et combien il en avait été proche\,; qu'on se
souvienne comment Monseigneur ne cessa depuis de le traiter\,; et qu'en
même temps on se souvienne des larmes et des sanglots cachés dans le
recoin de cet arrière-cabinet où je surpris M. le duc d'Orléans la nuit
de la mort de Monseigneur, de mon étonnement extrême, de la honte que
j'essayai de lui en faire, et de ce qu'il m'y répondit. Quel contraste,
grand Dieu\,! de cette douleur de la mort d'un ennemi près de devenir
son maître, avec la farce que M. du Maine donna à ses intimes au fond de
son cabinet, sortant de chez le roi qu'il venait de laisser presque à
l'agonie, livré aux remèdes d'un paysan grossier, que M. du Maine
contrefit et la honte de Fagon, avec tant de naturel et si plaisant que
les éclats de rire s'en entendirent jusque dans la galerie, et y
scandalisèrent les passants. C'est un fait célèbre et bien caractérisant
qui trouvera son détail en son lieu, si j'ai assez de vie pour pousser
ces Mémoires jusqu'à la mort du roi.

Mais une écorce funeste servit bien le duc du Maine, qu'il sut
puissamment manier, et avec un art qui lui était singulièrement propre.
M. le duc d'Orléans, marié par force, instruit de l'indignité de
l'alliance par les fureurs de Madame, par le cri public, jusque par la
faiblesse de Monsieur, fit en même temps ce qu'on appelle son entrée
dans le monde. Plus son éducation avait été jusqu'alors resserrée, plus
il chercha à s'en dédommager. Il tomba dans la débauche, il préfera les
plus débordés pour ses parties\,; sa grandeur et sa jeunesse lui firent
voir tout permis\,; et il se figura de réparer aux yeux du monde ce
qu'il crut y avoir perdu par son mariage, en méprisant son épouse, et en
se piquant de vivre avec et comme les plus effrénés. De là le désir de
l'irréligion et l'extravagante vanité d'en faire une profession
ouverte\,; de là un ennui extrême de toute autre chose que débauche
éclatante\,; les plaisirs, ordinaires et raisonnables, insipides\,;
l'oisiveté profonde à la cour où il ne pouvait traîner sa funeste
compagnie, et où pourtant il fallait bien qu'il demeurât souvent\,; nul
entregent pour s'en attirer d'autre, et dans une réciproque contrainte
avec son épouse et avec tout ce qui l'approchait, qui lui faisait
préférer sa solitude\,; et cette solitude, il était trop accoutumé au
bruit pour la pouvoir supporter.

Jeté par là dans la recherche des arts, il se mit à souffler, non pour
chercher à faire de l'or, dont il se moqua toujours, mais pour s'amuser
des curieuses opérations de la chimie. Il se fit un laboratoire le mieux
fourni, il prit un artiste de grande réputation, qui s'appelait Humbert,
et qui n'en avait pas moins en probité et en vertu qu'en capacité pour
son métier. Il lui fit suivre et faire plusieurs opérations, il y
travailla avec lui\,; mais tout cela très-publiquement, et il en
raisonnait avec tous ceux de la profession de la cour et de la ville, et
en menait quelquefois voir travailler Humbert et lui-même. Il s'était
piqué autrefois d'avoir cherché à voir le diable, quoiqu'il avouât qu'il
n'y avait pu réussir\,; mais épris de M\textsuperscript{me} d'Argenton,
et vivant avec elle, il y trouva d'autres curiosités trop approchantes
et sujettes à être plus sinistrement interprétées. On consulta des
verres d'eau devant lui sur le présent et sur l'avenir. J'en ai rapporté
des choses assez singulières, qu'il me raconta avant d'aller en Italie,
pour me contenter ici de rappeler seulement ces malencontreux
passe-temps, tout éloignés qu'ils fussent de la plus légère idée même de
crime. L'affaire d'Espagne dont il n'était jamais bien revenu\,; les
bruits affreux de lui et de sa fille par lesquels on essaya de rompre le
mariage de cette princesse avec M. le duc de Berry près d'être
déclaré\,; la publicité que la rage de cette grande affaire leur donna
ensuite, le trop peu de cas que l'un et l'autre en firent, et le trop
peu de ménagement là-dessus\,; enfin jusqu'à l'horrible opinion prise
sur Monsieur de la mort de sa première épouse, et que M. le duc
d'Orléans était le fils de Monsieur\,; tout cela forma ce groupe
épouvantable dont ils surent fasciner le roi, et aveugler le public.

Il en fut, comme je l'ai remarqué, si rapidement abreuvé que, dès le 17
février, que M. le duc d'Orléans fut avec Madame donner l'eau bénite à
la Dauphine, la foule du peuple dit tout haut toutes sortes de sottises
contre lui tout le long de leur passage, que lui et Madame entendirent
très-distinctement, sans oser le montrer, mais dans la peine, l'embarras
et l'indignation qui se peut imaginer. Il y eut même lieu de craindre
pis d'une populace excitée et crédule, lorsque, le 21 février, il alla
seul donner l'eau bénite au Dauphin. Aussi essuya-t-il sur son passage
les insultes les plus atroces d'un peuple qui ne se contenait pas, qui
lançait tout haut les discours les plus énormes, qui le montrait au
doigt avec les épithètes les plus grossières, que personne n'arrêtait,
et qui croyait lui faire grâce de ne se pas jeter sur lui et le mettre
en pièces. Ce fut la même chose au convoi. Les chemins retentissaient de
cris plus d'indignation et d'injures que de douleur. On ne laissa pas de
prendre sans bruit quelques précautions dans Paris pour empêcher la
fureur publique dont les bouillons se firent craindre en divers moments.
Elle s'en dédommagea par les gestes, les cris, et par tout ce qui se
peut d'atroce, vomi contre M. le duc d'Orléans. Vers le Palais-Royal,
devant lequel le convoi passa, le redoublement de huées, de cris,
d'injures, fut si violent, qu'il y eut lieu de tout craindre pendant
quelques minutes.

On peut imaginer le grand usage que M. du Maine sut tirer de la folie
publique, du retentissement des cafés de Paris, de l'entraînement du
salon de Marly, de celui du parlement, où le premier président lui
rendit religieusement ses prémices, de tout ce qui ne tarda pas à
revenir des provinces, ensuite des pays étrangers. On ne sème que pour
recueillir, et la récolte passa toutes les espérances. La mort du petit
Dauphin et le rapport de son ouverture fut un nouveau relais qui ranima
plus violemment la fureur et la licence, qui donna un nouveau jeu à M.
du Maine, à Bloin, aux affidés de l'intérieur, à M\textsuperscript{me}
de Maintenon, de les faire valoir\,; au roi, d'abattement, de crainte,
de haine et d'un malaise continuel. C'est la cruelle situation où ils le
voulaient pour se le rendre plus maniable, et disposer de lui plus
facilement. Le maréchal de Villeroy, quoique si distingué toute sa vie
par l'amitié de Monsieur et la considération de M. le duc d'Orléans,
n'avait garde de ne pas payer comptant son brillant retour à sa
protectrice. Il était fait pour ne penser et ne croire que comme
elle-même pensait et croyait, ou en faisait le semblant. Il avait été
trop avant dans l'intérieur de la cour, pour ignorer sa haine pour M. le
duc d'Orléans, et son aveuglement de mie pour M. le duc du Maine. Il
n'était pas rentré par elle pour les contredire, mais pour devenir leur
instrument et leur écho. Il se signala donc dans une occasion si
intéressante, et qui la lui devenait à lui-même par son ami Vaudemont,
Tessé le suivant de celui-ci, Tallard si longtemps le sien,
M\textsuperscript{me} d'Espinoy, les Rohan ses boussoles, Harcourt qui
l'était d'une autre façon, mais qui avec son esprit et son adresse sut
se mesurer dans le monde, sans cesser de plaire aux calomniateurs dont,
avec eux, il épousa les passions.

Le duc de Noailles tenait le loup par les oreilles. Il était en
quartier, par conséquent il se trouvait en des moments de privance chez
le roi et chez M\textsuperscript{me} de Maintenon. Plus il se sentait
mal avec eux, plus il craignait de leur déplaire, plus il passionnait de
s'y raccrocher. Il échappait souvent en sa présence des mots à l'un et à
l'autre où il n'osait prendre, parce qu'il ne voulait pas se rebrouiller
avec M. le duc d'Orléans. Il voiloit son silence du malaise où il était
avec eux\,; mais les occasions étaient continuelles. Il y avait
longtemps à attendre jusqu'au 1\^{}er avril\,; peut-être encore que
cette fatale tabatière lui pesait, quoique bien loin hors de sa poche.
Il eut une très-légère fluxion sur le visage qui ne fut accompagnée
d'aucun symptôme\,; il la donna pour une attaque d'apoplexie. Quoique
tout le monde ne cessât de le voir, et que personne ni les médecins n'en
aperçussent pas le moindre soupçon, lui, au contraire de tous les
apoplectiques, dont l'un des plus généraux effets de leur mal est de le
nier et de n'en vouloir jamais convenir, quitta le bâton les premiers
jours de mars et s'en alla à Vichy, où il demeura longtemps en panne, et
à laisser refroidir les fureurs et les propos, qui à la fin ne peuvent
toujours rouler sur la même chose. Il en revint parfaitement guéri,
parce qu'il n'était pas parti malade\,; et il n'a pas été question
depuis pour lui d'apoplexie ni de la moindre précaution pour la
prévenir.

\hypertarget{chapitre-vii.}{%
\chapter{CHAPITRE VII.}\label{chapitre-vii.}}

1712

~

{\textsc{Effiat avertit M. le duc d'Orléans et lui donne un pernicieux
conseil, qu'il se hâte d'exécuter.}} {\textsc{- Crayon d'Effiat.}}
{\textsc{- Conduite que M. le duc d'Orléans devait tenir.}} {\textsc{-
M. le duc d'Orléans totalement déserté et seul au milieu de la cour.}}
{\textsc{- Je lui reste unique.}} {\textsc{- Je l'empêche de faire un
cruel affront à La Feuillade.}} {\textsc{- Crises et bruits contre M. le
duc d'Orléans entretenus avec grand art et toujours.}} {\textsc{- Alarme
de mes amis sur ma conduite avec M. le duc d'Orléans.}} {\textsc{-
Service de Maréchal à M. le duc d'Orléans.}} {\textsc{- Deux cent trente
mille livres}} \footnote{Les pensions énumérées par Saint-Simon ne
  donnent que cent mille livres. Il y a probablement erreur dans le
  sommaire.} {\textsc{de pensions et vingt mille livres distribuées dans
la maison du Dauphin et de la Dauphine.}} {\textsc{- Mort de
Seignelay\,; son caractère.}} {\textsc{- Maillebois maître de la
garde-robe sans qu'il lui en coûte rien, et La Salle en tire le
double.}} {\textsc{- Douze mille livres de pension à Goesbriant.}}
{\textsc{- Survivance des gouvernements de Béarn, Bayonne, etc., au duc
de Guiche.}} {\textsc{- Tallard duc vérifié.}} {\textsc{- Appartement de
Monseigneur donné à M. {[}le duc{]} et M\textsuperscript{me} la duchesse
de Berry\,; le leur aux fils du duc du Maine\,: et au prince de Dombes,
la survivance du gouvernement de Languedoc.}} {\textsc{- Estaing vend sa
charge dans la gendarmerie.}} {\textsc{- Chimère de ce corps sur l'ordre
du Saint-Esprit.}} {\textsc{- Digression sur le prétendu droit des fils
de France, etc., de présenter au roi des sujets pour être faits
chevaliers de l'ordre.}} {\textsc{- Plaisante anecdote sur la promotion
d'Étampes à l'ordre du Saint-Esprit.}}

~

L'enchaînement naturel de toutes ces choses m'emporte, il faut se
ramener. Depuis l'extrémité du Dauphin, je ne sortis plus de ma chambre
qu'un moment pour voir le roi, et pour aller passer les après-dînées à
Versailles, dans celle du duc de Beauvilliers qui ne voyait presque du
tout personne, malade dans son lit, et pénétré de douleur au point où il
était. Un soir que j'en revenais, M\textsuperscript{me} la duchesse
d'Orléans me manda que M. le duc d'Orléans et elle s'ennuyaient fort de
ne me point voir, et que l'un et l'autre me priaient d'y aller, parce
qu'ils avaient quelque chose de pressé à me dire. Je ne les avais point
vus depuis le malheur public. Quoique Maréchal m'eût parlé, je n'avais
point été assez maître de ma douleur pour aller ailleurs que voir une
douleur pareille. Je ne me trouvais en état ni de parler ni encore moins
de raisonner\,; j'avais l'esprit si peu libre, et je ne voyais de plus
rien à faire sur une si atroce, mais si folle calomnie, et forgée dans
le sein de la plus tendre faveur. Je priai donc M. {[}le duc{]} et
M\textsuperscript{me} la duchesse d'Orléans de trouver bon que je
différasse à les voir au lendemain matin.

J'y allai en effet. Je trouvai M\textsuperscript{me} la duchesse
d'Orléans désolée. Elle m'apprit que le marquis d'Effiat était venu, la
veille au soir, de Paris les avertir des bruits affreux qui y étaient
universellement répandus, de l'effet général qu'ils y faisaient\,; que
le roi et M\textsuperscript{me} de Maintenon étaient non-seulement
persuadés par le rapport des médecins, mais qu'ils l'étaient aussi de
tout ce qui se disait contre M. le duc d'Orléans, et qui se débitait
avec tant d'emportement que d'Effiat ne le croyait pas en sûreté,
s'était déterminé malgré l'horreur de la chose à les venir avertir, et à
presser M. le duc d'Orléans d'avoir là-dessus avec le roi une
explication qui ne pouvait être différée, dont la plus naïve, la plus
nette et la plus persuasive était d'insister pour que le roi lui permît
de se remettre à la Bastille, de faire arrêter Humbert et tous ceux de
ses gens que le roi jugerait à propos, jusqu'à ce que cela fût éclairci.
«\,Madame, m'écriai-je, eh\,! que prétend faire M. le duc d'Orléans\,?
--- Monsieur, me dit-elle, il est allé parler au roi ce matin, qu'il a
trouvé fort sérieux et fort froid, même fort sec, et silencieux sur les
plaintes qu'il lui a faites et la justice qu'il lui a demandée. --- Et
la Bastille, madame, interrompis-je, en a-t-il parlé\,? --- Eh\,!
vraiment oui, monsieur, me répondit-elle, mais cela n'a pas été reçu. Il
y a eu un air de dédain, qui n'a pas changé, quoiqu'il ait fort insisté.
Enfin M. le duc d'Orléans s'est rabattu à demander au moins qu'Humbert y
fût mis, interrogé, et toutes les suites. Le roi a encore refusé d'assez
mauvaise grâce. Enfin, à force d'instances, il a dit qu'il ne le ferait
pas arrêter, mais qu'il donnerait ordre à la Bastille de l'y recevoir
s'il y allait se remettre lui-même.\,» Je m'écriai encore plus sur un si
pernicieux conseil, et si brusquement exécuté.

Il faut savoir que le marquis d'Effiat était un homme de beaucoup
d'esprit et de manége, qui n'avait ni âme ni principes, qui vivait dans
un désordre de mœurs et d'irréligion public, également riche et avare,
d'une ambition qui toujours cherchait par où arriver, et à qui tout
était bon pour cela, insolent au dernier point avec M. le duc d'Orléans
même qui, du temps qu'avec le chevalier de Lorraine, dont il était l'âme
damnée, il gouvernait Monsieur, sa cour et souvent ses affaires, à
baguette, s'était accoutumé à le craindre et à admirer son esprit. Avec
tant de vices si opposés au goût et au caractère du roi et de
M\textsuperscript{me} de Maintenon, il en était bien voulu et traité
avec distinction, parce qu'il avait eu part, avec le chevalier de
Lorraine, à réduire Monsieur au mariage de M. son fils, et ce dernier
par l'abbé Dubois\,; que, par conséquent, il s'était toujours entretenu
bien avec M\textsuperscript{me} la duchesse d'Orléans\,; qu'il s'était
sourdement livré et vendu à M. du Maine\,; et que par son ancienne
intimité avec le chevalier de Lorraine, l'ami le plus intime du maréchal
de Villeroy de tous les temps, il était devenu le sien jusqu'à s'en
faire admirer. Le conseil qu'il avait donné était si mauvais, pour un
homme surtout d'autant d'esprit et qui connaissoit si bien le monde,
qu'il me fut fort suspect.

Par cette conduite M. le duc d'Orléans se ravalait à la condition des
plus petites gens, d'un valet même d'une maison volée, au lieu de
l'avoir pris sur le haut ton, et en prince de son rang, sur qui aucun
soupçon ne saurait trouver prise, qui défie avec dignité d'en pouvoir
produire ni articuler le moindre appui, ni l'apparence la plus légère,
et qui, en faisant en public le parallèle exact et juste de son intérêt
et de celui de M. du Maine, tel qu'on vient de le voir, l'aurait fait
trembler avec toute sa faveur, l'aurait réduit à la défensive, et
peut-être, fait comme il était sur le courage, l'aurait forcé à jeter
l'éteignoir sur le feu qu'il avait allumé, et obligé le roi à le
ménager, et M\textsuperscript{me} de Maintenon à ne le pousser plus.
C'est ce que tout d'abord il fallait faire, après avoir demandé justice
au roi avec hauteur devant tout ce qui était après son souper dans le
cabinet, et ne l'avoir pas reçue\,; et, sans s'engager en accusation
directe, encore moins formelle, parler publiquement, assez fortement
pour donner toute cette peur à M. du Maine, et le mettre dans l'embarras
encore du côté du public, déjà si mal prévenu pour lui, et alors irrité
des pas de géant qu'il venait de faire\,; en même temps faire souvenir
le roi et ceux qui en étaient instruits, répandre pour l'apprendre à
tout le monde le fait, qui est raconté en son lieu, de la cassette de
Mercy prise lorsque du Bourg le battit en haute Alsace, n'oublier pas
les curés, les baillis et les officiers de terres de
M\textsuperscript{me} de Lislebonne en Franche-Comté, les uns
juridiquement exécutés, les autres en fuite\,; aussitôt après cette
affaire, et comme on n'était en nulle mesure avec la cour de Vienne, qui
s'opposait le plus à la paix et y traversait le plus les mesures de
celle de Londres, ne craindre pas de rappeler la facilité de la maison
d'Autriche, à s'aider du poison pour se défaire de qui l'embarrasse, la
mort du prince électeur de Bavière, et celle de la reine d'Espagne,
fille de Monsieur\,; et de là expliquer l'obscurité pourtant assez
claire de la lettre du prince Eugène à Mercy, trouvée dans sa cassette,
avec ses instructions sur l'intelligence en Franche-Comté\,: «\,Que si,
malgré toutes les mesures prises, il ne réussissait pas dans cette
expédition, et qu'eux d'ailleurs ne pussent réduire la France au point
qu'on s'était proposé, alors il faudrait en venir au grand remède\,;»
paraphraser bien aisément ce grand remède et l'expliquer des morts que
l'on pleurait, du péril extrême que le duc d'Anjou avait couru, et qui
n'était pas entièrement passé, pour forcer le roi, par le défaut de
toute sa ligne aînée, de rappeler le roi d'Espagne et ses enfants, et
d'en abandonner la monarchie à la maison d'Autriche\,; ajouter tout ce
qu'il convenait pour frapper sur l'insigne scélératesse d'oser répandre
des bruits exécrables, aussi opposés à son intérêt qu'à son honneur,
quand on en trouvait ailleurs de si conformes au crime habituel de la
maison d'Autriche, et annoncés même par le prince Eugène à Mercy, autant
que de telles horreurs sont susceptibles de l'être\,; appuyer là-dessus
avec d'autant plus de force, qu'en effet le soupçon était très-bien
fondé par la lettre du prince Eugène, précédée de si peu d'années des
deux exécutions que l'on vient de citer\,; que cette sorte d'accusation
de la cour de Vienne soulageait le roi et M\textsuperscript{me} de
Maintenon sur ce qu'ils avaient de plus cher, frappait le monde, les
neutres, les gens de bon sens\,; mais lâcher aussi des expressions
obscures qui eussent donné à courir à M. du Maine sur la conformité de
son intérêt, en autres vues, avec celui de la maison d'Autriche, qui
aurait ouvert les yeux au monde, toujours en évitant bien de s'engager
en rien de précis, et par là aurait tenu M. du Maine en effroi, en
grande peine, et le roi et M\textsuperscript{me} de Maintenon fort en
mesure.

Cela eût fait un violent éclat entre lui et M. du Maine\,; mais cet
éclat le désarmait\,: un ennemi public et déclaré est bien moins à
craindre que des mines chargées continuellement sous les pieds, un
ennemi surtout sur un trône branlant, qui indignait alors tout le monde,
un ennemi d'aussi peu de courage, et dont tout le danger ne se trouvait
que dans les ténèbres dont il savait s'envelopper et se faire un asile,
pour tout ce qu'il lui convenait d'attenter\,; et le roi, malgré son
abandon de tendresse pour lui et de faiblesse pour M\textsuperscript{me}
de Maintenon, n'aurait pu n'être pas en garde contre lui sur M. le duc
d'Orléans, et dans un grand embarras même de l'accroître davantage après
un si grand éclat. Toute son inquiétude se serait tournée à chercher à
l'apaiser entre eux, à empêcher les voies de fait. Elles n'étaient pas à
craindre de M. du Maine avec personne\,; combien moins avec un
petit-fils de France de la valeur de M. le duc d'Orléans\,! Le comte de
Toulouse n'aimait ni n'estimait son frère, et détestait sa belle-sœur,
desquels il était compté pour fort peu de chose. De la valeur et de
l'honneur il en avait beaucoup. Il est très-douteux que l'un lui eût
permis d'employer l'autre en cette occasion pour l'amour de son frère\,;
il ne l'est pas que le roi lui aurait imposé à temps et efficacement
dans un rang si inégal, dans une affaire si odieuse, où, par qui
d'où\footnote{Vieille locution qjui signifie \emph{de quelque côté que}.}
le bruit vînt, son neveu était l'attaqué et le plus cruellement, le roi
n'eût pas souffert que le comte de Toulouse en eût fait la folie, dont
les suites étaient sans fin et eussent fait le bourreau de ce qui lui
restait de vie\,; et plus que vraisemblablement à la fin et après lui
l'éradication de ses bâtards, avec le feu allumé pour la succession de
M. le Prince, qui eût jeté les princes du sang du côté de M. le duc
d'Orléans. Sa suite et sa maison étaient sans comparaison de celles des
bâtards. M. le duc de Berry était son gendre, abandonné alors d'amour à
son épouse qui était toute à son père et ce bas courtisan si avide de
plaire, quand il n'en coûte point de péril, et le gros du monde de même,
n'eût pas pris aisément parti contre M. le duc d'Orléans, dans de telles
extrémités, dans la position où il était, et dans celle où l'âge du roi
montrait en perspective M. le duc de Berry et lui.

Voilà sans doute ce que le duc du Maine redouta, et qu'il sut parer avec
adresse par le prompt usage du marquis d'Effiat et de ses salutaires
avis. Mais je parlais à sa sœur qui, en comparaison de lui, comptait
pour rien mari et enfants, et prodige d'orgueil, sans l'aimer ni
l'estimer. Je n'eus donc garde de lui montrer rien de ce sur quoi je
viens de m'étendre. Je me contentai de blâmer le conseil en gros par
d'autres raisons dont je pus m'aviser, et plus encore une résolution si
subite. Tandis que nous causions ainsi tous deux seuls, M. le duc
d'Orléans entra\,; jamais je ne vis homme si profondément outré et
abattu. Il me redit ce que je venais d'entendre qui s'était passé entre
le roi et lui, entre son lever et la messe, et l'ordre qu'il avait
envoyé, au retour de cette conversation, pour que Humbert s'allât
remettre à la Bastille. Je lui témoignai, comme j'avais fait à
M\textsuperscript{me} la duchesse d'Orléans, ce que je pensais
là-dessus, mais faiblement, parce que la chose était faite, et que
l'état où je le vis me fit plus de compassion qu'il ne me laissa espérer
des partis vigoureux. Je leur rendis ce que j'avais appris de Maréchal,
mais en supprimant le duc du Maine, duquel je ne parlai que
l'après-dînée tête à tête à M. le duc d'Orléans. Le lendemain, je sus
par lui que le roi avait dit sèchement qu'il avait changé d'avis sur
Humbert\,; qu'il était inutile qu'il allât se remettre à la Bastille, et
qu'il n'y serait pas reçu\,; qu'ayant voulu insister, le roi lui avait
tourné le dos, et s'en était allé dans sa garde-robe, et lui était sorti
du cabinet\,; en sorte qu'il venait de mander ce changement à Humbert,
que nous sûmes après être allé à la Bastille, sur l'ordre qu'il en avait
reçu de M. le duc d'Orléans, et y avoir été refusé.

De ces jours-là du premier éclat à Marly et dans le monde, M. le duc
d'Orléans fut non-seulement abandonné de tout le monde, mais il se
faisait place nette devant lui chez le roi et dans le salon, et, s'il y
approchait d'un groupe de courtisans, chacun sans le plus léger
ménagement faisait demi-tour à droite ou à gauche, et s'allait
rassembler à l'autre bout, sans qu'il lui fût possible d'aborder
personne que par surprise, et même aussitôt après, il était laissé seul
avec l'indécence la plus marquée. Jusqu'aux dames désertèrent un temps
M\textsuperscript{me} la duchesse d'Orléans, et il y en eut qui ne la
rapprochèrent plus. Après avoir si pitoyablement enfourné, il fallut
laisser passer l'orage\,; mais l'orage était trop soigneusement
entretenu pour passer. Il fut soutenu avec la même frayeur de son
approche, la même aliénation jusqu'au dernier Marly de la vie du roi, où
ce monarque menaça ouvertement ruine, et quand les bruits faiblissoient
dans Paris et dans les provinces, il s'y trouvait des émissaires adroits
et attentifs à les renouveler, et d'autres à en faire retentir l'écho à
la cour, et cela dura toujours, et bien après le roi, avec le même art.
En un mot, je fus le seul, je dis exactement l'unique, qui continuai à
voir M. le duc d'Orléans à mon ordinaire, et chez lui et chez le roi, à
l'y aborder, à nous asseoir tous deux en un coin du salon, où assurément
nous n'avions aucun tiers à craindre, à me promener avec lui dans les
jardins, et à la vue des fenêtres du roi et de M\textsuperscript{me} de
Maintenon. À Versailles je vivais dans le même commerce de tous les
jours. Il lui revint que La Feuillade tenait à Paris les propos les plus
injurieux sur lui\,; la furie le transporta, et j'eus toutes les peines
du monde de l'empêcher de le faire insulter, et de sa part, à grand
coups de bâton. C'est l'unique fois que je l'ai vu en furie, et se
porter à une telle extrémité.

Cependant M. de Beauvilliers, le chancelier, tous mes amis et amies,
m'avertissaient sans cesse que j'allais me perdre par une conduite si
opposée à l'universelle, et aux sentiments du roi et de
M\textsuperscript{me} de Maintenon pour M. le duc d'Orléans\,; que ne
rompre pas avec lui, par une entière cessation de le voir, était une
chose honnête et qui se pouvait souffrir\,; mais que de vivre
continuellement avec lui et publiquement, et dans les jardins de Marly
sous les yeux du roi et de toute la cour, c'était une folie inutile à M.
le duc d'Orléans, et qui ne pouvait que déplaire à un point qu'à la fin
elle me perdrait. Je tins ferme, je trouvai que le cas d'aussi rares
malheurs était celui non-seulement de n'abandonner pas ses amis quand on
ne les croyait pas coupables, mais celui encore de se rapprocher d'eux
de plus en plus pour son propre honneur, pour la consolation qu'on leur
devait et qu'ils ne recevaient de personne, et pour montrer au monde
l'indignation qu'on avait de la calomnie. On insista très-souvent, on me
fit entendre que le roi le trouvait mauvais, que M\textsuperscript{me}
de Maintenon en était piquée, on n'oublia rien pour me faire peur. Je
fus insensible à tout ce qu'on put me dire\,; et je ne cessai pas un
jour de voir M. le duc d'Orléans et d'ordinaire deux et trois heures de
suite. Cette matière reviendra bientôt\,; il est temps de reprendre la
suite des événements de cette année. Il faut seulement ajouter que ce
fut encore Maréchal qui empêcha que Humbert n'entrât à la Bastille.

Le roi, que M. le duc d'Orléans venait de quitter, quand il lui en fit
la proposition pour lui-même, et refusé au moins pour
Humbert\footnote{Nous avons reproduit exactement le texte du
  manuscrit\,; mais il y a une erreur évidente, puisque la proposition
  fut \emph{acceptée au moins pour Humbert}.}, entra dans sa garde-robe,
où, plein de la chose, il la conta à Fagon et à Maréchal qu'il y trouva.
Maréchal, avec sa vertueuse liberté, demanda au roi ce qu'il en avait
ordonné. Sur sa réponse, il loua la candeur et la franchise de M. le duc
d'Orléans, la prudence du roi de lui avoir refusé d'aller à la Bastille,
et improuva la permission donnée pour Humbert. «\,Que prétendez-vous par
là, sire, lui dit-il hardiment\,: afficher partout la honte prétendue de
votre plus proche famille\,? et quel en sera le bout\,? de ne trouver
rien, et d'en avoir la honte vous-même. Si par impossible, et je
répondrais bien que non, vous trouvez ce qu'on vous fait chercher,
feriez-vous couper la tête à votre neveu qui a un fils de votre fille,
et publier juridiquement son crime et son ignominie\,? Et si vous ne
trouvez rien, comme sûrement il n'y a rien à trouver, {[}irez-vous{]}
faire dire à tous ses ennemis et les vôtres, que c'est qu'on n'a pas
voulu trouver\,? Croyez-moi, sire, cela est horrible, épargnez-vous-le,
révoquez la permission tout à l'heure, et ôtez-vous de la tête des
horreurs, des noirceurs fausses qui ne sont bonnes qu'à abréger vos
jours et à les rendre très-misérables.\,» Cette vive et si prompte
sortie, d'un homme que le roi connaissoit vrai et réellement attaché à
sa personne, eut son effet pour Humbert. Le roi sur-le-champ dit qu'il
avait raison, qu'aussi ne s'était-il laissé aller pour Humbert que par
importunité, et qu'il ne le laisserait pas entrer à la Bastille\,; et
peu d'heures après que M. le duc d'Orléans se présenta devant lui il le
lui dit et lui ordonna de mander à Humbert de ne plus songer à la
Bastille. Maréchal me le conta le lendemain, et me dit que Fagon et
Bloin n'avaient pas dit un seul mot\,; je l'embrassai de sa vertueuse
bravoure qui avait si bien réussi, et je ne la laissai pas ignorer à M.
le duc et à M\textsuperscript{me} la duchesse d'Orléans.

Le roi donna douze mille livres de pension à la duchesse du Lude,
continua à la comtesse de Mailly les neuf mille livres qu'elle avait, à
toutes les dames du palais leurs six mille livres chacune, à
M\textsuperscript{me} Cantin, première femme de chambre, neuf mille
livres, et à presque toutes les autres femmes de chambre de la Dauphine
les gages qu'elles avaient, neuf mille livres à Boudin, son premier
médecin, et trois mille livres à Dionis, son premier chirurgien. Il
donna douze mille livres de pension à Dangeau, chevalier d'honneur,
autant au maréchal de Tessé, premier écuyer, conserva à tous les menins
les leurs de six mille livres\,; quatre mille livres de pension à
Bayard, écuyer particulier du Dauphin\,; dix mille livres à du Chesne,
son premier valet de chambre\,; cinq mille livres à Bachelier, son
premier valet de garde-robe\,; et neuf mille livres à Dodart, son
premier médecin. Il en donna aussi six mille à la nourrice du dernier
Dauphin, et mit toutes ces femmes auprès de celui qui restait, qui en
eut ainsi trente-deux. Le Fèvre, trésorier général de
M\textsuperscript{me} la Dauphine, eut vingt mille livres une fois
payées, que lui avait coûté sa charge.

Seignelay mourut fort brusquement d'une manière de pourpre. Il était
encore fort jeune, et quoique fort gros il excellait à danser. Il
s'était fait aimer et estimer à la guerre et à la cour, avait apprivoisé
La Salle, dont à la mort de son père, ministre et secrétaire d'État, on
lui avait acheté la survivance de sa charge de maître de la garde-robe
du roi, avec exercice en son absence, qui le regardait comme son fils,
et il était parvenu aux bontés du roi fort marquées. Ce fut un vrai
dommage. Il était gendre de la princesse de Furstemberg, dont il ne
laissa qu'une fille fort riche, aujourd'hui duchesse de Luxembourg. La
Salle y gagna une seconde fois sa charge, dont il fit aussitôt le marché
avec Desmarets pour son fils Maillebois, aujourd'hui chevalier de
l'ordre et maréchal de France, de la charge et non de la survivance,
moyennant cinq cent mille livres, et le payement actuel en outre de
trois années d'appointements de sa charge qui lui étaient dues, et
conserva son logement et les grandes entrées. Il n'en coûta rien à
Desmarets\,; le roi lui donna deux cent mille livres, et à son fils un
brevet de retenue du reste. Ce ne fut pas tout\,: il obtint en même
temps pour Goesbriant, son gendre, chevalier de l'ordre, et qui avait un
bon gouvernement, douze mille livres de pension. Peu de jours après, il
donna au duc de Guiche la survivance de son père des gouvernernents de
basse Navarre, Béarn, Bigorre, Bayonne et Saint-Jean-Pied-de-Port, qui
est un morceau de près de cent cinquante mille livres de rente, et où
sont toutes leurs terres. En même temps il fit le maréchal de Tallard
duc vérifié\,; de cette dernière grâce je n'en ai point su l'intrigue ni
l'anecdote. Peut-être fut-ce un fruit de la nouvelle faveur du maréchal
de Villeroy\,; au moins le nouveau duc fut déclaré un jour ou deux après
une fort longue audience que le roi avait donnée au maréchal de
Villeroy, le soir, chez M\textsuperscript{me} de Maintenon. En même
temps encore le roi donna, avec une légère augmentation, l'appartement
de Monseigneur, qu'occupait le Dauphin, à M. {[}le duc{]} et à
M\textsuperscript{me} la duchesse de Berry, et le leur aux deux fils du
duc du Maine, avec la survivance de son gouvernement de Languedoc à
l'aîné. Il y avait près de deux ans que son frère et lui avaient celles
de l'artillerie et des Suisses. L'aîné allait avoir douze ans, et le
cadet ne passait pas sept et demi.

Estaing, lieutenant général de mérite et de bonne maison, mort chevalier
de l'ordre, avait gardé jusqu'alors sa compagnie de gens
d'armes-Dauphin. La gendarmerie est féconde en chimères et en
prétentions. La Trousse, maréchal de camp avec la même compagnie, avait
été un des légers chevaliers de l'ordre de 1688, par la protection de
Louvois, dont il était le parent et l'affidé\,; Villarceaux, brigadier
avec la même charge, l'avait été aussi en la même promotion,
c'est-à-dire les chevau-légers-Dauphin, parce que M\textsuperscript{me}
de Maintenon, plus que très-amie de son père, l'était toujours demeurée,
l'avait fait nommer dans la promotion\,; et lui, qui était vieux et fort
peu de la cour, demanda et obtint que son fils fût fait chevalier de
l'ordre en sa place. De là la gendarmerie prit prétention que ces
charges donnaient l'ordre\,; parce que, le Dauphin, n'ayant point de
maison, ces deux charges faisaient toute la sienne. Ils voulaient
ignorer que le Dauphin n'a point de maison, parce qu'il n'est qu'un avec
le roi, dont tous les officiers grands et petits le servent, et que,
parce qu'il est un avec le roi, il est censé l'être en tout, et par
conséquent ne lui présente point de son chef de chevaliers de l'ordre à
faire, comme les fils de France qui ont une maison, et le premier prince
du sang qui en a une image. Ainsi d'Estaing, qui par sa naissance, son
mérite et ses services, n'avait pas besoin de ce chausse-pied pour être
chevalier de l'ordre, l'avait gardé pour cela, dans l'idée chimérique
que la gendarmerie s'était faite sur deux exemples auxquels Monseigneur
n'avait influé en rien\,; et la vendit dès qu'il ne vit plus qu'un
Dauphin dans la première enfance. Mais puisque l'occasion s'en présente
si naturelle, il est bon de dire un mot de ces présentations à l'ordre.

Les fils de France en prétendent deux, et voudraient aller jusqu'à
trois\,; les filles de France au moins un\,; les petits-fils de France
un\,; les petites-filles de France un\,; le premier prince du sang un\,;
et maintenant les autres princes du sang n'avouent plus qu'ils n'en ont
point\,; et ceux qui sont en usage d'en avoir se sont avisés, depuis le
ministère de M. le Duc, d'en prétendre en toutes les promotions qui sont
de plus de huit chevaliers, et ont trouvé la complaisance que le roi
s'est borné chaque fois à ce nombre pour ne les pas mécontenter, ou
plutôt le cardinal Fleury. Ces prétentions seront bientôt examinées.
Rien de cela ni qui ait le moindre trait dans les statuts de l'ordre
premier, second, troisième, qui sont les changements et les variations
qu'on a expliqués ailleurs\,; rien non plus dans aucun chapitre ni
règlement postérieur\,; ainsi rien d'écrit qui puisse appuyer quoi que
ce soit de cette prétention, en tout ni dans aucune de ses parties. Il
faut donc en venir à l'usage.

Henri III, instituteur de l'ordre, en a fait dix promotions, et en pas
une des dix on ne trouve aucun chevalier présenté à faire. Le duc
d'Alençon était pourtant son frère, qui avait une maison et une cour
nombreuse, qui par le malheur des temps figurait plus que n'a fait
Gaston du règne de Louis XIII, et incomparablement plus que n'a fait
Monsieur. Si on dit que le duc d'Alençon se moqua de l'institution de
l'ordre du Saint-Esprit, qu'il ne voulut jamais le prendre, et qu'il
affecta toujours de porter celui de Saint-Michel seul, pour des raisons
qui ne sont pas de notre sujet, on répondra que ce qui pouvait être bon
pour lui, que l'ordre nouveau ne pouvait honorer ni distinguer, ne
l'était pas pour ceux qui auraient pu être présentés par lui pour
l'avoir, qui en auraient été fort aises, et lui de nommer à un ordre
qu'il ne voulait pas recevoir. Mais outre ce raisonnement, le fait
parle. Le duc d'Alençon n'y a jamais nommé, et il ne paraît point qu'il
l'ait jamais prétendu. D'autres fils de France, il n'y en avait point\,;
mais la reine Marguerite était sœur d'Henri III, et ne fut brouillée
avec lui que pour y avoir été trop bien. Le roi de Navarre, son mari,
depuis successeur d'Henri III, était premier prince du sang. Il a été
catholique longtemps, et demeurant à la cour depuis la Saint-Barthélémy.
On ne voit nul vestige d'aucun chevalier de l'ordre fait à leur
nomination, ni d'aucune prétention là-dessus de leur part. Ainsi nul
usage en cette faveur sous Henri III, instituteur de l'ordre.

Henri IV, en six promotions qu'il a faites, est le premier qui ait pu
donner lieu à l'origine de cette prétention. Ce fut par une seule chose,
et qu'il n'a pas réitérée. Il faisait élever à sa cour le prince de
Condé, né posthume à Saint-Jean d'Angély, et l'avait ôté aux huguenots
et à Charlotte de La Trémoille, sa mère. Il mit auprès de lui tous
domestiques de son choix, lui fit une maison à part\,; et parce qu'Henri
IV n'avait point d'enfants, et qu'il vivait séparé de la reine
Marguerite sans dessein de la reprendre, il regardait alors le prince de
Condé comme l'héritier de la couronne. Il lui avait donné pour
gouverneur M. de Chevrières, à ce qu'il me semble, quoique le dernier
livre des armes, noms et qualités de l'ordre du Saint-Esprit dise que
c'était le comte de Belin, qui avait été gouverneur de Paris pour la
Ligue avant M. de Brissac. Quoi qu'il en soit, l'un était Mitte, avait
passé par divers emplois, et eut un fils aussi chevalier de l'ordre en
1619, lieutenant général de Provence, ambassadeur à Rome et ministre
d'État. L'autre était Faudoas, tous deux de qualité par eux-mêmes à être
chevaliers de l'ordre. Ce qui marque que celui des deux qui était
gouverneur du prince de Condé n'eut point l'ordre en cette qualité comme
présenté, ou comme ils prétendent encore, nommé par lui, c'est que, de
cette promotion qui fut de dix chevaliers, le duc de Ventadour fut le
premier, M. de Chevrières le second, M. de Belin le troisième\,; or
celui de M, le Prince eût été le dernier, comme on l'a vu depuis. Au
contraire, M. de Choisy, chevalier d'honneur de la reine Marguerite, qui
était L'Hôpital, fut le septième.

Il ne peut donc plus être question ici de la nomination de M. le Prince,
et quant à celle de la reine Marguerite, il n'est pas croyable que, n'en
ayant point prétendu sous Henri III, elle s'en fût avisée sous Henri IV.
Ce prince lui marqua toujours la plus grande considération depuis
qu'elle eut donné les mains à la dissolution de leur mariage, et il
n'est pas surprenant qu'il ait eu celle de faire chevalier de l'ordre
son chevalier d'honneur\,; on ne peut donc faire aucun usage de cette
promotion pour autoriser la prétention. Mais on la remonte à celle de
1595, où Claude Gruel, seigneur de La Prette, fut le vingt-cinquième et
le dernier. C'était véritablement un fort petit gentilhomme et dont les
emplois ne le portaient point à cette distinction. On dit qu'il était au
comte de Soissons, et qu'en recevant le collier, venant à dire suivant
la formule\,: \emph{Domine, non sum dignus}, Henri IV se mit à sourire,
et répondit\,: «\,Je le sais bien, je le sais bien, mais mon cousin le
comte de Soissons m'en a prié.\,» 1° René Yiau, sieur de Chanlivaut, qui
précéda immédiatement La Frette dans cette promotion, n'était pas
meilleur que lui ni plus brillant en emplois. 2° Il serait étrange
qu'Henri IV, qui s'était porté avec tant de partialité pour le prince de
Condé dans le procès que le comte de Soissons lui intenta, eût fait un
chevalier de l'ordre à sa nomination dans une promotion de vingt-cinq
chevaliers, et qu'il n'en eût fait aucun à celle du prince de Condé,
premier prince du sang, duquel il prenait un soin si particulier qu'il
le fit venir à sa cour pour l'élever sous ses yeux, et qu'en novembre de
la même année le parlement le vint saluer en corps à Saint-Germain comme
l'héritier de la couronne, en vertu d'une lettre de cachet qu'Henri IV
en avait expédié au camp de la Fère.

On pourrait dire qu'en janvier, que la promotion se fit, le prince de
Condé n'était peut-être pas encore à la cour\,: ce ne serait pas une
raison d'omettre son droit s'il en avait eu, mais au moins était-il à la
cour en janvier 1597 qu'en une promotion de vingt-deux chevaliers il
n'en eut aucun ni le comte de Soissons. 3° Ce conte porte à faux. Les
chevaliers du Saint-Esprit n'ont jamais dit en recevant l'ordre\,:
\emph{Domine non sum dignus. }Cette formule n'est ni dans les statuts ni
dans aucun règlement\,; elle n'a jamais été en usage et on n'en a ouï
parler que pour faire ce conte et la réponse d'Henri IV, qui peut être
plaisante, mais qui, outre qu'elle n'a pu être faite sur une formule
imaginaire qui n'a jamais été prononcée, serait trop cruelle aussi pour
être vraisemblable. De tout cela il résulte que sous Henri III ni sous
Henri IV nul usage de ces nominations, et que, si le comte de Soissons a
fait faire La Frette chevalier de l'ordre, ça été faveur et grâce
accordée à sa prière, et rien moins qu'un exercice et un droit qu'il
n'eut et ne prétendit jamais.

Louis XIII n'a fait que deux grandes promotions\,; l'une en 1619,
l'autre en 1633\,; le peu d'autres n'ont été que d'un chevalier à la
fois. En 1619 on n'en voit aucun pour Gaston, duc d'Orléans, son
frère\,; mais le père du maréchal de Rochefort, chambellan du prince de
Condé, qui des cinquante-neuf de la promotion fut le
cinquante-troisième\,; le baron de Termes, grand écuyer de France en
survivance de son frère, peut-être même en titre, car il y fut un
moment, et lorsqu'il fut tué devant Clérac, en 1621, la charge de grand
écuyer fut rendue à son frère\,; le baron de Termes, dis-je, le suivit
immédiatement\,; Hercule de Rohan, marquis de Marigny, puis de
Rochefort, frère de père et de mère du duc de Montbazon, vint après\,;
puis le comte de La Rocheguyon\,; Silly, qui fut ensuite duc à brevet\,;
le marquis de Portes vice-amiral, père de la première femme de mon
père\,; le comte de La Rochefoucauld, qui devint après le premier duc et
pair de sa maison\,; et le dernier marquis d'Étampes, grand maréchal des
logis de la maison du roi. Le roi aurait-il fait un chevalier de l'ordre
pour M. le Prince sans en donner un à Monsieur\,? Mais c'était le temps
des troubles et de l'évasion de la reine mère du château de Blois, où
elle avait été envoyée après la mort du maréchal d'Ancre. Cela
n'empêchait pas le droit de Monsieur, s'il en avait eu, et qui aurait vu
avec un juste dépit M. le Prince exercer le sien tandis que le sien à
lui demeurait inutile. Il n'est donc pas possible d'admettre le marquis
de Rochefort dans cette promotion, et au rang qu'il y tint, comme de la
nomination du prince de Condé. En celle de 1633 on ne voit en
quarante-trois chevaliers aucun pour Monsieur, qui alors était hors du
royaume, ni pour M. le prince de Condé\,; jusqu'ici donc nul usage de ce
prétendu droit.

Louis XIV n'a fait que deux grandes promotions, en 1661 et en 1688\,;
toutes les autres n'ont été que par occasions particulières de deux,
trois, rarement quatre à la fois, excepté celle de tous les maréchaux de
France qui ne l'étaient pas. C'est donc en ces deux grandes promotions
qu'il faut mettre l'époque du premier usage de ce prétendu droit,
c'est-à-dire après trois rois grands maîtres, après un grand nombre de
promotions, après quatre-vingt-deux ans de l'institution de l'ordre. Il
est vrai qu'en 1661, où la promotion fut de cinquante-trois chevaliers,
Monsieur eut deux chevaliers, les comtes de Clère et de Vaillac,
capitaine de ses gardes, qui se suivirent l'un l'autre immédiatement, et
le furent de quatre autres qui fermèrent la promotion, dont le dernier
fut Guitaut, premier gentilhomme de la chambre de M. le Prince. Mais ou
Monsieur n'en eut qu'un, ou bien Madame n'en eut point. On répète que
c'est le premier exemple, on va voir que Monsieur ne s'en tint pas là.
En 1688, où la promotion fut de soixante et dix, M. de La Vieuville, duc
à brevet et gouverneur de M. le duc de Chartres, ne le fut point sur le
compte de Monsieur, ni de M. le duc de Chartres, mais sur le compte du
roi, ce qui n'a jamais été mis en doute, et le marquis d'Arcy, aussi de
cette promotion, qui ne fut qu'après gouverneur du même prince, n'a pu
être mis sur le compte du Palais-Royal\,; mais Monsieur en eut deux,
Madame un et en fit passer un quatrième sur le compte de M. le duc de
Chartres, comme premier prince du sang, quoique petit-fils de France,
avec un rang fort supérieur à celui des princes du sang\,: c'était la
promotion de promesse d'avance du mariage de M. le duc de Chartres, dont
le chevalier de Lorraine avait répondu au roi, comme on le voit au
commencement de ces Mémoires, qui en eut la préséance sur les ducs. Il
fallait donc avoir aussi de la complaisance pour Monsieur, sans lui
montrer pourquoi, et distinguer le marquis d'Effiat, le
compersonnier\footnote{Vieux mot qui signifie \emph{associé}. Il
  s'appliquait surtout aux gens de mainmorte qui, daus la Bourgogne, le
  Nivernais, etc., mettaient leurs biens en commun.} du chevalier de
Lorraine, dans ce marché de la personne de M. de Chartres\,; ainsi
d'Effiat, quoique de la naissance qu'on n'ignorait pas, et le marquis de
Châtillon furent nommés par Monsieur. D'Effiat fut le
cinquante-troisième, et Châtillon le soixante-quatrième. D'Étampes, qui
prétendait l'emporter sur Châtillon, attendit Monsieur dans sa
garde-robe, caché, et quand Monsieur y fut entré, il lui dit mots
nouveaux sur son affection pour Châtillon, jusqu'à oser mettre l'épée à
la main et menacer Monsieur de courre sus à Châtillon partout.

Monsieur, qui craignait un scandale étrange et dont les suites pouvaient
être fâcheuses à son goût, fit tout ce qu'il put pour apaiser
d'Étampes\,; voyant enfin qu'il n'en pouvait venir à bout, d'Étampes
résolu à l'éclat le plus grand ou à être certain de l'ordre avant de
sortir ou de laisser sortir Monsieur de cette garde-robe, il lui en
renouvela parole, et, comme que ce fût, il l'assura qu'il le serait, le
fit nommer par M. le duc de Chartres, et c'est de ce prince que j'en
tiens l'histoire. D'Étampes fut le soixante-huitième, et précéda
immédiatement La Rongère, chevalier d'honneur de Madame, qu'elle nomma.
Lussan le suivit immédiatement, et fut le dernier de la promotion, non
pour M. le Prince ni de droit, mais par la prière de M. le Prince,
convenu qu'il n'avait nul droit, comme il est raconté.

Voilà donc le premier exemple en faveur des fils et filles de France et
du premier prince du sang. Il n'est pas étrange que M. le Duc, premier
ministre tout-puissant sous la jeunesse du roi, qui attenta le premier à
faire manger ses domestiques avec le monarque, et à les faire entrer
dans ses carrosses, se soit avantagé de l'exemple de 1688, pour la
promotion qu'il fit signer toute faite au roi, en 1724, et où il fourra
le chien, le chat et le rat. Il profita du nom de Tavannes et de sa
charge de lieutenant général de la plus considérable partie de la
Bourgogne, et qui était gentilhomme de sa chambre, titre nouveau pour
qui n'est pas premier prince du sang, et le mit le quarante-sixième de
cette promotion, disant même qu'il n'avait pas voulu {[}le{]} mettre le
dernier, comme s'il eût été de sa nomination. Il admit Simiane en
quarante-huitième, comme ayant parole à la nomination de feu M. le duc
d'Orléans, dont il était premier gentilhomme de la chambre, quoique sans
droit par la mort de ce prince\,; car cela fut dit ainsi, après force
allées et venues de la part de M. le duc d'Orléans d'aujourd'hui,
quoique fort mal ensemble. M. de Castries, chevalier d'honneur de
M\textsuperscript{me} la duchesse d'Orléans, veuve du régent, eut sa
nomination (et c'est l'unique d'une petite-fille de France), fut le
quarante-neuvième\,; et Clermont-Gallerande, premier écuyer de M. le duc
d'Orléans, premier prince du sang, ayant sa nomination, fut le
cinquantième et dernier. De ce détail, qui est exact, on peut juger de
la valeur de la prétention de nommer au roi des sujets pour les faire
chevaliers de l'ordre, de celle de l'extension de cette prétention, et
de celle encore tout idéale d'en prétendre en toute promotion qui passe
le nombre de huit chevaliers. On jugera aussi du nombre de ces
nominations qui, en promotions peu nombreuses et redoublées, égalerait
bientôt la nomination du roi, et rendrait l'ordre bien moins certain
auprès du roi qu'au service de ces princes.

\hypertarget{chapitre-viii.}{%
\chapter{CHAPITRE VIII.}\label{chapitre-viii.}}

1712

~

{\textsc{Arras bombardé par les ennemis.}} {\textsc{- L'Écluse emporté
par Broglio.}} {\textsc{- Ducasse arrive avec les galions.}} {\textsc{-
Son extraction, sa fortune, son mérite\,; est fait chevalier de la
Toison.}} {\textsc{- Mort et caractère du comte de Brionne.}} {\textsc{-
Monterey et Los Balbazes\,; quels\,; se font prêtres.}} {\textsc{-
Raison ordinaire de cette dévotion en Espagne.}} {\textsc{- Altesse
accordée en Espagne et â la princesse des Ursins et au duc de Vendôme,
avec les traitements à ce dernier des deux don Juan.}} {\textsc{-
Explication de ces traitements et de l'éclat qu'ils firent.}} {\textsc{-
Le roi à Marly, où il rétablit le jeu et la vie ordinaire avant
l'enterrement du Dauphin et de la Dauphine.}} {\textsc{- Lœwenstein fait
prince de l'empire.}} {\textsc{- Abbé de Vassé\,; son caractère\,;
refuse l'évêché du Mans.}} {\textsc{- Le roi d'Angleterre a la petite
vérole à Saint-Germain\,; répudie son confesseur jésuite.}} {\textsc{-
Mort de la princesse d'Angleterre à Saint-Germain.}} {\textsc{- Mort et
caractère de M\textsuperscript{lle} d'Armentières.}} {\textsc{- Sa
famille, sa fortune, sa maison.}} {\textsc{- Mort de
M\textsuperscript{me} de Villacerf, douairière.}} {\textsc{- Courageuse
opération de M\textsuperscript{me} Bouchu.}} {\textsc{- Mort, caractère
et famille de la marquise d'Huxelles.}} {\textsc{- Mort et caractère du
bailli de Noailles.}} {\textsc{- Le roi nomme le P. La Rue confesseur de
M. le duc de Berry, et retient le P. Martineau pour le petit Dauphin.}}
{\textsc{- Mémoire publié du Dauphin sur l'affaire du cardinal de
Noailles.}} {\textsc{- Service et enterrement du Dauphin et de la
Dauphine à Saint-Denis.}} {\textsc{- Queues étranges.}} {\textsc{- Bout
de l'an de Monseigneur à Saint-Denis.}} {\textsc{- Service à Notre-Dame
pour le Dauphin et la Dauphine.}} {\textsc{- Le clergé y obtient le
premier salut séparément de celui de l'autel.}} {\textsc{- Violet des
cardinaux.}} {\textsc{- Le cardinal de Noailles mange avec
M\textsuperscript{me} la duchesse de Berry.}} {\textsc{- Service à la
Sainte-Chapelle, où le P. La Rue fait l'oraison funèbre.}} {\textsc{- Je
vais passer un mois ou cinq semaines à la Ferté.}} {\textsc{- Causes de
ce voyage.}} {\textsc{- Chalais vient d'Espagne arrêter un cordelier en
Poitou\,; ce qu'il devient.}} {\textsc{- Renouvellement d'horreurs sur
M. le duc d'Orléans.}} {\textsc{- Adresse d'Argenson à son égard.}}
{\textsc{- M\textsuperscript{me} de Gesvres demande juridiquement la
cassation de son mariage pour cause d'impuissance.}} {\textsc{- Départ
des généraux\,: Villars en Flandre, Harcourt et Besons sur le Rhin,
Berwick aux Alpes, Fiennes en Catalogne.}} {\textsc{- Mariage de Bissy
avee M\textsuperscript{lle} Chauvelin.}} {\textsc{- Mariage de Meuse
avec M\textsuperscript{lle} de Zurlauben.}} {\textsc{- Mort, extraction,
caractère de l'abbé de Sainte-Croix.}} {\textsc{- Mort, famille et
caractère de Cominges, et sa dépouille.}} {\textsc{- Mort et caractère
de La Fare.}} {\textsc{- Mort du président Rouillé.}} {\textsc{- Mort de
l'abbé d'Uzès.}} {\textsc{- Rohan, évêque de Strasbourg, fait
cardinal.}} {\textsc{- Désordres de la Loire.}} {\textsc{- Duc de
Fronsac sort de la Bastille.}}

~

Il se passa deux bagatelles en Flandre dans le courant du mois de mars.
Les ennemis vinrent bombarder Arras pour brûler des amas de fourrages,
et ne causèrent presque aucun dommage. Le maréchal de Montesquiou apprit
qu'ils avaient mis huit cents hommes dans le bourg de l'Écluse. Broglio,
aujourd'hui maréchal de France et duc, eut ordre de les aller attaquer.
Il rencontra en chemin un parti de trois cents chevaux qui, à sa vue, se
retira sous le canon du château de l'Écluse. Il força les ennemis de se
retirer dans ce château, qu'il prit après avoir emporté ce bourg et les
retranchements\,; prit ou tua les huit cents hommes et les trois cents
chevaux. On était si peu accoutumé aux aventures heureuses qu'il fut
beaucoup parlé de celle-là.

Une beaucoup meilleure fut l'arrivée de Ducasse à la Corogne avec les
galions très-richement chargés qu'il était allé chercher en Amérique. On
les attendait depuis longtemps avec autant d'impatience que de crainte
des flottes ennemies dans le retour. Ce fut une grande ressource pour
l'Espagne qui en avait un extrême besoin, un grand coup pour le commerce
qui languissait, et où le désordre était près de se mettre, et un
extrême chagrin pour les Anglais et Hollandais qui les guettaient depuis
si longtemps avec tant de dépenses et de fatigues. Le duc de La
Rochefoucauld d'aujourd'hui, né quatrième cadet, qui portait lors le nom
de Durtal et qui était dans la marine, servait sur les vaisseaux de
Ducasse, qui l'envoya porter au roi cette grande nouvelle. Le roi
d'Espagne en fut si aise qu'il fit Ducasse chevalier de la Toison d'or,
au prodigieux scandale universel. Quelque service qu'il eût rendu, ce
n'était pas la récompense dont il dût être payé. Ducasse était connu
pour le fils d'un petit charcutier qui vendait des jambons à Bayonne. Il
était brave et bien fait\,; il se mit sur les bâtiments de Bayonne,
passa en Amérique, et s'y fit flibustier. Il y acquit des richesses et
une réputation qui le mit à la tête de ces aventuriers. On a vu en son
lieu combien il servit utilement à l'expédition de Carthagène, et les
démêlés qu'il eut avec Pointis qui la fit. Ducasse entra dans la marine
du roi où il ne se distingua pas moins. Il y devint lieutenant général,
et aurait été maréchal de France si son âge l'eût laissé vivre et
servir, mais il était parti de si loin qu'il était vieux lorsqu'il
arriva. C'était un des meilleurs citoyens et un des meilleurs et des
plus généreux hommes que j'aie connus, qui sans bassesse se
méconnaissait le moins, et duquel tout le monde faisait cas lorsque son
état et ses services l'eurent mis à portée de la cour et du monde.

Il mourut en ce même temps un homme de meilleure maison, mais d'un
mérite qui se serait borné aux jambons s'il fût né d'un père qui en eût
vendu. Ce fut le comte de Brionne, accablé d'une longue suite
d'apoplexies. Il était chevalier de l'ordre de 1688, et le premier
danseur de son temps, quoique médiocrement grand et assez gros. C'était
un assez honnête homme, mais si court et si plat que rien n'était
au-dessous. On ne le voyait jamais que dans les lieux publics de cour,
et chez lui {[}il{]} ne voyait personne\,; sa famille n'en faisait aucun
cas ni personne à la grande écurie. Son père, qui lui avait fait donner
autrefois ses survivances, l'avait comme forcé depuis deux ou trois mois
à s'en démettre, comme on l'a vu, de la charge pour son frère, de son
gouvernement pour son fils. M. le Grand, qui n'était pas tendre, disait
qu'il buvait tout son bon vin, et trouvait cela fort mauvais. Il n'eut
pas la peine d'avoir à s'en consoler.

Deux grands d'Espagne fort distingués se firent prêtres en ce
temps-ci\,: l'un fut le comte de Monterey, l'autre le marquis de Los
Balbazès. Monterey était second fils de don Louis de Ilaro y Guzman, qui
succéda à l'autorité, à la faveur et à la place de premier ministre du
comte-duc d'Olivarès, son oncle maternel\,; qui était grand écuyer de
Philippe IV, et qui traita et signa avec le cardinal Mazarin la paix des
Pyrénées et le mariage du roi, dans l'île des Faisans sur la rivière de
Bidassoa. Le marquisat et grandesse de Monterey passa en plusieurs
maisons par mariage d'héritières. La dernière était de la maison de
Tolède, qu'épousa le marquis de Monterey dont il s'agit ici, et qui en
prit le titre et fut par elle grand d'Espagne. Il fut gentilhomme de la
chambre, puis successivement vice-roi de Catalogne, gouverneur général
des Pays-Bas, du conseil de guerre, conseiller d'État (ce que nous
appelons ministre en France), président du conseil de Flandre, enfin
disgracié et chassé sous le ministère du duc de Medina-Celi, et n'eut
point d'enfants.

Los Balbazès fut érigé en marquisat en décembre 1621 pour le fameux
capitaine Ambroise Spinola de l'une des quatre premières maisons de
Gênes\,; un de ses fils fut cardinal, l'autre épousa une Doria de l'une
des quatre premières maisons de Gênes, qui était duchesse héritière del
Sesto, et eut la Toison. Son fils, gendre du connétable Colonne, fut
grand d'Espagne, du conseil de guerre, ambassadeur en France au mariage
du roi pour y accompagner la reine, conseiller d'État, c'est-à-dire
ministre, et majordome-major de la reine, seconde femme de Charles II.
Son fils, gendre du huitième duc de Medina-Celi, fut vice-roi de Sicile.
Il en partit pour venir à Gênes où il se fit prêtre. Son fils, gendre du
duc d'Albuquerque est grand écuyer de la princesse des Asturies, fille
du roi de Portugal, et a cinq sœurs toutes grandement mariées. Les
privilèges du clergé sont tels en Espagne qu'un particulier qui y entre
garantit sa famille de toutes recherches, parce que le droit de partage
qu'il conserve dans les biens en rend la discussion très-difficile et
presque toujours infructueuse\,; ils dérobent aussi à la justice
séculière les personnes du clergé, et rendent leurs punitions
impossibles. Ces considérations, beaucoup plus que la dévotion ni même
pour les grands seigneurs que l'ambition du cardinalat, y font entrer
ceux qui des grands emplois tombent en disgrâce, qui mettent ainsi leurs
biens à couvert et leurs personnes en sûreté.

L'Espagne avait ses Titans, sur le modèle de ceux de France, qui ne
gagnèrent pas moins que les nôtres à la mort du Dauphin. Ils se hâtèrent
encore plus d'en profiter. La princesse des Ursins, qui d'avance se
comptait déjà souveraine, eut impatience d'en faire sentir à l'Espagne
le poids, qui jusqu'alors lui était inconnu. Elle n'osa pourtant le
hasarder sans l'attache de la France, et elle n'ignorait pas le biais de
l'obtenir et de s'en faire soutenir dans son inouïe entreprise contre le
désespoir général qu'elle ne pouvait douter qu'elle n'allât exciter\,:
ce fut de rendre commun son intérêt avec celui du duc de Vendôme, et
d'acquérir pour une nouvelle grandeur l'appui certain et tout-puissant
de M\textsuperscript{me} de Maintenon et de M. du Maine. Sûre de ce
côté-là, elle obtint un ordre du roi d'Espagne aux grands, et par
conséquent à toute l'Espagne, de la traiter désormais d'Altesse et le
duc de Vendôme aussi, auquel on expédia une patente qui lui donnait tous
les rangs, honneurs et prérogatives dont avaient joui les deux don Juan.

Cette nouveauté fit en Espagne un éclat prodigieux, et y causa un dépit
et une consternation générale, dont il faut expliquer la raison. On a
vu, lorsqu'on a traité de la dignité des grands d'Espagne qu'elle va
d'égal avec tous les souverains non rois\,; qu'elle ne cède à pas un\,;
et que, si les ducs de Savoie, comme le fameux Charles-Emmanuel, ont eu
en Espagne quelque rare et très-légère préférence sur eux, elle a été
plutôt de distinction que de rang, et masquée de l'honneur de son
mariage avec l'infante qui à son tour était appelée à succéder à la
couronne\,; on ne parle point de ce qui s'y passa au voyage de Charles
I\^{}er, roi d'Angleterre, alors prince de Galles, parce que l'héritier
présomptif de la couronne de la Grande-Bretagne est hors de toute
parité. On n'a vu encore que, depuis la réunion des divers royaumes
d'Espagne par le mariage de Ferdinand et d'Isabelle, on n'a vu jusqu'à
Philippe V que deux fils d'Espagne, cadets, en âge d'homme\,: le frère
de Charles-Quint qui y fut régent en son absence, et qui passa depuis en
Allemagne où il fut roi et empereur et fonda la branche impériale
souveraine des États héréditaires d'Allemagne, et Ferdinand, fils de
Philippe III, né en 1609, cardinal et archevêque de Tolède sans avoir
été dans les ordres, et gouverneur des Pays-Bas, où il mourut en 1641
n'ayant pas trente-deux ans\,; ainsi nulle postérité et point de princes
de la maison d'Espagne. De bâtards reconnus, on n'y en a vu que deux,
tous deux du nom de don Juan d'Autriche, et tous deux personnages
surtout le premier fils de Charles-Quint, né d'une mère inconnue en
1543, célèbre par le gain de la bataille de Lépante, et qui commanda
presque toujours en chef les armées de terre et de mer\,; il mourut sans
alliance, en 1578, à trente-cinq ans\,; l'autre don Juan, fils de
Philippe IV, né d'une comédienne en 1629, mort sans alliance en 1579, à
cinquante ans, grand prieur de Castille, dignité qui donne la grandesse
et cent mille écus de rente, et général des armées d'Espagne.

Philippe IV étant mort en septembre 1665, et la reine sa veuve devenue
régente pendant la minorité de Charles II qui fut longue, don Juan fit
un parti contre elle qui, après une longue lutte lui arracha toute
l'autorité que don Juan exerça tout entière, et se fit grandement
compter jusqu'à sa mort. Il eut une espèce de maison, usurpa comme chef
de parti une grande supériorité sur les grands, et eut l'Altesse, à
quoi, outre la nécessité des temps, ils se ployèrent plus facilement à
cause de l'état des bâtards qui est particulier en Espagne, où s'est
conservé ce reste des mœurs et des coutumes mauresques. On a
vu\footnote{Voy. t. III, p.~249 et suiv., les passages auxquels renvoie
  Saint-Simon et que les précédents éditeurs avaient rejetés au XIX\^{}e
  volume\,; ce qui les avait forcés de modifier toutes ces phrases.},
lorsqu'on a parlé des grands et de l'Espagne, que les bâtards de gens
non mariés héritent à peu de chose près comme les enfants légitimes, à
leur défaut, et deviennent même grands d'Espagne par succession. Il y
faut garder pourtant certaines formalités faciles, et qu'il n'y ait
point d'obstacles de famille qui leur préfèrent les oncles, tantes, ou
cousins germains légitimes. Enfin cette première espèce de bâtards
diffère en Espagne fort peu des enfants légitimes. Les bâtards d'un
homme marié et d'une fille ne diffèrent des premiers que par plus de
formalités et de restrictions\,; mais ils succèdent aussi et héritent
des grandesses. Don Juan était de cette seconde sorte\,; ainsi son droit
de succession à la couronne lui facilita l'Altesse, la supériorité de
rang et tout ce qu'il voulut entreprendre, et qu'il soutint par les
troubles dont il fut toujours l'âme et le chef, et par toute l'autorité
et la réputation qui lui en demeurent après. Ce fut donc sur ce modèle
que M\textsuperscript{me} des Ursins voulut élever le duc de Vendôme, en
faire sa cour à M. du Maine, par un exemple pour lui en France, quoique
si différente de l'Espagne sur l'état des bâtards, plaire au roi et à
M\textsuperscript{me} de Maintenon par leur endroit le plus sensible, et
à l'appui de l'Altesse de M. de Vendôme faire passer la sienne, après
quoi elle n'était pas en peine d'arrêter les autres avantages que
Vendôme eût pu prétendre à l'exemple de don Juan, sous prétexte de ne
pas pousser à bout le mécontentement général. Il fut extrême.

On avait perdu don Juan de vue en Espagne\,; il était mort retiré dans
une commanderie, quelques années avant sa mort. Personne ne se souvenait
de l'avoir vu ni de son Altesse. M. de Vendôme n'était point bâtard de
leur dernier roi, il n'avait aucun droit à la couronne d'Espagne\,;
nulle parité donc avec don Juan. On le voyait traiter d'Altesse, lui et
M\textsuperscript{me} des Ursins, précisément comme les enfants actuels,
parce qu'en Espagne on ne connaît d'Altesse Sérénissime ni Royale pour
qui que ce soit sans aucune exception, et cette égalité de traitement
avec le prince des Asturies et les autres enfants était insupportable en
M\textsuperscript{me} des Ursins et M. de Vendôme, et bannit
volontairement beaucoup de gens de la cour et du service, pour éviter la
nécessité de la leur donner. On n'en fut pas moins indigné en France, où
M\textsuperscript{me} de Maintenon et M. du Maine ravis n'osèrent le
marquer. Le roi même fut très-sobre à en parler. Ils surent bien y
suppléer par les réflexions utiles du fruit à en tirer.

Le roi alla le mercredi 6 avril à Marly, où, quoique le Dauphin et la
Dauphine ne fussent pas encore enterrés, il rétablit son petit jeu chez
M\textsuperscript{me} de Maintenon dès le vendredi suivant, et voulut le
salon à l'ordinaire, et que M. {[}le duc{]} et madame la duchesse de
Berry y tinssent le lansquenet public et le brelan, et des tables de
différents jeux pour toute la cour. Il ne fut pas longtemps sans dîner
chez M\textsuperscript{me} de Maintenon une ou deux fois la semaine, et
à y entendre de la musique avec les mêmes dames familières.
M\textsuperscript{me} de Dangeau, qui en était une, eut la joie d'y
apprendre que le comte de Lœwenstein son frère, qui pendant l'occupation
de la Bavière, en était administrateur pour l'empereur, avait été fait
prince de l'empire. On y sut en même temps que l'abbé de Vassé refusait
l'évêché du Mans. C'était un grand homme de bien depuis toute sa vie,
qui ne s'était jamais soucié que de l'être, mais qui ne laissait pas de
voir bonne compagnie, et d'en être fort considéré\,: il avait plus de
soixante ans, et ne put être tenté de l'épiscopat à cet âge, quoique
placé au milieu des terres de sa maison. Je n'ai pas voulu omettre ce
refus pour la rareté dont il est, et pour celle encore d'avoir choisi un
homme de qualité et de ce mérite. C'était un phénomène pour le P.
Tellier.

Le roi d'Angleterre eut la petite vérole à Saint-Germain. On lui fit
recevoir les sacrements. On ne sait par quelle raison il fit comme
M\textsuperscript{me} la Dauphine, et ne voulut point de son confesseur
jésuite\,; il envoya chercher le cure de la paroisse, à qui il se
confessa. La reine sa mère s'enferma avec lui, et prit toutes les
précautions possibles pour séparer la princesse sa fille du mauvais air.
Elles furent inutiles\,; la petite vérole la prit, elle en mourut le
septième jour, qui fut le lundi 18 avril. Ce fut une grande affliction
pour la reine d'Angleterre, avec la triste perspective de sa séparation
prochaine d'avec le roi son fils par la nécessité de la paix et de
l'embarras de ce qu'il allait devenir. Le corps de la princesse
d'Angleterre fut porté sans cérémonie aux Filles Sainte-Marie de
Chaillot, où la reine sa mère se retirait souvent. La raison de la
petite vérole l'empêcha de recevoir aucunes visites.

M\textsuperscript{lle} d'Armentières mourut à Paris à plus de
quatre-vingts ans. C'était une fille de beaucoup de mérite, d'esprit et
de vertu, qui avait été longtemps fort pauvre, qui devint après fort
riche, et qui dans ces deux états eut quantité d'amis et d'amies
considérables. Elle avait été recueillie jeune et pauvre chez la
duchesse d'Orval, sœur de Palaiseau, chez qui elle logea la plus grande
partie de sa vie, et à qui à son tour elle fut fort utile quand elle la
vit tombée dans la pauvreté. Elles ne laissèrent pas de se séparer
d'habitation sur la fin, comme Saint-Romain et Courtin, deux conseillers
d'État fort connus par leurs ambassades, dont il a été quelquefois
mention, et qui avaient toujours logé ensemble par amitié.
M\textsuperscript{lle} d'Armentières laissa quatre mille livres de
pension à la duchesse d'Orval, l'usufruit de son bien à la duchesse du
Lude, son amie intime de tout temps, et le fonds à M. d'Armentières, son
plus proche parent, et l'aîné de sa maison. Sa mère n'était rien, son
père parut peu, quoique gouverneur de Saint-Quentin et avec un
régiment\,; mais le père de celui-là, aussi gouverneur de Saint-Quentin,
fut lieutenant général, député de la noblesse pour le bailliage de
Vermandois aux derniers états de Blois en 1588, ambassadeur vers les
archiducs en Flandre, chevalier du Saint-Esprit en 1597, et chevalier
d'honneur de la reine Marie de Médicis. Il eut la terre d'Armentières de
sa femme, qui était Jouvenel avec le sobriquet de des Ursins et
héritière\,; et le père de celui-là était ce vicomte d'Auchy, capitaine
des gardes du corps de Charles IX, qui garda le roi de Navarre à
Vincennes, qui y acquit son amitié, et que les Mémoires de Castelnau
appellent froid et sage, et l'un des plus hommes de bien de son temps.

M\textsuperscript{lle} d'Armentières n'avait qu'un frère dont la mère
était Pinart, héritière de grandes terres, entre autres de Louvois, et
dont le père était ce vicomte de Comblisy, fils du secrétaire d'État qui
trahit Henri IV et rendit à la Ligue Château-Thierry, dont il était
gouverneur, et qui était alors une place importante. Son petit-fils par
sa fille dissipa tout dans une vie obscure et inconnue, épousa une
gueuse des rues dont il n'eut point d'enfants, et mourut en 1604. Les
restes ne laissèrent pas d'être encore bons. M\textsuperscript{lle}
d'Armentières les recueillit, paya, s'arrangea, et devint riche, dans sa
vieillesse, dont elle sut faire un bon et honnête usage. Elle et le père
de celui à qui elle laissa le fonds de ses biens étaient enfants des
issus de germains. La branche de celui-là, distinguée par le nom de
Saint-Remi, était depuis longtemps dans l'indigence. Le père de ceux qui
se relevèrent et qui ont figuré pendant la régence de M. le duc
d'Orléans devint l'aîné de sa maison en 1690, par la mort de tout ce
qu'il restait d'aînés de toutes les branches, et n'en fut pas plus à son
aise. Il avait épousé la fille de d'Aguesseau, maître des comptes, dont
il eut la petite terre de Puysieux près de Beaumont vers Beauvais\,; et
ce maître des comptes, fort nouveau alors, est le grand-père de
d'Aguesseau, chancelier de France avec diverses fortunes, depuis le 2
février 1717, et {[}qui{]} l'est encore depuis vingt-six ans. Saint-Remi
mourut en 1712, à soixante-dix-neuf ans, et sa femme, en 1721, ayant eu
la joie de voir la fortune de son neveu, mais sans être jamais sortis de
leur village ni l'un ni l'autre, où leur maison ressemblait fort à une
hutte, et où ils avaient peine à subsister. Ils eurent trois fils\,:
l'aîné porta le nom d'Armentières\,; le second fut envoyé sans sou ni
maille page du grand maître à Malte\,; le troisième porta le nom de
Conflans, qui est celui de leur maison. Les deux aînés naquirent avec
beaucoup d'esprit et d'envie de faire. Ils se roidirent contre la
fortune, et, malgré leur pauvreté, ils trouvèrent le moyen de lire, de
s'instruire et de s'orner l'esprit de sciences et d'histoire, aidés tous
deux d'une fort belle mémoire, et assez avisés pour vivre tous trois
dans la plus grande union. Conflans et Armentières servirent.

Conflans, qui n'avait pas le sens commun, perdit sa jeunesse dans une
citadelle où il fut enfermé près de vingt ans, pour s'être battu contre
le fils unique de Pertuis, mort gouverneur de Courtrai, après avoir été
capitaine des gardes de M. de Turenne, et fort estimé. Le chevalier de
Conflans, revenu de ses caravanes, se battit en Angoumois, près de
Ruffec, avec un gentilhomme nommé Ponthieu, à coups de pistolet, et en
perdit le bras droit. Armentières se trouva dans un régiment employé à
la Rochelle, où le maréchal de Chamilly commandait. La maréchale, qui
avait beaucoup d'esprit et qui était la piété et la vertu même, trouva
de l'esprit et du savoir à d'Armentières. Ravie de rencontrer quelqu'un
à qui parler, elle s'en accommoda, mit le chevalier dans leurs milices,
les aida de tout ce qu'elle put\,; de retour les hivers à Paris, les y
fit venir, les vanta, les produisit chez elle à la meilleure compagnie
qui y était toujours, et les mit ainsi dans le monde\,; eux en surent
profiter et se faire connaître ailleurs.

Je ne sais comment Armentières fit connaissance, puis amitié, avec
M\textsuperscript{me} d'Argenton. M. le duc d'Orléans y soupait tous les
soirs quand il était à Paris, ses sociétés y étaient assez étranges, et
quoique chez sa maitresse, il ne laissait pas d'être difficile à amuser.
L'esprit fort orné d'Armentières, et sa religion à peu près de la trempe
de celle de M. le duc d'Orléans, firent juger à M\textsuperscript{me}
d'Argenton qu'il lui serait d'usage à amuser M. le duc d'Orléans. Elle
lui en parla comme de son ami dont il s'accommoderait\,; elle le lui
présenta\,; il fut de tous les soupers, et M. le duc d'Orléans le goûta.
Cela dura du temps, pendant lequel Armentières, qui cherchait à
s'accrocher, fit des connaissances au Palais-Royal, s'introduisit chez
M\textsuperscript{me} de Jussac, dans les temps qu'elle venait à Paris.

Cette M\textsuperscript{me} de Jussac, étant fille, avait été demoiselle
de la première femme de mon père, qui la donna par confiance à sa fille,
lorsqu'elle la maria au duc de Brissac, et elle ne l'a jamais oublié.
Elle passa de là à M\textsuperscript{me} de Montespan où elle vit le
grand monde et la plus fine compagnie. C'était une personne bien faite,
de bonne mine, qui n'avait pas été sans beauté, mais qui avec de
l'esprit avait encore plus de vertu et de sagesse, et qui avec une
grande douceur et beaucoup de circonspection se fit généralement aimer
et estimer. La confiance qu'on prit en elle lui fit donner le soin de
l'éducation de M\textsuperscript{lle} de Blois. Elle y fut continuée
après la retraite de M\textsuperscript{me} de Montespan, et le roi l'y
attacha de nouveau sans titre, lorsqu'il maria M\textsuperscript{lle} de
Blois à M. le duc d'Orléans, qu'elle suivait même au défaut de ses deux
dames. Elle avait épousé, chez M\textsuperscript{me} de Montespan,
Jussac qui était à M. du Maine sur le pied nouveau de premier
gentilhomme de sa chambre, qui fut tué au combat de Leuse, et qui lui
laissa un fils, tué aussi depuis dans la gendarmerie tout jeune, et deux
filles. M\textsuperscript{me} d'Orléans l'aima toujours tendrement. Sans
rien perdre de l'attachement le plus marqué pour M\textsuperscript{me}
de Montespan jusqu'à sa mort, ni de sa confiance, elle sut s'attirer
celle du roi et de M\textsuperscript{me} de Maintenon, sur ce qui
regardait M\textsuperscript{me} la duchesse d'Orléans, beaucoup d'amis
et de considération dans le monde.

Elle avait marié sa fille aînée à Chaumont, colonel d'infanterie, dont
le nom était d'Ambly, qui fut tué brigadier, sans enfants. Armentières,
qui tenait M. le duc d'Orléans par M\textsuperscript{me} d'Argenton,
crut ne pouvoir mieux faire que de s'assurer aussi M\textsuperscript{me}
la duchesse d'Orléans à cause de la cour et du service. Il songea donc à
épouser la seconde fille de M\textsuperscript{me} de Jussac, fort jolie,
et qui, avec moins d'esprit que la mère, mais un esprit de sagesse et de
conduite, lui ressemblent dans tous les points. Il tourna si bien qu'en
1709, tout au commencement de l'année, le mariage se fit par le concours
fort rare de l'épouse et de la maîtresse. Il en eut une charge de
chambellan de M. le duc d'Orléans, qu'il lui paya, et un régiment
d'infanterie avec des pensions. II avait déjà produit ses frères, et il
attrapa bientôt après une charge de chambellan pour l'aîné, qui, au
commencement de cette année-ci, épousa la fille aînée de
M\textsuperscript{me} de Jussac, veuve de Chaumont. Dans la suite ils
furent l'un après l'autre premiers gentilshommes de la chambre de M. le
duc d'Orléans, un peu avant et pendant la régence, et après leur mort à
tous deux, le chevalier leur frère leur succéda, qui, à la
recommandation de M. le duc d'Orléans, eut la commanderie de Pézenas
avec une autre. M\textsuperscript{me} de Conflans fut gouvernante des
deux dernières filles de M. le duc d'Orléans, se ruina au jeu, devint
aveugle, éleva son fils de façon qu'il ne fut qu'un garnement, et qu'il
passa enfin en Espagne. M\textsuperscript{me} d'Armentières fut dame de
M\textsuperscript{me} la duchesse de Berry, et de M\textsuperscript{me}
la duchesse d'Orléans ensuite, fit sagement une bonne maison, se fit
aimer et estimer, éleva bien son fils qui épousa la fille unique
d'Aubigny, ce fameux écuyer de M\textsuperscript{me} des Ursins, dont
j'ai parlé plus d'une fois, à qui il avait laissé beaucoup de biens, et
ce superbe lieu de Chanteloup, destiné par sa maîtresse à tenir sa cour
lorsqu'elle serait souveraine. Ce dernier Armentières est maréchal de
camp, et, avec peu d'esprit, songe fort à pousser sa fortune. Le
chevalier de Conflans, demeuré premier gentilhomme de la chambre de M.
le duc d'Orléans après la mort du régent son père, lui fut donné par
M\textsuperscript{me} la duchesse d'Orléans pour être son mentor.

Avec plus d'éducation et moins de naturel rustre, il en eût été fort
capable, mais un plus capable que lui n'y aurait pas réussi. Il eut de
fortes prises avec le bâtard du feu régent et de M\textsuperscript{me}
d'Argenton pour des droits qu'il prétendait comme grand prieur de
France, qui furent poussés loin, et qui ne réussirent ni à l'un ni à
l'autre. Tous deux répandirent des factums. M. le duc d'Orléans à la fin
les fit taire, et les remit en quelque bienséance, en sorte que le
bailli de Conflans résolut de ne se mêler plus de ce qui regarderait le
grand prieur. Il ne se tint pas parole à lui-même. Il fut à un chapitre
au temple\,; le grand prieur y présidait\,; le bailli de Conflans se
prit de bec contre lui\,; de part et d'autre la dispute s'échauffa au
point que l'un vint aux reproches, l'autre aux invectives, jusqu'à
insulter à sa bâtardise avec les plus gros mots. Le chapitre en fut
rompu, et l'éclat en fut si grand dans le monde, que le grand prieur
appelé par Conflans, neveu du bailli, et en sa place, parce qu'il était
manchot, se battit avec lui et fut dangereusement blessé. De cette
affaire, le bailli de Conflans fut congédié doucement par M. le duc
d'Orléans, et perdit beaucoup de la considération qu'il avait acquise
dans le monde, qui se choqua du peu d'égard, et encore pour des choses
de Malte que d'autres auraient plus décemment défendues, à la
reconnaissance que lui et les siens devaient de toute leur fortune au
père et à la mère du grand prieur. Il se retira chez
M\textsuperscript{me} d'Armentières, sa belle-sœur, en même temps
extrêmement du grand monde, et y vit dans la dévotion. Ces Conflans se
prétendent issus de mâles en mâles de la maison de Brienne si connue par
son antiquité, ses grands fiefs, ses grandes alliances, ses grands
emplois, ses connétables, ses chambriers\footnote{La charge des
  chambriers était a peu près la même que celle des chambellans\,; ils
  avaient en outre la garde du trésor royal.}, et par des rois de
Jérusalem et des empereurs de Constantinople, et ils sont donnés comme
tels dans la généalogie de cette maison, donnée parmi celle des
connétables par les continuateurs de du Fourny et du P. Anselme.

M\textsuperscript{me} de Villacerf, veuve de Villacerf qui avait eu les
bâtiments, et {[}qui avait{]} été si bien avec le roi, et mère du
premier maître d'hôtel de la Dauphine qu'on venait de perdre, mourut
fort vieille d'une saignée qui lui fut faite pour quelques légers accès
de fièvre, où on lui coupa le tendon.

M\textsuperscript{me} Bouchu, veuve du conseiller d'Élat et mère de la
comtesse de Tessé, fut plus heureuse. Elle cachait un cancer depuis
longtemps, dont une seule femme de chambre avait la confidence. Avec le
même secret elle mit ordre à ses affaires, soupa en compagnie, se fit
abattre le sein le lendemain de grand matin, et ne le laissa apprendre à
sa famille ni à personne que quelques heures après l'opération\,; elle
guérit parfaitement. Après tant de courage et de sagesse, {[}on la
vit{]}, pas longues années après, épouser le duc de Châtillon
cul-de-jatte, pour la rage d'être duchesse, pour ses grands biens, et
longtemps après mourir d'une fluxion de poitrine pour avoir voulu aller
jouir de son tabouret à Versailles par le grand froid.

La marquise d'Huxelles, mère du maréchal, mourut en ce même temps à
quatre-vingt-cinq ou six ans, avec la tête entière et la santé parfaite
jusqu'alors. Elle était fille du président Le Bailleul, surintendant des
finances\,; son père, son frère, son neveu et son petit-neveu, tous
présidents à mortier\,; et veuve en premières noces du frère aîné de
Nangis, père du maréchal de Nangis, dont elle a touché soixante et dix
ans durant six mille livres de douaire. C'était une femme de beaucoup
d'esprit, qui avait eu de la beauté et de la galanterie, qui savait et
qui avait été du grand monde toute sa vie, mais point de la cour. Elle
était impérieuse et s'était acquis un droit d'autorité. Des gens
d'esprit et de lettres, et des vieillards de l'ancienne cour,
s'assemblaient chez elle, où elle soutenait une sorte de tribunal fort
décisif. Elle conserva des amis et de la considération jusqu'au bout\,;
son fils, qu'elle traita toujours avee hauteur ne fut jamais trop bien
avec elle, et ne la voyait guère.

Le bailli de Noailles mourut aussi à Paris, à l'archevêché, où le
cardinal son frère l'avait retiré depuis quelque temps, que ses affaires
se trouvaient fort délabrées. Il avait deux belles commanderies, et il
était ambassadeur de Malte. C'était un très-bon homme et honnête homme,
tout uni, qui avait été fort libertin toute sa vie, et qui à la fin
pensait à son salut.

Le P. Tellier jugea que le P. La Rue avait besoin de quelque marque de
considération après ce qui lui était arrivé à la mort de la Dauphine. Le
roi le nomma donc confesseur de M. le duc de Berry, et déclara qu'il
réservait pour le petit Dauphin le P. Martineau, qui l'était de celui
dont la perte affligeait toute l'Europe. En même temps ces pères,
accoutumés à tirer parti de tout, firent grand bruit d'un mémoire trouvé
dans les papiers du Dauphin sur l'affaire du cardinal de Noailles, qui
ne lui était rien moins que favorable. Ils l'envoyèrent à Rome et le
firent imprimer. Ce mémoire au moins ne fut pas trouvé dans sa cassette,
à ce qu'on a pu voir plus haut\,; il put l'être ailleurs\,; c'est ce qui
ne peut se discuter avec exactitude. Je puis hardiment protester de la
mienne sur les sentiments de ce prince que j'ai rapportés et sur ce qui
s'est passé de lui à moi et encore si peu de jours avant la mort de
M\textsuperscript{me} la Dauphine\,; et c'est-à-dire avant la sienne. Ce
mémoire, s'il est tel qu'on l'a publié, a pu être des commencements de
l'affaire, dans l'esprit de M. de Cambrai et dans les préjugés des ducs
de Chevreuse et de Beauvilliers. Il a pu jeter sur le papier le pour en
attendant le contre\,; on a donné ce pour, et si le contre s'est trouvé
il a été bien supprimé. Ce qui me fait en juger ainsi est la différence
entière de ce mémoire avec les sentiments dans lesquels je ne puis
douter que ce prince soit mort, et qu'il était très-incapable de me
vouloir tromper ni personne en me mentant sans aucune raison ni besoin,
et voulant se servir de moi dans cette même affaire, où il aurait été
étrangement peu d'accord avec soi-même, ce qui était radicalement opposé
à son caractère. La cabale ennemie du cardinal de Noailles ne laissa pas
de triompher, armée de ce grand nom, mais ce triomphe, bâti sur un
fondement si peu solide par le tissu même de l'écrit tel qu'ils le
publièrent, ne fut pas de longue durée. Il tomba bientôt de lui-même,
mais c'en fut toujours assez pour éblouir et pour gagner du temps.

On fit à Saint-Denis, le lundi 18 avril, le service et l'enterrement des
deux Dauphins et de la Dauphine, épouse et mère. M. le duc d'Orléans et
M. le comte de Charolais y furent les princes du deuil. Il fut singulier
qu'il n'y en eut pas un troisième. Le roi qui avait envoyé le comte de
Toulouse à l'eau bénite, et le duc du Maine au convoi, comme princes du
sang, trouva apparemment trop fort d'y faire figurer un d'eux à
Saint-Denis. Il y eut pourtant trois princesses du deuil, parce que la
cérémonie était double pour prince et pour princesse\,:
M\textsuperscript{me} la duchesse de Berry, menée par Coettenfao, son
chevalier d'honneur, sa queue portée par le comte de Roucy, Biron et
Montendre\,; M\textsuperscript{me} la Duchesse, menée par le comte
d'Uzès, sa queue fort inégalement portée par Montpipeaux, qui était
Rochechouart, et L'Aigle, fils de sa dame d'honneur.
M\textsuperscript{lle} de Bourbon, menée par Blansac, sa queue portée
par Montboissier et d'Angennes. Les princes ne les menèrent point à
cause de l'inégalité du nombre\,; cela devait être ainsi\,; mais M. le
duc de Berry se résolut à y aller et fut ainsi le premier prince du
deuil. Néanmoins on ne changea rien et les princes ne menèrent point les
princesses. Le duc de Beauvilliers eut le courage d'y porter la queue de
M. le duc de Berry, assisté du marquis de Béthune, son premier
gentilhomme de la chambre, et de Sainte-Maure, son premier écuyer. Je ne
sais plus les deux autres queues. Quatre menins pour le dais du Dauphin,
quatre autres pour celui de la Dauphine\,; à celle de Bavière c'étaient
quatre chevaliers de l'ordre, en pointe avec le collier, MM. de Beuvron,
Lavardin, La Salle et La Vauguyon. Dangeau, chevalier d'honneur de l'une
et de l'autre à leur mort, avec un maréchal de France, premier écuyer,
eut le même dégoût à toutes les deux. Le maréchal de Bellefonds, premier
écuyer, porta la couronne au lieu de lui, et Montchevreuil le manteau à
la royale au lieu du maréchal à la mort de la dauphine de Bavière. À
celle-ci le maréchal de Tessé, premier écuyer, porta la couronne au lieu
de Dangeau, et d'O le manteau en la place du maréchal. Tout se fit avec
les cérémonies et l'assistance accoutumée.

On fut assez content de l'oraison funèbre prononcée par Maboul, évêque
d'Aleth. M. de Metz, premier aumônier, officia\,; la cérémonie commença
sur les onze heures. Comme elle est fort longue, on s'avisa de mettre
sur la crédence un grand vase rempli de vinaigre en cas que quelqu'un se
trouvât mal. M. de Metz, ayant pris la première ablution et voyant au
volume des petites burettes qu'il restait peu de vin pour la seconde, en
demanda davantage. On prit donc ce grand vase sur la crédence, pensant
que ce fût du vin, et M. de Metz, qui se voulut fortifier, dit, en
lavant ses doigts sur le calice, de verser tout plein. Il l'avala d'un
trait et ne s'aperçut qu'à la fin qu'il avait avalé du vinaigre\,; sa
grimace et sa plainte fit un peu rire autour de lui, et lui-même conta
après son aventure, dont il fut très-mécontent. J'allai voir le
lendemain M. de Beauvilliers, dont la santé souffrit de cette cruelle
cérémonie. Je lui dis en l'embrassant\,: \emph{Vous venez donc
d'enterrer la France. }Il en convint avec moi. Hélas\,! s'il était au
monde, combien plus en serait-il persuadé aujourd'hui. Achevons tout
d'un trait ce terrible calice en intervertissant peu les temps. La
présence des corps dans le chœur de Saint-Denis avait fait différer
l'anniversaire de Monseigneur. M. de Metz y officia le jeudi 21 avril
avec l'assistance accoutumée, où se trouvèrent M. le duc de Berry, M. le
duc d'Orléans, MM. les comtes de Charolais et prince de Conti\,; le roi
y fit aller aussi le duc du Maine et le comte de Toulouse.

Le mardi 10 mai, le service se fit à Notre-Dame pour M. {[}le Dauphin{]}
et M\textsuperscript{me} la Dauphine. M. le duc de Berry, M. le duc
d'Orléans et M. le comte de Charolais furent les trois princes du
deuil\,; M\textsuperscript{me} la duchesse de Berry,
M\textsuperscript{lle} de Bourbon et M\textsuperscript{me} de Charolais
furent les princesses. Là comme à Saint-Denis ce devaient être Madame et
M\textsuperscript{me} la duchesse d'Orléans, parce que le deuil doit
être du même rang que de ceux dont on fait le deuil, ou du plus
approchant quand le même est impossible. Mais jusqu'aux princesses du
sang en usent comme pour une garde de fatigue, et le roi ne s'en
souciait pas. La queue de M. le duc de Berry fut étrangement portée par
Sainte-Maure, son premier écuyer, Pons, maître de sa garde-robe, et, ce
qui surprit fort, par La Haye, très-mince gentilhomme, qui, de page du
roi, était devenu son écuyer particulier, et qui, depuis qu'il eut une
maison, commanda son équipage de chasse, chose même dont on fut d'autant
plus scandalisé que ce fut l'ouvrage de M\textsuperscript{me} la
duchesse de Berry. M. le duc d'Orléans eut la sienne portée par
d'Étampes, capitaine de ses gardes, et par le jeune Bréauté, maître de
sa garde-robe, qui mourut bientôt après sans alliance. M. de Charolais
lui fut égalé comme il l'avait été à Saint-Denis, et les princesses du
sang de même qui ne doivent avoir qu'un porte-queue. Jaucourt,
gouverneur de M. de Charolais, et un gentilhomme à lui portèrent la
sienne. Les princes ne menèrent point les princesses non plus qu'ils
avaient fait à Saint-Denis\,; M\textsuperscript{me} la duchesse de Berry
la fut par Coettenfao et le chevalier d'Hautefort, son chevalier
d'honneur et son premier écuyer, et sa queue portée par le comte de
Roucy, Biron et Montendre, les mêmes qu'à Saint-Denis\,;
M\textsuperscript{lle} de Bourbon menée par Blansac eut sa queue fort
inégalement portée par le comte de Roye, fils de Roucy et neveu de
Blansac, et par L'Aigle, fils de la dame d'honneur de
M\textsuperscript{me} la Duchesse, sa mère. M\textsuperscript{lle} de
Charolais, menée par le comte d'Uzès, eut sa queue portée par
Châteaurenauld et d'Angennes. Le clergé gagna d'être salué séparément de
l'autel et immédiatement après, et immédiatement avant le catafalque,
qui reçut deux saluts à cause des deux corps. M\textsuperscript{lle}s de
Bourbon n'eurent qu'un salut ensemble, comme étant de même rang. Le P.
Gaillard fit une belle oraison funèbre. Le cardinal de Noailles officia.
Sa personne seule était en ornements violets, parce que les cardinaux
n'en portent jamais de noirs\,: précision d'orgueil qui monte jusqu'à
l'autel. Il donna un superbe dîner aux princes et aux princesses du
deuil, et aux principales dames. M. {[}le duc{]} et
M\textsuperscript{me} la duchesse de Berry le firent mettre à table. En
retournant à Versailles, M. le duc de Berry alla voir M. le Duc à
l'hôtel de Condé, qui n'était pas encore en état de sortir de sa
chambre. La chambre des comptes fit faire, le mardi 24 mai, un grand
service à la Sainte-Chapelle, pour M. {[}le Dauphin{]} et
M\textsuperscript{me} la Dauphine. Le P. La Rue y fit l'oraison funèbre,
qui fut assez belle. On fut étonné qu'il s'en fût chargé après ce qui
lui était arrivé à la mort de cette princesse\,; indépendamment de cet
événement la fonction n'était guère celle d'un confesseur.

Retournons maintenant sur nos pas, c'est-a-dire à ce voyage de Marly où
les plaisirs recommencèrent comme je l'ai dit avant que l'enterrement de
la France fût fait à Saint-Denis. On a vu (ci-dessus, p.~161)
l'inquiétude de mes amis sur ma conduite unique avec M. le duc
d'Orléans. Elle ne fit que s'augmenter. Je ne pus me rendre à leurs
avis, que je pris longtemps pour des faiblesses de cour. À la fin leur
concert, sans avoir pu se concerter pour la plupart, me fit faire des
réflexions, sans toutefois mépriser moins les menaces de la colère du
roi et du dépit de M\textsuperscript{me} de Maintenon, que je ne pus
croire tels qu'ils m'en voulaient persuader, parce que je ne pouvais
comprendre que moi de plus ou de moins avec M. le duc d'Orléans que tout
homme et de toute espèce fuyait sans ménagement et avec l'indécence la
plus marquée, pût le rendre ou moins abandonné ou moins coupable aux
yeux du monde. C'était pourtant ce dernier point qui faisait mon crime
et la peine où étaient les deux ducs de Beauvilliers et de Chevreuse, le
chancelier et mes autres amis et amies. J'ai déjà dit que mon extrême
douleur de la perte du Dauphin avait éclaté. Elle éclatait encore par ma
retraite et ma tristesse\,; elle m'avait trahi. On se douta, et à la fin
on démêla en gros la grandeur de ma perte\,; on hasarda de m'en parler
en me faisant compliment, car j'en reçus peu a peu malgré moi d'une
infinité de gens qui la plupart vinrent chez moi où j'étais porte close
le plus que je pouvais, et qui me rencontrant me disaient qu'ils y
étaient venus pour me témoigner la part qu'ils prenaient à la grande
perte que j'avais faite. J'avais beau détourner, écarter, répondre enfin
avec la brièveté d'un homme qui glisse et qui ne veut point entendre, je
ne persuadai personne, et il demeura pour constant à la cour et d'une
manière publique que j'avais lieu d'être fort affligé comme un homme qui
a perdu la plus grande et la plus certaine fortune.

Cette idée qui en peu de temps devint générale, et qui est de celles
qu'on ne fortifie jamais mieux que lorsqu'on entreprend de les
combattre, ne cadrait pas en moi avec celle qui ne l'était pas moins
devenue, du prétendu crime de M. le duc d'Orléans que le duc du Maine
répandait de tout son art, et que M\textsuperscript{me} de Maintenon
soutenait de toute sa haine, de toutes ses affections, de toute sa
puissance. J'étais trop connu pour qu'on pût imaginer que quelque
considération ni quelque nécessité que ce pût être vînt jamais à bout de
me ployer à voir celui que je soupçonnerais d'un forfait si exécrable,
combien moins de vivre avec lui tous les jours en intimité et de braver
par cette conduite, dont la singularité m'était pour le moins inutile,
le cri public, appuyé de toute la faveur et de toute l'autorité qui
réduisaient le prince, que je voyais sans cesse, à la solitude la plus
entière et la plus humiliante au milieu du monde et de la cour, et dans
le sein de sa plus proche famille. J'étais aussi trop avant avec le
prince que tous les cœurs pleuraient, avec tout ce qui l'environnait de
plus intime, et d'autre part avec celui que de si puissantes raisons
d'intérêt et de haine voulaient résolument écraser de ce crime, pour
qu'il fût possible que je ne me doutasse de rien à son égard, pour peu
qu'il y eût quelque apparence, même légère, de soupçons, ce qui était
manifestement détruit par ma conduite avec lui, que ne détruisait point
celle du peu d'autres intimes entours du Dauphin, qui n'ayant nulle
habitude avec M. le duc d'Orléans ne changeaient rien en cette occasion
à leur conduite avec lui.

M. de Beauvilliers, comme je l'ai remarqué, avait dans tous les temps
évité de le voir, et M. de Chevreuse ne le voyait que de loin à loin et
toujours à des heures particulières. C'était donc le contraste que ma
conduite faisait avec l'opinion régnante et dominante et la brèche
qu'elle pouvait lui faire chez tous les gens indifférents, raisonnables
et raisonnants, qui choquaient directement l'intérêt si cher de M. du
Maine, et la volonté si déployée de M\textsuperscript{me} de Maintenon.
C'est ce que mes amis voyaient clairement, c'est ce qu'ils me foisoient
sentir tant qu'ils pouvaient, c'est ce que je fus quelque temps à ne
vouloir pas croire, c'est ce que j'aperçus enfin très-distinctement, et
que je méprisai aussi parfaitement. Ce n'est pas que j'ignorasse le
danger de me les attirer et que je ne visse le roi derrière eux en
croupe, et tout à leur disposition, mais je ne crus pas que mon intime
liaison avec M. le duc d'Orléans dût par frayeur et par bassesse leur
servir d'un nouveau poids pour l'accabler par mon changement de
conduite. J'étais plus qu'en tout abri de lui être associé dans les
clameurs élevées contre lui\,; je n'avais donc à craindre que des
querelles d'Allemand pour m'éloigner et me perdre sous d'autres
prétextes, et je me résolus à en courir les risques, en évitant avec
soin e sagesse toute prise sur moi. Je fus plusieurs fois averti que le
roi était mécontent, tantôt de m'avoir vu de ses fenêtres dans les
jardins avec son neveu, tantôt que M\textsuperscript{me} de Maintenon
était surprise de ce que seul en toute la cour j'osois l'aborder et le
voir. Elle-même et M. du Maine, qui se cachait sous ses ailes, étaient
bien aises de me faire revenir ces choses pour m'inquiéter et pour me
faire changer à l'égard de M. le duc d'Orléans, et cela dura entre les
deux voyages de Marly, et augmenta fort durant le second qui est celui
dont je parle, et pendant lequel se fit l'enterrement à Saint-Denis,
parce que l'éclat des cris et des insultes du peuple au convoi et les
échos du monde et de la cour redoublèrent, et que Marly est fait de
façon qu'on me voyait à découvert tous les jours avec M. le duc
d'Orléans. Tant fut procédé enfin que, quelque temps après l'enterrement
et sur la fin du voyage de Marly, M. de Beauvilliers me pressa d'aller à
la Ferté, même avant le retour à Versailles, et de laisser de loin
conjurer l'orage qu'il voyait se former contre moi. Je résistai quelques
jours, mais il vint un matin trouver M\textsuperscript{me} de
Saint-Simon pendant que j'étais à la messe du roi, à qui il dit qu'il
savait très-précisément que M\textsuperscript{me} de Maintenon allait
éclater contre moi, et que, sans en alléguer nulle cause, j'allais être
chassé, si de moi-même je ne me retirais pour un temps. Tout de suite il
se chargea de m'avertir du train que les choses prendraient à mon égard,
et de m'avertir de revenir dès qu'il y verrait sûreté. Il pria en même
temps M\textsuperscript{me} de Saint-Simon de penser à une sorte de
langage de chiffre, pourtant sans chiffre, dont elle se pût servir pour
me faire entendre ce qu'il lui dirait de me mander pendant mon absence,
et la conjura que cela fût fait dans la journée pour me faire partir le
lendemain comme ayant à la Ferté une affaire pressée qui m'y demandait,
et que lui se chargeait de le dire au roi et de lui faire trouver bon
que je n'achevasse pas les quatre ou cinq jours qui restaient à demeurer
à Marly. Je le trouvai encore en rentrant chez moi. L'alarme bien plus
vive où je le vis me fit moins d'impression que ses manières de parler
absolues et déterminées, et l'air d'autorité avec lequel il s'expliqua.
Rien n'était moins de son caractère, et depuis des années rien de si
nouveau avec moi. Le secret d'autrui était chez lui impénétrable. Son
ton et son expression me firent sentir ce qu'il ne disait pas, et {[}me
parurent{]} pris exprès pour, sous un conseil si vif, si pressé, si fort
impératif, me montrer un ordre qu'il n'avait pas la liberté d'avouer.
M\textsuperscript{me} de Saint-Simon et moi ne vîmes pas lieu à une plus
longue défense. J'employai le reste du jour à répandre doucement la
prétendue nécessité de mon voyage, à faire ma cour à l'ordinaire, à voir
M. {[}le duc{]} et M\textsuperscript{me} la duchesse d'Orléans, et me
disposer à partir, comme je fis le lendemain matin. Je ne vis jamais si
promptement changer un visage très-austère en un très-serein que fit
celui du duc, sitôt que j'eus lâché la parole de partir. Jamais il ne
m'en a dit davantage là-dessus, et je suis toujours demeuré persuadé que
le roi ou M\textsuperscript{me} de Maintenon me l'avaient envoyé, et lui
avaient dit que je serais chassé, si suivant son conseil je ne me
chassais pas de bonne grâce. Mon départ ni mon absence ne fit aucun
bruit, personne n'y soupçonna rien. Je fus soigneusement instruit, mais
toujours en énigme de conseil, de l'état où j'étais pour demeurer ou
revenir. J'ignorai de même ce que fit mon retour, qui me fut mandé de
même. Mon absence fut d'un mois ou cinq semaines, et j'arrivai droit à
la cour, où je vécus avec M. le duc d'Orléans tout comme j'avais fait
auparavant.

Il n'était pas au bout de ses malheurs. C'était trop que de s'être rendu
par un trop bon mot deux toutes-puissantes fées implacables. Chalais,
l'homme à tout faire de la prineesse des Ursins, fut dépêché par elle
pour un voyage si mystérieux que l'obscurité n'en a jamais été
éclaircie. Il fut dix-huit jours en chemin, inconnu, cachant son nom, et
passa à deux lieues de Chalais, où étaient son père et sa mère, sans
leur donner signe de vie, quoique fort bien avec eux. Il rôda
secrètement en Poitou, et enfin y arrêta un cordelier de moyen âge dans
le couvent de Bressuire, qui s'écria\,: «\,Ah\,! je suis perdu\,!» dès
qu'il se vit arrêté. Chalais le conduisit dans les prisons de Poitiers,
d'où il dépêcha à Madrid un officier de dragons qu'il en avait mené avec
lui, et qui connaissoit ce cordelier, dont on n'a jamais su le nom mais
bien qu'il était effectivement cordelier, revenant de plusieurs lieux
d'Italie et d'Allemagne, et même de Vienne. Chalais poussa à Paris, vint
à Marly chez Torcy, le 27 avril, un mercredi que le roi avait pris
médecine. Torcy le mena l'après-dînée dans le cabinet du roi, avec
lequel il fut une demi-heure, ce qui retarda d'autant le conseil d'État,
et Chalais s'en alla aussitôt à Paris. Tant d'apparat n'était pas fait
pour n'en pas tirer parti, et Chalais n'avait pas été prostitué au
métier de prévôt après un misérable moine, sans en espérer un grand
bruit. Tout fut incontinent après rempli des bruits les plus affreux
contre M. le duc d'Orléans qui, par ce moine, qui toutefois était bien
loin lors de la mort de nos princes, les avait empoisonnés, et en
prétendait bien empoisonner d'autres. En un instant Paris retentit de
ces horreurs\,; la cour y applaudit, les provinces en furent inondées,
et tôt après les pays étrangers avec une rapidité incroyable, et qui
montrait à découvert la préparation du complot, et une publicité qui
pénétra jusqu'aux antres. M\textsuperscript{me} des Ursins ne fut pas
moins bien servie en Espagne là-dessus que M. du Maine et
M\textsuperscript{me} de Maintenon en France. Ce fut un redoublement de
rage affreux. On fit venir le cordelier pieds et poings liés à la
Bastille, où il fut livré uniquement à d'Argenson.

Ce lieutenant de police rendait compte au roi directement de beaucoup de
choses, au désespoir de Pontchartrain, qui, ayant Paris et la cour dans
son département de secrétaire d'État, crevait très-inutilement de dépit
de se voir passer par le bec des plumes secrètes et importantes qui
faisaient de son subalterne une espèce de ministre plus craint, plus
compté, plus considéré que lui, et qui s'y conduisit toujours de façon à
s'acquérir des amis en grand nombre, et des plus grands, et à se faire
fort peu d'ennemis et encore dans un ordre obscur ou infime. M. le duc
d'Orléans laissa tomber cette pluie à verse faute de pouvoir l'arrêter.
Elle ne put augmenter la désertion générale\,; il s'accoutumait à sa
solitude, et comme il n'avait jamais ouï parler de ce moine, il n'en eut
pas aussi la plus légère inquiétude. Mais d'Argenson, qui l'interrogea
plusieurs fois et qui en rendait directement compte au roi, fut assez
adroit pour faire sa cour à M. le duc d'Orléans de ce qu'il ne trouvait
rien qui le regardât, et des services qu'il lui rendait là-dessus auprès
du roi\footnote{Le marquis d'Argenson confirme pleinement dans ses
  \emph{Mémoires} (p.~191 édit. 1825) ce que rapporte Saint-Simon.
  «\,Mon père, dit-il, garda la foi qu'il devoit au roi\,; mais il
  tourna la persuasion de telle sorte que sur cet interrogatoire M. le
  duc d'Orléans fut sauvé et innocenté.\,»}. Il vit en habile homme la
folie d'un déchaînement destitué de tout fondement, dont l'emportement
ne pouvait empêcher M. le duc d'Orléans d'être un prince très-principal
en France pendant une minorité que l'âge du roi laissait voir d'assez
près, et il sut profiter du mystère que lui offrit son ministère pour se
mettra bien avec lui de plus en plus, car il l'avait soigneusement,
quoique secrètement, ménagé de tout temps, et cette conduite, comme on
le verra en son temps, lui valut une grande fortune.

Ce cordelier demeura près de trois mois à la Bastille sans parler à qui
que ce soit qu'à d'Argenson, après quoi Chalais, prévôt de
M\textsuperscript{me} des Ursins, le ramena lui-même de Paris en
Ségovie, où il fut enfermé dans une tour tout au haut du château, d'où
il avait la plus belle vue du monde, l'élévation à pic des tours de
Notre-Dame de Paris, du côté où il était. Il y était encore plein de
santé et ne parlant à personne, dix ans après, lorsque j'allai voir ce
beau château. J'y appris qu'il jurait horriblement contre la maison
d'Autriche et les ministres de la cour de Vienne, avec des emportements
furieux de ce qu'ils le laissaient pourrir là\,; qu'il ne lisait que des
romans, qu'il demandait à celui qui avait soin de lui\,; et qu'il vivait
là avec tout le scandale que quatre murailles le peuvent permettre à un
scélérat. On prétendit qu'il avait fait son marché pour empoisonner le
roi d'Espagne et les infants. Ses fureurs contre Vienne sembleraient
favoriser cette opinion. Elle a prévalu dans les esprits les plus sages
delà et deçà des Pyrénées\,; mais le mystère de toute cette affaire
étant demeuré mystère, je me garderai d'en porter un jugement qui ne
pourrait être certain, ni même indiquer de fondement. Ce malheureux est
mort longtemps depuis mon retour d'Espagne, et dans sa même prison.
Chalais fit sa cour sans doute aux deux fées, de s'être chargé d'une
fonction si pénible et si peu décente à un homme de sa qualité. Si elle
servit, comme elles le prétendirent sans doute, à donner plus de poids
au mystère, et à leurs exécrables interprétations, ce voyage ne réussit
pas dans le monde, quoique si emmuselé par elles, à celui qui s'était
ravalé à leur servir de prévôt.

Il arrive assez souvent que les événements les plus tristes sont suivis
de quelque farce imprévue qui divertit le public lorsqu'il y pense le
moins. La maison du duc de Tresmes en fournit une qui fit un étrange
éclat, et qui amusa beaucoup le monde. Il avait marié son fils aîné à
M\textsuperscript{lle} Mascrani, comme je l'ai marqué en son temps.
C'était la fille unique d'un maître des requêtes qui avait des biens
immenses, qui n'avait plus ni père ni mère, qui était sous la tutelle de
l'abbé Mascrani, frère de son père lorsqu'elle se maria, et dont les
Caumartin, frères de sa mère et amis intimes du duc de Tresmes de tout
temps, avaient fait le mariage. Elle n'était plus enfant lorsqu'il se
fit. Avec ses richesses, elle crut qu'elle allait être heureuse\,; elle
ignorait que ce n'était pas le sort des femmes des Potier.
M\textsuperscript{me} de Revel, veuve sans enfants, et sœur peu riche du
duc de Tresmes, vint loger chez lui pour gouverner sa belle-fille, qui
ne se trouva pas facile à l'être, ni la tante bien propre à cet emploi.
Des mésaises on en vint aux humeurs, puis aux plaintes, après aux
querelles et aux procédés, enfin aux expédients. La jeune femme avait
plus d'esprit que les Gesvres, elle sut mettre toute sa famille dans ses
intérêts, jusqu'aux Caumartin qui s'embrouillèrent enfin avec les
Gesvres. Elle s'enfuit chez la vieille Vertamont, sa grand'-mère
maternelle, qui l'avait élevée, et qui en était idolâtre, et de cet
asile fit signifier une demande de cassation de son mariage pour cause
d'impuissance. Les factums de part et d'autre mouchèrent. On peut juger
ce qu'une telle matière fournit, et quelle source d'ordures et de
plaisanteries. L'affaire se plaida à l'officialité\footnote{Tribunal de
  l'évêque, tenu par un juge d'église appelé official.}. Le marquis de
Gesvres prétendit n'être point impuissant, et comme c'était chose de
fait, il fut ordonné qu'il serait visité par des chirurgiens, et elle
par des matrones, nommés par l'officialité, pour y faire leur rapport,
et tous deux en effet furent visités.

Il serait difficile de rendre les scènes que cette affaire produisit.
Les gens connus et même distingués allaient s'en divertir aux
audiences\,; on y retenait les places dès le grand matin, on s'y
portait, et de là des récits qui faisaient toutes les conversations. Les
pauvres Gesvres en pensèrent mourir de dépit et de honte, et se
repentirent bien de s'être engagés en un pareil combat. Il dura
longtemps et toujours avec de nouveaux ridicules, et ne finit qu'avec la
vie de la marquise de Gesvres. On se persuadait malignement qu'elle
n'avait pas tout le tort, et son mari en a confirmé la pensée en ne
songeant pas à se remarier depuis plus de trente ans. Il y a suppléé par
son frère qui a des enfants de la fille aînée du maréchal de
Montmorency.

Les généraux partirent chacun pour l'armée qu'ils devaient commander, et
les officiers généraux et particuliers qui y devaient servir\,: Villars
pour la Flandre, Harcourt et Besons pour le Rhin, Berwick pour le
Dauphiné et les Alpes\,; et Fiennes, lieutenant général, remplaça en
Catalogne le duc de Noailles qu'on ne songea pas à faire servir.

Bissy, fils du lieutenant général et petit-fils d'un de ces légers
chevaliers, de l'ordre de M. de Louvois en 1688, épousa la fille de
Chauvelin, conseiller d'État. Il vit bientôt après son oncle dans une
éclatante fortune, et longues années après toute puissance et les sceaux
entre les mains du frère de sa femme, qui finit comme Icare\,; et de ces
deux fortunes si proches de Bissy il n'en attrapa rien.

Meuse, de la maison de Choiseul, épousa la fille de Zurlauben, tué
lieutenant général distingué, et de la sœur de Sainte-Maure.

L'abbé de Sainte-Croix mourut à plus de quatre-vingt-dix ans. Il avait
six abbayes, un prieuré, un petit gouvernement, les chiens du roi pour
le chevreuil. Il était fils du célèbre Molé, premier président et garde
des sceaux, et n'avait jamais été que maître des requêtes, ni songé qu'à
chasser et à se divertir de toutes les façons, jusqu'à sa mort, dans une
santé parfaite. Il venait de temps en temps faire sa cour au roi, qui
toujours lui parlait et le distinguait, en considération des grands
services de son père que le roi n'a jamais oubliés, et qui ont toujours
et solidement porté sur tous ceux de ce nom.

Deux hommes d'une grosseur énorme, de beaucoup d'esprit, d'assez de
lettres, d'honneur, et de valeur, tous deux fort du grand monde, et tous
deux plus que fort libertins, moururent en ce même temps, et laissèrent
quelque vide dans la bonne compagnie\,: Cominges fut l'un, La Fare,
l'autre. Cominges était fils et neveu paternel de Guitaut et de
Cominges, tous deux gouverneurs de Saumur, tous deux capitaines des
gardes de la reine mère, tous deux chevaliers de l'ordre en 1661, tous
deux très-affidés du gouvernement, tous deux employés aux exécutions de
confiance les plus délicates. Guitaut mourut subitement au Louvre à
quatre-vingt-deux ans, en 1663, sans avoir été marié. Cominges, son
neveu, son survivancier, et père de celui dont il s'agit ici, fut un
homme important toute sa vie. Il fut envoyé en 1646, vers M. le Prince,
en Flandre, chargé d'arrêter et de conduire à Sedan, en août 1648, le
fameux conseiller Broussel\,; l'année suivante, d'arrêter les officiers
suspects du régiment de la reine\,; et la même année, de faire passer
par les armes, 1\^{}er et 8 juin, Chambretet d'autres officiers de
Bordeaux. Lui et son oncle arrêtèrent au Palais-Royal les princes de
Condé et de Conti, et le duc de Longueville, 18 janvier 1650. Il arrêta
aussi du Dognon, connu, depuis qu'il se fut fait faire maréchal de
France pour rendre Brouage, sous le nom du maréchal Foucault. Cominges
prit l'année 1660, en avril, Saumur, sur du Mont qui s'en était saisi
pour M. le Prince, et commanda en 1652 et 1653 en Italie, en l'absence
du comte d'Harcourt, et en Catalogne. Il alla depuis ambassadeur en
Portugal, en Angleterre, et mourut en mars 1670, à cinquante-sept ans.

Il avait épousé la fille d'Amalby, conseiller au parlement de Bordeaux.
Sa mère valait encore moins, comme toutes celles de ces Cominges, hors
une ou deux. Ils portaient en plein le nom et les armes de Cominges, se
prétendaient être descendus des comtes de ce nom. Ils n'en ont pourtant
jamais pu en aucun temps prouver aucune filiation ni jonction, et on ne
sait quels ils étaient avant 1440. Cominges son fils ne servit guère que
volontaire et toujours aide de camp du roi qui, malgré ses mœurs et son
peu d'assiduité, ne le voyait jamais sans lui parler et le traiter avec
distinction et familiarité à cause de la reine mère. Les courtisans,
pendant les campagnes du roi, appelèrent par plaisanterie les bombes et
les mortiers du plus gros calibre des Cominges, et si bien que ce nom
leur est demeuré dans l'artillerie. Cominges trouvait cette
plainsanterie très-mauvaise, et ne s'y accoutuma jamais. Il était fort
grand et de très-bonne mine. Il passait pour avoir secrètement épousé
M\textsuperscript{lle} Dorée, qui avait été fille d'honneur de
M\textsuperscript{me} la Duchesse, qui, depuis qu'elle ne l'était plus,
logeait chez sa sœur, femme de Tambonneau, président en la chambre des
comptes, et longtemps ambassadeur en Suisse, fils de la vieille
Tambonneau si fort du grand monde, et de laquelle j'ai parlé.

Cominges n'avait qu'un frère qui était un fort honnête garçon, qui avait
servi sur mer et sur terre, qui avait de l'esprit, qui s'attacha fort
d'amitié au comte de Toulouse. Il avait été fort du grand monde et
bienvoulu partout. Il se retira les dernières années de sa vie qu'il
passa dans une grande piété. Il était chevalier de Malte et avait une
commanderie et une abbaye. Leur sœur, vieille fille de beaucoup d'esprit
aussi, de vertu et assez du monde, voulut faire une fin, comme les
cochers. Elle épousa La Traisne, premier président du parlement de
Bordeaux, qui était un très-digne magistrat fort ami de mon père, dont
elle fut la seconde femme, et n'en eut point d'enfants. Le gouvernement
de Saumur fut donné à d'Aubigny, neveu de l'archevêque de Rouen et
cousin prétendu de M\textsuperscript{me} de Maintenon, quoique tout
jeune et ce gouvernement fort gros, et indépendamment de celui de la
province. Cominges l'avait eu à la mort de son père.

La Fare fut l'autre démesuré en grosseur. Il était capitaine des gardes
de M. le duc d'Orléans, après l'avoir été de Monsieur, et croyait avec
raison avoir fait une grande fortune. Qu'aurait-il dit s'il avait vu
celle de ses enfants\,: l'un avec la Toison et le Saint-Esprit, l'autre
très-indigne évêque-duc de Laon\,? Il avait trop d'esprit pour n'en
avoir pas été honteux. La Pare était un homme que tout le monde aimait,
excepté M. de Louvois, dont les manières lui avaient fait quitter le
service. Aussi souhaitait-il plaisamment qu'il fût obligé de digérer
pour lui. Il était grand gourmand\,; et au sortir d'une grande maladie,
il se creva de morue et en mourut d'indigestion. Il faisait d'assez
jolis vers, mais jamais en vers ni en prose rien contre personne. Il
dormait partout les dernières années de sa vie. Ce qui surprenait c'est
qu'il se réveillait net, et continuait le propos où il le trouvait,
comme s'il n'eût pas dormi.

Rouillé, président en la chambre des comptes, des ambassades\footnote{Les
  précédents éditeurs ont fait de Rouillé \emph{un président en la
  chambre des comptes des ambassadeurs}. Cette prétendue chambre des
  comptes des ambassadeurs n'a jamais existé.} duquel j'ai parlé
plusieurs fois, où il avait toujours fort bien fait, fut trouvé mort
dans son lit à Paris par ses valets allant l'éveiller le matin du 30
mai. Il s'était couché en bonne santé ayant soupé chez la princesse
d'Espinoy. C'était un homme sec et sobre autant que son frère le
conseiller d'État était gourmand, ivrogne et débauché, et aussi sage que
l'autre l'était peu.

Le duc d'Uzès perdit aussi l'abbé d'Uzès, son frère, chanoine de
Strasbourg.

Le dimanche 29 mai, il arriva un courrier de Rome avec la nouvelle d'une
promotion de onze cardinaux que le pape venait de faire\,: c'était celle
des couronnes, dans laquelle le cardinal de Rohan fut compris. Ce fut le
plus beau cardinal du sacré collège\,; aussi était-il le fils de
l'amour. Mais sa mère n'en eut pas la joie, peut-être en eut-elle la
douleur où elle était. C'est de quoi il ne nous appartient pas de juger.

Le débordement de la Loire désola encore cette année l'Orléanais et la
Touraine, noya beaucoup de gens et de bestiaux, et entraîna quantité de
maisons. C'étaient les fruits du crédit qu'avait eu La Feuillade du
temps de Chamillart, comme je l'ai remarqué en son temps.

Le duc de Richelieu, qui avait fait mettre le duc de Fronsac, son fils,
à la Bastille, il y avait quelque temps, paya ses dettes et l'en fit
sortir le croyant bien corrigé.

\hypertarget{chapitre-ix.}{%
\chapter{CHAPITRE IX.}\label{chapitre-ix.}}

1712

~

{\textsc{La reine d'Espagne accouche d'un prince.}} {\textsc{-
L'empereur couronné roi de Hongrie à Presbourg.}} {\textsc{- Mort du duc
de Vendôme.}} {\textsc{- Éclaircissement sur la sépulture du duc de
Vendôme.}} {\textsc{- Dames du palais en Espagne.}} {\textsc{- Mort, fin
et dernier bon mot d'Harlay, ci-devant premier président.}} {\textsc{-
Singularité du roi sur ses ministres.}} {\textsc{- Course d'un gros
parti ennemi en Champagne.}} {\textsc{- Trêve publiée entre la France et
l'Angleterre.}} {\textsc{- Porto-Ercole pris par les ennemis.}}
{\textsc{- La Badie rend le Quesnoy\,; est mis à la Bastille.}}
{\textsc{- Broglio défait dix-huit cents chevaux.}} {\textsc{- Emo ne
peut raccommoder la république de Venise avee le roi.}} {\textsc{-
Voyage de Fontainebleau par Petit-Bourg.}} {\textsc{- Rohan, évêque de
Strasbourg, fait cardinal, en reçoit la calotte et le bonnet.}}
{\textsc{- M\textsuperscript{me} la grande-duchesse en apoplexie.}}
{\textsc{- Siége de Landrecies par le prince Eugène.}} {\textsc{- Combat
de Denain.}} {\textsc{- Montesquiou prend Marchiennes.}} {\textsc{-
Prince Eugène lève le siège de Landrecies.}} {\textsc{- Villars prend
Douai}} {\textsc{- Nos lignes de Weissembourg inutilement canonnées.}}
{\textsc{- Cantons catholiques, battus par les cantons protestants, font
la paix.}} {\textsc{- Cassart prend, rase, pille et brûle Santiago au
cap Vert.}} {\textsc{- Échange du marquis de Villena et de Cellamare
avec Stanhope et Carpenter.}} {\textsc{- Mort du fils aîné du duc de La
Rocheguyon.}} {\textsc{- Mort de l'abbé Tallemant.}} {\textsc{- Mort du
frère du maréchal de Villars et du fils unique de du Bourg\,; leur
caractère.}} {\textsc{- Albemarle, pris à Denain, renvoyé sur sa
parole.}} {\textsc{- Mort, conduite, fortune, famille de M. de
Soubise.}} {\textsc{- Injure espagnole qui ne se pardonne jamais.}}
{\textsc{- Mort du marquis de Saint-Simon.}} {\textsc{- Mort de
M\textsuperscript{me} de La Fayette.}} {\textsc{- Mort de Cassini, grand
astronome.}} {\textsc{- Mort, caractère et savoir de Refuge.}}
{\textsc{- Mort de M\textsuperscript{me} Herval.}} {\textsc{- Abbé
Servien chassé, et pourquoi\,; son caractère et sa fin.}} {\textsc{-
Désordres des loups en Orléanais.}}

~

On eut la nouvelle que la reine d'Espagne était accouchée le 6 juin d'un
prince à Madrid, qu'on nomma don Philippe, et que le 22 mai l'empereur
avait été couronné roi de Hongrie à Presbourg avec grande magnificence.

Vendôme triomphait en Espagne, non des ennemis de cette couronne, mais
des Espagnols et de nos malheurs. À son âge et à celui de ceux que nous
pleurions, il se comptait expatrié pour le reste de sa vie. Leur mort le
rendit aux plus flatteuses espérances d'en revenir jouir à notre cour,
et d'y redevenir un personnage qui y ferait de nouveau bien compter avec
lui. L'Altesse avait été un fruit aussi prompt que délicieux d'une si
surprenante délivrance\,; l'assimilation aux don Juan en fut un autre
coup sur coup qui acheva de l'enivrer des larmes de la France, où, porté
sur ce nouveau piédestal, il projetait de venir faire le prince du sang
en plein par le titre d'en avoir désespéré l'Espagne. Sa paresse, sa
liberté de vie, ses débauches avaient prolongé son séjour sur la
frontière, où il se trouvait plus commodément pour satisfaire à tous ses
goûts qu'à Madrid, où, bien qu'il ne se contraignît guère, il ne pouvait
éviter quelque sorte de contrainte de représentation et de paraître à la
cour. Il y arriva pour recevoir les profusions intéressées de la
toute-puissance de la princesse des Ursins\,; mais, comme je l'ai
remarqué, son dessein se bornait à l'Altesse commune et au leurre plutôt
qu'à l'effet bien établi des traitements des deux don Juan qu'elle lui
avait fait donner. Elle se hâta donc de faire expédier avec lui ce qui
pour le militaire demandait nécessairement sa présence, et de le
renvoyer promptement à la frontière. Lui-même, comblé de distinctions où
il n'avait osé prétendre, embarrassé de la solitude où le laissait
l'extrême dépit des grands et des seigneurs de leur subite humiliation à
son égard, et rappelé dans ses quartiers par sa paresse et ses infâmes
délices, il s'en retourna volontiers très-promptement. Il n'y avait rien
à y faire. Les Autrichiens, étonnés et affaiblis du départ des Anglais,
se trouvaient bien éloignés de l'offensive\,; et Vendôme, nageant dans
les charmes de son nouveau sort, ne pensait qu'à en jouir dans une
oisiveté profonde, sous prétexte que tout n'était pas prêt pour
commencer les opérations.

Pour être plus en liberté, il se sépara des officiers généraux et alla
s'établir avec deux ou trois de ses plus familiers et ses valets, qui
faisaient partout sa compagnie la plus chérie, à Vignarez, petit bourg
presque abandonné et loin de tout, au bord de la mer, dans le royaume de
Valence, pour y manger du poisson tout son soûl. Il tint parole et s'y
donna de tout au cœur joie près d'un mois. Il se trouva incommodé, on
crut aisément qu'il ne lui fallait que de la diète\,; mais le mal
augmenta si promptement et d'une façon si bizarre, après avoir semblé
assez longtemps n'être rien, que ceux qui étaient auprès de lui, en
petit nombre, ne doutèrent pas du poison et envoyèrent aux secours de
tous côtés\,; mais le mal ne les voulut pas attendre\,; il redoubla
précipitamment avec des symptômes étranges. Il ne put signer un
testament qu'on lui présenta, ni une lettre au roi par laquelle il lui
demandait le retour de son frère à la cour. Tout ce qui était autour de
lui s'enfuit et l'abandonna, tellement qu'il demeura entre les mains de
trois ou quatre des plus bas valets, tandis que les autres pillaient
tout et faisaient leur main et s'en allaient, Il passa ainsi les deux ou
trois derniers jours de sa vie sans prêtre, sans qu'il eût été seulement
question d'en parler, sans autre secours que d'un seul chirurgien. Les
trois ou quatre valets demeurés auprès de lui, le voyant à la dernière
extrémité, se saisirent du peu de choses qui restaient autour de lui,
et, faute de mieux, lui tirèrent sa couverture et ses matelas de dessous
lui. Il leur cria pitoyablement de ne le laisser pas mourir au moins à
nu sur sa paillasse, et je ne sais s'il l'obtint. Ainsi mourut, le
vendredi 10 juin, le plus superbe des hommes, et pour n'en rien dire
davantage après avoir été obligé de parler si souvent de lui, le plus
heureux jusqu'à ses derniers jours. Il avait cinquante-huit ans, sans
qu'une faveur si prodigieuse et si aveugle ait pu faire qu'un héros de
cabale d'un capitaine qui a été un très-mauvais général, d'un sujet qui
s'est montré le plus pernicieux, et d'un homme dont les vices ont fait
en tout genre la honte de l'humanité. Sa mort rendit la vie et la joie à
toute l'Espagne.

Aguilar, l'ami du duc de Noailles, revenu d'exil pour servir sous lui,
fut fort accusé de l'avoir empoisonné, et se mit aussi peu en peine de
s'en défendre, comme on s'y mit peu de faire aucune recherche. La
princesse des Ursins, qui pour sa grandeur particulière avait si bien su
profiter de sa vie, ne profita pas moins de sa mort. Elle sentit sa
délivrance d'un nouveau don Juan à la tête des armées d'Espagne, qui n'y
était plus en refuge et en asile souple par nécessité sous sa main, et
qui au contraire, délivré de tout ce qui l'y avait relégué, recouvrait
en plein toutes ses anciennes forces en France, d'où il tirerait toute
sorte de protection et d'autorité. Elle ne se choqua donc point de la
joie qui éclata sans contrainte, ni des discours les plus libres de la
cour, de la ville, de l'armée, de toute l'Espagne\,; ni par conséquent
le roi et la reine, qui n'en firent aucun semblant. Mais pour soutenir
ce qu'elle avait fait, et faire à bon marché sa cour à M. du Maine, à
M\textsuperscript{me} de Maintenon, au roi même, elle fit ordonner que
le corps de ce monstre hideux de grandeur et de fortune serait porté à
l'Escurial. C'était combler la mesure des plus grands traitements. Il
n'était point mort en bataille, et de plus on ne voit aucun particulier
enterré à l'Escurial, comme il y en a plusieurs à Saint-Denis. Cet
honneur fut donc déféré à ceux qui venaient d'être donnés à sa
naissance. C'est aussi ce qui enfla M. du Maine jusqu'à ne pouvoir s'en
contenir. Mais en attendant que je parle du voyage que j'ai fait à
l'Escurial, si j'ai assez de vie pour pousser ces Mémoires jusqu'à la
mort de M. le duc d'Orléans, il faut expliquer ici cette illustre
sépulture.

Le panthéon est le lieu où il n'entre que les corps des rois et des
reines qui ont eu postérité. Un autre lieu séparé, non de plain-pied,
mais proche, fait en bibliothèque, est celui où sont rangés les corps
des reines qui n'ont point eu de postérité, et des infants. Un troisième
lieu, qui est comme l'antichambre de ce dernier, s'appelle proprement le
pourrissoir, quoique ce dernier en porte aussi improprement le nom. Il
n'y paraît que les quatre murailles blanches avec une longue table nue
au milieu. Ces murs sont fort épais\,; on y fait des creux où on met un
corps dans chacun, qu'on muraille par-dessus, en sorte qu'il n'en paraît
rien. Quand on juge qu'il y a assez longtemps pour que tout soit assez
consommé et ne puisse plus exhaler d'odeur, on rouvre la muraille, on en
tire le corps, on le met dans un cercueil qui en laisse voir quelque
chose par les pieds. Ce cercueil est couvert d'une étoffe riche, et on
le porte dans la pièce voisine. Le corps du duc de Vendôme était encore
depuis neuf ans dans cette muraille lorsque j'entrai dans ce lieu, où on
me montra l'endroit où il était, qui était uni comme tout le reste des
quatre murs et sans aucune marque. Je m'informai doucement aux moines
chargés de me conduire et de me faire les honneurs dans combien il
serait transporté dans l'autre pièce. Ils ne répondirent qu'en évitant
de satisfaire cette curiosité, en laissant échapper un air
d'indignation, et ne se contraignirent pas de me laisser entendre qu'on
ne songeait point à ce transport, et que, puisqu'on avait tant fait que
de l'emmurailler, il y pourrait demeurer. Je ne sais ce que M. du Maine
fit du testament non signé qui lui fut envoyé et dont il fit son
affaire, mais il ne put obtenir du roi aucune démonstration en faveur de
M. de Vendôme, ni le retour du grand prieur qui demeura à Lyon jusqu'à
la mort du roi\,; mais le roi prit le deuil quelques jours en noir.
M\textsuperscript{me} de Vendôme recueillit les grands avantages qui lui
avaient été faits par son contrat de mariage, dont Anet et Dreux ont
passé d'elle à M\textsuperscript{me} du Maine, et les autres terres
réparties de même aux autres héritiers de la duchesse de Vendôme après
elle\,; mais le roi reprit aussitôt Vendôme et ce qui se trouva de
réversible à la couronne. Le grand prieur ne prétendit rien et n'eut
rien aussi, comme exclu de tout héritage par ses vœux de l'ordre de
Malte. On paya les créanciers peu à peu, et les valets devinrent ce
qu'ils purent. Il n'est pas encore temps de parler de ce que devint
Albéroni. Ce fut à peu près en ce temps-ci que la reine, n'ayant plus de
filles ni de menines, prit des dames du palais à peu près comme celles
de M\textsuperscript{me} la Dauphine et de la reine.

Harlay, ci-devant premier président, dont j'ai eu tant d'occasion de
parler, mourut à Paris fort peu de temps après. Je n'ai plus à le faire
connaître. J'ajouterai seulement l'humiliation où fut réduit ce superbe
cynique. Il loua une maison dont la muraille du jardin était mitoyenne à
celui des Jacobins du faubourg Saint-Germain, mais dans la rue de
l'Université, qui n'était point à eux comme celles de la rue
Saint-Dominique et de la rue du Bac, où pour les mieux louer ils donnent
des portes dans leur jardin, et ces mendiants en tirent cinquante mille
livres de rente. Harlay, accoutumé à l'autorité, leur demanda une porte
dans leur jardin. Il fut refusé. Il insista, leur fit parler et ne
réussit pas mieux. Cependant on leur fit entendre que, encore que ce
magistrat naguère si puissant ne pût plus rien par lui-même, il avait un
fils et un cousin conseillers d'État, auxquels ils ne pouvaient se
promettre de n'avoir jamais affaire, et qui, sans se soucier de la
personne, pourraient bien par orgueil leur faire sentir leur
mécontentement. L'argument d'intérêt est le meilleur avec les moines.
Ceux-ci se ravisèrent. Le prieur, accompagné de quelques notables du
couvent, alla faire excuse à Harlay et lui dire qu'il était le maître de
faire percer la porte. Harlay, toujours lui-même, les regarda de
travers, répondit qu'il s'était ravisé et qu'il s'en passerait. Les
moines, fort en peine du refus, insistèrent\,; il les interrompit et
leur dit\,: «\,Voyez-vous, mes pères, je suis petit-fils d'Achille du
Harlay, premier président du parlement, qui a si bien servi l'État et
les rois, et qui, pour soutenir la cause publique, fut traîné à la
Bastille où il pensa être pendu par ces scélérats de ligueurs\,; il ne
me convient donc pas d'entrer ni d'aller prier Dieu chez des gens de la
robe de votre Jacques Clément.\,» Et tout de suite leur tourna le dos et
les laissa confondus. Ce fut son dernier trait. Il tomba dans l'ennui et
dans la misère des visites\,; et comme il conservait toujours toutes ses
mêmes manières de gravité empesée, de compliments de fausse humilité, de
discours recherchés, d'orgueil le plus incommode, il désolait tous ceux
qu'il allait voir, et il allait jusque chez des gens qui s'étaient
souvent morfondus dans ses antichambres. Peu à peu des apoplexies
légères mais fréquentes lui embarrassèrent la langue, en sorte qu'on
avait grand'peine à l'entendre et lui beaucoup à marcher\footnote{Nous
  avons reproduit le texte de Saint-Simon\,; les anciens éditeurs ont
  substitué \emph{parler} à \emph{marcher}.}. En cet état il ne cessait
point de visiter, et ne s'apercevait point qu'il trouvait beaucoup de
portes fermées. Il mourut enfin dans cette misère et dans le mépris, au
grand soulagement du peu qui par proximité le voyait, surtout de son
fils et de son domestique.

Une bagatelle ne doit pas être oubliée ici, qui montrera combien le roi
croyait {[}devoir{]} et avait soin de tenir ses ministres de court. Le
comte d'Uzès, qui, depuis les funestes obsèques dont j'ai parlé et où je
l'ai nommé, était allé en Espagne, s'était arrêté à Madrid sur la mort
de M. de Vendôme, sous lequel il devait servir. À peine y fut-il huit
jours, que le roi d'Espagne le renvoya au roi avec une lettre, par
laquelle il lui demandait un général pour commander ses armées. De
quatre généraux français qu'il lui nommait, il n'y en eut point de
nommé, parce que le roi d'Espagne se ravisa bientôt et n'en voulut plus.
Le comte d'Uzès arriva chez Torcy le 21 juin, à Marly, qui le mena au
roi, lequel, après qu'ils furent sortis de son cabinet, passa chez
M\textsuperscript{me} de Maintenon, et y travailla avec Voysin et
Desmarets ensemble, chose assez rare qu'il y travaillât avec deux en
même temps. Pendant ce travail, il arriva à Torcy un courrier
d'Angleterre, attendu avec impatience\,; Torcy en alla porter les
dépêches au roi. Voysin et Desmarets sortirent, et attendirent avec les
courtisans que Torcy sortît à son tour. Cependant ils étaient ministres
l'un et l'autre. Torcy très-sûrement rendit compte de ces mêmes
dépêches, le lendemain matin, au conseil d'État, en leur présence, et
apparemment les lut entières, puisqu'elles étaient importantes\,; Voysin
et Desmarets y en dirent leur avis, comme le duc de Beauvilliers et le
chancelier, et Torcy même\,; peut-être, et il y a toute apparence,
qu'étant rentrés avec le roi, comme ils firent, dès que Torcy fut sorti,
le roi lui-même leur dit ce qu'il venait d'apprendre. Mais ils n'en
quittèrent pas moins la place à Torcy\,; le roi ne les retint point, et
le courtisan, répandu dans les salons, fut témoin de cette cérémonie. Le
17 juillet la trêve fut publiée en Flandre entre la France et
l'Angleterre, à la tête des troupes des deux couronnes. Un mois
auparavant, le prince Eugène avait envoyé près de deux mille chevaux
faire une course en Champagne, qui pensèrent prendre l'archevêque de
Reims qui faisait ses visites. Ils brûlèrent un faubourg de Vervins,
passèrent près de Sainte-Menehould, firent beaucoup de désordres en
Champagne et autour de Metz, passèrent la Meuse à Saint-Mihiel, la
Moselle auprès de Pont-à-Mousson, emmenèrent grand nombre d'otages, et
se retirèrent à Traarbach, sans que Saint-Frémont ni Coigny, détachés
après, chacun de leur côté, eussent pu les joindre.

Zumzungen, général de l'empereur, se rendit maître de Porto-Ercole après
une belle défense du gouverneur.

Le prince Eugène ouvrit la tranchée devant le Quesnoy, la nuit du 20 au
21 juin, malgré l'inaction déclarée des Anglais, qui précéda la trêve
avec eux. Jarnac en apporta la capitulation au roi le 8 juillet à Marly.
La Badie, qui y commandait, s'étant rendu prisonnier de guerre avec sa
garnison, fut fort chargé de s'être mal défendu par le maréchal de
Villars et par toute l'armée\,; il obtint la permission du prince Eugène
de venir se justifier à la cour, mais en arrivant à Paris il fut mis à
la Bastille. Broglio cependant défit dix-huit cents chevaux des ennemis,
presque tous tués ou pris. Ces bagatelles soutenaient.

Emo, sage, grand, était à Paris depuis quelques mois, envoyé sans
caractère par la république de Venise, pour tâcher d'accommoder la
brouillerie causée par le choix du cardinal Ottoboni, Vénitien, pour
être protecteur de France à Rome, et l'acceptation qu'il en avait faite
contre la loi de sa patrie. Mais l'affaire n'était pas encore mûre, et
il s'en retourna sans avoir rien obtenu.

Le roi partit le mercredi, 13 juillet, de Marly après le conseil d'État,
s'arrêta un peu à Versailles, alla coucher à Petit-Bourg et le lendemain
à Fontainebleau. Il y donna, le 20 du même mois, au cardinal de Rohan la
calotte rouge, qu'il avait reçue la veille de Rome, et qu'il lui vint
présenter, et cinq jours après le bonnet que le camérier Bianchini lui
avait apporté. Quelques jours auparavant, M\textsuperscript{me} la
grande-duchesse était tombée en apoplexie au Palais-Royal, où elle fut
obligée de demeurer assez longtemps. M. {[}le duc{]} et
M\textsuperscript{me} la duchesse d'Orléans l'y laissèrent lorsqu'elle
fut hors de danger, et allèrent à Fontainebleau.

Le prince Eugène assiégea Landrecies. Le roi, piqué des avantages qu'il
ne laissait pas de prendre quoique destitué du secours des Anglais,
voulait en profiter, et trouvait fort mauvais que Villars laissât
assiéger et prendre les places de la dernière frontière sans donner
bataille pour l'empêcher. Villars en avait des ordres réitérés. Il
mandait force gasconnades, il en publiait, mais il tâtonnait et reculait
toujours, et manqua plus d'une occasion de prêter le collet au prince
Eugène, dont quelques-unes furent si visibles, et même d'une apparence
si avantageuse, que toute l'armée en murmura publiquement. Il cherchait,
disait-il, les moyens de faire lever le siége de Landrecies, et le roi
attendait tous les jours des courriers de Flandre avec la dernière
impatience. Montesquieu vit jour à donner un combat avec avantage. Il
était fort connu du roi pour avoir été longtemps major du régiment des
gardes, inspecteur puis directeur d'infanterie, et beaucoup plus par ses
intimes liaisons avec les principaux valets de l'intérieur. Il dépêcha
secrètement un courrier au roi avec un plan de son dessein, en lui
marquant qu'il était sûr que Villars ne l'approuverait pas, et en
représentant la nécessité de profiter des conjonctures. La réponse fut
prompte. Il eut ordre de suivre, d'exécuter son projet, même malgré
Villars, mais de faire cela par rapport à lui avec adresse. L'extrême
mépris que le prince Eugène avait conçu du maréchal de Villars lui fit
commettre une lourde faute, qui fut de s'éloigner de Marchiennes, et
même de Denain où étaient ses magasins principaux, pour subsister plus
commodément derrière l'Escaillon qui se jette dans l'Escaut près de
Denain, qu'il avait retranché, et y avait laissé dix-huit bataillons et
quelque cavalerie. Sur ces nouvelles, le maréchal Montesquiou pressa
Villars d'y marcher.

Dans la marche, Montesquiou s'avança avec une tête, quatre lieutenants
généraux et quatre maréchaux de camp, et envoya Broglio, depuis maréchal
de France, avec la réserve qu'il commandait, enlever cinq cents chariots
de pain pour l'armée ennemie, ce qu'il exécuta fort bien et avant
l'attaque de Denain. Montesquiou avec cette tête de l'armée arriva
devant Denain à tire-d'aile, fit promptement sa disposition, et attaqua
tout de suite les retranchements. Villars marchait doucement avec le
gros de l'armée, déjà fâché d'en voir une partie en avant avec
Montesquiou sans son ordre, et qui le fut bien davantage quand il
entendit le bruit du feu qui commençait. Il lui dépécha ordre sur ordre
d'arrêter, de ne point attaquer, de l'attendre, le tout sans se hâter le
moins du monde, parce qu'il ne voulait point de combat. Son confrère lui
renvoya ses aides de camp, lui manda que le vin était tiré et qu'il
fallait le boire, et poussa si bien ses attaques qu'il emporta les
retranchements, entra dans Denain, s'y rendit le maître de toute
l'artillerie et des magasins, tua beaucoup de monde, en fit noyer
quantité en tâchant de se sauver, entre lesquels se trouva le comte de
Dohna qui y commandait, et se mit en posture de s'y bien maintenir s'il
prenait envie au prince Eugène de l'y attaquer, qui arrivait avec son
armée par l'autre côté de la rivière, qui fut témoin de l'expédition,
qui recueillit les fuyards, et qui s'arrêta, parce qu'il ne crut pas
pouvoir attaquer Denain emporté avec succès.

Tingry cependant, depuis maréchal de Montmorency, averti d'avance par
Montesquiou, était sorti de Valenciennes et avait si bien défendu un
pont, qui était le plus court chemin du prince Eugène pour tomber sur le
maréchal de Montesquiou, qu'il l'empêcha d'y passer, le força à prendre
le grand tour par l'autre côté de la rivière, par où je viens de dire,
et fit qu'il arriva trop tard. Villars, arrivant avec le reste de
l'armée comme tout était fait, enfonça son chapeau, dit merveilles aux
tués et aux ennemis delà l'eau qui se retiraient, et dépêcha Nangis au
roi qui avait été un des quatre maréchaux de camp de l'attaque, que
Voysin mena au roi le mardi 26 juillet à huit heures du matin, et qui
eut force louanges et douze mille livres pour sa course. Les ennemis y
perdirent extrêmement, et le maréchal de Montesquiou fort peu. Le fils
unique du maréchal de Tourville y fut tué à la tête de son régiment,
dont ce fut grand dommage, et laissa sa sœur héritière qui épousa depuis
M. de Brassac et fut dame de M\textsuperscript{me} la duchesse de Berry
quand on lui en donna.

Villars, fort étourdi d'une action faite malgré lui, s'en voulait tenir
là\,; mais Montesquiou, sûr du roi, se moqua de lui, détacha le soir
même du combat, qui était le dimanche 24 juillet, Broglio avec douze
bataillons sur Marchiennes, où était le reste et la plus grande partie
des magasins des ennemis, et le suivit en personne avec dix-huit autres
bataillons et quelque cavalerie, sans que Villars osât s'y opposer
formellement, après ce qui venait d'arriver. Il prit Saint-Amand en
passant, où il y avait huit cents hommes, et l'abbaye d'Hannon, où il y
en avait deux cents. Villars, aide-major du régiment des gardes et
aide-major général de l'armée, arriva le dernier juillet à Fontainebleau
avec force drapeaux, par qui on apprit qu'un fils d'Overkerke avait été
tué à Denain, qui était officier général fort estimé parmi les
Hollandais. Le lundi 1\^{}er août, Artagnan arriva à une heure
après-midi à Fontainebleau, de la part du maréchal de Montesquiou, son
oncle, avec la nouvelle qu'il avait pris Marchiennes et tout ce qui s'y
était trouvé prisonniers de guerre. Il y avait dans la place six
bataillons, un détachement de cinq cents hommes de la garnison de Douai,
et le régiment de cavalerie entier de Waldec, qui allait joindre l'armée
du prince Eugène, et qui n'en put sortir avant d'y être enfermé\,;
soixante pièces de canon\,; et, outre ce qu'il y avait de munitions de
guerre et de bouche en magasin, cent cinquante bélandres qui en étaient
chargées sur la rivière, six desquelles avaient chacune deux cents
milliers de poudre, le tout sans avoir presque perdu personne à ce
siége. Un fils du maréchal de Tessé avait été fort blessé à Denain à la
tête du régiment de Champagne, et le marquis de Meuse à la tête du sien.

Montesquiou eut dans l'armée et à la cour tout l'honneur de ces deux
heureuses actions, qui levèrent, pour ainsi dire, le sort dont nous
étions si misérablement enchantés, qui parurent avec raison un prodige
de la Providence, et qui mirent fin à tous nos malheurs. Montesquieu eut
le sens d'être sage et modeste, de laisser faire le matamore à Villars
qui se fit moquer de soi, de respecter la protection ouverte de
M\textsuperscript{me} de Maintenon, et de se contenter de la gloire, à
laquelle personne ne se méprit. Ce fut à Fontainebleau un débordement de
joie dont le roi fut si flatté qu'il en remercia les courtisans pour la
première fois de sa vie. Le prince Eugène, manquant de pain et de toutes
choses, leva aussitôt après le siége de Landrecies, et une désertion
effroyable se mit dans ses troupes.

Le roi envoya ordre en même temps de faire le siége de Douai. Le samedi
10 septembre, Aubigny, ce prétendu cousin de M\textsuperscript{me} de
Maintenon qui venait d'avoir le gouvernement de Saumur, et qui était
brigadier et colonel du régiment royal, arriva à Fontainebleau, et fut
mené par Voysin dans le cabinet du roi après son souper. Il lui apprit
que Vieux-Pont ayant emporté les demi-lunes le 7, la chamade avait été
battue le 8, et la garnison se rendit prisonnière de guerre. Albergotti
qui commandait au siége fit entrer huit bataillons dans la place avec
Vieux-Pont pour y commander, et permit aux officiers d'emmener leurs
équipages. La descente du fossé n'avait pas encore été faite. Aubigny
eut douze mille livres pour sa course. Le prince Eugène se tenait
toujours près de Mons avec une armée hors d'état de rien faire, et celle
du roi alla faire le siége du Quesnoy. Mais il faut retourner sur nos
pas. Il y avait du temps que le fort de Scarpe s'était rendu, la
garnison de quatre cents hommes prisonnière de guerre. Saint-Pierre en
apporta la nouvelle au roi. Le duc de Wurtemberg, général de l'armée de
l'empereur sur le Rhin, avait eu ordre d'attaquer nos lignes de
Weissembourg\,; il s'en approcha, les canonna deux jours durant sans y
faire aucun mal, y perdit assez de monde, et se retira, après quoi on
brûla leurs batteries. Ce fut tout l'exploit qu'il y eut de part et
d'autre en Allemagne.

Il y eut du bruit en Suisse entre les cantons catholiques et
protestants. Ils prirent les armes\,; les derniers furent victorieux.
Quoique la guerre fût fort courte, il en coûta cher aux cantons
catholiques. La paix entre eux fut signée à Arrau.

Cassart, avec une escadre armée à Toulon, prit dans la principale île du
cap Vert le fort et la ville de Santiago aux Portugais, où il y avait
douze mille hommes en état de porter les armes, et on n'en avait
débarqué que mille. Le gouverneur s'était rendu à condition qu'en payant
soixante mille piastres, la ville ni les forts ne seraient point
endommagés. Cependant le gouverneur, l'évêque et les principaux
habitants se sauvèrent dans les montagnes. Cette fuite irrita Cassart.
Il en prit prétexte de prendre quatre cents nègres et deux vaisseaux qui
se trouvèrent à la rade, d'emporter les principales marchandises de la
ville, puis de la piller et brûler.

Enfin le marquis de Villena, connu quelquefois sous le nom de duc
d'Escalone, et le prince de Cellamare, prisonniers de guerre, furent
échangés\,: le premier contre Stanhope, comme je l'ai rapporté en son
lieu, à Bruhuega\,; l'autre contre le général Carpenter. J'aurai tant à
parler dans la suite de tous les deux, pendant la régence de M. le duc
d'Orléans et lors de mon ambassade extraordinaire en Espagne, si j'ai
assez de vie pour conduire ces Mémoires à leur terme, que j'ai voulu
marquer leur échange ici. Incontinent après, le roi d'Espagne donna à
Villena la charge de son majordome-major qu'il lui gardait depuis
longtemps. J'ajouterai en passant que c'était en tout genre un des
premiers et des plus grands seigneurs d'Espagne, et orné de toutes
sortes de vertus.

Le duc de La Rocheguyon perdit son fils aîné, de la petite vérole, chez
l'archevêque de Cambrai où on l'avait transporté. Ce fut le troisième
aîné de suite que cette maladie lui emporta. Il lui restait trois
garçons, l'aîné desquels était comblé d'abbayes. Le second était M. de
Durtal, qu'on a vu il n'y a pas longtemps revenir des Indes avec
Ducasse, et apporter ici la nouvelle de l'arrivée des galions, à qui le
roi donna le régiment de son frère, et qui est aujourd'hui duc de La
Rochefoucauld, et le chevalier de La Rochefoucauld, qui avait dès
l'enfance la commanderie de Pézenas. Cette mort causa un grand trouble
dans la famille.

L'abbé de Tallemant mourut en même temps assez vieux, regretté de tous
les gens de lettres, et même d'assez de gens de considération dans
l'Église, et d'autres du grand monde.

Le maréchal de Villars perdit son frère de maladie, qui servait de
lieutenant général dans son armée, et était gouverneur de Gravelines.
C'était un fort honnête homme et modeste, qui rougissait souvent des
incartades du maréchal. Il était chef d'escadre, fort estimé. Son frère,
prenant le grand vol, l'avait fait passer du service de la marine à
celui de terre, où, bien qu'assez novice, il était devenu bon officier,
et fort aimé et personnellement considéré. Quelque temps après, le comte
du Bourg, depuis maréchal de France, perdit son fils unique, brigadier
de cavalerie, et mestre de camp du régiment royal. Il avait acquis de la
réputation, et ne laissa point d'enfants. Ce fut une grande douleur pour
son père. Albemarle, lieutenant général dans les troupes ennemies, et
fils du favori du roi Guillaume, avait été pris à Denain. Le prince de
Rohan fit grande connaissance avec lui, et le fît loger à Paris dans la
superbe maison que son père avait achetée. Il y eut le choix d'aller
demeurer à Chartres ou à Orléans, à lui et à cinq ou six prisonniers de
considération venus avec lui, mais il faisait grande instance d'avoir la
permission d'aller sur sa parole dans une de ses terres en Gueldre. Il
n'eut point celle de paraître à la cour. Le cardinal de Rohan, retourné
à Fontainebleau pour le serment que les cardinaux prêtent pour leurs
bénéfices, obtint, pour lui et pour les autres prisonniers qui étaient
avec lui, la liberté de s'en aller chez eux sur leur parole, et le roi
fit au cardinal la galanterie de vouloir que ce fût lui qui leur en
mandât la première nouvelle. L'état de son père le rappela promptement à
Paris.

M. de Soubise ne jouit pas longtemps du plaisir de voir son fils revêtu
de la pourpre romaine. Il mourut à Paris le 24 août, à plus de
quatre-vingt-un ans, prince avec quatre cent mille livres de rente,
étant né gentilhomme avec quatre mille livres de rente, comme il lui est
échappé quelquefois de lâcher cette parole à quelques amis particuliers
dans le transport de sa prodigieuse fortune. Elle fut le fruit d'une
prudence que peu de gens voudraient imiter, du mépris qu'il fit des
préjugés qui ont acquis le plus de force, de la leçon qu'il reçut de
l'exemple de M. de Montespan, et de la préférence qu'il donna sur un
affront obscur et demi-caché à la plus énorme fortune que lui valut la
beauté de sa seconde femme, son concert secret avec elle, l'art
merveilleux par lequel elle sut se conserver le premier crédit après que
les temps de l'acquérir furent passés, et la conduite de l'un et de
l'autre toute dressée à ce but, dont j'ai assez parlé en divers endroits
de ces Mémoires\,; et des immenses biens, établissements et grandeurs
qu'elle leur valut et par quels degrés, pour n'avoir à ajouter ici que
quelques éclaircissements sur M. de Soubise, qui était le plus beau
gendarme et un des hommes le mieux faits de son temps de corps et de
visage jusque dans sa dernière vieillesse, et qui se soucia le moins
d'encourir la plus mortelle injure qu'un Espagnol puisse dire à un
autre, qui jusque dans la lie du peuple ne se pardonne jamais.

Je me souviens qu'étant à Madrid, le marquis de Saint-Simon, qui
apprenait l'espagnol, se fâcha par la ville contre un de mes cochers\,;
et, voulant dire autre chose, l'appela\ldots{} À l'instant le cocher
arrêta, descendit de son siége, jeta son fouet au nez du jeune homme
dans le carrosse, et s'en alla sans qu'il fût possible de l'engager à
continuer de mener. On fut quatre ou cinq jours à lui faire entendre que
c'était méprise, et faute de savoir la langue ni ce que ce mot
signifiait\,; et ce ne fut qu'à force de l'en persuader qu'on parvint à
l'apaiser. Je pense bien aussi que M. de Soubise, qui se trouvait si
bien de mériter ce nom, n'eût pas souffert qu'on l'en eût appelé, car il
était fort brave homme et bon lieutenant général.

Il était fils du second duc de Montbazon et de sa seconde femme\,;
lequel était frère cadet du premier duc de Montbazon, propre neveu
paternel du marquis de Marigny, depuis comte de Rochefort, chevalier du
Saint-Esprit en 1619 parmi les gentilshommes, et le cinquante-quatrième
de cette promotion qui fut en tout de cinquante-huit, frère de père de
la connétable de Luynes, si fameuse depuis sous le nom de duchesse de
Chevreuse par son second mariage, et de ce prince de Guéméné qui eut
tant d'esprit, et qui ne fut duc et pair qu'en 1654 par la mort de leur
père\,; par conséquent, fils de cette belle M\textsuperscript{me} de
Montbazon, et beau-frère de cette princesse de Guéméné qui attrapa le
tabouret par l'adresse que j'ai racontée (t. II, p.~153, 154)\,; toutes
deux si considérées parmi les frondeurs, et dont la beauté et l'intérêt
a tant causé de cabales, les a tant fait figurer dans la minorité de
Louis XIV, et tant gouverner les premiers personnages d'alors. Cette
belle duchesse de Montbazon, mère de M. de Soubise, était Avaugour des
bâtards de Bretagne, qui ont été aussi connus sous les noms de Goello et
de Vertus, et la mère de cette Avaugour était Fouquet de La Varenne,
fille de ce fameux La Varenne, qui de fouille-au-pot, puis cuisinier,
après portemanteau d'Henri IV, et Mercure de ses plaisirs, se mêla
d'affaires jusqu'à devenir considérable, à avoir procuré le
rétablissement des jésuites en France, et avoir partagé la Flèche avec
eux, qui durent ce beau et riche collège à sa protection\,; qui devint
puissamment riche, qui se retira à la Flèche après la mort d'Henri IV,
qui fut follement frappé, volant une pie, de l'entendre dire, crier et
répéter\,: \emph{Maquereau}, d'un arbre où elle s'était relaissée, sans
qu'on pût jamais lui persuader que c'était quelque pie privée, échappée
d'un village où elle avait appris à parler\footnote{Saint-Simon a déjà
  raconté cette anecdote (t. II, p.~66).}. Il prit cela pour un miracle
pareil à celui de l'âne de Balaam, que c'était le reproche de sa vie et
des péchés qui lui avaient valu sa fortune. Il quitta la chasse, se mit
au lit\,; la fièvre lui prit, et il en mourut en deux ou trois jours.
C'est ce quartier\footnote{Ce quartier de La Varenne est celui que
  Saint-Simon appelle \emph{orde} (ignoble) dans un passage
  singulièrement altéré par les anciens éditeurs. Voy. t. II, p.~397.}
si honteux et si proche qui fit l'embarras pour Strasbourg, dont
M\textsuperscript{me} de Soubise sortit par l'adresse et l'autorité que
j'ai racontées, en faisant passer cette La Varenne, dont le nom est
Fouquet, non du surintendant Fouquet, pour être La Varenne d'une
très-bonne maison d'Anjou, éteinte lors depuis plus d'un siècle. M. de
Soubise était frère de père et de mère de la seconde femme du duc de
Luynes, qui épousa la sœur de sa mère, dont il eut le comte d'Albert,
M\textsuperscript{me} de Verue, et nombre d'autres enfants. Il était
oncle propre du duc de Montbazon, mort fou et enfermé à Liége en 1699,
et du chevalier de Rohan, décapité à Paris devant la Bastille, 27
novembre 1674\,; enfin grand-oncle du prince de Guéméné et du prince de
Montauban, lequel prince de Guéméné était père de celui d'aujourd'hui,
gendre du prince de Rohan fils de M. de Soubise, de l'archevêque de
Reims, et de plusieurs autres enfants. On n'osa hasarder, à la mort de
M. de Soubise, ce qu'il avait osé à celle de sa femme, de la faire
porter droit à la Merci, sous prétexte que cette église était vis-à-vis
de sa porte, où il la fit enterrer. Son corps fut porté à la paroisse
comme tous, et de là à la Merci tant qu'ils voulurent. Le cardinal de
Noailles avait mis ordre à ce que cette entreprise et surprise ne fût
pas réitérée.

Je perdis en même temps le marquis de Saint-Simon, aîné de la maison.
Son père et son frère avaient mangé obscurément plus de quarante mille
livres de rente, sans sortir de chez eux. Ce cadet s'était mis, faute de
mieux, dans le régiment des gardes, où par ancienneté il était devenu
capitaine et brigadier, fort aimé et estimé. Il avait dîné avec moi à
Fontainebleau quatre ou cinq jours auparavant. Je présentai son fils
tout jeune au roi, qui n'était pas encore dans le service\,; le roi
sur-le-champ lui donna une lieutenance aux gardes.

M\textsuperscript{me} de La Fayette mourut assez jeune d'une longue
apoplexie, fille unique fort riche de Marillac, doyen du conseil. Elle
ne laissa que la duchesse de La Trémoille, sa fille unique.
M\textsuperscript{me} de La Fayette, si connue par son esprit, était
belle-mère de celle-ci.

Cassini, le plus habile mathématicien et le plus grand astronome de son
siècle, mourut à l'Observatoire de Paris, à quatre-vingt-six ans avec la
tête et la santé entière. M. Colbert, qui voulait relever en France les
sciences et les arts, et qui avait fait bâtir l'Observatoire, attira par
de grosses pensions plusieurs savants étrangers. Celui-ci florissait à
Bologne sa patrie\,; il avait déjà rendu son nom célèbre par de grandes
découvertes, lorsque M. Colbert le fit venir avec sa famille\,; il les
augmenta depuis beaucoup, et fort utilement pour la navigation. Il
demeura à l'Observatoire toute sa vie, qu'il gouverna. Son fils y
remplit sa place avec presque autant de réputation en France et dans les
pays étrangers, où ils furent l'un et l'autre agrégés aux plus célèbres
académies. Ce rare savoir fut également rehaussé en l'un et en l'autre
par leur modestie et leur probité. Ce P. Cassini, capucin prédicateur du
pape, que Clément XI (Albani) fit cardinal en cette année, était du même
nom, et parent éloigné de ces illustres astronomes.

Refuge mourut en même temps\,: c'était un très-honnête homme et
très-vertueux, avec de l'esprit, parfaitement modeste, d'une grande
valeur, avec de la capacité à la guerre. Il était ancien lieutenant
général, gouverneur de Charlemont, et commandait à Metz. C'était le plus
savant homme de l'Europe en toutes sortes de généalogies, et de tous les
pays, depuis les têtes couronnées jusqu'aux simples particuliers, avec
une mémoire qui ne se méprenait jamais sur les noms, les degrés ni les
branches, sur aucune date, sur les alliances, ni sur ce que chacun était
devenu. Il était fort réservé la-dessus, mais sincère quand il faisait
tant que de parler. Il se peut dire que sa mémoire épouvantait. Un
courrier, qu'il reçut à Metz d'un de ces seigneurs allemands du Rhin, en
pensa tomber à la renverse en lui rendant son paquet de la part de son
maître. «\,J'ai bien l'honneur de le connaître,\,» lui dit Refuge, et
tout de suite lui en détailla toute la généalogie. Il était honorable,
mais sobre et fort distrait. Ses valets quelquefois en abusaient, et lui
portaient tout de suite des sept ou huit verres de vin qu'il ne
demandait point et qu'il avalait sans y penser. Il se grisait de la
sorte\,; et quand cela était passé, il ne comprenait pas comment cela
lui était arrivé. Il était vieux, et laissa une fille mariée au fils
unique du comte du Luc, et un fils unique non marié, aussi vertueux que
lui, aussi brave, et qui sert d'officier général avec réputation, mais
qui, avec la même modestie, n'est pas si généalogiste. Il ne faut pas
omettre la mort de M\textsuperscript{me} Herval, quoique personne
purement de la ville. On a rarement vu ensemble tant de vertu, de
sagesse, de piété également soutenue toute sa vie, dans la plus simple
modestie, avec une si parfaite et si durable beauté. Elle était sœur de
Bretonvilliers, lieutenant de roi de Paris, qui venait de mourir
subitement, et veuve d'Herval, fort enrichi sous M. Fouquet, depuis
intendant des finances, fort dans le grand monde, et le plus gros joueur
de son temps. Elle n'avait point d'enfants\,; c'était une femme qui avec
du monde, de l'esprit et de la politesse, s'était toujours fort retirée,
qui avait refusé de grands mariages pour sa beauté, sa vertu, et ses
biens dont elle faisait de grandes aumônes, et qui depuis longtemps
s'était mise dans un couvent où elle voyait à peine sa plus proche
famille.

L'abbé Servien fut chassé de Paris, et envoyé je ne me souviens plus où.
Il était frère de Sablé et de la feue duchesse de Sully, tous enfants du
surintendant des finances. Rien de si obscur ni de si débordé que la vie
de ces deux frères, tous deux d'excellente compagnie et de beaucoup
d'esprit. L'abbé était à l'Opéra, où on chantait au prologue un refrain
de louange excessive du roi, qui se répéta plusieurs fois. L'abbé,
impatienté de tant de servitude, retourna le refrain fort plaisamment à
contre-sens, et se mit à le chanter tout haut d'un air fort ridicule,
qui fit applaudir et rire à imposer silence au spectacle. L'exil ne dura
pas\,; il y fit le malade, et le mépris que faute de mieux on voulut
montrer aida fort à la liberté de son retour. Il ne paraissait jamais à
la cour, et peu à Paris, en compagnies honnêtes. Ses goûts ne l'étaient
pas, quoique l'esprit fût orné et naturellement plaisant, de la fine et
naturelle plaisanterie, sans jamais avoir l'air d'y prétendre. Il mourut
comme il avait vécu, d'une misérable façon, chez un danseur de l'Opéra
où il fut surpris. Il est pourtant vrai qu'avec cette vie il disait
exactement son bréviaire, ainsi que le cardinal de Bouillon. Il y eut en
ce temps-ci un débordement de loups qui firent de grands désordres dans
l'Orléanais\,; l'équipage du roi pour le loup y fut envoyé, et les
peuples furent autorisés à prendre des armes et à faire quantité de
grandes battues.

\hypertarget{chapitre-x.}{%
\chapter{CHAPITRE X.}\label{chapitre-x.}}

1712

~

{\textsc{Renonciations exigées par les alliés en la meilleure et plus
authentique et sûre forme pour empêcher à jamais la réunion sur la même
tête des monarchies de France et d'Espagne.}} {\textsc{- Mesures sur ces
formes,}} {\textsc{- Formes des renonciations traitées entre les ducs de
Chevreuse, de Beauvilliers et moi, puis avec le duc de Noailles, qui
s'offre à en faire un mémoire, et qui le fait faire, et enfin le donne
pour sien.}} {\textsc{- Intérêt de M. le duc de Berry et de M. le duc
d'Orléans à la solidité des Renonciations et de leurs formes, qui n'ont
que moi pour conseil là-dessus.}} {\textsc{- Sentiments de M. le duc de
Berry à l'égard du duc de Beauvilliers.}} {\textsc{- Aux instances du
duc de Beauvilliers, je fais un mémoire sur les formes à donner aux
renonciations\,; le voir parmi les Pièces.}} {\textsc{- Division de
sentiment sur un point des formes entre le duc de Noailles et moi.}}
{\textsc{- Sa conduite là-dessus.}} {\textsc{- Le duc de Noailles gagne
à son avis le duc de Chevreuse.}} {\textsc{- Danger de sa manière de
raisonner.}} {\textsc{- Le duc de Chevreuse nous propose d'en passer par
l'avis du duc de Beauvilliers, qui nous assemble chez le duc de
Chevreuse.}} {\textsc{- Le duc de Chevreuse, et moi après, exposons à la
compagnie nos différentes raisons.}} {\textsc{- Le duc de Beauvilliers
se déclare de mon avis et malmène fort le duc de Chevreuse, qui se rend,
et le duc de Noailles aussi.}}

~

La paix se trouvait à peu près arrêtée entre la France et l'Angleterre
qui se faisait fort d'y faire passer ses alliés. J'ai déjà averti plus
d'une fois que je passais le détail de ce grand événement sous silence,
parce qu'il se trouvera de main de maître dans les Pièces, depuis le
voyage de Torcy à la Haye inclusivement, jusqu'à la signature de la paix
à Utrecht. Torcy lui-même en a fait toute la relation qu'il m'a
communiquée, et c'en est la copie fidèle qu'on verra dans les Pièces. Je
n'ai donc à ajouter à ce morceau si curieux de l'histoire de nos jours
que ce qui n'a pu être dans cette importante relation, parce que, ne
faisant pas partie de la négociation, Torcy n'a pas été en état de
l'écrire quoique ayant un rapport direct à l'affaire de la paix, qu'il
n'a pas ignorée, comme on le verra\footnote{Voy. les Mémoires de Torcy,
  qui font partie de toutes collections de Mémoires relatifs à
  l'histoire de France.}. Nos malheurs domestiques et redoublés firent
naître une difficulté qui accrocha la paix déjà réglée à Londres, et qui
la retarda beaucoup. La reine Anne et son conseil furent arrêtés par la
considération du droit du roi d'Espagne de succéder à la couronne de
France, si l'auguste et précieux filet qui seul l'en excluait venait à
se rompre, et de ce qu'il n'était pas possible à l'Angleterre, ni à
aucune autre des puissances en guerre, de consentir à voir sur une même
tête les deux premières couronnes de l'Europe. La difficulté fut donc
proposée\,; le roi n'était pas en état de ne s'y pas rendre\,; il fallut
donc travailler à la lever d'une manière si solide que le cas ne pût
jamais arriver, et que toutes les puissances pussent être là-dessus en
entière sûreté. Elles étaient justement alarmées de l'exemple récent du
succès des renonciations du roi, si solennellement faites par le traité
des Pyrénées et par celui de son mariage conclu en même temps par les
deux premiers ministres de France et d'Espagne, assemblés en personne et
qui les avaient signées en public après vingt-quatre conférences tenues
ensemble aux frontières des deux royaumes, dans l'île des Faisans, sur
la rivière de Bidassoa, jurées ensuite par les deux rois en personne, en
présence l'un de l'autre et en public, à leur entrevue dans la même île,
en accomplissant le mariage.

Le testament de Philippe V ne leur était pas une réponse. On n'avait pas
oublié les écrits que le roi fit publier, quatre ou cinq ans après la
paix des Pyrénées, lorsqu'à la mort du roi d'Espagne il se saisit d'une
grande partie des Pays-Bas espagnols et de la Franche-Comté, sous
prétexte des droits de la reine\,; et le traité de partage, auquel
l'empereur, seul de toute l'Europe, avait refusé de consentir, était une
autre raison bien forte pour faire tout craindre là-dessus. Une
troisième n'était pas oubliée\,: les mêmes renonciations avaient été
faites par le traité du mariage de Louis XIII, et néanmoins peu de temps
après que Philippe V fut arrivé en Espagne, il y fit reconnaître et
rétablir, au préjudice de ces mêmes renonciations, le droit à la
couronne d'Espagne de M. le duc d'Orléans, tiré par lui de la reine sa
grand'mère, épouse de Louis XIII. En effet, c'en était trop pour ne pas
engager toute l'Europe à prendre ses précautions, et à s'assurer d'une
manière solide. Mais c'était là où consistait l'embarras\,; les traités,
les renonciations, les serments, parurent une faible ressource après ces
exemples. On chercha donc quelque chose de plus fort\,; on ne le put
trouver dans la chose même parce qu'il n'y en a point de plus sacrées
parmi les hommes que celles-là auxquelles on ne croyait pas pouvoir se
fier\,; il fallut donc se tourner du côté des formes pour suppléer par
la plus grande solennité qu'on y pourrait donner.

On fut longtemps là-dessus, et, bien que le roi offrît tout ce qu'on lui
pourrait demander pour rassurer l'Europe contre le danger de voir jamais
les deux couronnes sur la même tête, il ne voulait rien accorder en
effet, non pour réserver aux siens une porte de derrière, mais par
l'entêtement de son autorité, à laquelle il croyait que toute forme
donnait atteinte, puisqu'on en désirait pour appuyer cette même autorité
et y ajouter une solidité entière. Il était blessé là-dessus dans sa
partie la plus sensible, absolu sans réplique comme il s'était rendu, et
ayant éteint et absorbé jusqu'aux dernières traces, jusqu'aux idées,
jusqu'au souvenir de toute autre autorité, de tout autre pouvoir en
France qu'émané de lui seul. Les Anglais, peu accoutumés chez eux à de
pareilles maximes, et qui voulaient leur sûreté et celle de leurs alliés
à qui, quand ils l'auraient voulu, ils n'auraient pas persuadé de passer
légèrement ce grand article, insistèrent, et proposèrent les états
généraux du royaume pour y déclarer et y faire accepter les
renonciations. Ils disaient avec raison qu'il ne suffisait pas à la
vérité de la chose, ni par conséquent à la sûreté de l'Europe, que le
roi d'Espagne renonçât au royaume de France, si le royaume de France ne
renonçait aussi à lui et à sa postérité en acceptant et ratifiant sa
renonciation\,; que cette formalité était nécessaire pour rompre en même
temps le double lien qui attachait la branche d'Espagne à la France,
comme la France l'était aussi à la branche d'Espagne.

Les Anglais, accoutumés à leurs parlements qui sont en effet leurs états
généraux, croyaient aux nôtres la même autorité. Ils en ignoraient la
nature, et la mesurant à celle des leurs, ils en voulaient appuyer et
consolider les renonciations par une autorité, dans leur idée, légale,
la plus grande qui pût être réclamée, et qui appuyât le plus solidement
l'autorité du roi. Lui montrer se défier de la faiblesse de la sienne,
il est inexprimable l'effet de ce doute dans l'âme d'un prince presque
déifié à ses propres yeux, et dans l'usage intérieur et constant du plus
illimité despotisme. Lui faire apercevoir qu'on croyait trouver dans ses
sujets une autorité confîrmative de la sienne, c'était un attentat au
premier chef le plus sensible, qu'une couronne ne pouvait courir. On fit
entendre aux Anglais la faiblesse et l'inutilité du secours d'autorité
qu'ils demandaient. On leur expliqua la nature et l'impuissance de nos
états généraux, et ils comprirent enfin combien leur concours serait
vain quand même il serait accordé.

On leur disait vrai, mais on se gardait bien en même temps de leur
enseigner où résidait par nature, par droit, et par exemple, ce qu'ils
cherchaient sans pouvoir le trouver, ou peut-être sans le vouloir, à
cause de Philippe de Valois et de la loi salique. Quoi qu'il en soit, on
fut longtemps à battre l'eau\,: la France à dire qu'un traité des
renonciations, une déclaration du roi expresse et confirmative
enregistrée au parlement, suffisait\,; les Anglais à répliquer par
l'événement des renonciations, traités, contrats de mariage de Louis
XIII et de Louis XIV\,; et cependant la paix, toute convenue avec les
Anglais, et fort au-dessus de nos espérances, demeurait accrochée. Les
renonciations étaient consenties en France et en Espagne, où il n'y
avait point de difficulté pour la forme, comme il sera expliqué en son
lieu\,; mais tout était arrêté sur celles de France. C'est ce qui fit
dépêcher de Londres Bolingbroke à Fontainebleau, dont tout le personnel,
voyage, jusqu'à la réception et les moindres particularités, sont si
bien expliquées dans les Pièces, que je m'abstiendrai d'en rien dire
ici.

Dès la naissance de la difficulté, elle avait été traitée entre les ducs
de Chevreuse, de Beauvilliers et moi. Le duc d'Humières y fut admis
quelque temps après en quatrième, et le duc de Noailles, qui les
cultivait avec grand soin depuis que je l'avais raccommodé avec eux,
avait si bien fait qu'ils voulurent bien qu'il entrât en cinquième dans
cette grande affaire. II se piquait de lecture, de bibliothèque, de
commerce de gens instruits à fond dans notre histoire, et de l'être fort
lui-même\,; et pour en dire la vérité, il était quelquefois difficile de
n'être pas souvent ébloui de son esprit, de son débit et de sa vaste
superficie. Mais dans ces cinq personnes il n'y avait que M. de
Chevreuse de véritablement instruit. M. de Beauvilliers ne s'était
jamais adonné à fond à cette étude, et il y avait longues années qu'il
n'avait pas même le temps de lire par le nombre de ses fonctions. M.
d'Humières s'en piquait encore moins\,; et M. de Noailles, qui écorchait
la superficie de tout, n'avait jamais pu rien approfondir en aucun
genre. Je n'aurai pas la hardiesse ni la fatuité de me nommer\,; je me
soumets très-sincèrement au jugement qu'on voudra porter en examinant ce
qui s'en trouvera dans les Pièces. Toutefois nous tombâmes aisément
d'accord sur ce que je représentai, qui fut approuvé et appuyé par le
duc de Chevreuse. Mais il fallut après entrer dans le détail, et ce fut
un travail qui ne convenait pas au peu de loisir du duc de Chevreuse
qui, comme on l'a vu, ministre en effet sans le paraître, était tout
occupé des affaires d'État. M. de Beauvilliers en son genre, et M.
d'Humières au sien, s'en pouvaient encore moins charger. Je me trouvai
les reins trop faibles\,; tellement que le duc de Noailles s'offrit de
lui-même de faire un mémoire qui embrassât toute la matière, et qui
expliquât toute la forme, par preuves et par raisons, de consolider les
renonciations au gré des Anglais d'une manière ferme, stable et légale,
et il promit aux ducs de Chevreuse et de Beauvilliers, en notre
présence, qu'il serait fait, et en état de le donner à eux et à nous
avant le départ de la cour pour Fontainebleau, pour l'examiner et le
lire après entre nous cinq ensemble.

Ce fut dans cet intervalle que le duc de Charost fut admis en sixième
par MM. de Chevreuse et de Beauvilliers, et ce fut le dernier qu'on y
reçut. Il y avait encore du temps jusqu'au voyage. De fois à autre je
demandais au duc de Noailles des nouvelles de son travail, les autres
lui en parlaient aussi\,; il nous assurait toujours qu'il avançait et
qu'il tiendrait parole. Restait pourtant la plus grande difficulté\,:
c'était d'amener le roi à consentir à ces formes\,; et MM. de Chevreuse
et de Beauvilliers, dont ce devait être l'ouvrage particulier par leur
familiarité et surtout par leur caractère de ses ministres, en étaient
fort en peine. Mais, persuadés qu'il n'y en avait point d'autres qui
pussent opérer validité et sûreté, que celles-là étaient les seules,
qu'elles ne seraient même employées que par l'expresse volonté du roi,
ils se flattèrent qu'il pourrait se laisser persuader que par là son
autorité serait à couvert, et que, pressé à l'excès comme il l'était de
la nécessité de la paix et de la fermeté des Anglais à ne passer pas
outre sans être pleinement satisfaits sur la stabilité légale des
renonciations, il pourrait à la fin se résoudre, en faveur d'un si grand
bien que ses forces épuisées ne lui permettaient plus de différer, et à
des conditions si disproportionnées de toutes les précédentes, dont les
offres étaient encore si présentes à son esprit.

Dans cet état de choses, j'étais en presse avec M. le duc de Berry et M.
le duc d'Orléans. Celui-ci me croyait instruit des formes nécessaires
pour la validité des renonciations, et il en avait aisément persuadé
l'autre. L'un, isolé et fui depuis le paquet des poisons, n'avait que
moi à qui parler et à qui consulter. Indépendamment de l'état où M. du
Maine et M\textsuperscript{me} de Maintenon l'avaient réduit avec la
cour et le monde, il n'avait personne avec qui traiter une matière si
délicate\,; et M. le duc de Berry timide à l'excès, sous le joug dur et
jaloux du roi, avait encore moins à qui parler là-dessus. Il n'avait pas
pour M. de Beauvilliers l'ouverture et la confiance de son incomparable
frère. Il avait toujours présente une éducation qui lui avait paru dure
par son peu de goût pour l'étude\,; par la sévérité avec laquelle il
était contenu dans le respect pour son aîné, avec lequel, sans préjudice
de la plus tendre et de la plus réciproque amitié, il était enclin à
s'échapper\,; et par le sérieux d'un gouverneur toujours en garde, et
qui, dans la crainte de ce qui pouvait arriver un jour, était
particulièrement occupé de le tenir bas, pour qu'il s'accoutumât à se
tenir dans les bornes de la dépendance à l'égard d'un frère destiné à
devenir son roi. Il ne voyait pas en même temps tout ce que le
gouverneur faisait auprès de ce frère pour entretenir l'égalité entre
eux, lui faire sentir celle que la nature y avait mise jusqu'à ce que
l'aînesse eût à user de son droit, et alors même la bienséance, la
douceur, la solidité de repos et de sûreté à vivre avec son cadet en
père, en frère, en ami tout à la fois. Il n'y avait pas assez longtemps
que M. le duc de Berry était sorti d'entre ses mains pour voir cette
conduite telle qu'elle était, et telle qu'elle devait être considérée.
Meudon, par où il avait commencé à respirer quelque air de liberté,
n'était pas une cour propre à lui donner là-dessus des idées
raisonnables\,; aussi peu les jeunes dames de la cour de sa délicieuse
belle-sœur avec qui il avait passé ses moments les plus libres\,; et
M\textsuperscript{me} la duchesse de Berry, telle qu'on a pu la voir en
quelques endroits de ces Mémoires, n'était bonne qu'à l'écarter de plus
en plus du duc de Beauvilliers. Dans cette situation de ces deux
princes, j'étais le seul qu'ils pussent et voulussent consulter.

La confiance de M. le duc d'Orléans en moi, communiquée par lui à M. le
duc de Berry, était aidée de la commodité à son égard de ma position,
par la place que le roi avait forcé M\textsuperscript{me} de Saint-Simon
de prendre auprès de M\textsuperscript{me} la duchesse de Berry. Tous
deux avaient le plus grand intérêt à ne pas renoncer à la couronne
d'Espagne d'une manière solide et sans retour par les lois du pays, sans
que toutes les précautions fussent également prises pour leur assurer la
couronne de France par une renonciation aussi solide et aussi sans
retour du roi d'Espagne et de sa postérité\,; et c'était là sur quoi ils
me consultaient. J'avais temporisé avec eux aisément, sous prétexte de
la difficulté de la matière qu'il fallait approfondir, discuter, étudier
à fond\,; mais à la fin ils me pressèrent, pressés eux-mêmes par les
nouvelles d'Angleterre.

J'avais eu occasion trop souvent, dans des temps d'oisiveté et de
loisir, de causer et de raisonner d'histoire avec M. le duc d'Orléans,
pour qu'il me pût croire absolument neuf sur ces matières. Il ne le
laissa pas ignorer à M. le duc de Berry, et tous deux se mirent à me
presser vivement. Je ne laissai pas de tergiverser encore\,; mais
lorsque je vis que nous étions d'accord, les cinq que j'ai nommés, sur
la forme à proposer, et qu'il ne s'agissait plus que du mémoire dont le
duc de Noailles s'était chargé, je ne crus pas devoir amuser plus
longtemps deux princes si fort intéressés, qui prenaient en moi toute
confiance là-dessus, et qui n'avaient personne autre en qui la pouvoir
prendre. J'expliquai donc ce que je pensais là-dessus à M. le duc
d'Orléans, qui était fort instruit lui-même de notre histoire\,; et la
discussion de cette importante matière dura plusieurs conversations
longues entre lui et moi. Je voyais peu M. le duc de Berry et comme
point en particulier, et comme il était peu instruit il aurait fallu
plus de temps avec lui. Je ne voulus rien qui pût être remarqué\,; ainsi
M. le duc d'Orléans, bien persuadé de la solidité unique de ce que je
lui proposai, se chargea d'en informer M. le duc de Berry, qu'il
persuada parce qu'il l'était lui-même. Je ne voulus point que M. le duc
de Berry m'en parlât, parce que ce n'aurait pu être qu'en particulier,
ni M\textsuperscript{me} la duchesse de Berry par la même raison, et,
comme je l'ai dit ailleurs, que je ne voyais plus que très-rarement, et
un moment en public. M. le duc d'Orléons et M\textsuperscript{me} de
Saint-Simon étaient des canaux qui y suppléaient aisément, et par qui je
sus aussi combien ils étaient contents, et persuadés qu'il n'y avait
aucun autre moyen solide que celui que j'avais proposé à M. le duc
d'Orléans.

Ces choses en étaient là aux approches du voyage de Fontainebleau, et M.
le duc de Noailles n'avait pas encore achevé son mémoire. Il s'excusa
sur l'importance de la matière et le nombre de choses qu'il fallait
examiner, puis choisir et ranger\,; mais il nous assura toujours qu'il
serait en état de nous montrer le mémoire dans les premiers jours que le
roi serait à Fontainebleau, où nous allions tous en même temps que lui,
à deux ou trois jours près. Les détails se prolongèrent, et nous
découvrîmes qu'il avait des gens obscurs cachés tout au haut de son
logement dans la galerie de Diane qui donne sur le jardin, qu'il faisait
travailler, dont il refondait continuellement l'ouvrage, qui par là ne
finissait jamais. La découverte ne lui fut point cachée, il ne put si
bien la dissimuler que la chose ne demeurât comme avouée, dont il
demeura fort embarrassé.

M. de Beauvilliers, extrêmement pressé par les instances des Anglais, ne
voulut plus s'attendre au duc de Noailles. Il me pria de faire le
mémoire. Je m'en défendis par beaucoup de raisons, et en effet, je
n'avais apporté à Fontainebleau que peu de livres, et aucun qui pût me
servir à un travail auquel je n'avais aucun lieu de m'attendre. J'eus
beau dire et alléguer les meilleures excuses, il fallut céder à
l'autorité qu'il avait sur moi. Je me mis donc à travailler dans un lieu
où je n'avais aucun secours, et où je n'avais pas la liberté de le
faire. Il fallait être assidu aux heures de cour que j'avais accoutumé
de prendre, manger en compagnie\,; et Fontainebleau était le lieu du
monde où on se rassemblait, et où on s'invitait le plus à dîner et à
souper. J'avais encore à faire face au monde et à mes sociétés
ordinaires, parce qu'il ne fallait pas laisser soupçonner que je fusse
occupé à rien de sérieux. Mon travail était donc fort interrompu\emph{,
}qui est la chose du monde la plus nuisible à bien faire, surtout en
telles matières. J'avais souvent recours aux nuits.

Je ne sais pourquoi alors j'étais épié plus qu'à l'ordinaire, quoique je
le fusse toujours. M\textsuperscript{me} de Saint-Simon ne put venir à
Fontainebleau cette année, à cause des suites d'une rougeole. Nous nous
écrivions tous les jours\,; et quoique nous ne nous mandassions jamais
que des riens par la poste, nous ne reçûmes pas une seule lettre, moi
d'elle, elle de moi, par la poste que très-visiblement décachetée. C'est
ce qui me fit tenir encore plus soigneusement sur mes gardes pour éviter
de paraître retiré, et ce qui rendit mon travail plus coupé et plus
difficile. M. de Beauvilliers logeait dans la galerie de Diane,
vis-à-vis du duc de Noailles, et ces deux logements leur appartenaient
de tous temps. J'étais à l'autre bout du château, au-dessus d'une partie
de l'appartement de la reine mère, et j'avais des fenêtres qui donnaient
sur la cour du Cheval-Blanc, et de l'autre côté sur la cour des
Fontaines. Tous les soirs M. de Beauvilliers traversait tout cet espace
seul, sans laquais, ni flambeau, ni personne avec lui, montait mon degré
assez court à tâtons, et pendant le souper du roi me faisait lui lire ce
que j'avais écrit depuis la veille. Il était environ une heure avec moi,
et s'en retournait seul comme il était venu. Le duc de Noailles, seul de
nous cinq, ignorait que je travaillasse\,; et le duc de Beauvilliers fut
le seul qui vit ce que je faisais avant que ce fût achevé. Il en fut
content, et il le dit aux trois autres. Cependant le duc de Noailles
faisait suer ses inconnus dans son grenier\,: et il en sortit enfin un
assez court mémoire, comme le mien était tout près de s'achever.

Je ne ferai point ici d'analyse de l'un ni de l'autre\,; mais je dirai
d'autant plus franchement que celui du duc de Noailles était, à la
diction près, fort médiocre, pour en parler modestement, et qu'il n'y
avait de lui que la seule diction. Le sien et le mien convenaient pour
le principal et l'essentiel. Le mien se trouve dans les Pièces. Je
l'avais intitulé\,: \emph{Mémoire succinct sur les formes}, etc.
L'abondance de la matière et la nécessité des preuves m'emportèrent
tellement que, de succinct que je comptais qu'il serait, je fis un gros
ouvrage. La longueur dont en serait même l'extrait m'empêche d'en rien
insérer ici, mais il faut le voir dans les Pièces, pour entendre la
dispute dont je vais parler et dont l'explication serait ici trop
longue. Ainsi je suppose que je la vais raconter à qui a lu le
\emph{Mémoire}, prétendu succinct, \emph{sur les formes}, etc., qui est
dans les Pièces.

Le duc de Noailles et moi, raisonnant sur la matière, nous aperçûmes
bientôt tous deux qu'il y avait un point sur lequel nous n'étions pas
d'accord. J'estimais qu'on ne pouvait employer que les ducs-pairs, et
même vérifiés, et aussi les officiers de la couronne. Le duc de Noailles
croyait, ou voulait croire, qu'il y fallait joindre les gouverneurs de
province et les chevaliers de l'ordre, en faveur de la noblesse, auprès
de laquelle je n'ai que trop reconnu depuis qu'il s'en voulait dès lors
faire un mérite.

Nous disputâmes. Je lui objectai l'impuissance, par lui-même avouée, des
états généraux, par conséquent celle de la noblesse, qui n'en est que le
second des trois ordres qui les forment, encore plus d'un extrait aussi
peu nombreux de ce second ordre. Je lui représentai que les ducs et les
officiers de la couronne étaient eux-mêmes de ce même second ordre,
quoique, par leurs fiefs et leurs offices, nécessairement capables de ce
qui passait le pouvoir des états généraux, qui n'avaient que celui de
porter au roi les représentations et les supplications des provinces qui
les députaient, et les remèdes aux besoins et aux maux que les provinces
les avaient chargés de présenter au roi pour être examinés. Je lui fis
remarquer le peu de poids personnel que ceux qu'il voulait admettre,
quand bien même ils seraient admissibles, ajouteraient, non qu'ils
dussent être exclus, s'ils pouvaient ne le pas être, mais qui, n'étant
pas de nature admissible, ne laissaient rien à regretter, et qu'il se
trompait grandement, s'il croyait flatter la noblesse par l'admission
qu'il prétendait, puisqu'elle ne le pourrait être qu'autant qu'elle
serait elle-même admise, non en la personne de ceux qui le seraient
comme nés par leur état de gouverneurs de province et de chevaliers de
l'ordre, mais seulement en celles de ceux qu'il lui serait permis à
elle-même de choisir et de députer. J'ajoutai que le premier des trois
ordres, qui était le clergé, voudrait dès lors ne se pas contenter des
pairs ecclésiastiques, puisque la noblesse ne se contenterait pas des
ducs et des officiers de la couronne, quoique de son même ordre\,; que,
par une suite nécessaire le tiers ordre, surtout les parlements,
auraient la même prétention, avec d'autant plus d'apparence qu'à la
différence des deux premiers ordres il ne s'y trouvait de leur personne
d'admis que le seul chancelier, qui même n'en était comme plus par son
office de la couronne\,; que cela retomberait donc dans les états
généraux, c'est-à-dire dans ce qui n'avait nulle autorité, et dans ce
qui se trouvait impraticable. À ces raisons nulle réponse de M. de
Noailles que la convenance d'honorer les gouverneurs de province et les
chevaliers de l'ordre\,; et moi de répondre qu'il ne s'agissait, en
chose de cette qualité, ni de convenance, ni de complaisance, mais de la
stabilité immuable par sa légalité d'un acte à faire pour assurer le
repos du royaume, l'état des princes de la maison royale sur la
succession à la couronne, la foi des puissances avec qui la paix ne se
pouvait conclure qu'en assurant pour toujours la tranquillité de
l'Europe\,; ce qui ne se pouvait qu'en se restreignant, pour la loi à
faire, à ceux qui en avaient le pouvoir, et en se gardant de la rendre
nulle en y admettant comme législateurs ceux qui n'avaient rien qui les
pût rendre tels.

Beaucoup d'esprit, de discours et de paroles éloquentes servirent à M.
de Noailles à la place de réponses et de raisons. Il convint qu'on s'en
pouvait tenir à mon avis\,; et néanmoins il voulut, deux jours après,
m'en reparler encore. Voyant qu'il ne réussissait pas en raisons, il
prit le parti de tenter l'autorité. Il alla parler au duc de Chevreuse
sans m'en dire mot. Il espéra de le gagner par son bien-dire, et que,
l'ayant pour lui, le duc de Beauvilliers serait emporté, après quoi la
chose demeurerait décidée. En effet, il persuada M. de Chevreuse, qui,
avec tout son savoir, n'avait pas présentes des choses depuis si
longtemps oubliées, parce qu'on n'avait pas eu besoin d'y avoir recours.
M. de Chevreuse m'en parla\,; et ce fut ce qui m'apprit que M. de
Noailles l'avait informé de notre dispute, dont pourtant il n'avait osé
lui demander de me faire un secret.

M. de Chevreuse, avec tout le savoir, toutes les lumières, toute la
candeur que peut avoir un homme, était sujet à raisonner de travers. Son
esprit, toujours géomètre, l'égarait par règle, dès qu'il partait d'un
principe faux\,; et comme il avait une facilité extrême et beaucoup de
grâce naturelle à s'exprimer, il éblouissait et emportait, lors même
qu'il s'égarait le plus, après s'être ébloui lui-même, et persuadé qu'il
avait raison. C'est ce qui lui arriva dans la conduite particulière de
ses affaires domestiques, qu'il crut sans cesse augmenter, puis
raccommoder, et qu'il détruisit géométriquement par règles, par
démonstrations, qui le menèrent à une ruine tellement radicale qu'il
serait mort de faim sans le gouvernement de Guyenne, et
M\textsuperscript{me} de Chevreuse après lui, à qui il ne resta rien que
les trente mille livres de pension que le roi mit pour elle sur les
appointements de ce gouvernement. En autres affaires on l'a vu, en leur
lieu, être pour M. de Luxembourg, pour d'Antin, pour les prétentions les
plus chimériques, se bercer soi-même de l'ancienneté de Chevreuse, du
cardinal de Lorraine, et de sa succession à la dignité de Chaulnes, et
cela à force de faux raisonnements entés l'un sur l'autre, toujours à la
manière des géomètres, et de la meilleure foi du monde. C'est donc ce
qui lui arriva sur cette affaire. Nous disputâmes, nous ne nous
persuadâmes point\,; il fut néanmoins question de nous fixer tous à
l'une ou à l'autre opinion, pour marcher après en conséquence. Le duc de
Noailles n'insista plus avec moi, comptant sur M. de Beauvilliers par
avoir gagné M. de Chevreuse. De mon côté je ne recherchai pas une
dispute inutile, mais je crus devoir rendre compte aux trois autres de
cette division d'avis. Quelque grande que fût la liaison des ducs de
Charost et d'Humières avec le duc de Noailles, depuis l'alliance du
premier par le mariage de sa fille unique avec le duc de Grammont, et de
Charost depuis surtout qu'il était capitaine des gardes, je n'eus pas de
peine à les avoir de mon côté. Le duc de Noailles se consola aisément de
n'avoir pas persuadé deux hommes qu'il ne regardait pas comme pouvant
emporter la balance\,; et il avait raison de croire que nous nous
rendrions tous trois à l'autorité, si le duc de Beauvilliers, comme il
n'en doutait pas, était emporté par le duc de Chevreuse. Ce dernier me
proposa donc que la chose fût discutée en sa présence, et que, de
quelque côté qu'il se rangeât, tous y acquiesçassent. J'y consentis avec
plaisir, et je répondis pour MM. de Charost et d'Humières. Le duc de
Noailles, qui comptait l'emporter par là, accepta pareillement. J'avais
déjà parlé à M. de Beauvilliers de cette dispute, mais légèrement\,; M.
de Chevreuse aussi. M. de Beauvilliers, qui alors se trouvait fort
occupé des affaires, ne voulait point perdre inutilement son temps, et
nous avait dit à l'un et à l'autre qu'il fallait nous assembler, et là
décider et convenir sur les raisons de part et d'autre\,; et ç'avait été
là-dessus que M. de Chevreuse nous avait proposé séparément, au duc de
Noailles et à moi, d'en passer par l'avis dont serait M. de
Beauvilliers. Le duc de Noailles me parla après de cette proposition de
M. de Chevreuse. Lui et moi nous la fîmes aux ducs de Charost et
d'Humières, qui en convinrent aisément. L'affaire pressait, et les
Anglais voulaient savoir à quoi s'en tenir. Ainsi M. de Beauvilliers,
comme le plus occupé, ne tarda pas à nous donner l'après-dînée qu'il se
prévoyait la plus libre, et voulut que nous nous assemblassions dans la
petite chambre de l'appartement du duc de Chevreuse, qui était de
plain-pied à la cour des Fontaines, du côté le plus proche de la
chapelle, sous une partie de l'appartement de la reine mère. Nous
arrivâmes tous presque en même temps.

M. de Beauvilliers ne voulut pas qu'on dît un mot de ce qui nous
assemblait que tous ne fussent arrivés. Alors il pria la compagnie
d'entrer en matière. C'était à qui voulait inclure à ouvrir pour en
proposer les raisons, et à qui voulait exclure à les réfuter, qui par
conséquent ne pouvaient parler qu'après les autres. Ainsi, après un
petit mot en gros de ce qui nous assemblait, M. de Beauvilliers regarda
les ducs de Chevreuse et de Noailles, et les pria d'exposer ce qu'ils
avaient à dire. Il y eut entre eux un court combat de civilité à qui
prendrait la parole. M. de Chevreuse la voulait laisser à M. de
Noailles, de qui venait l'avis qu'il avait embrassé. M. de Noailles, par
déférence à l'âge et à l'ancienneté, aux lumières, et encore plus à
l'effet qu'il en attendait sur le duc de Beauvilliers, voulut absolument
lui laisser la parole. M. de Chevreuse la prit donc\,; et, pour ne pas
allonger ce récit, je dirai tout court que je ne vis jamais soutenir une
mauvaise cause avec tant de grâce, d'esprit, d'éloquence et
d'élégance\,; et, si tout manquait dans les raisons, la perfection du
débit, et de tout le secours que peut donner l'esprit et le savoir, y
fut entière.

Entre nous trois de même avis, je dirai franchement que ce fut à moi à
répondre\,; j'étais l'ancien, j'avais fait le mémoire, c'était mon avis
qui était devenu celui des deux autres. Je pris donc la parole à mon
tour, et je commençai par l'embarras et la honte où j'étais de me voir
forcé à soutenir une opinion contraire à celle du duc de Chevreuse, à
qui j'épargnai d'autant moins les louanges, les déférences et les
respects, que j'étais mieux résolu à ne le pas épargner sur les raisons.
Je dis aussi un petit mot léger de politesse à M. de Noailles, après
quoi j'entrai en matière. Je la possédais assez pour me posséder
moi-même. Le ton, les expressions, tout fut mesuré et modeste\,; mais
les raisons, les réponses, les réfutations furent décochées avec la
dernière force, et par-ci par-là respects et compliments courts à M. de
Chevreuse, et rien au duc de Noailles. Je n'oubliai pas, entre autres
raisons, de leur faire remarquer que les gouverneurs de province et les
chevaliers de l'ordre, desquels le roi se faisait accompagner en son lit
de justice, n'y étaient placés que sur le banc des baillis, c'est-à-dire
derrière les conseillers du parlement, du côté des fenêtres\,; qu'ils y
étaient sans voix, même consultative, c'est-à-dire absolument sans
parole\,; et qu'ils y demeuraient toujours découverts. Ce contraste avec
les simples conseillers du parlement de place et de voix fut exposé avec
étendue ainsi que celui d'un simple lit de justice, où il ne s'agit que
d'enregistrement d'édits et de déclarations du roi tout au plus, et bien
rarement encore de quelque interprétation ou de légère législation sur
des points de droit ou de coutume qui se prennent en divers sens dans
les divers tribunaux, avec une législation de l'importance de celle-ci,
qui ne regardait rien moins que la succession à la couronne, et un ordre
à y établir inconnu depuis tant de siècles, contraire à la pratique de
tant de siècles constante et continuelle, et qui, au préjudice de toutes
les lois des États et des familles particulières, excluait de la
couronne toute une branche aînée et bien reconnue telle, en faveur des
cadettes.

Quoique je me restreignisse le plus qu'il me fût possible, l'importance
de la matière, et plus encore la nécessité de démêler, de rendre
palpables et de répondre aux sophismes, aux inductions et aux
entortillements où le duc de Chevreuse excellait, et qu'il savait
masquer d'une apparence de simplicité et de justesse par la netteté, la
facilité et la grâce naturelles de son élocution, me rendirent plus long
que je n'aurais voulu. Le silence fut entier pendant nos deux discours,
et l'application des assistants extrême. M. de Beauvilliers surtout n'en
perdit pas un mot. Quand j'eus fini, M. de Noailles voulut dire quelque
chose\,: ce ne fut rien qui méritât réponse. M. de Chevreuse reprit la
parole, mais en légère répétition de ce qu'il avait déjà dit. M. de
Beauvilliers ne le laissa pas aller loin, il l'interrompit, lui dit
qu'on avait déjà entendu ce qu'il répétait, et lui demanda s'il avait
quelque chose de nouveau à dire. M. de Chevreuse convint qu'il n'avait
point de raisons nouvelles. M. de Noailles, sans attendre de question,
témoigna par un geste de salut qu'il n'en avait pas non plus.

Le duc de Beauvilliers regarda les ducs de Charost et d'Humières, comme
pour leur demander leur avis, qui dirent en deux mots qu'ils étaient du
mien plus que jamais. Alors je vis un prodige qui me combla d'embarras,
et qui, en effet, me couvrit de confusion. M. de Beauvilliers reprit en
très-peu de mots le précis de la chose et de la diversité des deux
avis\,; puis tout d'un coup cet homme si mesuré, si sage, si modeste, si
accoutumé à n'être qu'un en sentiment et en tout avec le duc de
Chevreuse, et à lui déférer, se changea en un autre homme. Il rougit, et
parut avoir peine à se contenir. Il dit qu'il ne comprenait pas comment
on pouvait penser comme M. de Chevreuse sur ce qui nous divisait, en
expliqua les raisons courtement, mais sans rien oublier d'essentiel,
dévoila les sophismes avec une justesse, une précision extrême\,; et de
là (et c'est le prodige, et où la honte m'accabla) il tomba sur M. de
Chevreuse comme un faucon, et le traita comme un régent fait un jeune
écolier qui apporte un thème plein des plus gros solécismes et les lui
fait tous remarquer en le réprimandant. Je ne m'étendrai pas davantage
sur un discours si animé et dans lequel rien ne fut oublié. La
conclusion fut à mon avis. M. de Chevreuse, petit comme l'écolier devant
son maître, embarrassé, confus, mais sans altération, acquiesça tout
court. M. de Noailles, étourdi à ne savoir où il en était, demeura muet.

En se levant, M. de Beauvilliers nous regarda tous pour confirmer le
jugement, en disant\,: «\,Messieurs, voilà donc que tout est convenu
entre nous, et qu'il passe à l'avis de M. de Saint-Simon,\,» d'un air
plus approchant de son air ordinaire. MM. de Chevreuse et de Noailles
répondirent qu'ils s'y rendaient\,; et ce mot ne fut pas plus tôt dit
que je sortis sans dire mot à personne, et gagnai ma chambre dans le
dernier étonnement, non de ce que mon avis avait prévalu, mais de la
manière dont la chose s'était passée. Peu de temps après que je fus dans
ma chambre, les ducs de Charost et d'Humières y vinrent pleins du même
étonnement, et assez aises de la longue et forte boutade. Pour moi, à
l'occasion de qui elle s'était faite, j'en étais peiné au dernier point.
Le duc de Noailles, à qui M de Beauvilliers ne s'était jamais adressé en
tout son discours, mais lui avait laissé voir auparavant que ce mémoire
donné comme de lui, et qu'il avait fait tant faire et refaire, lui
paraissait pitoyable, fut outré d'avoir été si fortement battu en la
personne de M de Chevreuse, ce qu'avec tout son art il ne put nous bien
cacher. Pour M. de Chevreuse, que j'évitai un jour ou deux, il n'y parut
jamais, et il demeura toujours le même avec M. de Beauvilliers et avec
moi, avec une douceur, une simplicité, une vérité, un naturel vraiment
respectables.

\hypertarget{chapitre-xi.}{%
\chapter{CHAPITRE XI.}\label{chapitre-xi.}}

1712

~

{\textsc{Conférences sur les formes des renonciations entre le duc de
Beauvilliers et moi.}} {\textsc{- Différence essentielle de validité
entre celle du roi d'Espagne et celle des ducs de Berry et d'Orléans.}}
{\textsc{- Le roi non susceptible d'aucune autre forme que d'un
enregistrement ordinaire.}} {\textsc{- Peine extrême du duc de
Beauvilliers là-dessus, sur ce que je lui représente.}} {\textsc{- Le
duc de Beauvilliers de plus en plus en peine.}} {\textsc{- Je lui
propose une façon inouïe d'en sortir.}} {\textsc{- Je m'anéantis au duc
de Beauvilliers.}} {\textsc{- Puissants moyens des ducs de Berry et
d'Orléans d'appuyer les justes formes valides en leur faveur.}}
{\textsc{- Je ramène les ducs de Berry et d'Orléans à laisser le roi
régler sans nulle résistance la forme des renonciations.}} {\textsc{-
Caractère, état et friponnerie de Nancré.}} {\textsc{- Il ne tient pas à
lui et à Torcy de me faire une affaire cruelle auprès du roi sur les
renonciations.}} {\textsc{- Ducs d'Hamilton et d'Aumont ambassadeurs en
France et en Angleterre.}} {\textsc{- Grand traitement de ce dernier,
qui, avant son départ, est fait seul chevalier de l'ordre.}} {\textsc{-
Extraction et mort du duc d'Hamilton.}} {\textsc{- Duc de Shrewsbury
ambassadeur en France.}} {\textsc{- Bailli de La Vieuville ambassadeur
de Malte, au lieu du feu bailli de Noailles.}} {\textsc{- Course de
l'électeur de Bavière à Fontainebleau.}} {\textsc{- Retour du roi par
Petit-Bourg à Versailles.}} {\textsc{- Départ de la duchesse d'Albe pour
l'Espagne.}} {\textsc{- Abbé de Castillon\,; quel.}} {\textsc{- Il
l'épouse, et sa fortune.}} {\textsc{- La Salle\,; son extraction, son
caractère, sa fortune, son mariage.}} {\textsc{- Quelques anciennes et
courtes anecdotes.}}

~

Ce fut après à MM. de Chevreuse et de Beauvilliers, mais à celui-ci
surtout, à voir comment ils s'y prendraient pour oser faire au roi une
proposition qu'il trouverait si choquant cette autorité dont il était
idolâtre, à la déification de laquelle il avait employé tout son règne.
Ils m'ont laissé ignorer ce qui se passa là-dessus\,; et je n'ai pas cru
devoir crocheter des amis si respectables, et qui d'ailleurs avaient en
moi la plus parfaite confiance, soit qu'au fait et au prendre ils
n'aient osé faire la proposition après avoir bien tâté et reconnu le
terrain, qui est ce que le secret à mon égard m'a fait soupçonner, soit
qu'ils aient été repoussés sans espérance. Vers la fin de Fontainebleau,
M. de Beauvilliers me déclara que le roi n'entrerait jamais dans ces
formes, et qu'il ne voulait ouïr parler que d'un simple enregistrement
des renonciations au parlement et tout au plus d'y appeler les deux
princes intéressés et les pairs\,; encore n'en voudrait-il pas répondre.

Je lui dis qu'en cela comme en tout le roi était le maître, mais que
cela n'aurait nulle validité\,; que les alliés seraient bien simples
s'ils s'en contentaient, et les deux princes intéressés encore plus, à
qui cela coupait la gorge. Ce terme l'effraya, et je m'expliquai. Je lui
dis donc que ces renonciations étaient doubles et réciproques\,; qu'en
Espagne la forme de toute espèce de législation était certaine et
reconnue\,; que cette même forme servait encore pour la reconnaissance
d'un roi et de son héritier, pour son inauguration, pour les serments à
lui faire, en un mot, pour tout ce qu'il y avait de plus grand et de
plus auguste à traiter\,; que cette forme était les états généraux
connus sous le nom de \emph{las cartes}, où les grands, les prélats, la
noblesse, les conseils, les tribunaux et les députés des villes se
trouvaient, où le roi présidait, et où tout ce qui passait était
immuable\,; que c'était là où les renonciations de M. le duc de Berry et
de M. le duc d'Orléans passeraient et seraient admises et enregistrées
en loi, sans retour pour eux et leur postérité, outre que le pouvoir des
rois d'Espagne, peu ou point astreint aux formes, les pouvait exclure de
la succession, comme le simple testament de Charles II avait appelé
Philippe V à ses couronnes\,; qu'il est clair par là qu'il ne manquerait
rien à l'exclusion de M. le duc de Berry et de M. le duc d'Orléans de la
succession d'Espagne, pour avoir toute la légalité et la certitude qui
la pouvait opérer, tandis que celle du roi d'Espagne et de sa postérité
à la couronne de France ne recevrait pas le moindre degré de validité.
Je lui retraçai les raisons qui l'avaient persuadé de la nécessité des
formes que j'avais proposées, et qui avaient été si approuvées de lui
chez le duc de Chevreuse, lequel était aussi du même avis, à cette
petite augmentation près que le duc de Noailles avait imaginée, et que
lui avait si fort rejetée\,; que de tout cela il résulterait que les
deux princes et leur postérité demeureraient exclus sans retour de toute
prétention à la couronne d'Espagne, tandis que le roi d'Espagne et la
sienne demeureraient dans tous leurs droits sur celle de France, parce
que sa renonciation, faite de bonne foi de sa part, se trouverait
destituée de celle de la nation française à lui et aux siens, et par
conséquent ne serait qu'un vain leurre qui ne pouvait jamais acquérir
aucun droit aux ducs de Berry et d'Orléans, au préjudice de la branche
d'Anjou aînée de la leur. La conversation fut longue\,; M. de
Beauvilliers demeura persuadé, mais sans espérance du côté du roi.

Le lendemain nous nous revîmes. Il me représenta la nécessité pressante
de la paix, les instances continuelles des Anglais sur les
renonciations, l'impossibilité de vaincre le roi sur un article qui lui
était aussi sensible que celui de son autorité unique\,; que
l'enregistrement des traités de paix étant en usage, et allant, non à
confirmer son autorité par une autre, mais simplement à la promulguer,
il consentirait par cette raison à l'enregistrement des renonciations
comme d'une partie intégrante du traité de paix\,; qu'on aurait même
peine à lui faire goûter qu'il se fît séparément de l'enregistrement du
traité même, c'est-à-dire qu'il se fît deux enregistrements au lieu d'un
seul du traité\,; et qu'il prévoyait une extrême difficulté à y faire
appeler, non les deux princes, parce qu'il s'agissait d'eux, et
d'autoriser leur renonciation de leur présence, et que les Anglais ne
s'en contenteraient pas autrement, mais d'y faire appeler les pairs, par
cette délicatesse extrême d'autorité qui l'effaroucherait en lui
proposant une chose non usitée aux enregistrements des traités, et qui
le hérisserait par le soupçon d'une autorité confirmative de la sienne.
M. de Beauvilliers ajouta qu'en différant on ne persuaderait pas le roi
davantage sur les formes effectivement nécessaires\,; que cependant tout
était à craindre pour la paix du chagrin extrême d'Heinsius et de son
parti, qui gouvernait les Provinces-Unies, qui ne voulaient point la
paix, et du désespoir de la maison d'Autriche et de tout ce qui avait
épousé ses intérêts, qui faisaient l'impossible pour accrocher et
rompre\,; que, par toutes ces considérations si pressantes dans
lesquelles il me conjurait d'entrer, il me conjurait en même temps d'y
faire entrer les deux princes, et de leur persuader de se rendre à
l'absolue nécessité. Je répondis que c'était à eux, que la chose
regardait, à prendre leur parti d'eux-mêmes, non à moi à me sentir ou
plutôt à abuser de leur confiance, dans l'affaire la plus grande et la
plus principale qui pût les regarder et toute leur postérité\,; que je
leur avais démontré quelles étaient les formes de renonciation du roi
d'Espagne à la couronne de France, auxquelles seules ils se pussent fier
de validité et de stabilité\,; que je ne pouvais leur tenir un autre
langage\,; que tout ce que je pouvais était de regretter qu'ils
n'eussent pas en main un autre conseil que le mien sur une affaire si
capitale, qui pourrait leur proposer mieux\,; mais qui, mes faibles
lumières ne me montrant de sûr que les formes dont il s'agissait, je ne
pouvais leur en dissimuler toute la nécessité.

Le duc de Beauvilliers revint à l'impossibilité à l'égard du roi\,; moi,
que ce n'était pas mon affaire, mais celle des deux princes\,; et que
s'ils faisaient instruire les Anglais, qu'ils les persuadassent, comme
il était facile et certain, eux-mêmes princes ne trouveraient de sûreté
que dans les formes proposées, et pour la sûreté de l'Europe et de la
paix tiendraient ferme, et obligeraient enfin le roi à les contenter,
tant par la nécessité pressante de la paix que pour ne laisser pas
persuader l'Europe que, par cette feinte de délicatesse d'autorité, il
se voulait moquer de toute l'Europe, et en particulier des Anglais, à
qui il devait une paix si inespérée et si nécessaire, et les éblouir
d'un enregistrement vain qui laissait la branche d'Anjou dans tous ses
droits, et en état, si le cas en arrivait, de porter à la fois les deux
couronnes de France et d'Espagne, après tant de sang répandu pour
l'empêcher. Ce propos, vrai et solide, effraya étrangement le duc de
Beauvilliers\,; il me dit tout ce qu'il put\,; moi de me taire. Nous
nous séparâmes de la sorte.

Comme je m'habillais le lendemain matin, il m'envoya prier d'aller chez
lui. Il me dit qu'il n'avait point pu dormir de la nuit dans le détroit
où je l'avais laissé. Il m'exhorta de nouveau, je demeurai ferme, et la
conversation ne finit que par l'heure du conseil. En nous quittant, il
me pria qu'il pût m'entretenir encore le lendemain chez lui à la même
heure. J'étais dans une vraie angoisse de résister ainsi, pour la
première fois, à un homme que je regardais comme mon père et mon oracle
depuis toute ma vie, et pour lequel mon estime intime, la tendresse de
mon cœur, l'admiration de mon esprit, et la reconnaissance de tout ce
qu'il avait fait pour me porter au plus haut point auprès du Dauphin,
n'avaient fait qu'accroître la plus entière déférence pour lui. Je le
trouvai dans un état encore plus peiné que je ne l'avais laissé la
veille. Il reprit les mêmes raisons. Tandis qu'il parlait je me parlais
à moi-même, et je résolus enfin de sortir du déchirement où je me
trouvais.

Tout à coup je l'interrompis, et le regardant avec feu\,: «\,C'est
battre l'eau, monsieur, lui dis-je, que répéter toujours les mêmes
choses\,; épargnez-vous-en la peine, parce que je vous déclare que
jamais elles ne me persuaderont\,; mais prenez une autre voie. Vous êtes
un ancien ministre d'État et un très-homme de bien, et je ne dirai guère
en avouant que je suis bien loin au-dessous de proportion avec vous sur
ces deux points. Toute ma vie je vous ai regardé comme mon père, parce
que vous avez bien voulu m'en servir, et mon respect et ma confiance
vous ont aussi toujours rendu mon oracle. Je veux vous en donner la plus
insigne marque, et la preuve la plus unique qui se puisse en donner à un
homme, et que je ne donnerais sans exception quelconque à nul autre
homme sur la terre, en quelque chose que ce fût. Tenez, monsieur,
finissons\,; quittez tout raisonnement, parce qu'encore une fois, vous
ne me persuaderez jamais\,; mais prenez la voie de l'autorité, et sans
nulle sorte de raisonnement, dites-moi crûment et nettement en deux
mots\,: «\,Je veux que vous fassiez telle chose.\,» Je ne répliquerai
pas un seul mot\,; et contre mon sens, contre ma conviction la plus
intime, contre tout l'ouvrage que j'ai bâti et qui est pleinement
achevé, j'obéirai comme un enfant, et je n'oublierai rien pour détruire
tout ce que j'ai édifié et persuadé, sans cesser un instant de l'être
tout autant que je le fus jamais, et je mettrai tout ce qui est en moi
pour ramener les deux princes à tout ce que vous voudrez me prescrire\,;
mais rien sans un \emph{je le veux, et je l'exige}. Vous en savez plus
que moi de bien loin en affaires, vous êtes encore plus s'il se peut
au-dessus de moi en piété et en lumières, je me reposerai dessus et vous
sacrifierai mes sentiments les plus chers et ma conviction la plus
intime.\,» J'avais pendant ce discours les yeux fichés sur les siens\,;
ils se mouillèrent de larmes. Jamais je ne vis homme si concentré ni si
touché. Il se jeta à mon cou, et parlant à peine\,: «\, Non, me dit-il,
c'en est trop, cela n'est pas juste, je n'y puis consentir. ---
Toutefois, repris-je, ce qui est en débat entre vous et moi ne peut
finir que par là. N'espérez rien du raisonnement, mais comptez sur tout
par l'autorité. » Mille choses tendres et d'un homme touché jusqu'au
plus profond du cœur, succédèrent de sa part à cette nouvelle reprise de
déclaration\,; et finalement il me dit qu'il prendrait cette journée
pour y bien penser, et me dire le lendemain, à même heure, en même lieu,
à quoi il serait arrêté. Je retournai donc à ce rendez-vous. Il commença
par tout ce qu'il est possible à l'amitié d'exprimer, et à l'humilité
d'un si grand homme de bien, qui était effrayé de la grandeur de mon
sacrifice, et qui en sentait toute l'étendue. Il me dit qu'il n'avait
pensé à autre chose la veille, et toute la nuit qu'il n'avait pu
dormir\,; qu'il ne savait comment se résoudre de prendre sur soi ce que
je lui proposais, et d'abuser de ma déférence à un point aussi inouï\,;
et de là voulut revenir à raisonner. Je l'interrompis\,: «\,Je m'en
vais, lui dis-je, monsieur,\,» en faisant un mouvement comme pour me
lever\,; «\,de raisonnement je n'en écoute plus\,; c'est votre décision
que j'attends\,: ou laissez-moi dans ma liberté avec les deux princes,
ou prononcez en deux mots avec autorité\,; et ôtez-vous bien de l'esprit
que ceci puisse avoir une autre issue.\,» Il fut quelques moments sans
répondre, et moi en silence. Ses yeux se baignèrent encore. Il se jeta à
moi sans rien dire, tout retiré en lui-même. Puis me regardant avec
tendresse\,: «\,Puisqu'il n'y a donc point d'autre voie, et que vous le
voulez absolument,\,» me dit-il, mais avec un air de modestie, même de
honte qui ne se peut décrire, «\,il faut bien que je prenne l'unique
parti que vous me laissez, quelque peine qu'il me fasse. J'exige donc de
vous que vous tâchiez à détruire ce que vous avez fait, non qu'il ne
soit bon, mais parce que le roi n'y passera jamais, et qu'il nous faut
finir la paix, et que vous rameniez les deux princes à se contenter de
l'enregistrement en leur présence et en celle des pairs. --- Vous le
voulez, monsieur, repris-je, vous serez obéi. De ma part je n'y
oublierai rien\,; je vous rendrai compte de temps en temps de ce que
j'aurai fait en conséquence. Demeurons-en là fermement, et surtout plus
de raisonnements inutiles,\,» Il m'embrassa encore tendrement, me dit
tout ce qui me pouvait exprimer l'effet que son cœur et son esprit
ressentaient d'un si extraordinaire abandon de déférence, et combien il
en demeurerait pénétré toute sa vie. Cette conversation fut la plus
courte de beaucoup, et nous nous séparâmes.

La besogne que j'entreprenais était fort étrange\,; j'avais soufflé le
chaud, j'avais parlé raison, règle, lois, droits, justice, intérêt le
plus palpable, et j'avais pleinement persuadé et affermi\,; il n'y avait
plus qu'à en faire usage avec les Anglais, qui ne pouvaient goûter un
sceau aussi informe et aussi superficiel, pour des renonciations si
importantes à toute l'Europe et à eux-mêmes, qu'un simple enregistrement
usité pour tous les traités, et qui n'en avait rendu aucun plus stable.
Ils alléguaient sans cesse le violement des renonciations de la reine,
aussitôt après la mort du roi Philippe IV son père, qui avait coûté à
l'Espagne un si grand démembrement des Pays-Bas et de toute la
Franche-Comté, quoique ces renonciations eussent été enregistrées au
parlement dans le traité des Pyrénées, que le roi en personne les eût
jurées, et signées, face à face du roi son beau-père, en présence de
leurs deux premiers ministres et des deux cours, qui en furent acteurs
et témoins dans l'île des Faisans ou de la Conférence. On ne pouvait
disconvenir que cette solennité n'eût tout une autre force que le simple
enregistrement du traité au parlement, ni que celui des renonciations à
part qu'il s'agissait de faire\,; et néanmoins on ne pouvait disconvenir
non plus de l'irruption subite du roi en Flandre et en Franche-Comté,
aussitôt après la mort du roi son beau-père, pour se mettre en
possession des droits de la reine, dont il fit publier des écrits,
nonobstant la renonciation.

Les Anglais eux-mêmes avaient vu, par le traité de partage dont leur roi
Guillaume III avait été le principal promoteur, ce qu'on pensait en
France des renonciations de la reine, lorsqu'il ne s'agissait plus comme
autrefois de simples droits à prétendre sur le roi son frère, malgré
l'universalité de ses renonciations, mais de la succession à la
monarchie entière\,; et toute l'Europe, à l'exception de l'empereur,
avait regardé ce traité de partage comme fort avantageux, en ce que la
France s'y contentait d'une portion de la monarchie d'Espagne, qu'elle
croyait pouvoir prétendre entière nonobstant les renonciations. Elle y
était revenue par le testament inespéré de Charles II, et par le vœu de
toute la nation espagnole\,; et il s'agissait au moins d'empêcher d'une
manière solide, à laquelle ces exemples rendaient les Anglais et leurs
alliés d'autant plus délicats et circonspects, qu'un même prince
français ne pût en aucun cas posséder les deux monarchies, et dominer
l'Europe par une si formidable puissance. Les Anglais n'avaient pas
oublié par quelle forme de jugement Philippe de Valois avait emporté la
couronne de France, en vertu de la loi salique, sur leur roi Edouard
III, bien plus proche par sa mère, fille de Philippe le Bel, et sœur des
rois Louis X le Hutin, Philippe V le Long, et Charles IV le Bel, morts
sans postérité masculine, lesquels étaient cousins germains de Philippe
de Valois, fils des deux frères. Les Anglais n'avaient pu oublier
qu'Edouard III reconnut si bien le pouvoir des juges et la validité du
jugement qu'il ne songea pas à contester, qu'il rendit personnellement
hommage à Philippe de Valois, 6 juin 1329, dans l'église d'Amiens, pour
ce qu'il tenait de la couronne de France, et que ce ne fut qu'au bout de
quelque temps qu'il s'avisa de vouloir revenir par les armes contre le
droit qu'il avait reconnu, excité par les pratiques du fameux Robert
d'Artois outré d'avoir été juridiquement débouté du comté-pairie
d'Artois, dans la dignité et possession duquel sa tante paternelle
Mahaut avait été maintenue, et déshonoré de plus par la preuve de faux,
et le jugement en conséquence de quatre pièces qu'il avait fait
fabriquer et produire, ce qui le jeta entre les bras d'Edouard III, pour
se venger de sa mauvaise fortune contre son roi et sa patrie. Il n'en
fallait pas tant avec des gens aussi accoutumés et attachés que le sont
les Anglais aux formes légales et juridiques, pour les porter à demander
toutes celles qui uniquement pouvaient valider solidement des
renonciations si importantes à eux et à toute l'Europe, et dont leurs
alliés se reposaient sur eux et sur leur propre intérêt, dans un traité
dont ils s'étaient enfin rendus les maîtres.

Eux instruits et bien persuadés, c'était à M. le duc de Berry et à M. le
duc d'Orléans à les laisser faire, à ne se montrer en rien, à laisser au
roi les soupçons qu'il aurait voulu prendre, mais à se bien garder de
tout ce qui aurait pu lui en donner lieu à cet égard\,; en tout cas, en
éditant bien attentivement toutes preuves possibles, l'un son
petit-fils, l'autre son neveu, se consoler des reproches sans preuves et
des humeurs, par la solidité avec laquelle ils s'assuraient une
réciproque validité de leurs renonciations et de celles du roi
d'Espagne, puisque le roi n'aurait eu en ce cas d'autre choix que celui
de souffrir les formes que les anglais auraient exigées, ou de rompre la
paix, auquel cas il n'y aurait point de renonciations, et de continuer
une guerre que toutefois il ne lui était plus possible de soutenir.

Toutes ces choses m'étaient bien présentes, je les avais bien inculquées
aux deux princes, et ils étaient bien persuadés. Défaire ce même ouvrage
était une triste entreprise. Persuader contre sa propre conviction est
un étrange embarras. Il fallut pourtant travailler en conformité de ce
que le poids immense de M. de Beauvilliers sur moi m'avait fait lui
promettre. Le récit en détail en serait long et ennuyeux\,; je me
contenterai de dire que je commençai par éloigner, et empêcher après,
toute instruction et tout concert des Anglais. Je revins auprès des deux
princes à des réflexions de prudence et de timidité sur le danger que le
roi pût découvrir ce commerce, et qu'il se prît à eux de la roideur des
Anglais, et de leurs propositions de formes, qui, selon ses délicates et
si sensibles préventions, attaqueraient aux yeux de toute l'Europe son
autorité si chérie, et lui feraient recevoir l'affront de souffrir que
celle de ses sujets la confirmât, et y parût nécessaire. Je les pressai
sur le désespoir où le roi se trouverait d'acheter la paix à ce prix, ou
de continuer une guerre qu'il savait si précisément ne pouvoir soutenir,
et dont le poids l'avait forcé aux conditions les plus honteuses et les
plus dommageables, qu'il avait même vu mépriser, et de laquelle il
sortait par le moyen de l'Angleterre, sans qu'il fût plus question de
lui en imposer que d'honnêtes. J'avais affaire à deux princes fort
différents, mais tout semblables pour l'excès de la timidité. M. le duc
de Berry, tenu de très-court depuis son enfance, était accoutumé à
dépendre du roi jusque pour les choses les plus ordinaires et les plus
indifférentes, et à trembler sous son moindre sérieux. M. le duc
d'Orléans ne le craignait guère moins. Il était de plus si battu de
l'oiseau par les diverses aventures de sa vie, qu'il était tout aussi
éloigné que M. le duc de Berry de s'exposer à sa colère. Ce furent les
armes dont je me servis contre moi-même, et pour les ramener à ce que je
voulus, en ruinant ce que j'avais édifié.

C'était à quoi j'étais occupé, lorsque, tout à la fin du voyage de
Fontainebleau, je fus averti de la chose du monde que pour lors je
méritais le moins. Nancré y avait fait quelques tours\,; il avait écumé
quelques mots de fins de conversations, interrompues par son arrivée
deux ou trois fois, entre M. le duc d'Orléans et moi. Il avait eu, comme
je l'ai dit en son lieu, la charge de capitaine de ses Suisses, par
M\textsuperscript{me} d'Argenton, sur Saint-Pierre, pour qui
M\textsuperscript{me} la duchesse d'Orléans la voulait alors, qui de
pique le fit depuis son premier écuyer, contre le gré de M. le duc
d'Orléans\,; et cela avait fait de grandes brouilleries. Nancré était un
bourgeois de Paris qui s'appelait Dreux, de même famille que le gendre
de Chamillart\,; mais son père avait servi, il était devenu officier
général avec estime et gouverneur de \footnote{Le nom est en blanc dans
  le manuscrit.}\ldots{} Il avait épousé en secondes noces une fille de
La Bazinière, dont j'ai parlé ailleurs, et qui était sœur de la mère du
premier président de Mesmes qui vivait intimement avec eux. Nancré avait
beaucoup d'esprit. Il s'était lassé de l'emploi de lieutenant-colonel de
je ne sais plus quel régiment, où il était parvenu par ancienneté. Il
trouva cette porte pour en sortir. Il vivait dans la liaison la plus
étroite avec sa belle-mère, vieille beauté riche et fort du grand monde
de Paris. Elle alla loger avec lui au Palais-Royal, et elle y tint le
dé. Lui se fourra tant qu'il put dans le monde. Il avait ce qu'il
fallait pour en être goûté, et la probité ne l'arrêtait sur rien. Il
voulait cheminer et être de quelque chose\,; les moyens ne lui coûtaient
pas. Il s'était fourré chez M. de Torcy. Il y chercha commission de
parler à M. le duc d'Orléans sur les renonciations. Chagrin de n'en pas
avoir l'honneur auprès de Torcy, il alla lui dire que c'était moi qui,
entêté de pairie, lui tournais la tête sur les formes, et arrêtais la
paix.

Torcy, avec qui je n'avais pas la plus légère habitude, et qui était ami
de beaucoup de gens avec qui je ne frayais pas, alla rendre au roi ce
que Nancré lui avait rapporté. Le roi en colère en parla à M. le duc de
Berry, et lui cita ses auteurs. J'en fus incontinent averti par M. le
duc de Berry même. Cela m'engagea à le prier de trouver bon que je ne le
visse plus du tout pour ôter au roi tout prétexte, et que notre commerce
se continuât par M\textsuperscript{me} de Saint-Simon et M. le duc
d'Orléans, par qui il avait toujours passé, en sorte même que je n'avais
vu que peu et rarement M. le duc de Berry en particulier. Je ne pouvais
en user de même sans éclat avec M. le duc d'Orléans, ainsi je me résolus
à ce qui pourrait en arriver. Je me plaignis amèrement à lui de la
scélératesse de Nancré, qui s'enfuit à Paris aussitôt, et ne reparut de
longtemps. Le roi néanmoins ne me fit semblant de rien\,; et comme en
effet je parvins à ramener les deux princes à se contenter de
l'enregistrement fait en présence des pairs, cette friponnerie de Nancré
et ce mauvais office de Torcy n'eurent aucune suite. Je le laissai
tomber et ne crus pas devoir dire ni faire dire au roi quoi que ce soit
là-dessus.

Quelque dépit et quelques obstacles que les alliés apportassent à la
paix, les choses étaient tellement avancées avec l'Angleterre, que le
duc d'Aumont fut nommé pour y aller en ambassade, sur ce que le duc
d'Hamilton fut déclaré ambassadeur en France. M. d'Aumont était alors
fort en liaison avec le duc de Noailles et moi, et j'aurai lieu d'en
parler dans les suites. Il eut vingt-quatre mille écus d'appointements
par an, vingt-quatre mille livres pour dédommagement de la perte du
change, et cinquante-quatre mille livres pour ses équipages et pour
trois mois d'avance. Il eut de plus cinq cent mille livres de brevet de
retenue sur sa charge de premier gentilhomme de la chambre, et fut
chevalier de l'ordre, seul et extraordinairement à une messe basse avant
son départ. C'est le dernier que le roi ait fait.

Le duc d'Hamilton était un assez jeune seigneur, fort du parti de la
reine et considéré. Il était Douglas. Anne Hamilton, fille aînée du
dernier Jacques, marquis d'Hamilton, avait épousé Guillaume Douglas,
comte de Selkirk. Le marquis d'Hamilton fut fait duc et chevalier de la
Jarretière par Charles I\^{}er, et après diverses fortunes eut la tête
coupée peu de jours après cet infortuné monarque. Charles II, son fils,
après son rétablissement, fit duc d'Hamilton ce comte de Selkirk, gendre
du dernier duc d'Hamilton, qui n'avait point laissé de garçons\,; et ce
nouveau duc d'Hamilton eut avec la dignité presque tous les biens de son
beau-père qui lui furent restitués, dont il prit le nom et les armes.
C'est le grand-père ou le bisaïeul de celui dont il s'agit ici. Le parti
contraire à la reine, outré de n'avoir pu empêcher la paix, se rabattit
faute de mieux à lui faire toutes les sortes de dépits qu'il put.
Hamilton avait gagné un procès depuis peu en plein parlement contre
milord Mohun, du parti contraire. Ce parti le piqua tant qu'il put, et
le força presque malgré lui à se battre avec Hamilton. Mohun fut tué sur
la place, mais Macartnay, qui lui servit de second, enfila sur-le-champ
le duc Hamilton par derrière et s'enfuit. La reine, qui sentit d'où le
coup partait, en fut également affligée et offensée, et nomma à
l'ambassade de France le duc de Shrewsbury, chevalier de la Jarretière,
l'un de ses plus confidents ministres, aîné de la maison Talbot.

Le bailli de La Vieuville, beau-frère de la dame d'atours de
M\textsuperscript{me} la duchesse de Berry, succéda au feu bailli de
Noailles à l'ambassade de Malte et y fit tout fort noblement.

L'électeur de Bavière fit une légère apparition à Fontainebleau. Il y
vint de Petit-Bourg, vit le roi un quart d'heure dans son cabinet, dit
en sortant à d'Antin qu'il partait beaucoup plus content qu'il ne
l'avait espéré en venant, et s'en retourna à Petit-Bourg.

Quinze jours après, c'est-à-dire le mercredi 14 septembre, le roi, après
le conseil d'État, alla coucher à Petit-Bourg, et le lendemain à
Versailles, où peu de jours après la duchesse d'Albe prit congé de lui
chez M\textsuperscript{me} de Maintenon. Elle partit peu de jours après,
sans avoir laissé un sou de dettes de leur longue et magnifique
ambassade en des temps très-malheureux. Elle emmena avec elle un petit
abbé de Castillon qui n'avait pas de chausses, et qui n'avait de
ressource que les lieux et les heures publics, où il ennuyait même
beaucoup de sa présence qui était aussi assez vilaine. Il était
Gonzague, mais arrière-cadet, et il cherchait ici fortune depuis
quelques années. Je ne sais comment il fit connaissance avec la duchesse
d'Albe\,; mais fort peu après être arrivé en Espagne, il quitta le petit
collet et elle l'épousa. Il parvint en cette considération, peu après, à
la grandesse et à la clef de gentilhomme de la chambre. Ils n'ont point
eu d'enfants. Elle venait de mourir lorsque j'arrivai en Espagne, où je
le vis sans meubles, avec un châlit et un capucin, qui en voulait
prendre l'habit. La douleur ne fut pas de durée\,; il s'était déjà
remarié, lorsque j'en partis, à une beauté fille du prince de
Santo-Buono-Caraccioli, chose infiniment rare en Espagne.

La Salle, qu'on a vu avoir vendu pour la seconde fois sa charge de
maître de la garde-robe, par un hasard unique, s'ennuya de son oisiveté.
C'était un fort honnête homme, qui avait du sens, et qui ne manquait pas
d'esprit, bien fait et de fort bonne mine, qui, pour le petit-fils d'un
vendeur de sabots dans la forêt de Senonches, avait fait une grande
fortune, n'en était pas encore content, et se rendait peu de justice. Un
ancien bailli de la Ferté que j'y ai vu longtemps, et qui a survécu mon
père de beaucoup d'années, nous en mit au fait pour l'extraction.
J'étais à la Ferté avec ma mère lorsque mon père, mandé pour le
chapitre, nous envoya la liste de la promotion de 1688. Ce bailli se
trouva à la réception des lettres et à la lecture de la liste. Au nom de
La Salle, il demanda qui il était, et, sur la réponse, se mit à rire et
dit que cela ne se pouvait pas, et enfin ajouta qu'étant jeune il avait
connu son grand-père qui vendait des sabots en gros après en avoir fait
dans sa jeunesse. Il nous dit qu'étant devenu à son aise sur ses vieux
jours il avait acquis une petite terre qui jamais n'a valu mille écus de
rente, et sans aucune étendue dans la lisière de la forêt de Senonches
qui s'appelle la Salle. J'y ai passé plusieurs fois\,; ils y ont fait un
petit castel de cartes, proportionné à la valeur de ce petit bien. Le
fils du sabotier voulut aller à la guerre, il s'y distingua\,; il
parvint par son ancienneté à la tête des gens d'armes de la garde.

Caillebot avait quitté ce nom et s'appelait La Salle\,; il vivait dans
un temps où on se battait beaucoup\,; il était fort sur la hanche, et
passa pour un brave à quatre poils qu'il ne fallait pas choquer. Ce fut
par ces bravades que le cardinal Mazarin, qui en avait aisément peur, et
qui voulait aussi s'en attacher partout, le poussa dans les gens d'armes
que Miossens commandait, si connu depuis sous le nom de maréchal
d'Albret, et si compté à la cour et dans le monde. La Salle sut si bien
lui faire sa cour et se faire passer d'ailleurs pour un brave important,
qu'il eut la compagnie quand le maréchal d'Albret la quitta en 1666. Il
poussa son fils dans cette compagnie quoique jeune, car il était de
1646\,; il se trouva de la valeur et de l'honneur, et il monta assez
vite. M. de Soubise était dans la même compagnie\,; il y était entré
pauvre gentilhomme, et fort éloigné d'imaginer de devenir prince et fort
riche\,; la beauté de sa seconde femme et la bonté du roi firent ce
miracle. Il était en son plus doux mouvement lorsque La Salle mourut et
laissa la compagnie des gens d'armes vacante en 1672. M. de Soubise
l'obtint, mais le fils de son prédécesseur l'y importuna. Il pensa
toujours de loin pour fonder des établissements avec son grand secours
domestique. Il voulut ranger de bonne heure tout obstacle à pouvoir
assurer sa charge à sa famille. La Salle servait bien, ne voulait point
quitter, et il avait la fantaisie d'espérer de succéder à M. de Soubise.
Cette folie fit sa fortune\,; il y en avait au crédit où était
M\textsuperscript{me} de Soubise\,; d'ailleurs cette espérance aurait pu
être fondée sur l'âge de M. de Soubise qui avait quinze ans plus que
lui, et sur les hasards de la guerre. La conjoncture heureuse qui se
présenta fit l'affaire de tous les deux.

Il y avait plusieurs années que Vardes était chassé pour avoir eu une
part principale dans l'affaire qui perdit la comtesse de Soissons et le
comte de Guiche, et qui touchait le roi si fort
immédiatement\footnote{L'aventure à laquelle Saint-Simon fait allusion a
  été racontée en détail par les contemporains. Voy. notes à la fin du
  volume.}. Vardes était un favori qui par sa trahison attira sur soi
plus de colère\,; il fut envoyé à Aigues-Mortes dont il était
gouverneur, avec défense d'en sortir et d'y voir personne, et ordre de
se défaire de sa charge de capitaine des Cent-Suisses de la garde. C'est
le même qui se battit avec mon père. Il était chevalier de l'ordre, de
la promotion de 1661, et si gâté de la fortune, que j'ai ouï dire aux
contemporains qu'il regarda pour la première fois son cordon bleu avec
quelque complaisance en chemin de son exil. On espère toujours. Tardes
se flatta du pardon après un châtiment de quelques années, et il
s'obstina à garder sa charge pour ne se pas trouver dépouillé à son
retour. À la fin on lui fit si bien entendre que son espèce de prison ne
finirait que par sa démission, qu'il se résolut à ce calice. M. de
Louvois, ennemi terrible et implacable, mais également bon ami et bon
parent, fut bientôt averti\,; il fit parler à Vardes par Tilladet, son
cousin germain, qu'il avait déjà fait maître de la garde-robe, et
Vardes, dans la nécessité de vendre, crut se faire un protecteur de
Louvois. M\textsuperscript{me} de Soubise, instruite de la première
main, saisit la charge de maître de la garde-robe que Tilladet allait
vendre pour se défaire de La Salle, et s'en délivrer par une fortune si
fort au-dessus de lui. Vouloir et pouvoir fut pour elle la même chose.
Ainsi La Salle quitta les gens d'armes et le service militaire pour
celui de la cour et de la personne du roi, en 1678. Ce service était
d'une assiduité extrême\,: lever, coucher, changement d'habits pour la
chasse ou la promenade tous les jours, en y allant et au retour, et cela
de deux années l'une tout de suite, avec un prince qui voulait une
entière régularité. Celle de La Salle la fut, et plut fort au roi, mais
elle devint continuelle pendant bien des années que Lyonne, fils du
secrétaire d'État, fut son camarade, qui ne mettait jamais le pied à la
cour, et que les services importants de feu son père, et la
considération des Estrées, dont le duc neveu du cardinal avait épousé sa
sœur, faisait passer au roi, jusqu'à ce qu'enfin il vendit à Souvré,
fils de feu M. de Louvois. Une vie si coupée et si nécessairement
occupée de riens, déplaisait souvent à La Salle. Il était fort glorieux
et entêté de son mérite, et quoique j'eusse peu d'habitude avec lui, et
en général c'était un homme chagrin, particulier, sauvage, avec qui on
n'en avait guère, je lui ai ouï regretter les gens d'armes, et sa charge
qui l'avait tiré du service, disait-il, malgré lui, et l'avait empêché
d'être maréchal de France. Désœuvré, par n'avoir plus de fonctions et
n'avoir jamais eu beaucoup de commerce, il s'en était allé auprès de
Dreux, dans une petite terre appelée Montpinçon, dont la maison était au
bord de la rivière d'Eure, dont les jardins étaient souvent inondés. Il
l'accommoda pour habiter et pour s'amuser\,; il s'y ennuya, il s'alla
promener en basse Normandie chez des gens de sa connaissance. Il trouva
dans une de ses visites une fille de vingt ans, jolie et bien faite avec
sa mère, qui était du voisinage, et qui s'appelait
M\textsuperscript{lle} de Bénouville. II les vit le soir qu'il y arriva,
et y dîna le lendemain avec elles. Quelqu'un à table demanda à la mère
si elle ne songeait point à la marier. Elle répondit qu'elle y pensait
bien, mais que cela n'était pas facile quand on n'avait rien à donner.
De propos en propos elle dit que ce qu'elle voudrait trouver, ce serait
quelque homme âgé qui ne songeât point au bien, mais à se donner une
compagnie et une femme qui eût soin de lui et qui en fût tout occupée\,;
que sa fille avait la raison de penser de même et d'aimer mieux un
mariage comme celui-là, qui la mettrait à son aise, que d'épouser un
jeune homme. La conversation changea, La Salle ne parut pas y prendre la
moindre part, mais il y fit ses réflexions. Elles ne furent pas longues.
Dans la fin de la journée il s'informa au maître de la maison de ce que
c'était que M. M\textsuperscript{me} et M\textsuperscript{lle} de
Bénouville\,; ce qu'il en apprit ne lui déplut pas, et la demoiselle lui
avait donné dans les yeux. Il crut bannir l'ennui de sa vie en
l'épousant, et tout de suite pria celui à qui il s'en informait d'en
faire la proposition à la fille et à la mère. Toutes deux, le lendemain
matin, crurent rêver, et eurent peine à se persuader que la chose fût
sérieuse. Le cordon bleu du vieux galant qui la demandait sans dot
quelconque, uniquement à condition de demeurer à Montpinçon sans jamais
aller à Paris, leur parut les cieux ouverts. Elles envoyèrent bien vite
chercher le père, et dans le jour tout fut d'accord et réglé. La Salle
partit là-dessus pour le venir dire au roi, et s'en retourna tout
aussitôt en Normandie où le mariage se fit. Il a été très-heureux, et
cette jeune femme a vécu avec lui à merveilles\,; vertu, complaisance,
soin d'attirer du monde, et pourtant avec économie. Ils se firent aimer
et considérer chez eux. La Salle avait soixante-six ans. Il lui tint
parole sur Paris, mais lui-même ne faisait que deux ou trois apparitions
par an à Versailles, et encore moins à Paris. Ils ont eu un fils qui est
dans le service et marié.

\hypertarget{chapitre-xii.}{%
\chapter{CHAPITRE XII.}\label{chapitre-xii.}}

1712

~

{\textsc{Le roi à Rambouillet.}} {\textsc{- Mort de Ribeire, conseiller
d'État\,; sa place donnée à La Bourdonnaie, son gendre.}} {\textsc{-
Mort de Godolphin.}} {\textsc{- Le Quesnoy rendu à discrétion.}}
{\textsc{- Bouchain\,; la garnison prisonnière.}} {\textsc{- Valory et
Varennes gouverneurs.}} {\textsc{- Châtillon brigadier, depuis duc et
pair et gouverneur de Mgr le Dauphin.}} {\textsc{- Perte de la
Quenoque.}} {\textsc{- Les campagnes finies.}} {\textsc{- Retour des
généraux d'armée à la cour.}} {\textsc{- Montesquiou demeure à commander
en Flandre.}} {\textsc{- Princesse des Ursins aux eaux de Bagnères\,;
Chalais l'y va trouver\,; pompe de cette dame.}} {\textsc{- Survivance
du gouvernement de Lyon, etc., au duc de Villeroy, et les lieutenances à
ses fils.}} {\textsc{- Villars gouverneur de Provence\,; Saillant
gouverneur de Metz\,; Tessé général des galères.}} {\textsc{- Les frères
Broglio gouverneurs de Gravelines et du Mont-Dauphin.}} {\textsc{-
Dangeau donne à son fils son gouvernement de Touraine.}} {\textsc{-
Comte de Toulouse et d'Antin achètent leurs maisons à Paris.}}
{\textsc{- Quatre cent mille livres d'augmentation de pension à M. le
duc de Berry\,; il entre au conseil de dépêches.}} {\textsc{- La musique
du roi à la messe de M\textsuperscript{me} la duchesse de Berry.}}
{\textsc{- Hammer à la cour\,; merveilleusement reçu\,; quel est cet
Anglais.}} {\textsc{- Duchesses, etc., conservent leur nom et leur rang
en se remariant au-dessous de leur premier mari en Angleterre.}}
{\textsc{- Marlborough se retire en Allemagne\,: quelle y était sa
principauté de l'empire.}} {\textsc{- Renonciation du roi d'Espagne à la
couronne de France en pleines cortès.}} {\textsc{- Lettre tendre qu'il
écrit là-dessus à M. le duc de Berry.}} {\textsc{- Mort de l'abbé
d'Armagnac.}} {\textsc{- Mort du duc de Chevreuse.}} {\textsc{-
Anecdotes sur sa famille, sur lui, sur la duchesse sa femme.}}
{\textsc{- Mort du duc Mazarin.}} {\textsc{- Anecdotes sur lui, sur sa
famille, sur leur fortune.}} {\textsc{- Mort de la duchesse de
Charost.}} {\textsc{- Mort du duc de Sully.}} {\textsc{- Berwick en
Roussillon, etc.}} {\textsc{- Chamillart revoit le roi.}} {\textsc{-
Plénipotentiaires d'Espagne.}} {\textsc{- Besons joué par
M\textsuperscript{me} la duchesse de Berry.}} {\textsc{-
M\textsuperscript{me} de Pompadour gouvernante des enfants de M. le duc
de Berry.}} {\textsc{- La Mouchy et son mariage.}} {\textsc{- Mariage de
Meuse avec M\textsuperscript{lle} de Zurlauben.}} {\textsc{- Musiques et
scènes de comédies chez M\textsuperscript{me} de Maintenon.}} {\textsc{-
Le maréchal de Villeroy y est admis.}} {\textsc{- Dessein sur lui.}}
{\textsc{- Gouvernement de Guyenne donné au comte d'Eu.}} {\textsc{-
Conduite des ducs de La Rochefoucauld dans leur famille.}} {\textsc{-
État de cette famille.}} {\textsc{- Désir, jalousie, vains efforts des
ducs de La Rochefoucauld pour le rang de prince étranger.}} {\textsc{-
Duc de La Rochefoucauld obtient la distraction du duché de La Rocheguyon
avec la dignité pour son second petit-fils et sa postérité, au préjudice
de l'aîné.}} {\textsc{- Ce cadet duc par démission de son père.}}
{\textsc{- Nouveaux efforts inutiles sur l'abbé de La Rochefoucauld,
qui, moyennant un bref, prend l'épée et va mourir à Bude.}}

~

Le roi alla les premiers jours d'octobre passer une semaine chez M. le
comte de Toulouse à Rambouillet, avec un très-court accompagnement.
Excepté sa propre table, M. le comte de Toulouse fit et magnifiquement
la dépense de tout le reste. Le roi y fit une chose contre sa coutume.
Ce fut de permettre à La Bourdonnaie d'y venir lui parler, et de lui
donner la place de conseiller d'État, vacante par la mort de Ribeire,
son beau-père, car il évitait toujours ces espèces de successions dans
les familles. Le beau-père était d'une grande réputation et parfaitement
intègre\,; le gendre s'en était acquis dans les grandes intendances.

Ce fut aussi où on apprit la mort de Godolphin, naguère grand trésorier
d'Angleterre, espèce de premier ministre, et le chef du parti whig dont
le fils avait épousé la fille du duc de Marlborough, chez qui il mourut
de la taille, à la campagne, et ces deux hommes ne furent jamais qu'un.
Ce fut un grand soulagement pour la reine et pour son nouveau ministère,
un grand abattement pour le parti qui lui était opposé, et le dernier
coup du revers de la fortune pour le duc de Marlborough.

Le roi y reçut aussi la nouvelle de la prise du Quesnoy par M. de
Châtillon, qui a fait depuis une si grande fortune et si peu espérée,
que Voysin, son beau-père, lui amena à son travail. La place se rendit à
discrétion. Ils étaient encore onze à douze cents hommes sous les armes,
et il s'y trouva un grand amas d'artillerie et de munitions. Châtillon
fut fait brigadier pour la nouvelle, et Valory eut le gouvernement de la
place dont il avait conduit les travaux du siége. Aussitôt après, le
maréchal de Villars fit le siége de Bouchain, qui se rendit peu de jours
après, la garnison prisonnière de guerre. Villars envoya la nouvelle par
le comte de Choiseul, son beau-frère, et la garnison à Reims, avec le
gouverneur, parce que c'était lui qui avait fait, cette même campagne,
une course en Champagne qui avait fort effrayé ce pays. Le gouvernement
de Bouchain fut rendu à Varennes qui l'avait auparavant. Cette conquête
fut une consolation de la perte de la Quenoque, qui venait d'être
surpris par un partisan d'Ostende à l'ouverture des portes, qui s'était
faite par l'aide-major, sans découverte ni la moindre précaution. Ainsi
finit la guerre cette année. Les armées d'Allemagne et de Savoie
venaient de se séparer, et les maréchaux d'Harcourt et de Berwick
arrivèrent à la cour incontinent après, et en même temps le maréchal de
Villars. Montesquiou demeura à commander en Flandre.

M\textsuperscript{me} des Ursins fit en même temps un voyage à Bagnères
pour une enflure de genou, escortée par un détachement des gardes du
corps du roi d'Espagne, en avant-goût de la souveraineté dont elle se
flattait. Chalais l'y alla trouver de Paris. Son retour à Madrid ne fut
pas moins pompeux.

En ce même temps-ci le roi fit plusieurs grâces. Le maréchal de Villeroy
eut pour le duc de Villeroy la survivance de son gouvernement\,; la
lieutenance générale qu'il en avait, pour le marquis de Villeroy son
petit-fils, et la lieutenance de roi de celui-ci à son frère. Le
maréchal de Villars obtint le gouvernement de Provence\,; celui de Metz
qu'il avait fut donné à Saillant\,; la charge de général des galères au
maréchal de Tessé, absent, et qui ne l'avait pas demandée, avec le
pareil brevet de retenue de M. de Vendôme et les appointements échus
depuis sa mort. Le gouvernement de Mont-Dauphin et celui de Gravelines
aux deux Broglio, l'un gendre de Voysin, l'autre qui a fait une si
grande fortune, et Dangeau eut permission de céder à son fils le
gouvernement de Touraine en en retenant l'autorité et les appointements.
La Vrillière, assez mal dans ses affaires, vendit sa magnifique maison
vis-à-vis la place des Victoires au comte de Toulouse, et d'Antin en
acheta une autre fort belle à peu près dans le même quartier, qui avait
été bâtie pour Chamillart. On ne laissa pas d'être surpris que ces deux
hommes qui tenaient de si près au roi, l'un par ce qu'il lui était,
l'autre par sa charge, et plus encore par sa faveur, et courtisan au
suprême, fissent ces acquisitions dans Paris. Peu de temps après, le roi
suppléa à la modicité de l'apanage de M. le duc de Berry par une pension
de quatre cent mille francs, et ordonna à sa musique de se trouver tous
les jours à la messe de M\textsuperscript{me} la duchesse de Berry,
comme à la sienne, qui fut une très-sensible distinction pour elle et
pour M. le duc de Berry. Il en reçut une plus touchante par l'entrée au
conseil des dépêches qui était le chemin des autres conseils.

Il parut à la cour un personnage singulier qui y fut reçu avec des
empressements et des distinctions surprenantes. Le roi l'en combla, les
ministres s'y surpassèrent, tout ce qui était de plus marqué à la cour
se piqua de le festoyer. C'était un Anglais d'un peu plus de trente ans,
de bonne mine et parfaitement bien fait, qui s'appelait le chevalier
Hammer, et qui était fort riche. Il avait épousé aussi la fille unique
et héritière de milord Harrington, secrétaire d'État, veuve du duc de
Grafton, qui s'en était éprise, et qui conserva de droit son nom et son
rang de duchesse de Grafton, comme il se pratique toujours en Angleterre
en faveur des duchesses, marquises et comtesses qui étant veuves se
remarient inégalement. Hammer passait pour avoir beaucoup d'esprit et de
crédit dans la chambre des communes. Il était fort attaché au
gouvernement d'alors, et fort bien avec la reine qui l'avait tenu toute
la campagne auprès du duc d'Ormond pour être un peu son conseil. De
Flandre il vint ici\,; il y demeura un mois ou six semaines, également
couru et recherché, et s'en alla d'ici en Angleterre pour l'ouverture du
parlement. Je n'ai point su alors ce qu'il était venu faire, ni même
s'il était chargé de quelque chose, comme l'accueil qu'il y reçut porte
à le croire, et j'ai oublié à m'en informer depuis. On n'en a guère ouï
parler dans la suite. Il faut qu'il n'ait fait ni figure ni fortune sous
ce règne en Angleterre, et qu'il ne se soit pas accroché au suivant. Il
ne trouva plus le duc de Marlborough, qui venait enfin d'en sortir avec
permission et de passer à Ostende avec très-peu de suite. Son dessein
était de se retirer en Allemagne, où il était prince de l'empire ou
plutôt de l'empereur Léopold, qui lui avait donné le titre de prince de
Mindelen, sans la principauté, mais de l'argent pour acheter des terres
en Souabe, auxquelles on devait donner le titre et le nom de Mindelen\,;
mais il avait gardé l'argent et n'avait point acquis de terres.

Il arriva un courrier d'Espagne avec la copie de l'acte de renonciation
du roi d'Espagne passé le 5 novembre en pleines cortès, en présence de
l'ambassadeur d'Angleterre. Ce courrier apporta aussi un projet pour
celle de M. le duc de Berry, et une lettre de la main du roi d'Espagne à
ce Prince, la plus tendre, la plus forte, la plus précise, pour lui
témoigner sa sincérité dans cet acte qui l'avançait en sa place à la
succession à la couronne de France, et avec quelle joie son amitié pour
lui le lui avait fait faire. Lui et M. le duc d'Orléans me la
montrèrent, parce que je demandai à la voir. Elle me parut si
importante, que je leur recommandai beaucoup de la conserver
soigneusement comme une pièce tout à fait importante pour eux\,; ils ne
s'en étaient seulement pas avisés. Ils me l'avouèrent et trouvèrent que
j'avais grande raison.

Plusieurs personnes considérables moururent dans la fin de cette année.
L'abbé d'Armagnac étant allé voir sa sœur à Monaco y mourut de la petite
vérole\,: il avait trente ans, de bonnes moeurs, deux grosses abbayes en
attendant mieux, et M. le Grand comptait qu'il aurait pour lui la
nomination du Portugal au chapeau, que son frère avait autrefois perdue
par l'avarice de M\textsuperscript{me} d'Armagnac qui fit l'éclat
étrange qui l'ôta de toutes sortes {[}de{]} portées.

La mort de M. de Chevreuse qui arriva à Paris le samedi 5 novembre,
entre sept et huit heures du matin, me donne occasion de m'étendre sur
un personnage qui a tant, toujours et si singulièrement figuré, et avec
qui j'ai vécu tant d'années dans la plus intime confiance d'affaires, et
dans la plus libre privance d'amitié et de société. Quoique j'en aie
rapporté diverses choses en plusieurs occasions, il en reste bien plus
encore que la longueur m'empêchera de dire, quoiqu'il y eût en toutes à
s'amuser, et peut-être plus encore à profiter. Né avec beaucoup d'esprit
naturel, d'agrément dans l'esprit, de goût pour l'application et de
facilité pour le travail et pour toutes sortes de sciences, une justesse
d'expression sans recherche et qui coulait de source, une abondance de
pensées, une aisance à les rendre et à expliquer les choses les plus
abstraites ou les plus embarrassées avec la dernière netteté, et la
précision la plus exacte, il reçut la plus parfaite éducation des plus
grands maîtres en ce genre, qui lui donnèrent toute leur affection et
tous leurs rares talents.

Le duc de Luynes, son père, n'avait ni moins d'esprit ni moins de
facilité et de justesse à parler et à écrire, ni moins d'application et
de savoir. Il s'était lié par le voisinage de Dampierre avec les
solitaires de Port-Royal des Champs, et après la mort de sa première
femme, mère du duc de Chevreuse, s'y était retiré avec eux\,; il avait
pris part à leur pénitence et à quelques-uns de leurs ouvrages, et il
les pria de prendre soin de l'instruction de son fils, qui, né le 7
octobre 1646, n'avait que sept ans à la mort de sa mère, qui fut
enterrée à Port-Royal des Champs. Ces messieurs y mirent tous leurs
soins par attachement pour le père, et par celui que leur donna pour
leur élève le fonds de douceur, de sagesse et de talents qu'ils y
trouvèrent à cultiver. La retraite du duc de Luynes à Port-Royal des
Champs dura plusieurs années. Sa mère, si fameuse dans toutes les
grandes cabales et les partis de son temps, sous le nom de son second
mari le duc de Chevreuse, mort sans postérité en 1657, elle en 1679,
suivant le siècle par son âge, était très-peinée de voir son fils comme
enterré. M. de Chevreuse, dernier fils du duc de Guise, tué aux derniers
états de Blois en 1588, avait toujours vécu avec elle dans la plus
grande union, et comme elle avait toujours passionnément aimé le duc de
Luynes, qui logea toujours avec eux, M. de Chevreuse l'aima de même, et
leur fit à tous deux tous les avantages qu'il put. Il donna même au duc
de Luynes sa charge de grand fauconnier, et son gouvernement d'Auvergne
que M. de Luynes ne garda pas longtemps. Sa famille ne souffrait guère
moins que M\textsuperscript{me} de Chevreuse d'une retraite qui rendait
ses talents inutiles pour le monde. Ils s'adressèrent à mon père qui
était son ami intime. Il fut plus heureux qu'eux dans ses
remontrances\,: M. de Luynes sortit de Port-Royal, mais il en conserva
l'affection et la piété. Il retourna loger avec sa mère, où toute sa
piété ne put le défendre de l'amour pour sa propre tante.

M\textsuperscript{me} de Chevreuse était fille du second duc de
Montbazon, frère du premier et d'une Lenoncourt, et sœur de père et de
mère du prince de Guéméné, depuis troisième duc de Montbazon, si connu
par son esprit, père du quatrième duc de Montbazon, mort fou et enfermé
à Liége, et du chevalier de Rohan, décapité à Paris devant la Bastille,
pour crime de lèse-majesté, en 1674, le 17 janvier. Le père de
M\textsuperscript{me} de Chevreuse épousa en secondes noces une Avaugour
ou Vertus, des bâtards de Bretagne, de laquelle il eut M. de Soubise,
dont la mort a été rapportée il n'y a pas longtemps, et deux filles,
dont l'aînée fut abbesse de la Trinité de Caen, puis de Malnoue, et
mourut en 1682, et c'est la cadette dont il s'agit ici. Elle avait
quarante ans juste moins que sa sœur la duchesse de Chevreuse, qui était
de 1600, et elle de 1640. Elles avaient perdu leur père commun en 1654,
et sa mère à elle en 1657. M\textsuperscript{me} de Chevreuse l'avait
élevée, et pris soin d'elle comme de sa fille. Elle eut envie d'être
religieuse, et elle entra même au noviciat. Le duc de Luynes, éperdument
amoureux, oublia tout ce qu'il avait appris au Port-Royal sur les
passions, et songea encore moins à tout ce que ces saints et savants
solitaires auraient pu lui dire sur une novice et sœur de sa mère.
M\textsuperscript{me} de Chevreuse, qui craignait toujours son retour
dans la retraite dont on avait eu tant de peine à le tirer, eut tant de
peur que le désespoir de ne pouvoir obtenir l'objet de sa passion ne le
précipitât de nouveau dans la solitude, qu'elle pressa sa sœur de
quitter le voile blanc, et qu'avec de l'argent, qui fait tout à Rome,
elle eut dispense pour ce mariage, qu'elle fit en 1661, et qui fut fort
heureux. M\textsuperscript{me} de Luynes était également belle et
vertueuse. Elle eut deux fils et cinq filles, et mourut fort saintement
à la fin de 1684, six ans avant le duc de Luynes, qui se remaria encore
une fois.

M. de Chevreuse, qui était assez grand, bien fait, et d'une figure noble
et agréable, n'avait guère de bien. Il en eut d'immenses de la fille
aînée et bien-aimée de M. Colbert, qu'il épousa en 1667. Outre la dot et
les présents les plus continuels et les plus considérables, il tira de
la considération de ce mariage l'érection nouvelle de Chevreuse en duché
vérifié en sa faveur, la substitution des biens du duc de Chaulnes,
cousin germain paternel de son père, sa charge de capitaine des
chevau-légers de la garde, et finalement le gouvernement de Guyenne.
M\textsuperscript{me} de Chevreuse était une brune, très-aimable femme,
grande et très-bien faite, que le roi fit incontinent dame du palais de
la reine\,; elle sut plaire à l'un et à l'autre, être très-bien avec les
maîtresses, mieux encore avec M\textsuperscript{me} de Maintenon,
souvent, malgré elle, de tous les particuliers du roi, qui s'y trouvait
mal à son aise sans elle, et tout cela sans beaucoup d'esprit, avec une
franchise et une droiture singulière, et une vertu admirable qui ne se
démentit en aucun temps. J'ai parlé ailleurs de l'union de ce mariage\,;
de celle qui fit un seul cœur et une même âme des duchesses de Chevreuse
et de Beauvilliers sœurs, et des deux ducs beaux-frères\,; du voyage
d'Italie et d'Allemagne\,; de M. de Chevreuse et du rang dont il y
jouit\,; de la part qu'ils eurent tous deux à l'orage du quiétisme, qui
les pensa perdre, et qui leur rendit pour toujours M\textsuperscript{me}
de Maintenon ennemie\,; de leur abandon à la fameuse Guyou et à
l'archevêque de Cambrai, dont rien ne les put déprendre\,; du ministère
effectif mais secret du duc de Chevreuse jusqu'à sa mort, et de beaucoup
d'autres choses, surtout sur Mgr le duc de Bourgogne, M. le duc
d'Orléans, et M. le prince de Conti\,; et on a pu voir le danger où il
fut de perdre sa charge.

J'ai eu lieu aussi, en plusieurs endroits, de parler du caractère de son
esprit, de sa dangereuse manière de raisonner, de la droiture de son
cœur, et avec quelle effective candeur il se persuadait quelquefois des
choses absurdes et les voulait persuader aux autres, dont j'ai marqué
plusieurs exemples, mais toujours avec cette douceur et cette politesse
insinuante qui ne l'abandonna jamais, et qui était si sincèrement
éloignée de tout ce qui pouvait sentir domination ni même supériorité en
aucun genre. Les raisonnements détournés, l'abondance de vues, une
rapide mais naturelle escalade d'inductions dont il ne reconnaissoit pas
l'erreur, étaient tout à fait de son génie et de son usage. Il les
mettait si nettement en jour et en force avec tant d'adresse, qu'on
était perdu si on ne l'arrêtait dès le commencement, parce qu'aussitôt
qu'on lui avait passé deux ou trois propositions qui paraissaient
simples, et qu'il faisait résulter l'une de l'autre, il menait son homme
battant jusqu'au bout, lequel en sentait le faux qui éblouissait, et qui
pourtant n'y trouvait point de jointure à opposer un mot. Amoureux par
nature des voies obliques en matière de raisonnement, mais toujours de
la meilleure foi du monde, il se déprit pourtant assez tard de la
doctrine de Port-Royal jusqu'à un certain point, car il savait ajuster
des mixtions étranges, sans en quitter l'estime, le goût, l'éloignement
secret mais ferme des jésuites, surtout les mœurs, la droiture, l'amour
du vrai, les vertus, la piété. C'est ce même goût de raisonnements peu
naturels qui le livra avec un abandon qui dura autant que sa vie aux
prestiges de la Guyon et aux fleurs de M. de Cambrai\,: c'est encore ce
qui perdit ses affaires et sa santé, et ce qui très-certainement l'eût
entêté plus que personne, mais sans aucun intérêt, du système de Law,
s'il avait vécu jusque-là.

Dampierre, dont il fit un lieu charmant, séduit par le goût et le
secours de M. Colbert, qui lui manqua au milieu de l'entreprise,
commença à l'incommoder. Sa déférence pour son père le ruina, par
l'établissement de toutes ses sœurs du second lit dont il répondit, et
les avantages quoique légers auxquels il consentit pour ses frères aussi
du second lit, et qui ne pouvaient rien prétendre sans cette bonté. Il
essuya des banqueroutes des marchands de ses bois\,; il avait tous ceux
de Chevreuse et de la forêt de Saint-Léger et d'autres contigus. Il
imagina de paver un chemin qui déblayât facilement ces bois, mais il ne
s'en trouva pas plus avancé quand ce pavé fut achevé. Il se tourna
ensuite à former un canal qui pût flotter à bois perdu jusqu'à la Seine.
Il en fit bien les deux tiers, et vit après qu'il n'y passerait jamais
un muid d'eau. Les acquisitions, les dédommagements, les frais furent
immenses\,; il se trouva accablé d'affaires et de dettes, et obligé à la
fin à vendre la forêt de Saint-Léger et beaucoup de terres et d'autres
bois au comte de Toulouse, qui en décupla sa terre de Rambouillet, mais
qui firent presque de Dampierre une maison sans dépendances. Il fit
aussi et refit, à diverses reprises, des échanges avec Saint-Cyr, et
c'est ce qui fit transporter le titre et l'ancienneté de Chevreuse sur
Montfort-l'Amaury\,; en un mot, il était presque sans ressource lorsque
le gouvernement de Guyenne lui tomba de Dieu et grâce, sans qu'il y eût
pensé, comme on l'a vu en son temps. Sa santé, il la conduisit de même.
Il avait eu la goutte dès l'âge de dix-neuf ans, sans l'avoir jamais
méritée, mais elle lui venait de race. L'exemple de son père lui fit
peur\,; il ne l'avait pas méritée davantage, et il en était accablé, et
dans la suite ses frères le furent encore davantage. Il se réduisit donc
à un régime qui lui réussit pour la goutte qu'il n'eut que rare et
faible, et pour le préserver de maladies, mais qu'il outra et qui le
tua. M. de Vendôme, qui avait quelquefois mangé avec lui à Marly, dans
les premiers temps que le roi aimait qu'on allât à la table du grand
maître, disait plaisamment au roi que M. de Chevreuse s'empoisonnoit
d'eau de chicorée pendant tout un repas, pour avoir le plaisir de boire
à la fin une rasade de vin avec du sucre et de la muscade. En effet,
c'était sa pratique. En affaires et en santé, le mieux chez lui était le
plus grand ennemi du bien.

Jamais homme ne posséda son âme en paix comme celui-là\,; comme dit le
psaume, il la portait dans ses mains. Le désordre de ses affaires, la
disgrâce de l'orage du quiétisme qui fut au moment de le renverser, la
perte de ses enfants, celle de ce parfait Dauphin, nul événement ne put
l'émouvoir ni le tirer de ses occupations et de sa situation ordinaire
avec un cœur bon et tendre toutefois. Il offrait tout à Dieu, qu'il ne
perdait jamais de vue\,; et dans cette même vue, il dirigeait sa vie et
toute la suite de ses actions. Jusque avec ses valets il était doux,
modeste, poli\,; en liberté dans un intérieur d'amis et de famille
intime, il était gai et d'excellente compagnie, sans rien de contraint
pour lui ni pour les autres, dont il aimait l'amusement et le plaisir\,;
mais si particulier par le mépris intime du monde, et le goût et
l'habitude du cabinet, qu'il n'était presque pas possible de l'en tirer,
et que le gros de la cour ignorait qu'il eût une table également
délicate et abondante. Il n'y arrivait jamais que vers l'entremets. Il
se hâtait d'y manger quelque pourpoint de lapin, quelque grillade, enfin
ce qui avait le moins de suc, et au fruit quelques sucreries qu'il
croyait bonnes à l'estomac, avec un morceau de pain pesé dont on avait
ôté la mie. Il voulait manger en sorte qu'il pût travailler en sortant
de table, avec la même facilité qu'avant de s'y mettre\,; et en effet,
il rentrait bientôt après dans son cabinet. Le soir, peu avant minuit,
il mangeait quelque œuf ou quelque poisson à l'eau ou à l'huile, même
les jours gras. Il faisait tout tard et assez lentement. Il ne
connaissoit pour son usage particulier ni les heures ni les temps, et il
lui arrivait souvent là-dessus des aventures qui faisaient notre
divertissement dans l'intime particulier, et sur lesquelles M. de
Beauvilliers ne l'épargnait pas, malgré toute sa déférence dans le
courant ordinaire de la vie.

Les chevaux de M. de Chevreuse étaient souvent attelés douze ou quinze
heures de suite. Une fois que cela arriva à Vaucresson, d'où il voulait
aller dîner à Dampierre, le cocher, puis le postillon se lassèrent de
les garder\,; c'était en été. Sur les six heures du soir, les chevaux
{[}furent{]} ennuyés à leur tour, et on entendit un fracas qui ébranla
tout. Chacun accourut\,; on trouva le carrosse brisé, la grande porte
fracassée, les grilles des jardins qui fermaient les côtés de la cour
enfoncées, les barrières en pièces, enfin un désordre qu'on fut
longtemps à réparer. M. de Chevreuse, que ce vacarme n'avait pu
distraire un instant, fut tout étonné quand il l'apprit, et M. de
Beauvilliers se divertit longtemps à le lui reprocher et à lui en
demander les frais. Une autre aventure, à laquelle M. de Chevreuse ne
tenait point, lui arriva encore à Vaucresson, et le mettait dans un
embarras véritable, mais plaisant à voir, toutes les fois qu'on la lui
remettait. Sur les dix heures du matin, on lui annonça un M. Sconin, qui
avait été son intendant, qui s'était mis à choses à lui plus utiles, où
M. de Chevreuse le protégeait. Il lui fit dire de faire un tour de
jardin, et de revenir dans une demi-heure. Il continua ce qu'il faisait
et oublia parfaitement son homme. Sur les sept heures du soir, on le lui
annonce encore\,: «\,Dans un moment,\,» répondit-il sans s'émouvoir. Un
quart d'heure après, il l'appelle et le fait entrer. «\,Ah\,! mon pauvre
Sconin, lui dit-il, je vous fais bien des excuses de vous avoir fait
perdre votre journée. --- Point du tout, monseigneur, répond Sconin\,;
comme j'ai l'honneur de vous connaître il y a bien des années, j'ai
compris ce matin que la demi-heure pourrait être longue, j'ai été à
Paris, j'y ai fait, avant et après dîner, quelques affaires que j'avais,
et j'en arrive.\,» M. de Chevreuse demeura confondu. Sconin ne s'en tut
pas, ni les gens mêmes de M. de Chevreuse. M. de Beauvilliers s'en
divertit, et quelque accoutumé que M. de Chevreuse fût à ces badinages,
il ne résistait point à voir remettre ce conte sur le tapis. J'ai
rapporté ces deux-là dont je me suis plutôt souvenu que de cent autres
de même nature, sur lesquels on ne finirait point, mais que j'ai voulu
écrire ici parce qu'ils caractérisent.

Le chancelier disait de ces deux beaux-frères qu'ils n'étaient, comme en
effet, «\,qu'un cœur et qu'une âme\,; que ce que l'un pensait, l'autre
le pensait de même tout aussitôt, mais que, pour l'exécution, M. de
Beauvilliers avait un bon ange qui le préservait d'agir en rien comme M.
de Chevreuse, quelque conformément à lui qu'il pensât toujours. Le fait
était exactement vrai. On le verra lorsqu'il sera question de M. de
Beauvilliers\,; et il est inconcevable que deux hommes, si opposés en
actions communes mais continuelles, aient passé leur vie ensemble, sans
se quitter, dans la plus intime et la plus indissoluble union, et jamais
interrompue un seul instant. Ils vivaient dans les mêmes lieux,
logeaient ensemble à Marly et fort proche à Versailles, mangeaient
continuellement ensemble, et il n'y avait jour qu'ils ne se vissent
deux, trois et quatre fois\,; en un mot, cette union était telle, que
l'intimité de l'un, même l'admission à une société particulière, ne
pouvait être avec l'un qu'elle ne fût en même temps avec l'autre, et
pareillement avec leurs épouses.

M. de Chevreuse écrivait aisément, agréablement et admirablement bien et
laconiquement, pour le style et pour la main, et ce dernier est aussi
rare. Il était, non pas aimé, mais adoré dans sa famille et dans son
domestique, et toujours affable, gracieux, obligeant. À qui ne le
connaissoit pas familièrement, il avait un extérieur droit, fiché,
composé, propre, qui tirait sur le pédant, et qui, avec ce qu'il n'était
point du tout répandu, éloignait. Pendant le Fontainebleau de cette
année, lui et M\textsuperscript{me} de Chevreuse me proposèrent une
promenade à Courance. J'allai chez lui, et comme j'entrais dans sa
chambre dans la dernière familiarité, je l'y surpris devant une armoire
qui prenait à la dérobée un verre de quinquina\,; il rougit et me
demanda en grâce de n'en rien dire. Je le lui promis, mais je lui
représentai qu'il se tuait avec du quinquina sans manger. Il m'avoua, à
force de le presser, qu'il s'était mis à cet usage depuis plusieurs mois
pour son estomac, et je voyais et savais qu'il diminuait encore sa
nourriture. Je lui dis sur cela tout ce que je pus, et je lui prédis
qu'il se percerait l'estomac. Le pis était qu'il s'était mis à tendre
peu à peu à la diète de Cornaro, qui avait été fort bonne à ce Vénitien,
mais qui en avait tué beaucoup d'autres, M. de Lyonne entre autres, le
célèbre ministre d'État. Cela n'alla pas loin\,; il tomba malade à
Paris, il souffrit d'extrêmes douleurs avec une patience et une
résignation incroyables, reçut les sacrements avec la plus ardente
piété, et mourut paisible et tranquille dans ses douleurs, et à soi
comme en pleine santé, au milieu de sa famille. On l'ouvrit, et on lui
trouva l'estomac percé.

Si M. de Chevreuse avait, comme on l'a vu ailleurs, essayé d'alléger ses
chaînes ne les pouvant rompre, d'allonger ses séjours de Dampierre aux
dépens des voyages de Marly, pour y vivre à Dieu et à lui-même avec plus
de loisir et de liberté\,; et si, après divers reproches du roi qu'il
coulait en douceur sans se détourner de son but, il avait fallu que le
roi lui eût enfin parlé en ami qui le voulait sous sa main, à la suite
de ses affaires, et en maître qui voulait être obéi et servi,
M\textsuperscript{me} de Chevreuse n'était pas plus éblouie des
distinctions et des particuliers où le roi la voulait toujours. Le bel
âge, la figure, la danse, l'air et le jeu de la table l'avaient initiée
dans tout, aussitôt après son mariage\,; et, avec une droiture et une
franchise qui à la cour lui étaient uniques, elle avait su plaire en
même temps à la reine, au roi, à ses maîtresses, non-seulement sans
bassesse, mais sans courir après. Sa vertu et sa piété, qui fut aussi
vraie qu'elle dans tous les temps de sa vie, fut une autre source de
faveur, lorsque le roi et M\textsuperscript{me} de Maintenon se furent
piqués d'une dévotion qui fit de cette femme le prodige qu'on a vu si
longtemps, sans presque pouvoir le croire. Malgré la haine que, depuis
l'affaire du quiétisme, elle avait prise et conservée pour MM. de
Chevreuse et de Beauvilliers, quoique auparavant elle eût toujours bien
plus goûté ce dernier que son beau-frère, elle n'avait pu s'empêcher
d'aimer toujours M\textsuperscript{me} de Chevreuse\,; et, depuis
qu'elle eut perdu toute espérance de les culbuter, elle n'avait pas
moins d'empressement que le roi de l'attirer dans leurs parties
particulières. M\textsuperscript{me} de Chevreuse, qui n'était pas moins
détachée que son mari, ni moins désireuse que lui de vivre pour Dieu et
pour elle-même, profita d'une fort longue infirmité pour se séquestrer,
sous prétexte qu'elle ne pouvait plus mettre de corps, sans quoi, en
robe ou en robe de chambre, les dames ne pouvaient se présenter nulle
part devant le roi. Lassé de son absence, il fît pour la rappeler de ses
particuliers ce qu'il n'a jamais fait pour aucune autre. Il voulut
qu'elle revînt à Marly avec dispense de tout ce qui était public, et que
là, et à Versailles, elle vînt les soirs le voir chez
M\textsuperscript{me} de Maintenon sans corps, et tout comme elle
voudrait, pour sa commodité\emph{, }à leurs dîners particuliers et à
toutes leurs parties familières. Il lui donna, comme on l'a dit, trente
mille livres de pension sur les appointements du gouvernement de la
Guyenne. Fort peu en avaient de vingt mille, et pas une seule dame de
plus forte. Sa douleur, qui fut telle qu'on la peut imaginer, mais qui
comme elle fut courageuse et toute en Dieu, lui fut une raison légitime
de séparation, mais qu'il fallut pourtant interrompre par des
invitations réitérées, non pour des parties, mais pour voir le roi en
particulier. Après son deuil elle tira de longue, mais elle ne put
éviter les parties et les particuliers. La mort du roi rompit ses
chaînes\,; elle se donna pour morte\,; elle s'affranchit de tous devoirs
du monde\,; elle vécut à l'hôtel de Luynes et à Dampierre dans sa
famille, avec un cercle fort étroit de parents qui ne se pouvaient
exclure, et d'amis très-particuliers. Elle dormait extrêmement peu,
passait une longue matinée en prières et en bonnes œuvres, rassemblait
sa famille aux repas, qui étaient toujours exquis sans être fort grands,
toujours surprise des devoirs que le monde ne cessa jamais de lui
rendre, quoiqu'elle n'en rendît aucun. C'était un patriarche dans sa
famille, qui en faisait les délices, l'union, la paix, et qui rappelait
la vie des premiers patriarches. Jamais femme si justement adorée des
siens, ni si respectée du monde jusqu'à la fin de sa vie, qui passa
quatre-vingts ans, en pleine santé de corps et d'esprit, et qui fut trop
courte pour ses amis et pour sa famille. Après elle on sentit ce qu'on
avait prévu. Cette famille, si unie et si rassemblée autour d'elle, fut
bientôt séparée. Elle mourut dans l'été de 1732, dans la vénération
publique, aussi saintement et aussi courageusement qu'elle avait vu
mourir M. de Chevreuse, parmi les larmes les plus amères de tous les
siens.

Le duc Mazarin mourut dans ses terres, où il s'était retiré depuis plus
de trente ans. Il en avait plus de quatre-vingts, et ce ne fut une perte
pour personne, tant le travers d'esprit, porté à un certain point,
pervertit les plus excellentes choses. J'ai ouï dire aux contemporains
qu'on ne pouvait pas avoir plus d'esprit ni plus agréable\,; qu'il était
de la meilleure compagnie et fort instruit\,; magnifique, du goût à
tout, de la valeur\,; dans l'intime familiarité du roi qui n'a jamais pu
cesser de l'aimer et de lui en donner des marques, quoi qu'il ait fait
pour être plus qu'oublié\,; gracieux, affable et poli dans le
commerce\,; extraordinairement riche par lui-même\,; fils du maréchal de
la Meilleraye, un des hommes du plus grand mérite, de la plus constante
faveur, et le plus compté de son temps, à qui il succéda au gouvernement
de Bretagne, de Nantes, de Brest, du Port-Louis, de Saint-Malo, et dans
la charge de grand maître de l'artillerie lors absolue. Son père résista
tant qu'il put à la volonté du cardinal Mazarin, son ami intime, qui
choisit son fils comme le plus riche parti qu'il connût pour en faire
son héritier en lui donnant son nom et sa nièce. Le maréchal qui avait
de la vertu, disait que ces biens lui faisaient peur, et que leur
immensité accablerait et ferait périr sa famille\,; à la fin il fallut
céder.

Dans un procès que M. Mazarin eut avec son fils à la mort de sa femme,
il fut prouvé en pleine grand'chambre qu'elle lui avait apporté
vingt-huit millions. Il eut en outre le gouvernement d'Alsace, de
Brisach, de Béfort, et le grand bailliage d'Haguenau qui seul était de
trente mille livres de rente. Le roi le mit dans tous ses conseils, lui
donna les entrées des premiers gentilshommes de la chambre, et le
distingua en tout. J'oublie le gouvernement de Vincennes.

Il était lieutenant général dès 1654, et avait beau jeu à devenir
maréchal de France et général d'armée. La piété, toujours si utile et si
propre à faire valoir les bons talents, empoisonna tous ceux qu'il
tenait de la nature et de la fortune, par le travers de son esprit. Il
fit courir le monde à sa femme avec le dernier scandale\,; il devint
ridicule au monde, insupportable au roi par les visions qu'il fut lui
raconter avoir sur la vie qu'il menait avec ses maîtresses\footnote{Le
  duc de Mazarin déclara un jour au roi que l'ange Gabriel l'avait
  averti qu'il lui arriverait malheur, s'il ne rompait vite avec
  M\textsuperscript{lle} de La Vailière. \emph{Mémoires de l'abbé de
  Choisy}, coll. Petitot, t. LXIII, p.~207. Voy. les notes à la fin du
  volume.}. Il se retira dans ses terres, où il devint la proie des
moines et des béats, qui profitèrent de ses faiblesses et puisèrent dans
ses millions. Il mutila les plus belles statues, barbouilla les plus
rares tableaux, fit des loteries de son domestique, en sorte que le
cuisinier devint son intendant et son frotteur secrétaire. Le sort
marquait selon lui la volonté de Dieu. Le feu prit au château de Mazarin
où il était. Chacun accourut pour l'éteindre, lui à chasser ces coquins
qui attentaient à s'opposer au bon plaisir de Dieu.

Sa joie était qu'on lui fît des procès, parce qu'en perdant il cessait
de posséder un bien qui ne lui appartenait pas\,; s'il gagnait, il
conservait ce qui lui avait été demandé, en sûreté de conscience. Il
désolait les officiers de ses terres par les détails où il entrait, et
les absurdités qu'il leur voulait faire faire. Il défendit dans toutes
aux filles et femmes de traire les vaches, pour éloigner d'elles les
mauvaises pensées que cela pouvait leur donner. On ne finirait point sur
toutes ses folies. Il voulut faire arracher des dents de devant à ses
filles parce qu'elles étaient belles, de peur qu'elles y prissent trop
de complaisance. Il ne faisait qu'aller de terre en terre\,; et il
promena pendant quelques années le corps de M\textsuperscript{me}
Mazarin qu'il avait fait apporter d'Angleterre, partout où il allait. Il
vint à bout, de la sorte, de la plupart de tant de millions, et ne
conserva que le gouvernement d'Alsace et deux ou trois gouvernements
particuliers. C'était un assez grand et gros homme, de bonne mine, qui
manquait de l'esprit, à ce qu'il me parut une fois que je le vis chez
mon père, lorsqu'il fut chevalier de l'ordre en 1688. Depuis sa retraite
dans ses terres, il ne fit plus que trois ou quatre apparitions de
quelques jours à Paris et à la cour où le roi le recevait toujours avec
un air d'amitié et de distinction marquée. Il faut maintenant ajouter un
mot de curiosité sur un homme et une fortune aussi extraordinaires. Son
nom de famille était La Porte. On prétend qu'il leur est venu de ce que
leur auteur était portier d'un conseiller au parlement, dont le fils
devint un très-célèbre avocat à Paris, lequel très-certainement était le
grand-père du maréchal de La Meilleraye. Cet avocat La Porte était
avocat de l'ordre de Malte, et le servit si utilement que l'ordre, en
reconnaissance, reçut de grâce son second fils, qui devint un homme d'un
mérite distingué, et commandeur de la Madeleine près de Parthenay. Ce La
Porte, qui s'était fort enrichi, était aussi avocat de M. de Richelieu.
Il acquit quelque bien dans son voisinage, et s'affectionna tellement à
sa famille, que, voyant qu'après avoir mangé tout son bien\footnote{Reproduction
  textuelle du manuscrit. Le sens est \emph{voyant que M. de Richelieu
  avoit mangé tout son bien}, etc.} et laissé sa maison ruinée, il prit
un fils qu'il avait laissé pour son gendre, qui, avec ce secours, se
releva, et mourut en 1590 à quarante-deux ans, chevalier du
Saint-Esprit, capitaine des gardes du corps et prévôt de l'hôtel, qui
est ce que mal à propos on nomme grand prévôt de France. Sa femme était
morte dès 1580. Ce furent le père et la mère du cardinal de Richelieu,
et d'autres enfants dont il ne s'agit pas ici. L'avocat La Porte
survécut son gendre et sa fille. Il avait chez lui un clerc qui avait sa
confiance, qu'il avait fait recevoir avocat, et qui s'appelait
Bouthillier. En mourant il lui laissa sa pratique\,; et lui recommanda
ses petits-enfants de Richelieu qui n'avaient plus de parents.
Bouthillier en prit soin comme de ses propres enfants, et c'est d'où est
venue la fortune des Bouthillier.

Barbin, qui a tant fait parler de lui sous la régence de Marie de
Médicis, était un petit procureur du roi, de Melun, homme d'esprit et
d'intrigue. Henri IV était souvent à Fontainebleau\,; il {[}Barbin{]}
mourait d'envie de se fourrer dans quelque chose, mais était trop petit
compagnon pour pénétrer chez les ministres. À ce défaut il se mit à
faire sa cour à Léonora Galigaï, femme de Concini depuis maréchal
d'Ancre, laquelle était venue d'Italie avec la reine, était sa première
femme de chambre, et pouvait dès lors tout sur elle. Il courtisa Léonora
par de petits présents de fruits, l'attira par des collations à sa
petite maison près de Melun, et s'insinua si bien dans son esprit qu'il
devint dans la suite son principal confident. Elle devint dame d'atours
de la reine, son mari marquis d'Ancre, et, après la mort d'Henri IV,
tous deux devinrent les maîtres de la reine et de l'État. Au
commencement de 1616, la cour étant à Tours, il se fit un grand
changement dans le ministère. Le chancelier de Sillery, Villeroy et le
président Jeannin, qu'on appelait les barbons, furent chassés, et avec
eux Puysieux, secrétaire d'État, fils du chancelier et petit-gendre de
Villeroy. Du Vair, premier président du parlement de Provence, eut les
sceaux, Mangot fut secrétaire d'État, et Barbin mis en la place de
Jeannin, sous le titre de contrôleur général des finances. Étant encore
petit procureur du roi de Melun, il avait fait amitié avec l'avocat
Bouthillier, et logeait chez lui quand il allait à Paris. Il y vit
souvent M. de Luçon, qui fit habitude avec lui, et à qui il plut tant
qu'il le fit connaître à Léonora, ce qui fut le fondement de l'amitié et
de la confiance que Marie de Médicis prit en lui, et qui le conduisit à
une si haute fortune. Il était aussi bon parent et ami qu'ennemi sans
mesure et sans bornes. Il n'oublia pas la mémoire de son grand-père
maternel, l'avocat La Porte, et il trouva dans son oncle maternel et
dans son cousin germain La Porte un mérite qu'il put élever. L'oncle
devint commandeur de Braque, bailli de la Morée, ambassadeur de sa
religion en France, grand prieur de France, gouverneur d'Angers et du
Havre de Grâce, lieutenant général au gouvernement d'Aunis et des îles
de Ré et d'Oléron, et un des hommes d'alors avec lequel il fallut le
plus compter pour les grâces, et souvent pour les affaires. Il avait de
la capacité, mais trop de hauteur dans ses manières. Il mourut à la fin
de 1644\,; ainsi il jouit de toute la fortune de son neveu.

Son autre neveu La Porte, qui s'appelait le marquis de La Meilleraye,
fut un homme de grand sens dans le cabinet, de grande valeur et de
grande capacité à la guerre, tellement que lui et le commandeur furent
fort utiles au cardinal de Richelieu. La Meilleraye était homme
d'honneur et de vertu, doux, affable, poli, obligeant, à ce que j'ai ouï
dire à mon père, dont il était ami particulier, et n'avait pas la
rudesse et la hauteur de son oncle. Il eut le gouvernement de Bretagne,
Nantes, Port-Louis, et fut chevalier de l'ordre en 1633, fit la charge
de grand maître de l'artillerie par commission après le maréchal
d'Effiat son beau-père, l'eut après en titre, lorsqu'en 1634 le célèbre
duc de Sully, après la mort de son fils, consentit enfin a en donner la
démission pour un bâton de maréchal de France, et M. de La Meilleraye
reçut de la main même de Louis XIII le bâton de maréchal de France sur
la brèche de Hesdin qu'il venait de prendre d'assaut. Il mourut en 1664,
fort goutteux, à soixante-deux ans. Il ne laissa qu'un fils de sa
première femme, et n'eut point d'enfants de la seconde, fille du duc de
Brissac. Le maréchal de La Meilleraye et son fils furent tous deux
séparément faits ducs et pairs parmi les quatorze que le roi érigea, et
qu'il enregistra, et reçut en son lit de justice de décembre
1663\footnote{Voy., t. 1\^{}er, p.~449, le récit de la réception de ces
  ducs et pairs.}.

La duchesse de Charost mourut en même temps, à cinquante et un ans,
après plus de dis ans de maladie, sans avoir pu être remuée de son lit,
voir aucune lumière, ouïr le moindre bruit, entendre ou dire plus de
deux mots de suite, et encore rarement, ni changer de linge plus de deux
ou trois fois l'an, et toujours à l'extrême-onction après cette fatigue.
Les soins et la persévérance des attentions du duc de Charost dans cet
état furent également louables et inconcevables\,: et elle les sentait,
car elle conserva sa tête entière jusqu'à la fin avec une patience, une
vertu, une piété, qui ne se démentirent pas un instant, et qui
augmentèrent toujours. Le duc de Charost avait épousé en 1680, étant
fort jeune, la fille du prince d'Espinoy et de la sœur de son père, qui
avait valu, comme on l'a vu ailleurs, le tabouret de grâce à son mari.
M\textsuperscript{me} de Charost mourut trois ans après, et laissa deux
fils. Charost se remaria en 1692 à cette femme-ci, qui était Lamet et
héritière. Le marquis de Baule, son père, tué lieutenant général à
Neerwinden, avait le gouvernement de Dourlens, qui passa à Charost et au
fils unique qu'il eut de cette femme. Il l'avait perdu depuis un an, âgé
de seize ans, et le gouvernement lui revint\,; et pour le dire tout de
suite, le duc de Sully fut trouvé mort dans son lit par ses valets tout
à la fin de l'année, à quarante-huit ans, qui entraient dans sa chambre
pour l'éveiller. Il y avait longtemps qu'il en était menacé, et qu'il
s'endormait partout et à toute heure. C'eût été un honnête homme et de
mise s'il n'eût point été si étrangement et si obscurément débauché. Il
se ruina avec des gueuses. Il était gendre et beau-frère des ducs de
Coislin, et n'eut point d'enfants. Il avait peu servi et paraissait peu
à la cour. Le chevalier de Sully son frère hérita de sa dignité, et eut
les bagatelles qu'il avait du roi. C'étaient les gouvernements de Gien
et de Mantes, et une petite lieutenance de roi de Normandie. Tout cela
ensemble de huit mille livres de rente, mais cela convenait à leurs
terres.

Le roi fit partir le duc de Berwick le 28 novembre, et marcher en
Roussillon quarante bataillons et quarante escadrons, pour faire lever
le blocus que Staremberg faisait de Girone, où le marquis de Brancas,
longtemps depuis maréchal de France, etc., commandait et n'avait plus de
vivres dans la place que pour jusqu'à la fin de décembre. Deux jours
auparavant il avait vu pour la première fois Chamillart dans son
cabinet, depuis sa disgrâce. Bloin l'amena par les derrières au retour
du roi de Marly. Il lui fit mille amitiés, et lui permit de le voir de
temps en temps. Il est plaisant à dire que le roi le désirait depuis
longtemps, et qu'il l'avait mandé plus d'une fois à Chamillart, qui fut
extrêmement sensible à ce zeste de retour qui ne fut pas du goût de
M\textsuperscript{me} de Maintenon. L'audience ne fut guère qu'un quart
d'heure, mais seul. Il sortit par les derrières, ne se montra qu'à peu
de gens, et s'en retourna aussitôt à Paris, où il avait toujours grande
et bonne compagnie de la cour et de la ville. J'y soupais presque tous
les soirs dans le peu que j'allais à Paris.

Des trois plénipotentiaires venus d'Espagne pour aller à Utrecht, il n'y
eut que le duc d'Ossone qui demeura à Paris, en attendant de pouvoir
être admis au congrès. Bergheyck retourna en Espagne, et Monteléon passa
en Angleterre avec le caractère d'ambassadeur. C'est le même qu'on a vu
Vaudemont donner pour évangéliste à Tessé lorsqu'il alla négocier en
Italie, puis à Rome. M\textsuperscript{me} la duchesse de Berry était
grosse depuis plusieurs mois. Il fut question d'une gouvernante. Elle en
usa là-dessus comme elle avait fait pour la charge de premier écuyer de
M. le duc de Berry. Besons était pauvre et vieux, cette place était
utile, il désirait de plus de laisser après lui sa femme en situation de
pouvoir protéger sa famille\,; il nous en parla à moi, et à
M\textsuperscript{me} de Saint-Simon qui le rendit de sa part à
M\textsuperscript{me} la duchesse de Berry. Elle parut ravie de la
vanité d'avoir la femme d'un officier de la couronne, et qui devait son
bâton à M. le duc d'Orléans, quoique d'ailleurs il l'eût bien mérité.
Elle ne laissa rien à dire à tout ce qui pouvait prouver la convenance
de ce choix, elle combla Besons, et le pressa fort de parler au roi. La
vérité est que, tandis qu'elle se montrait si empressée d'avoir la
maréchale de Besons, d'Antin et Sainte-Maure l'avaient tonnelée pour
leur cousine de Pompadour, qui cherchait à toutes restes\footnote{Vieille
  locution qui signifie \emph{à défaut de tout}.} à s'accrocher quelque
part. Rien ne convenait moins à M\textsuperscript{me} la duchesse de
Berry, à la conduite qu'elle avait, et à la situation où elle s'était
mise, qu'une précieuse du premier ordre, affolée de la cour jusqu'à
avoir marié sa fille unique au fils de Dangeau pour s'y fourrer sans y
avoir été de sa vie, toute sous leur coupe, et dans la main de
M\textsuperscript{me} de Maintenon par M\textsuperscript{me} de Dangeau,
par sa sœur à elle la duchesse d'Elbœuf, et par être fille de
M\textsuperscript{me} de Noailles, et petite-fille de
M\textsuperscript{me} de Neuillant, qui avait pris chez elle
M\textsuperscript{me} de Maintenon arrivant des îles, laquelle se
piquait de quelque souvenir.

Pompadour, de son chef, ne convenait pas davantage. On pouvait dire ce
contraste de lui que c'était un sot de beaucoup d'esprit et aussi entêté
de la cour que sa femme, où il ne tenait plus à rien depuis que la place
de menin qu'il avait eue de Dangeau par le mariage de sa fille, et celle
de dame du palais que sa fille avait eue de M\textsuperscript{me} de
Dangeau, n'existaient plus par la mort des Dauphins et de la Dauphine.
Il était frère de la mère de Chalais, et par là lié tant qu'il put à la
princesse des Ursins. Cela était directement opposé à M. le duc
d'Orléans et à M\textsuperscript{me} sa fille, et c'était avec ce qui le
leur était le plus dans la cour qu'ils cherchaient à s'appuyer. D'Antin,
courtisan jusque dans les moelles, ne songea qu'à son fait, dans
l'espérance de plaire à M\textsuperscript{me} de Dangeau, et par ce
service à M\textsuperscript{me} de Maintenon, qu'elle lui ferait
valoir\,; et M\textsuperscript{me} la duchesse de Berry en fut la dupe
de plus d'une façon. Besons, de plus en plus pressé par elle, alla
parler au roi, qui fut bien étonné de se voir demander une chose
accordée à une autre. Le maréchal ne le fut pas moins quand il entendit
le roi lui répondre que M\textsuperscript{me} la duchesse de Berry
s'était moquée de lui, qu'elle et M. le duc de Berry lui avaient demandé
la place pour M\textsuperscript{me} de Pompadour, à qui il avait trouvé
bon qu'ils la donnassent, comme il l'aurait trouvé tout aussi bien
remplie par la maréchale de Besons s'ils la lui avaient proposée. Besons
fut outré d'être joué de la sorte, et si gratuitement, et ne le laissa
pas ignorer à M\textsuperscript{me} la duchesse de Berry, qui se trouva
confondue. M\textsuperscript{me} de Saint-Simon pour sa vade\footnote{Terme
  de jeu qui s'employait au figuré dans le sens de \emph{pour son
  compte, pour son intérêt}.}, lui dit son avis du procédé, et la mit
après au fait de ce qu'elle avait si bien choisi. Elle ignorait, non
l'alliance de Dangeau qui ne le pouvait pas être, mais celle de Chalais,
le fait de M\textsuperscript{me} de Neuillant, et le caractère des
personnes. Elle fut outrée, mais il n'était plus temps. Quatre ou cinq
jours après, M\textsuperscript{me} de Pompadour fut déclarée.
M\textsuperscript{me} de Saint-Simon fit donner la place de
sous-gouvernante à M\textsuperscript{me} de Vaudreuil qui était une
femme d'un vrai mérite. Cela était fort au-dessous d'elle. Son mari
était de bon lieu, et gouverneur général de Canada\,; mais elle avait
peu de bien, beaucoup d'enfants à placer, puis à pousser, qui se sont
depuis avancés par leur mérite, et avec beaucoup d'affaires qui
l'avaient fait revenir de Québec.

M\textsuperscript{me} la duchesse de Berry avait auprès d'elle une
petite favorite de bas étage, bien faite, jolie, d'esprit, qui avait été
élevée auprès d'elle. Elle était fille de Porcadel, commis aux parties
casuelles \footnote{Le nom est en blanc dans le manuscrit.}, et d'une
mère femme de chambre principale de M\textsuperscript{me} la duchesse de
Berry, qui était fille de\ldots{}\footnote{Le nom est en blanc dans le
  manuscrit.}, premier chirurgien de feu Monsieur. Elle l'avait gardée
depuis son mariage, et cherchait à la marier. Elle trouva Mouchy, homme
de qualité, avancé en âge, et dans le service, franc bœuf d'ailleurs à
embâter. Il était parent des Estrées, et cette parenté ne leur faisait
pas déshonneur. Ils en firent leur cour à M\textsuperscript{me} la
duchesse de Berry\,; le mariage fut bâclé en un moment. Elle voulait y
être et s'en amuser, et elle ne savait où le faire. Elle pria tant et si
bien M\textsuperscript{me} de Saint-Simon qu'elle en eut la
complaisance. Le festin très-nombreux, le coucher, le dîner du lendemain
se fît dans notre appartement, et nous n'eûmes que vingt-quatre heures
pour le nommer. Ils ne laissèrent pas d'être magnifiques. Comme il était
tout près et de la tribune et du plain-pied, M\textsuperscript{me} la
duchesse de Berry en eut tout l'amusement qu'elle s'en était proposé.
Cette Mouchy fut une étrange poulette, comme on le verra en son temps.

Le marquis de Meuse, de la maison de Choiseul, qui avait un régiment,
épousa en même temps chez la duchesse d'Antin une fille de feu
Zurlauben, lieutenant général et, bien que Suisse, homme de qualité, et
de la sœur de Sainte-Maure.

L'ennui gagnait le roi chez M\textsuperscript{me} de Maintenon, dans les
intervalles de travail avec ses ministres. Le vide qu'y laissait la mort
de la Dauphine ne se pouvait remplir par les amusements de ce très-petit
nombre de dames qui étaient quelquefois admises. Les musiques, qui y
devenaient fréquentes, par cela même languissaient. On s'avisa de les
réveiller par quelques scènes détachées des comédies de Molière, et de
les faire jouer par des musiciens du roi vêtus en comédiens.
M\textsuperscript{me} de Maintenon, qui avait fait revenir le maréchal
de Villeroy sur l'eau pour amuser le roi par les vieux contes de leur
jeunesse, l'introduisit seul aux privances de ces petites ressources,
pour les animer de quelque babil. C'était un homme de tout temps dans sa
main, et qui lui devait son retour. Il était propre à hasarder certaines
choses qui n'étaient pas de la sphère des ministres, qu'elle voulait qui
lui revinssent après par le roi pour la sonder\,; s'il y avait lieu, les
appuyer, et les pousser d'autant plus délicatement et sûrement qu'elles
sembleraient moins venir d'elle. La mort des princes du sang qui n'en
avaient laissé que d'enfants, celle des Dauphins et de la Dauphine, le
pis que néant où la plus noire et fine politique avait réduit M. le duc
d'Orléans, et le tremblement inné de M. le duc de Berry sous le roi
soigneusement entretenu, ouvraient un vaste champ à l'ambition démesurée
de M. du Maine et à l'affolement pour lui de sa toute-puissante
gouvernante. Le maréchal de Villeroy était un vil courtisan et rien de
plus, nul instrument ne leur était plus propre\,; M\textsuperscript{me}
de Maintenon ne songea donc plus qu'à le mettre à toute portée de s'en
pouvoir servir.

Peu de jours après, le roi déclara, allant à la messe, qu'il avait donné
le gouvernement de Guyenne au comte d'Eu. Ainsi les deux fils du duc du
Maine, revêtus déjà des survivances de Languedoc, des Suisses et de
l'artillerie, se trouvèrent passablement pourvus. Le maréchal de
Villeroy n'y influa point, que je pense\,; il ne pouvait encore en être
là. Quelque accoutumée que fût la cour à des accroissements gigantesques
de ses bâtardeaux, elle ne laissa pas d'être également surprise et
consternée de cette énorme augmentation, et de le laisser apercevoir à
travers ses flatteries, dont M. du Maine fut assez embarrassé. Une autre
surprise bien plus grande suivit celle-ci de fort près et termina cette
année. Les ducs de La Rochefoucauld s'étaient accoutumés depuis
longtemps à ne vouloir chez eux qu'un successeur pour recueillir tous
les biens et toute la fortune du père, à ne marier ni filles ni cadets,
qu'ils comptaient pour rien, et à les jeter à Malte et dans l'Église\,;
le premier duc de La Rochefoucauld fit son second et son quatrième fils
prêtres. L'aîné mourut évêque de Lectoure, l'autre se contenta
d'abbayes, le second fut chevalier de Malte. De six filles qu'il eut,
quatre furent abbesses, la dernière religieuse. La troisième, plus
coriace que les autres, voulut absolument un mari. On ne lui voulait
rien donner. M\textsuperscript{me} de Puysieux, qui a depuis été si en
faveur auprès de la reine mère pendant sa régence, languissait dans la
disgrâce et l'exil où était mort le chancelier de Sillery, son
beau-père, et qui avait fait perdre à son mari sa charge de secrétaire
d'État et sa fortune. Elle était Valencey, glorieuse à l'excès, et
faite, comme on le vit depuis, pour le monde et pour l'intrigue.
L'alliance l'éblouit avec raison, elle tint lieu de dot. Cette raison
courba l'orgueil des La Rochefoucauld\,; le duc donna sa fille à
Sillery. Tous deux sont morts longues années depuis à Liancourt, ruinés,
et M\textsuperscript{me} de Sillery, qui n'avait rien eu, y a passé la
plupart de sa vie défrayée, pour se servir d'un terme honnête, par son
frère et par son neveu.

Le second duc de La Rochefoucauld, qui a tant figuré dans les troubles
contre Louis XIV, et si connu par son esprit, eut cinq fils et trois
filles. Des quatre cadets, trois furent chevaliers de Malte\,; et le
dernier, prêtre, fort mal appelé\,; et tous quatre avec force abbayes.
Les trois filles moururent sibylles dans un coin de l'hôtel de La
Rochefoucauld, où on les avait reléguées, ayant à peine de quoi vivre,
et toutes trois dans un âge très-avancé.

Le troisième duc de La Rochefoucauld, le favori du roi, et que nous
verrons bientôt mourir, n'eut que deux fils\,: l'aîné qui fut fait duc
cinquième de La Rocheguyon, en épousant la fille aînée de Louvois\,; et
le marquis de Liancourt qui ne s'est point marié. Du père et de ses deux
fils on en a souvent parlé.

Le duc de La Rocheguyon ne fut pas si discret que son père\,: il eut
huit garçons et deux filles. Le second ne vécut que dix ans\,; l'aîné et
le troisième moururent en entrant dans le monde\,; le quatrième fut
chargé des abbayes de ses oncles et grands-oncles à mesure qu'elles
vaquèrent\,; le cinquième mourut aussi à dix ans\,; le sixième fut jeté
sur mer sous le nom de comte de Durtal. C'est lui qui fut du voyage des
galions que ramena Ducasse, que ce général envoya porter au roi la
nouvelle de leur arrivée, et qui est aujourd'hui cinquième duc de La
Rochefoucauld. Le septième mourut encore à neuf ou dix ans. Le huitième
et dernier fut chevalier de Malte, et eut, tout enfant, la commanderie
magistrale de Pézénas à la recommandation du roi. L'aînée des deux
filles mourut fille de Sainte-Marie\,; la cadette tint bon jusqu'à
vingt-cinq ans, et fut enfin mariée, en 1725, au duc d'Uzès
d'aujourd'hui, qui voulut bien se contenter de peu de chose. Ce tableau
expliqué, voici ce qui arriva.

M. de La Rocheguyon ne se trouva plus que trois fils. L'aîné avait
vingt-cinq ans alors et plus de soixante mille livres de rente en
bénéfices, le comte de Durtal, et le commandeur. Cela se trouvait fort
mal arrangé. Pour bien faire il eût fallu que Durtal eût été l'aîné,
c'est ce que voulurent les père et mère. L'abbé n'avait jamais voulu
ouïr d'entrer dans les ordres. Tant qu'il avait eu des aînés ç'avait été
son affaire, mais l'étant devenu, cela devint l'affaire de ses parents.
Ils le pressèrent de s'engager, ils lui détachèrent dévots, docteurs,
prélats\,; on ne put le déprendre de l'expectative sûre des dignités et
des biens qui alors le regardaient uniquement. Il en voulait jouir quand
ils viendraient à lui échoir. Il n'avait eu de vocation à l'état qu'on
lui avait fait embrasser que celle des cadets de cette maison.

Outre le désir d'accumuler toujours tout sur la même tête, une autre
raison puissante y tenait MM. de La Rochefoucauld attachés. Le père de
celui-ci n'avait jamais pu digérer le rang de prince donné à MM. de
Bouillon. Il se croyait d'aussi bonne maison qu'eux, et il n'avait pas
tort\,; il croyait aussi l'avoir aussi bien mérité, et par les mêmes
voies. Il ne se trompait pas encore, et ces voies n'étaient pas
étrangères à sa maison. Mais il ne put parier de mérite à la guerre ni
dans le cabinet avec MM. de Bouillon et de Turenne. Quoique plus galant
qu'eux, et d'un esprit plus propre aux manéges des ruelles et aux essais
des beaux esprits, il ne put atteindre à la considération de leurs
alliances, à leur autorité dans les partis, à leur réputation fondée sur
les choses qu'ils avaient ourdies et exécutées, à l'opinion que le
cardinal Mazarin en conçut, et à l'espérance d'amitié, de conseil et de
protection qu'il se figura de trouver en eux en se les attachant, comme
il fit par tout ce qu'il leur prodigua. Ce ver rongeur de princerie
passa du père au fils. Il espéra ce rang d'une faveur constante qui
obtint sans cesse tout ce qu'il voulut\,; mais ce rang, qu'il demanda
souvent à un maître qui était son ami, il ne put jamais l'arracher,
quelques efforts qu'il ait faits\,; et ce dépit ajouta encore à la
disgrâce des puînés et des filles de la maison, qu'on ne voulait ni
établir ni montrer à la cour. Ce fut donc une chose bien dure, à des
gens si absolus dans leur famille, de trouver une résistance invincible
dans leur aîné d'entier dans les ordres et de renoncer à son aînesse.

À bout d'espérance de ce côté-là, ils prirent une autre route. Ils lui
proposèrent de quitter le petit collet, puisque c'était un état qu'il ne
voulait pas suivre. Mais à ce petit collet tenaient soixante mille
livres de rente. Il avait vu tous ses frères constamment traités comme
de petits garçons, et manquer à tout âge du plus nécessaire. La douceur,
l'onction, la tendresse n'étaient pas le faible de leurs parents.
L'extrême épargne l'était davantage. Il ne crut donc pas {[}devoir{]} se
livrer à leur merci en quittant ses bénéfices. Il tergiversa, il essuya
prières, menaces, conseils. Poussé enfin au pied du mur, il déclara
qu'il demeurerait abbé et aîné, pour faire en temps et lieu ce qui lui
conviendrait davantage\,; qu'il était trop jeune pour n'avoir point
d'état, et trop vieux pour se faire mousquetaire, puis capitaine en
attendant un régiment. Rien n'était plus sensé, mais ce n'était pas le
compte de sa famille. On en vint aux gros mots, on lui chassa des
domestiques principaux auxquels il prenait le plus de confiance, on lui
détacha toutes les personnes qu'on crut qui lui feraient plus
d'impression. Il écouta tout, il souffrit tout avec toute la douceur, la
patience et le respect possible, sans laisser échapper une plainte ni
une parole qu'on pût reprendre, mais sans pouvoir être ébranlé. La
famille, rugissant et ne sachant plus que faire, eut recours au dernier
remède.

M. de La Rochefoucauld, aveugle et retiré au chenil, se fit mener dans
le cabinet du roi, à qui il raconta avec sa véhémence ordinaire, malgré
son âge, l'état déplorable où sa famille allait être réduite par
l'opiniâtreté de son petit-fils qui voulait manger à deux râteliers. Il
cria, il pleura, il se désespéra, il se dit bien misérable de survivre à
la perte de sa maison. Cette perte était imaginaire avec trois
petits-fils, tous trois jeunes et en état d'avoir des enfants. Mais
marier des cadets et les voir sans rang vis-à-vis ceux des Bouillon,
était l'enclouure qui faisait faire tant de vacarmes. Ils voulaient de
plus, en habiles gens, profiter de leur prétendu malheur pour tirer du
roi une grâce inouïe et qui n'avait jamais été imaginée que pour les
bâtards du roi par l'édit de 1711, qui sous d'autres prétextes n'avait
été fait que pour eux, et qui de plus abroge même rétroactivement les
duchés femelles. Cet édit, par une de ses plus énormes nouveautés,
permet aux bâtards du roi revêtus de plusieurs duchés, qui vont toujours
à l'aîné des fils, d'en donner à leurs cadets, et de les faire ainsi
ducs et pairs, par une exception à eux particulière et privativement à
tous autres. M. de La Rochefoucauld ramassa donc toutes les forces qu'il
put tirer de son ancienne et constante faveur, de son ascendant sur le
roi, de son âge, de son aveuglement du désespoir où il était, et de la
désolation de sa maison. Il redoubla ses cris, ses pleurs, ses furies\,;
et il étourdit si bien le roi que, moitié compassion de ce vieillard
qu'il avait si longtemps aimé, moitié désir de finir une scène si
importune, il lui accorda ce qu'il lui demanda, contre toutes les lois
et les règles, contre les termes de l'érection et de l'enregistrement de
tous les duchés, et de celui de La Rochegnyon comme de tous les autres,
contre l'orgueil d'assimiler quelqu'un à ses bâtards\,; et il permit au
duc de La Rocheguyon de céder ce duché vérifié à M. de Durtal, son
second fils, et de faire de ce cadet tige nouvelle de ducs de La
Rocheguyon, de la même ancienneté de l'érection faite pour le père, en
en dépouillant son aîné et sa postérité qui y était uniquement et
distinctement appelé. L'étonnement de la cour, pour ne rien dire de
plus, surpassa encore la joie de MM. de La Rochefoucauld père et fils.
Ce dernier se démit, dès que les patentes furent faites, de la terre et
de la dignité de La Rocheguyon, en faveur du comte de Durtal, qui prit
aussitôt le nom et le rang de duc de La Rocheguyon. Ce fut par donation
entre vifs pour la terre, dont le père retint les revenus qui sont de
quatre-vingt mille livres de rente, avec un superbe château, et les plus
beaux droits du monde, au bord de la Seine et près de Paris. L'abbé, qui
se voyait si étrangement frustré, espéra bien y revenir en d'autres
temps, et les ducs postérieurs aussi.

L'affaire consommée, M. de La Rochefoucauld se fit encore conduire dans
le cabinet du roi. Il y recommença ses plaintes et ses douleurs, et il
obtint encore que le roi parlerait à son petit-fils qu'il n'avait jamais
vu, pour l'engager à opter. L'abbé fut donc obligé de venir trouver le
roi, dont il ne douta pas d'être maltraité. Il y fut heureusement
trompé\,: le roi lui parla avec une bonté de père, et l'abbé lui
répondit avec tant de respect, de sagesse et de raison qu'il le désarma.
Tout tenait au revenu, et à l'indépendance d'en toucher suffisamment. Le
roi le sentit et n'ignorait pas à qui il avait affaire. Ses parents,
ainsi sans ressource, se tournèrent d'un autre côté. Ils voulaient avant
tout demeurer maîtres de leur bourse, et l'abbé de ses bénéfices pour
n'être pas à leur discrétion. Pour accommoder l'un et l'autre, ils
imaginèrent un bref du pape qui permît à l'abbé d'aller à la guerre en
conservant ses bénéfices. Ils le lui proposèrent\,; il n'osa pas y
résister, parce que toute la difficulté sur laquelle il s'était tenu
jusqu'alors était par là levée. De ces brefs, il y en avait mille
exemples, même parmi les simples particuliers. Forbin, capitaine des
mousquetaires gris avant Maupertuis, en avait un, et il était mort abbé
et lieutenant général des armées du roi\,; et plusieurs autres comme
lui. Rome ne fit aucune difficulté. Le pauvre abbé de La Rochefoucauld
prit donc l'épée. La guerre de Hongrie fit partir les enfants de M. du
Maine et plusieurs autres. L'abbé y alla\,; mais en arrivant à Bude, la
petite vérole le prit en 1717, à trente ans, et en délivra son père, et
son frère duc à ses dépens. Ce qui est arrivé depuis dans cette famille
n'a pas donné lieu de croire que Dieu ait béni ces arrangements.

\hypertarget{chapitre-xiii.}{%
\chapter{CHAPITRE XIII.}\label{chapitre-xiii.}}

1713

~

{\textsc{1713.}} {\textsc{- Victoire de Steinbok sur les Danois, qui
brûle Altona.}} {\textsc{- La Porte secourt le roi de Suède d'argent, et
change à son gré son ministère.}} {\textsc{- Ragotzi en France.}}
{\textsc{- Digression sur sa manière d'y être\,; son extraction, sa
famille, sa fortune et de ses proches, de Serini et Tékéli\,; son
traitement\,; son caractère.}} {\textsc{- Trente mille livres de pension
à M\textsuperscript{lle} d'Armagnac.}} {\textsc{- Trois mille livres de
pension rendues à M\textsuperscript{lle} de Chausseraye.}} {\textsc{-
Trois mille livres de pension à M\textsuperscript{me} de Vaugué.}}
{\textsc{- Girone délivré et ravitaillé.}} {\textsc{- Berwick de retour
à la cour.}} {\textsc{- Bockley brigadier.}} {\textsc{- Brancas
chevalier de la Toison d'or et ambassadeur en Espagne.}} {\textsc{-
Amusements multipliés chez M\textsuperscript{me} de Maintenon.}}
{\textsc{- Matignon cède à son fils ses charges de Normandie.}}
{\textsc{- Mariage de Maillebois avec une fille d'Alegre.}} {\textsc{-
Mariage de Châteaurenauld avec une fille de la maréchale de Noailles.}}
{\textsc{- Mariage de M. d'Isenghien avec M\textsuperscript{lle} de
Rhodes.}} {\textsc{- Arias, Polignac, Odescalchi, Sala, expectorés
cardinaux\,; quels les trois étrangers\,; pourquoi \emph{in petto}\,;
pourquoi expectorés.}} {\textsc{- Polignac, seul rappelé d'Utrecht,
arrive et reçoit de la main du roi sa calotte rouge.}} {\textsc{-
Jacques II}} \footnote{Saint-Simon veut parler de Jacques III, fils de
  Jacques II, et prétendant au trône d'Angleterre.} {\textsc{, sous le
nom de chevalier de Saint-Georges, se retire pour toujours de France par
la paix, et va en Lorraine.}} {\textsc{- Faiblesse du roi pour les
cardinaux, qui leur marque une place à la chapelle pour le sermon.}}
{\textsc{- Adoucissements sur les preuves pour entrer dans le chapitre
de Strasbourg, et ses causes.}} {\textsc{- Bévue à l'égard des ducs.}}
{\textsc{- Mort de la marquise de Mailly et sa conduite dans sa
famille.}} {\textsc{- Mort de l'évêque de Lavaur, son fils.}} {\textsc{-
Mort de Brissac, ci-devant major des gardes du corps.}} {\textsc{- Sa
fortune\,; son caractère.}} {\textsc{- Plaisant tour de Brissac aux
dames dévotes de la cour.}}

~

La cour, dans les premiers jours de cette année, apprit la victoire de
Steinbock sur les Danois, dans le pays de Mecklembourg, qui fut
complète. Ce comte, à la tête de ce qu'il était resté de troupes
suédoises depuis la défaite du roi son maître à Pultawa, s'était
toujours soutenu, et battit enfin complètement une armée fort supérieure
à la sienne. Il marcha ensuite à Altona, à qui il demanda six cent mille
livres de contribution. Cette ville, qui est considérable mais sans
fortifications, est vis-à-vis de Hambourg, l'Elbe entre deux. Elle eut
l'imprudence de refuser de payer\,; aussitôt après les Suédois y mirent
le feu. Il y eut trois mille maisons brûlées, et tout ce qui peut
accompagner ces sortes de malheurs. Cette ville est au roi de Danemark,
dont le territoire sert de fort près Hambourg, des deux côtés de l'Elbe,
et tient toujours cette ville impériale dans une grande jalousie et dans
la crainte de ses prétentions. Steinbok eut cinq mille prisonniers et
quantité d'officiers. Après l'exécution d'Altona, il alla tirer de
grandes contributions du Holstein danois. Le roi de Suède reçut beaucoup
d'argent en ce même temps de Constantinople, où il fit faire tous les
changements dans le ministère que ce prince désira.

Ragotzi, échappé de son étroite prison de Neustadt à force d'argent et
d'adresse, avait gagné la Pologne, s'était enfin embarqué à Dantzick, et
arriva à Rouen. Il avait pris le titre de prince de Transylvanie,
reconnu du pays, du Turc et de tous les mécontents hongrois, qui le
voulaient faire roi de Hongrie, lorsque le prodigieux succès de la
bataille d'Hochstedt changea toute la face des affaires. La France
l'avait aussi reconnu et stipendié. Des Alleurs avait été longtemps
auprès de lui, et à la fin y avait pris caractère public d'envoyé du
roi, d'où il était passé à l'ambassade de Constantinople. Ragotzi, qui
n'avait de ressource qu'en France, comprit bien que son titre y serait
embarrassant et l'exclurait de tout\,; il prit donc le parti de
l'incognito, ne voulut et ne prétendit rien, et prit le nom de comte de
Saroz. M. de Luxembourg, qui était à Rouen, le reçut sans honneurs, mais
avec les civilités les plus distinguées, le logea, le défraya et lui
prêta sa maison à Paris, où il vint peu de jours après. En dernier lieu
il venait d'Angleterre, où il était peu resté. Ce chef si chéri des
mécontents de Hongrie mérite bien une petite digression.

Son trisaïeul, Sigismond Ragotzi, fut élu prince de Transylvanie après
la mort du fameux Botskay en 1606. C'était un homme sans ambition,
tranquille et paisible, également bien avec le Grand Seigneur Achmet et
l'empereur Matthias. Il ne se souciait point de la principauté\,; et dès
l'an 1608 il la céda à Gabriel Bathori, que ses cruautés firent chasser
par Bethlem Gabor, qui devint prince de Transylvanie.

Georges Ragotzi fut fait prince de l'empire, et fut élu prince de
Transylvanie, en 1631, par la protection de la maison d'Autriche. Il
épousa la fille d'Étienne, frère de Bethlem Gabor, prince de
Transylvanie\,; en secondes noces, Suzanne Lorantzi, dont il eut
Sigismond, duc de Mongatz, qui n'eut point d'enfants d'Henriette, fille
de Frédéric V, électeur palatin.

Du premier lit vint autre Georges, prince Ragotzi, prince de
Transylvanie après son père, mort en 1648. Ce second Georges fut fort
malmené des Turcs, et mourut à Waradin, en juin 1660, des blessures
qu'il avait reçues, un mois auparavant, en un combat qu'il perdit contre
eux à Plansenberg, près d'Hermanstadt, où il fit des prodiges de valeur.
Il avait épousé Sophie, héritière de la maison Bathori, dont il
laissa\,: Frédéric, prince Ragotzi, qui passa toute sa vie particulier.
Il épousa Hélène Esdrin, fille de Pierre, comte de Serin, vice-roi ou
ban de Croatie, qui fut un des principaux chefs de la révolte qui
commença en 1665 contre l'empereur. Les Hongrois se plaignaient des
garnisons allemandes et de l'infraction de leurs privilèges. Serin, au
lieu d'exécuter les ordres de l'empereur pour les fortifications des
places frontières, ne songea qu'à les traverser. Il leva des troupes en
1666 avec le comte Nadasti, président du conseil souverain de Hongrie,
sous prétexte de s'opposer aux Turcs. Leur dessein était de se défaire
de l'empereur Léopold à son passage près de Puttendorf, place de
Nadasti, allant avec douze gentilshommes seulement et Lobkowitz, grand
maître de sa maison, au-devant de l'infante d'Espagne, qu'il allait
épouser. Le commandant de l'embuscade devait l'envelopper et le
poignarder\,; mais elle ne fut placée qu'après qu'il fut passé. Ce grand
coup manqué, et Serin, irrité du refus du gouverneur de Carlstadt qui
l'aurait rendu tout à fait maître de la Croatie, il résolut de
soustraire la Hongrie à l'empereur. Il gagna le comte Frangipani dont il
avait épousé la sœur, le comte de Tattenbach et son propre gendre le
prince Ragotzi, qui est père de celui qui donne lieu à cette digression.
Tout ceci se passa en 1669.

Ces chefs sentirent qu'ils ne pouvaient se passer des Turcs\,; ils leur
firent des propositions. Le Grand Seigneur voulut des places de sûreté
en Hongrie pour leur donner des troupes\,; ils firent ce qu'ils purent
pour lui en livrer. Cependant, soit que le Grand Seigneur, peu porté à
la guerre, en révélât le secret, soit qu'il eût été découvert par un
Grec nommé Panagiotti, qui servait d'interprète au résident de
l'empereur à Constantinople, l'empereur sut tout ce qui s'y était passé.
En 1670, il envoya le général-major Spanckaw avec six mille hommes en
Croatie, où Serin, trop faible pour résister, implora la clémence de
l'empereur, et lui envoya son fils unique pour otage de sa fidélité
future. Cela n'empêcha point Spanckaw d'assiéger Schackthom, où Serin et
Frangipani, son beau-frère s'étaient retirés, et de s'en rendre maître,
où il prit la comtesse Serin, sœur de Frangipani. Les deux beaux-frères
s'étaient évadés par une porte secrète\,; ils se retirèrent dans un
château du comte Keri qu'ils comptaient leur ami, mais qui se saisit
d'eux, et les fit conduire à Vienne, où ils furent mis en prison. Serin
y éprouva le sort ordinaire des grands criminels malheureux. Frangipani,
pour avoir grâce et obtenir ses charges, n'oublia rien pour le perdre.
Ragotzi même livra toutes les lettres qu'il avait reçues de lui. Le
capitaine Tcholnitz, qui était de leur secret, et qui s'en repentit,
porta à l'empereur une lettre que Serin lui avait donnée pour Frangipani
dès avant leur emprisonnement, depuis lequel Nagiferentz fut arrêté\,:
c'était le secrétaire de la ligue. On trouva chez lui les pièces de la
conjuration, les divers traités, et cinq cassettes pleines de lettres,
d'instructions, d'actes, qu'on envoya à Vienne. Nadasti avait déjà été
arrêté. Le procès fut juridiquement instruit\,; les plus grands
seigneurs furent nommés juges\,; les prisonniers, qui avaient été
transférés à Neustadt, y eurent la tête coupée publiquement le 30 avril
1671. La comtesse Serin, sœur de Frangipani, l'eut deux ans après, 18
novembre 1673. Leur fils unique perdit le nom et les armes de sa
famille\,; on lui donna le nom de Gadé, et on le renferma pour toute sa
vie dans le château de Rattenberg. L'irruption de l'électeur de Bavière
dans le Tyrol le fit transférer en 1703 à Gratz en Styrie, où il mourut,
la même année, de maladie. Sa sœur unique, veuve Ragotzi en 1681, et
mère de notre Ragotzi, était ainsi devenue puissante héritière.

Le fameux Tékéli avait eu envie de l'épouser lorsqu'elle était fille. Le
comte Étienne, son père, était fort puissant en Hongrie, et y jouissait
de trois cent mille livres de rente. Les ministres de l'empereur furent
accusés de l'avoir injustement enveloppé dans l'affaire du comte Serin,
pour s'emparer de ses grands biens. Après l'exécution du comte Serin et
des autres chefs, le général Sporck alla assiéger les places de Tékéli,
qui, ne se trouvant pas en état de leur résister, l'amusa, et fit évader
cependant son fils unique Émeric Tékéli, travesti en paysan, avec deux
gentilshommes déguisés de même, qui le conduisirent heureusement en
Pologne. Son père ne survécut guère. Ses biens furent confisqués. Il
avait trois filles qui furent menées à Vienne\,; elles s'y firent
catholiques\,; l'empereur en prit soin. Deux épousèrent les princes
François et Paul Esterhazy, ce dernier était palatin de Hongrie\,;
l'autre le baron Letho.

Émeric, leur frère, qui se rendit depuis si fameux, vint de Pologne, où
il s'était retiré d'abord, en Transylvanie. Il s'y rendit si agréable au
prince Abaffi, par son esprit et sa valeur, qu'il le mit à la tête de
son conseil et de ses troupes, et l'envoya au secours des mécontents de
Hongrie, dont il fut fait généralissime en 1778, quoiqu'il n'eût encore
que vingt ans. Il se rendit si redoutable par ses conquêtes et ses
progrès, que l'empereur le fit rechercher d'accommodement, dont on ne
put convenir. Il le fut encore en 1680 pendant une trêve de deux mois.
Il offrit de se faire catholique pour épouser la fille du comte Serin,
veuve du prince Ragotzi, mère de celui qu'on vient de voir arriver à
Paris. L'empereur n'y put consentir, dans la crainte de le rendre trop
puissant par les grands biens de cette dame, et qu'elle ne voulût venger
la mort de son père. Les états de Hongrie furent assemblés par
l'empereur pour traiter\,; mais Tékéli, irrité du refus de ce grand
mariage, déclara qu'il ne pouvait rien faire sans les Turcs. Tandis que
l'empereur envoya le baron de Kaunitz à Constantinople, Tékéli
recommença les hostilités avec des succès qui s'augmentèrent par les
secours qu'il reçut de la Porte. Il fut encore question
d'accommodement\,; il se rompit et se renoua. Le Grand Seigneur ayant
appris que Tékéli pensait sérieusement à rentrer sous l'obéissance de
l'empereur, lui envoya offrir l'assurance de la principauté de
Transylvanie après Abaffi. Lui et les autres chefs promirent
quatre-vingt mille écus de tribut annuel, au nom de la Hongrie, si les
Turcs les voulaient assister puissamment. Cela n'empêcha pas Tékéli de
convenir, en octobre 1681, d'une suspension d'armes qui devait finir au
dernier juin 1682, avec l'empereur, qui en avait besoin pour faire
couronner l'impératrice-reine de Hongrie. Tékéli, qui devait agir
incontinent après, alla cependant prendre des mesures avec le bacha de
Bude, qui le reçut superbement, et à tel point qu'on prétendit qu'il
l'avait revêtu de la couronne et des autres ornements royaux de Hongrie,
en présence de plusieurs autres bachas. Le secrétaire de Tékéli était
cependant à Vienne pour obtenir la permission d'épouser la comtesse
Serin. Il la dut à l'opinion qu'on eut à Vienne qu'il était en état de
le faire, malgré le refus, et au désir extrême de le gagner. De Bude il
alla donc au château de Montgatz, qui était à la comtesse et sa
résidence ordinaire, où leur mariage fut incontinent célébré avec grande
magnificence. Il y fit entrer de ses troupes et dans toutes les autres
places de sa nouvelle épouse, se joignit aux Turcs au commencement
d'août 1682, porta la terreur partout, et fit frapper des médailles sur
lesquelles il prit le titre de prince de Hongrie. Il y eut encore des
propositions d'accommodement à la diète de Cassovie, qui n'eurent aucun
effet.

Tékéli, voyant approcher les Turcs, répandit un manifeste qui ouvrit aux
mécontents les portes de la plupart des villes. Le siége de Vienne fut
formé par les Turcs, que le fameux Jean Sobieski, roi de Pologne, fit
lever par la victoire complète qu'il remporta. Il s'entremit ensuite de
l'accommodement des mécontents, mais inutilement par la hauteur de la
cour de Vienne. Tékéli, apprenant que ces pourparlers le rendaient
suspect à la Porte, alla à Constantinople, eut l'adresse de pénétrer
jusqu'au Grand Seigneur, lui dit qu'il lui apportait sa tête. Cette
hardiesse, soutenue de ce qu'il sut dire, lui réussit si bien, que le
Grand Seigneur l'assura de sa protection et de ses secours. Il fut
depuis constamment attaché à la Porte, et à la tête des mécontents.
Cette même année elle le fit prince de Transylvanie par la mort
d'Abaffi. Il y défit entièrement le général Heusler, et le prit
prisonnier. Il continua depuis divers exploits, jusqu'à ce que, brouillé
avec les Transylvains, et accablé de goutte, il se retira à
Constantinople. Il y fut reçu et traité en grand prince, avec de grands
revenus, et divers palais du Grand Seigneur pour sa demeure. Il mourut
dans ce brillant état le 13 septembre 1705, n'ayant pas encore cinquante
ans, et catholique. Son épouse était morte le 10 février 1703. Revenons
maintenant à son fils du premier lit, le prince Ragotzi. Elle n'eut
point d'enfants de ce fameux comte Tékéli.

Léopold-François, prince Ragotzi, avait apporté en naissant plus qu'il
ne fallait pour être suspect à la cour de Vienne. Ses liaisons et ses
droits ne le rendirent pas innocent. Il fut arrêté en avril 1701, et
conduit à Neustadt, accusé d'avoir tenté de soulever la Hongrie. Il
vendit tout ce qu'il put avoir à Neustadt, gagna avec cinq cents ducats
d'or Leheman, capitaine au régiment de Castelli, qui lui fournit un
habit de dragon, se familiarisa avec ses gardes, officiers et soldats,
les régala, les enivra, se sauva dans un faubourg, le 7 novembre de la
même année 1701, où il trouva trois chevaux qu'on lui tenait tout prêts,
et gagna Raab et la Pologne, d'où il alla joindre le comte Berzini, l'un
des chefs des mécontents de Hongrie, On détacha tout ce qu'on put après
lui dès qu'on s'aperçut de son évasion. On afficha dans Vienne des
placards de proscription, où sa tête fut mise à prix. Sa femme, qui
était à Vienne, fut enfermée dans un couvent. On exécuta à mort le
capitaine qui avait fourni l'habit de dragon, et tous ceux qu'on crut
avoir favorisé sa fuite. En avril 1703, il fut condamné à Vienne d'avoir
la tête coupée. Sa femme eut permission en 1705 de se retirer en Bohême.
Elle y fut arrêtée en 1707, mais elle trouva bientôt après moyen de se
sauver en Saxe, d'où elle se retira à Dantzick. Ses deux fils furent mis
à la garde du maître d'hôtel de l'évêque de Raab. En 1704 Ragotzi fut
proclamé prince de Transylvanie. Il le fut de nouveau en 1707. On a vu
en divers endroits de ces Mémoires plusieurs de ses exploits, et qu'il
fit trembler l'empereur dans Vienne, dont la campagne fut plus d'une
fois ravagée, et le feu des villages vu des fenêtres du palais. La
malheureuse bataille d'Hochstedt arrêta tous ses progrès\,; les
mécontents se dissipèrent. Leurs chefs pour la plupart firent leur
accommodement l'un après l'autre. Lui, qui n'y pouvait espérer ni
honneur ni sûreté, se retira en Pologne, et vint en France, qui lui
avait fourni des subsides, et tenu un ministre près de lui avec
caractère public.

Il avait épousé, en septembre 1694, Charlotte-Amélie, fille de Charles
landgrave de Hesse-Rinfels, Wanfried, et d'Alexandrine-Julie, comtesse
de Linange. Ce landgrave était frère puîné du landgrave Guillaume de
Hesse-Rinfels, mari d'une sœur de M\textsuperscript{me} de Dangeau, et
père du landgrave de Hesse-Rinfels, dont trois filles ont épousé le roi
de Sardaigne M. le Duc dont elle a laissé M. le prince de Condé, et le
jeune prince de Carignan d'aujourd'hui. Ragotzi était donc gendre du
beau-frère de M\textsuperscript{me} de Dangeau. Elle était tout
Allemande et fort attachée à sa parenté. Cette alliance de Ragotzi était
fort proche, quoique sans parenté effective, mais elle fit sur elle la
même impression. Elle était favorite de M\textsuperscript{me} de
Main-tenon, fort bien avec le roi, et de toutes leurs parties et
particuliers. Dangeau, répandu de toute sa vie dans le plus grand monde
et dans la meilleure compagnie de la cour, en était enivré. Il se mirait
dans tout ce à quoi il était parvenu. Il nageait dans la grandeur de la
proche parenté de sa femme. Tous deux firent leur propre chose de
Ragotzi, qui ne connaissait personne ici, et qui eut le bon esprit de se
jeter à eux. Ils le conduisirent très-bien. Non-seulement il ne
prétendit rien, mais il n'affecta quoi que ce soit\,: et par là il se
concilia tout le monde en le mettant à son aise avec lui, et soi avec
tous. On lui en sut gré dans un pays si fort en prise aux prétentions,
et il en reçut cent fois plus de considération et de distinction.

Dangeau, qui tenait chez lui une grande et bonne table, et qui vivait
avec le plus distingué et le plus choisi, mit peu à peu, mais
promptement, Ragotzi dans la bonne compagnie. Il prit avec elle, et
bientôt il fut de toutes les parties, et de tout avec tout ce qu'il y
avait de meilleur à la cour, et sans mélange. M\textsuperscript{me} de
Dangeau lui gagna entièrement M\textsuperscript{me} de Maintenon, et par
elle M. du Maine. Le goût à la mode de la chasse, avec quelque soin, lui
familiarisa M. le comte de Toulouse jusqu'à devenir peu à peu son ami
particulier. Il vint ainsi à bout de faire de ces deux frères son
conseil pour sa conduite auprès du roi, et les canaux pour tout ce qu'il
en put désirer de privances, et de ces sortes de distinctions de
familiarité personnelle et de distinctions d'égards qui sont
indépendantes de rang. Avec ces secours, et qui ne tardèrent pas, il fut
de toutes les chasses, de toutes les parties, de tous les voyages de
Marly, mais demandant comme les autres courtisans, ne sortit presque
point de la cour, y voyait le roi assidûment, mais sans contrainte aux
heures publiques, et très-rarement sans que le roi cherchât à lui
parler, et seul dans son cabinet dès qu'il en désirait des audiences,
mais sur quoi il était fort discret. Ragotzi était d'une très-haute
taille, sans rien de trop, bien fournie sans être gros,
très-proportionné et fort bien fait, l'air fort, robuste et très-noble
jusqu'à être imposant sans rien de rude\,; le visage assez agréable, et
toute la physionomie tartare. C'était un homme sage, modeste, mesuré, de
fort peu d'esprit\,; mais tout tourné au bon et au sensé\,; d'une grande
politesse, mais assez distingué selon les personnes\,; d'une grande
aisance avec tout le monde, et en même temps, ce qui est rare ensemble,
avec beaucoup de dignité sans nulle chose dans ses manières qui sentît
le glorieux. Il ne parlait pas beaucoup, fournissait pourtant à la
conversation, et rendait très-bien ce qu'il avait vu sans jamais parler
de soi. Un fort honnête homme, droit, vrai, extrêmement brave, fort
craignant Dieu sans le montrer, sans le cacher aussi, avec beaucoup de
simplicité. En secret il donnait beaucoup aux pauvres, des temps
considérables à la prière, eut bientôt une nombreuse maison qu'il tint
pour les mœurs, la dépense et l'exactitude du payement dans la dernière
règle, et tout cela avec douceur. C'était un très-bon homme et fort
aimable et commode pour le commerce, mais après l'avoir vu de près on
demeurait dans l'étonnement qu'il eût été chef d'un grand parti, et
qu'il eût fait tant de bruit dans le monde. En arrivant à Versailles il
descendit chez Dangeau, où se trouva le baron de Breteuil, introducteur
des ambassadeurs qui devait le mener chez le roi. Breteuil se retira
sans entrer dans le cabinet où Torcy était, et demeura seul en tiers. Il
vit Madame ensuite sans y être mené, et dîna chez Torcy qui le traita
magnifiquement. Il ne vit aucun prince ni princesse du sang en
cérémonie. Il ne les fréquenta que selon que la familiarité s'en
présenta. M\textsuperscript{me} la Duchesse fut celle avec qui il en eut
davantage, un peu aussi avec M\textsuperscript{me} la princesse de
Conti. Le roi lui donna six cent mille livres sur l'hôtel de ville, et
lui paya d'ailleurs six mille livres par mois et l'Espagne trente mille
livres par an. Cela lui fit autour de cent mille livres de rente. Sa
maison était à Paris uniquement pour son domestique, lui toujours à la
cour, sans y donner jamais à manger. Le roi lui faisait toujours meubler
un bel appartement à Fontainebleau. Il portait la Toison que le roi
d'Espagne lui avait envoyée lorsqu'il était encore à la tête des
mécontents.

L'orgueil de M. le Grand ne put supporter longtemps la distinction
unique d'une pension de trente mille livres donnée à la duchesse de
Chevreuse. Il se fit porter chez le roi, car il ne pouvait presque plus
se soutenir depuis longtemps par l'accablement de la goutte, et là en
diminutif de M. de La Rochefoucauld, il se mit à parler de ses maux, de
sa fin prochaine, de l'état de ses affaires, de la façon la plus
touchante, qu'il finit par demander une grâce sans l'expliquer, avec
toute l'instance possible. Le roi, de longue main accoutumé à ne lui
refuser rien, lui demanda ce qu'il voulait. Alors il étala le mérite de
M\textsuperscript{lle} d'Armagnac, sa tendresse pour elle, et sa
désolation de se voir sur le point de la laisser sans pain. Avec ses
prosopopées, il eut pour elle une pension de trente mille livres.

M\textsuperscript{lle} de Chausseraye rattrapa en même temps une pension
de mille écus, qu'elle avait perdue moyennant une grosse affaire de
finance, que le roi lui avait permis de faire. Elle prétendit n'en avoir
rien tiré, et raccrocha sa pension. On peut voir, t. VIII, p.~57 et
suiv., quelle était cette maîtresse poulette, de laquelle il sera encore
parlé. Le maréchal de Villars obtint aussi une pareille pension pour sa
sœur, M\textsuperscript{me} de Vaugué, dont il avait fait la duègne et
l'Argus de sa femme. Il la logeait et la nourrissait pour cela\,; mais
d'ailleurs il ne donnait pas un sou à elle ni à ses enfants qui
mouraient de faim. C'étaient de petits gentilshommes tout au plus de
Dauphiné, et des plus minces, dont on n'avait jamais ouï parler.

Bockley, frère de la duchesse de Berwick, apporta au roi, le 12 janvier,
la nouvelle de la retraite de Staremberg, le 3 au soir, vers Ostalric,
qui avait levé le blocus de Girone, voyant arriver le duc de Berwick
avec ses troupes. Berwick envoya aussitôt relever la garnison, et tout
le pays s'empressa d'y porter toutes sortes de vivres. On y mit aussi
force munitions et des vivres pour un an. Berwick observa les ennemis
jusqu'à ce que tout fût entré dans Girone, et qu'ils fussent retirés à
demeure\,; il revint aussitôt après à la cour, où il fut parfaitement
bien reçu. Brancas en eut la Toison, et fort peu après fut nommé
ambassadeur en Espagne, où on l'envoya sans le laisser revenir à Paris.
Bockley en fut brigadier.

Les parties particulières devinrent de plus en plus fréquentes chez
M\textsuperscript{me} de Maintenon. Dîners, musiques, scènes de
comédies, actes d'opéra, loteries toutes en billets noirs, mêmes dîners
à Marly, quelquefois à Trianon, et toujours le même très-petit nombre et
les mêmes dames, toujours le maréchal de Villeroy aux musiques et aux
pièces\,; très-rarement M. le comte de Toulouse qui aimait la musique,
presque jamais M. du Maine, et nul autre homme sans aucune exception,
que des moments le capitaine des gardes en quartier, quand il venait
dire au roi que son souper était servi, et que la musique n'était pas
achevée.

Matignon obtint la permission de se démettre en faveur de son fils de
ses charges de Normandie, en retenant le commandement et les
appointements toute sa vie. C'était un masque en usage depuis quelque
temps pour suppléer aux survivances en les déguisant si grossièrement
ainsi.

D'Alègre, mort longtemps depuis maréchal de France, point du tout
corrigé de l'alliance des ministres par toutes les indignités qu'il
avait essuyées de celle de Barbezieux, maria sa fille à Maillebois, avec
sa lieutenance générale de Languedoc de vingt mille livres de rente. Le
roi donna deux cent mille livres, Desmarets peu de chose\,: la noce fut
magnifique à Paris.

La maréchale de Noailles avait encore une fille à marier, fort laide,
qui commençait à monter en graine, et que pour cette raison ils
appelaient la douairière. Elle obtint, pour la marier au fils du
maréchal de Châteaurenauld bien plus jeune qu'elle, la lieutenance
générale de Bretagne qu'avait le maréchal, et lui donna d'ailleurs fort
peu de chose. Châteaurenauld était fort riche, et n'avait que ce fils
qu'il mit ainsi dans une grande alliance dont il avait grand besoin.

M. d'Isenghien épousa peu après M\textsuperscript{lle} de Rhodes, malgré
M\textsuperscript{me} de Rhodes. La fille était en âge, et ses parents
la soutinrent. Elle était riche, et je crois la dernière Pot, qui était
une bonne, illustre et très-ancienne maison. Quelque temps après,
Vieuxpont, officier général, veuf d'une fille de la princesse de
Montauban et de Rannes, son premier mari, tué colonel général des
dragons, épousa une fille de Beringhen, premier écuyer.

Le pape avait réservé quatre chapeaux \emph{in petto }dans la promotion
qu'il avait faite en 1712, pour les couronnes\,: il les déclara au
commencement de cette année. Ce furent don Manuel Arias, archevêque de
Séville, l'abbé de Polignac, Benoît Sala, bénédictin, évêque de
Barcelone, et Benoît Erba, archevêque de Milan, à qui son oncle don
Livio Odescalchi, neveu d'Innocent XI, qui n'avait plus personne de son
nom, l'avait fait prendre avec l'assurance d'une partie de ses grands
biens, et qui s'appela le cardinal Odescalchi. Arias, avancé dans
l'ordre de Malte, et avec le caractère public de sa religion auprès du
feu roi d'Espagne, était une des meilleures têtes et un des plus
vertueux hommes d'Espagne. Il était entré dans les conseils, et il eut
une part principale au testament. Il fut après gouverneur du conseil de
Castille\,; et, lorsque M\textsuperscript{me} des Ursins se sentit en
force d'écarter tous ceux qui avaient le plus contribué à faire appeler
Philippe V à la couronne et qui avaient le plus de part au gouvernement,
elle éloigna celui-ci par l'archevêché de Séville, et la nomination du
roi d'Espagne au cardinalat. Je ne fais que rappeler ces choses, parce
que j'ai parlé d'Arias avec étendue à l'occasion et au temps du
testament de Charles II.

L'archiduc, reconnu par force à Rome, comme on l'a vu du temps que le
marquis de Prié et le maréchal de Tessé y étaient ambassadeurs,
s'opposait à ce que Philippe V eût un chapeau. Il avait nommé Sala comme
roi d'Espagne, et avait employé les menaces pour s'assurer de son
chapeau. La nonciature était fermée en Espagne depuis cette
reconnaissance de l'archiduc. Philippe V insistait pour le chapeau de sa
nomination, et protestait d'injure contre celui de Sala comme étant,
lui, roi d'Espagne de droit et d'effet, et non pas l'archiduc, et par le
personnel de Sala à son égard. Ce moine était de la lie du peuple,
cocher en son jeune temps, puis bénédictin pour avoir du pain et devenir
quelque chose. C'était un drôle d'esprit et d'entreprise, qui excita le
peuple puis les magistrats de Barcelone contre le roi d'Espagne, et qui
figura assez parmi eux pour avoir eu grande part à la révolte de la
Catalogne, et être regardé comme l'âme du parti de l'archiduc, lequel en
récompense le fit évéque de Barcelone. Avec ce caractère, Sala se
signala de plus en plus et mérita enfin la nomination de l'archiduc. Ces
oppositions réciproques firent garder \emph{in petto} le chapeau de la
nomination d'Espagne à la promotion des couronnes. Polignac, qui avait
celle du roi Jacques, n'essuyait point de contradiction\,; mais la
fonction d'Utrecht, incompatible avec le chapeau, fit que le roi désira
qu'il fût réservé \emph{in petto}, mais il le sut, et fut ainsi assuré
de l'avoir dès que la paix serait conclue. Erba, j'ignore quelle raison
le retint dans ce purgatoire.

La paix sur le point d'être conclue par toutes les puissances, excepté
l'empereur, ce prince, qui l'était élu et couronné, mais qu'on ne
traitait encore que d'archiduc en France et en Espagne, voulut que Sala
fût cardinal sans plus attendre, et le roi d'Espagne ne pressa pas moins
pour que sa nomination fût remplie. Le pape ainsi tourmenté des deux
côtés, et qui voyait qu'à la fin l'Italie demeurerait à l'empereur,
n'osa l'amuser plus longtemps, et se flatta de faire passer Sala au roi
d'Espagne, en déclarant Arias en même temps. Il fit donc avertir le roi
qu'il allait expectorer Polignac avec les autres, et que cela ne se
pouvait plus différer. Il ne restait plus que des bagatelles à ajuster à
Utrecht, et l'espérance de finir alors avec l'empereur était perdue\,:
le roi consentit donc à l'expectoration, et dépêcha en même temps un
courrier à Polignac, pour le faire revenir sur-le-champ. Il laissa donc
ce qui restait à achever et la paix à signer au maréchal d'Huxelles et à
Ménager, et accourut à sa barrette. Le courrier chargé de sa calotte le
trouva à mi-chemin. Il la mit dans sa poche et continua son voyage. Il
arriva le 22 février à Paris, et le jeudi 23, il alla l'après-midi à
Marly chez Torcy, qui, entre la fin de la musique et le souper, le mena
chez M\textsuperscript{me} de Maintenon.

Polignac, qui avait reçu en passant les compliments et les empressements
du salon, présenta au roi sa calotte, qui la lui mit sur la tête, et lui
donna une chambre à Marly. Ce fut une chose assez étrange qu'un cardinal
\emph{in petto} de la nomination du roi Jacques traitât et conclût à
Utrecht la consommation dernière des malheurs de ce prince et son
expulsion de France, avec tout ce qu'il plut aux Anglais de prescrire à
cet égard. Sa visite de remercîment à Saint-Germain et de retour dut
être bien embarrassante, mais quand on est cardinal rien n'embarrasse
plus\,: au moins ne le put-il être que de la reine d'Angleterre. En
conséquence de ce qui avait été arrêté avec les Anglais, le roi
d'Angleterre était déjà parti avec une petite suite sous le nom de
chevalier de Saint-Georges, pour se retirer à Bar, dont M. de Lorraine
avait fait meubler le château, et l'y vint voir. Il alla aussi à
Lunéville voir M. et M\textsuperscript{me} de Lorraine, et s'arrêta à
Bar, à Commercy, chez M. de Vaudémont, et dans tous ces environs assez
longtemps.

Le roi, qui n'avait jamais pu se défaire du respect que le cardinal
Mazarin lui avait imprimé pour les cardinaux, régla avec les cardinaux
de Rohan et de Polignac la place que les cardinaux occuperaient au
sermon à la chapelle, et avec tant d'égards qu'il prit la peine de la
dessiner sur du papier devant eux et à leur gré. Il n'y avait
jusqu'alors rien de marqué là-dessus. Les places des cardinaux de
Bouillon et de Coislin étaient fixes par leurs charges\,; le cardinal de
Janson n'avait presque point demeuré à la cour cardinal que depuis qu'il
fut grand aumônier\,; Bonzi l'était de la reine, et depuis sa mort
presque toujours en Languedoc\,; Le Camus ne vit jamais Paris ni la cour
depuis sa promotion\,; Estrées, souvent à Rome, puis en Espagne, ne
s'était point soucié de place réglée au sermon\,; Furstemberg encore
moins, qui ne s'y trouvait presque jamais. Le roi entretint après le
cardinal de Polignac des matières d'Utrecht près de deux heures tête à
tête.

On a vu en son lieu par quel tour de passe-passe, aidé de tout l'art et
de l'or de M\textsuperscript{me} de Soubise, secondée de toute
l'autorité du roi, le cardinal de Rohan avait été reçu chanoine de
Strasbourg, et en était devenu coadjuteur et enfin évêque. La
multiplicité et l'excès des mésalliances que la longue suite du même
esprit de gouvernement a forcé toute la noblesse du royaume de
contracter pour vivre, l'excluait toute d'entrer dans le chapitre de
Strasbourg, à commencer par les princes du sang, et à continuer par tout
ce qu'il y a de plus grand et de plus illustre. Il n'y en avait plus dès
lors qui en pussent faire les preuves que MM. d'Uzès qui y mirent
bientôt obstacle par leurs mariages, M. de Duras et le comte de Roucy,
dont le fils en déchut. On considéra cependant qu'il était de l'intérêt
très-essentiel du roi que des François y pussent être admis, parce qu'il
en était que l'évêque fût François et qu'il n'est élu que par le
chapitre et tiré du chapitre. Le roi chercha donc à apporter quelque
tempérament là-dessus. Le cardinal de Rohan l'y servit, mais, comme il
n'était là question que du chapitre, ce ne fut qu'avec le chapitre qu'on
négocia. Il députa au roi pour cette affaire le comte de Lœwenstein,
frère de M\textsuperscript{me} de Dangeau, grand doyen de Strasbourg,
chanoine de Cologne et d'autres grandes églises, que nous verrons
bientôt évêque de Tournai, sans être dans les ordres. Ce comte eut une
longue audience du roi, tête à tête. Le chapitre consentit par degrés à
des adoucissements sur les mères, même pour les Allemands, et peu à peu
enfin à recevoir les François sans preuves, qui auraient trois
ascendants masculins ducs. Ces trois ascendants furent une fort mauvaise
idée, c'était la date qu'il fallait fixer. Je suis par exemple duc et
pair trente ans avant M. d'Aumont, pour ne citer que celui-là et en
laisser beaucoup d'autres\,; je ne suis pourtant que le second, car
c'est mon père qui le fut fait, et qui fut enregistré, reçu le 1\^{}er
février 1635. M. d'Aumont est le cinquième\,; son grand-père pourrait
donc, s'il vivait, mettre de ses enfants dans le chapitre de Strasbourg,
tandis que je n'y ferais pas recevoir les miens, et le maréchal d'Aumont
n'est duc et pair que de la fin de décembre 1665.

La vieille Mailly mourut à quatre-vingt-cinq ou six ans, aussi entière
de tête et de santé qu'à quarante. C'est celle que la longueur de son
visage étroit et la singularité de son nez faisait nommer la Bécasse.
Elle était Montcavrel, et longtemps depuis son mariage elle devint
héritière de sa maison qu'elle rendit très-puissante en biens, de
très-pauvres qu'étaient son mari et elle, à force de travail,
d'assiduité, d'art et de procès. J'ai parlé en son lieu de la
substitution qu'ils firent. Elle traita toute sa vie ses enfants à la
baguette, en jeta un à Saint-Victor dont il se serait bien passé. Il en
devint pourtant prieur, puis évêque de Lavaur, et fut homme de bien. Il
était mort à Montpellier un mois ou deux avant elle. Elle força un autre
de ses fils à se faire prêtre, dont il ne pouvait se consoler, et le
laissa les coudes percés pourrir à Saint-Victor sans y être religieux,
jusqu'à ce que le mariage de son autre fils avec la nièce à la mode de
Bretagne de M\textsuperscript{me} de Maintenon, qui fut dame d'atours de
la Dauphine, fit cet abbé de Mailly archevêque d'Arles, puis de Reims,
que nous verrons cardinal. Ses deux filles, l'une s'échappa et se maria
malgré elle à l'aîné des Mailly\,; l'autre, elle la fit religieuse, qui,
de nécessité vertu, la devint bonne, et a été une excellente abbesse de
Poissy, adorée et respectée au dernier point dans cette communauté si
grande et si jalouse de l'élection qu'elle a perdue. On n'a pas vu que
Dieu ait béni cette conduite dans tout ce qui est arrivé depuis de toute
cette famille.

Le vieux Brissac mourut aussi à pareil âge, retiré chez lui depuis
plusieurs années. Il était lieutenant général et gouverneur de Guise, et
avait été longtemps major des gardes du corps. C'était un très-petit
gentilhomme qui avait percé tous les grades des gardes du-corps, qui
avait plu au roi par son application, par ses détails, par son
assiduité, par ne compter que le roi et ne ménager personne. Il en avait
tellement acquis la familiarité et la confiance sur ce qui regardait les
gardes du corps, que les capitaines des gardes, tout grands seigneurs et
généraux d'armée qu'ils étaient, le ménageaient et avaient à compter
avec lui, à plus forte raison tous les officiers des gardes. Il était
rustre, brutal, d'ailleurs fort désagréable et gâté à l'excès par le
roi, mais homme d'honneur et de vertu, de valeur et de probité, et
estimé tel quoique haï de beaucoup de gens, et redouté de tout ce qui
avait affaire à lui, même de toute la cour et des plus importants, tant
il était dangereux. Il n'y avait que lui qui osât attaquer Fagon sur la
médecine. Il lui donnait des bourrades devant le roi qui mettaient Fagon
en véritable furie, et qui faisaient rire le roi et les assistants de
tout leur cœur. Fagon, aussi avec bien de l'esprit, mais avec fougue,
lui en lâchait de bonnes qui ne divertissaient pas moins, mais en tout
temps Fagon ne le pouvait voir ni en ouïr parler de sang-froid.

Un trait de ce major des gardes donnera un petit crayon de la cour. Il y
avait une prière publique tous les soirs dans la chapelle de Versailles
à la fin de la journée, qui était suivie d'un salut avec la bénédiction
du saint sacrement tous les dimanches et les jeudis. L'hiver, le salut
était à six heures\,; l'été, à cinq, pour pouvoir s'aller promener
après. Le roi n'y manquait point les dimanches et très-rarement les
jeudis en hiver. À la fin de la prière, un garçon bleu en attente dans
la tribune courait avertir le roi, qui arrivait toujours un moment avant
le salut\,; mais qu'il dût venir ou non, jamais le salut ne l'attendait.
Les officiers des gardes du corps postaient les gardes d'avance dans la
tribune, d'où le roi l'entendait toujours. Les dames étaient soigneuses
d'y garnir les travées des tribunes, et, l'hiver, de s'y faire remarquer
par de petites bougies qu'elles avaient pour lire dans leurs livres et
qui donnaient à plein sur leur visage. La régularité était un mérite, et
chacune, vieille et souvent jeune, tâchait de se l'acquérir auprès du
roi et de M\textsuperscript{me} de Maintenon. Brissac, fatigué d'y voir
des femmes qui n'avaient pas le bruit de se soucier beaucoup d'entendre
le salut, donna le mot un jour aux officiers qui postaient\,; et pendant
la prière il arrive dans la travée du roi, frappe dessus de son bâton,
et se met à crier d'un ton d'autorité\,: *Gardes du roi,
**retirez-vous\,; le, roi ne vient point au salut*. À cet ordre tout
obéit, les gardes s'en vont, et Brissae se colle derrière un pilier.
Grand murmure dans les travées, qui étaient pleines\,; et un moment
après chaque femme souffle sa bougie, et s'en va tant et si bien qu'il
n'y demeura en tout que M\textsuperscript{me} de Dangeau et deux autres
assez du commun.

C'était dans l'ancienne chapelle. Les officiers, qui étaient avertis,
avaient arrêté les gardes dans l'escalier de Bloin et dans les paliers
où ils étaient bien cachés, et quand Brissac eut donné tout loisir aux
dames de s'éloigner et de ne pouvoir entendre le retour des gardes, il
les fit reposter. Tout cela fut ménagé si juste que le roi arriva un
moment après, et que le salut commença. Le roi, qui faisait toujours des
yeux le tour des tribunes et qui les trouvait toujours pleines et
pressées, fut dans la plus grande surprise du monde de n'y trouver en
tout et pour tout que M\textsuperscript{me} de Dangeau et ces deux
autres femmes. Il en parla, dès en sortant de sa travée, avec un grand
étonnement. Brissac, qui marchait toujours près de lui, se mit à rire et
lui conta le tour qu'il avait fait à ces bonnes dévotes de cour, dont il
s'était lassé de voir le roi la dupe. Le roi en rit beaucoup, et encore
plus le courtisan. On sut à peu près qui étaient celles qui avaient
soufflé leurs bougies et pris leur parti sur ce que le roi ne viendrait
point, et il y en eut de furieuses qui voulaient dévisager Brissac, qui
ne le méritait pas mal par tous les propos qu'il tint sur
elles\footnote{Cette anecdote a déjà été racontée par Saint-Simon, t.
  VI, p.~205-206\,; mais les variantes des deux récits sont nombreuses,
  et nous n'avons pas cru devoir supprimer ce passage, comme l'ont fait
  les précédents éditeurs.}.

\hypertarget{chapitre-xiv.}{%
\chapter{CHAPITRE XIV.}\label{chapitre-xiv.}}

1713

~

{\textsc{Mort, état et caractère du comte de Nassau-Saarbrück.}}
{\textsc{- Mort et singularité de Chambonas, évêque de Viviers.}}
{\textsc{- Singularité étrange de Desmarets, archevêque d'Auch.}}
{\textsc{- Mort du connétable de Castille.}} {\textsc{- Villena,
majordome-major du roi d'Espagne, en sa place.}} {\textsc{- Chalais
reconduit son cordelier prisonnier en Espagne.}} {\textsc{- Duc et
duchesse de Shrewsbury à la cour.}} {\textsc{- État et nom de cet
ambassadeur et de l'ambassadrice\,; caractère de la duchesse, qui change
entièrement les coiffures des femmes, dont le roi n'avait pu venir à
bout.}} {\textsc{- Maison du duc d'Aumont, à Londres, brûlée.}}
{\textsc{- Caractère du duc d'Aumont.}} {\textsc{- L'incendie coûte cinq
cent cinquante mille livres au roi.}} {\textsc{- Bout de l'an à
Saint-Denis du Dauphin et de la Dauphine.}} {\textsc{- \emph{Histoire de
la compagnie de Jésus}, du P. Jouvency.}} {\textsc{- Scandale de ce
livre, dont les jésuites se tirent à bon marché.}} {\textsc{- Abbé de
Castries premier aumônier de M\textsuperscript{me} la duchesse de
Berry.}} {\textsc{- Son caractère\,; sa fortune.}} {\textsc{-
Longepierre secrétaire des commandements de M\textsuperscript{me} la
duchesse de Berry\,; son caractère.}} {\textsc{- Mort de l'électeur de
Brandebourg, premier roi de Prusse.}} {\textsc{- Électeurs de Cologne et
de Bavière à Paris et à Suresne\,; voient le roi.}} {\textsc{- Règlement
en vingt-cinq articles, fait par le roi, entre les gouverneurs ou
commandants généraux de Guyenne et le gouverneur de Blaye, dont je gagne
vingt-quatre articles, de l'avis du duc du Maine, contre le maréchal de
Montrevel.}} {\textsc{- Ténébreuse noirceur de Pontchartrain, qui me
fait éclater.}} {\textsc{- La Chapelle\,; quel\,; je lui fais une
étrange déclaration.}} {\textsc{- Conversation étrange entre le
chancelier et moi.}} {\textsc{- Même conversation avec la chanceliere.}}
{\textsc{- M\textsuperscript{me} de Saint-Simon vainement attaquée.}}
{\textsc{- L'intimité entière subsiste entre le chancelier, la
chanceliere, et M\textsuperscript{me} de Saint-Simon et moi.}}

~

Le comte de Nassau-Saarbrük mourut dans son château de Saarbrück, où il
s'était comme retiré depuis quelques années. Il avait toujours servi,
était lieutenant général, et il avait le régiment Royal-Allemand, qui
est de vingt-cinq mille livres de rente. C'était l'homme du monde le
mieux fait, du plus grand air et imposant, fort poli, fort brave, fort
honnête homme, avec peu d'esprit et considéré. Il était aussi fort
riche, mais luthérien, et point vieux. Le roi lui-même lui avait fait
diverses attaques sur sa religion avec bonté, et ne lui avait pas laissé
ignorer qu'il irait à tout en se faisant catholique, sans l'avoir pu
ébranler.

Une autre mort dont je ne parlerais pas sans la singularité de l'homme,
est celle de l'évèque de Viviers. Il était frère de Chambonas, qui était
à M. du Maine. C'est sans doute cette protection qui le fit souffrir dix
ans de suite à Paris dans un logis garni auprès de ma maison. Il
écrivait toute la nuit jusqu'à épuiser plusieurs secrétaires, et se
levait à une heure ou deux après midi. Il mandait tous les ordinaires
des nouvelles des fanatiques de Languedoc et d'autres nouvelles de la
province, de Paris, où il était, à Bâville, intendant ou plutôt roi du
Languedoc, qui était à Montpellier, qui ne put jamais détruire ce
commerce que Viviers grossissait de force mémoires et instructions. Avec
cinquante mille livres de rente de son évêché et d'une abbaye, il laissa
six cent mille livres. Cela me fait souvenir d'une singularité d'un
autre genre. L'archevêque d'Auch, frère de Desmarets, passait sa vie à
Paris en hôtel garni, et en robe de chambre, sans voir personne, ni
ouvrir aucune lettre qu'il reçût, qu'il laissait s'amasser en monceaux.
À la fin le roi se lassa et dit à Desmarets de le renvoyer à son église.
L'embarras fut d'autant plus grand d'en entreprendre le voyage, qu'il en
était depuis assez longtemps aux emprunts pour vivre, et aux expédients.
Refusé partout où il s'adressa, et pressé sans relâche, son secrétaire
s'avisa de lui proposer d'attaquer cette montagne de lettres et de
paquets fermés, pour voir s'il ne s'y trouverai point quelque lettre de
change\,; faute de ressource, il y consentit. Le secrétaire se mit en
besogne, et trouva pour cent cinquante mille livres de lettres de change
de toutes sortes de dates, dans l'ignorance desquelles il mourait de
faim. Il s'en alla donc, et ne fut plus en peine de payer sa dépense.

Le connétable de Castille mourut en ce même temps dans sa prison à
Bayonne. Il était majordome-major du roi d'Espagne, qui est la plus
grande charge. Elle fut donnée sur-le-champ au marquis de Villena, qui
avait été vice-roi de Naples et pris les armes à la main à Gaëte par les
Impériaux. Le choix ne pouvait être plus digne, jusqu'à honorer le roi
qui le fit. J'ai déjà parlé de ce seigneur, et j'en aurai occasion
encore, et d'expliquer ce que c'est que la charge qu'il eut.

Chalais, qui avait vu M\textsuperscript{me} des Ursins à Bagnères, et
qui en était revenu à Paris, en repartit en ce même temps avec son
cordelier prisonnier, qu'il conduisit en Espagne. Ce métier de recors ne
lui réussit pas dans le monde.

Le duc et la duchesse de Shrewsbury étaient arrivés depuis quelque
temps. J'ai marqué en deux mots (p.~256 ci-dessus), quel était cet
ambassadeur d'Angleterre. On le trouvera plus expliqué dans les Pièces
concernant le traité de Londres\footnote{On a déjà dit que ces Pièces
  n'avaient pas été remises à M. le duc de Saint-Simon en même temps que
  les Mémoires dont elles sont le complément.}. Il eut sa première
audience particulière à l'ordinaire. Comme il n'y avait ni reine ni
Dauphine, la duchesse alla saluer le roi dans son cabinet entre le
conseil et le dîner, menée par la duchesse d'Aumont, et accompagnée du
baron de Breteuil, introducteur des ambassadeurs. Le soir, la duchesse
d'Aumont la mena prendre son tabouret au souper du roi. Les Anglais sont
grands voyageurs. Celui-ci, qui avait porté l'épée de l'État au
couronnement de Jacques II, qui avait eu sa confiance, et été son grand
chambellan, le quitta en 1680, et passa en Hollande, où il offrit ses
services au prince d'Orange. Il se promena ensuite en Italie, fut à
Rome, où il épousa la fille du marquis Paleotti, Bolonois, et de
Catherine Dudley, fille du duc de Northumberland, et de Marie-Madeleine
Gouffier de Brazeux. Voilà bien du mélange. La religion ne contraignit
point l'Italienne. Elle suivit son mari en Angleterre, où le prince
d'Orange régnait, qui le fit duc et chevalier de la Jarretière. Il fut
aussi secrétaire d'État. La reine Anne le mit de son conseil privé, et
le fit son grand chambellan. Il fut vice-roi d'Irlande au retour de son
ambassade de France, et il mourut à Londres en 1718.

Sa femme était une grande créature et grosse, hommasse, sur le retour et
plus, qui avait été belle et qui prétendait l'être encore\,; toute
décolletée, coiffée derrière l'oreille, pleine de rouge et de mouches,
et de petites façons. Dès en arrivant elle ne douta de rien, parla haut
et beaucoup en mauvais français, et mangea dans la main à tout le monde.
Toutes ses manières étaient d'une folle, mais son jeu, sa table, sa
magnificence, jusqu'à sa familiarité générale la mirent à la mode. Elle
trouva bientôt les coiffures des femmes ridicules, et elles l'étaient en
effet. C'était un bâtiment de fil d'archal, de rubans, de cheveux et de
toutes sortes d'affiquets de plus de deux pieds de haut qui mettait le
visage des femmes au milieu de leur corps, et les vieilles étaient de
même, mais en gazes noires. Pour peu qu'elles remuassent, le bâtiment
tremblait, et l'incommodité en était extrême. Le roi, si maître jusque
des plus petites choses, ne les pouvait souffrir. Elles duraient depuis
plus de dix ans sans qu'il eût pu les changer, quoi qu'il eût dit et
fait pour en venir à bout. Ce que ce monarque n'avait pu, le goût et
l'exemple d'une vieille folle étrangère l'exécuta avec la rapidité la
plus surprenante. De l'extrémité du haut, les dames se jetèrent dans
l'extrémité du plat, et ces coiffures plus simples, plus commodes et qui
siéent bien mieux durent jusqu'à aujourd'hui. Les gens raisonnables
attendent avec impatience quelque autre folle étrangère qui défasse nos
dames de ces immenses rondaches de paniers, insupportables en tout à
elles-mêmes et aux autres.

L'hôtel de Powis à Londres, où logeait le duc d'Aumont, fut entièrement
brûlé, et il fallut abattre une maison voisine pour empêcher que
l'incendie ne se communiquât aux autres. Sa vaisselle fut sauvée. Il
prétendit avoir perdu tout le reste. Il prétendit aussi avoir reçu
plusieurs avis qu'on le voulait brûler et même assassiner, et que la
reine, à qui il l'avait dit, lui avait offert de lui donner des gardes.
Le monde en jugea autrement à Londres et à Paris, et se persuada que
lui-même avait été l'incendiaire, pour gagner sur ce qu'il en tirerait
du roi, et pour couvrir une contrebande monstrueuse dont les Anglais se
plaignirent ouvertement dès son arrivée, et où il gagna infiniment\,:
c'est au moins ce qui se débita publiquement dans les deux cours et dans
les deux villes, et ce que presque tous en crurent.

M. d'Aumont avait toute sa vie été un panier percé qui avait toujours
vécu d'industrie\,; il avait eu longtemps affaire à un père fort dur, et
à une belle-mère qui le haïssait fort, et qui était une terrible dévote.
Il s'était marié malgré eux par amour réciproque à
M\textsuperscript{lle} de Piennes, dont la mère était Godet, comme
l'évêque de Chartres qui y fit à la fin entrer M\textsuperscript{me} de
Maintenon, et le roi par elle, lequel imposa enfin et obligea le père à
consentir, après plusieurs années que ce mariage demeurait accroché, et
que tous deux étaient résolus à n'en jamais faire d'autre. Le duc
d'Aumont était d'une force prodigieuse, d'une grande santé, débauché à
l'avenant, d'un goût excellent, mais extrêmement cher en toutes sortes
de choses, meubles, ornements, bijoux, équipages\,; il jetait à tout, et
tira des monts d'or des contrôleurs généraux et de son cousin
Barbezieux, avec qui, pour n'en pas tirer assez à son gré, il se
brouilla outrageusement. Il prenait à toutes mains et dépensait de même.
C'était un homme de beaucoup d'esprit, mais qui ne savait rien, à
paroles dorées, sans foi, sans âme, de peu de réputation à la guerre
pour en parler sobrement, et à qui son ambassade ne réussit ni en
Angleterre ni en France. Avant la mort de son père, logeant dans une
maison de louage, il l'ajusta et la dora toute, boisa son écurie comme
un beau cabinet, avec une corniche fort recherchée tout autour, qu'il
garnit partout de pièces de porcelaine. On peut juger par là de ce qu'il
dépensait en toutes choses. Le roi donna deux cent cinquante mille
livres à milord Powis, et au duc d'Aumont cent mille francs, et
cinquante mille par an pendant quatre ans, tant en considération de son
incendie que de la dépense de son ambassade.

On fit à Saint-Denis le bout de l'an du Dauphin et de la Dauphine, je
n'oserais dire de la France. Tout ce qui a suivi une telle perte ne le
prouve que trop évidemment. Il n'y eut que leurs maisons, les princes et
princesses de la maison royale, du sang et légitimés, et M. de Metz qui
officia, et cela ne dura guère plus d'une heure.

Le livre du jésuite Jouvency fit alors grand bruit. C'est une histoire
latine de sa compagnie depuis son origine jusqu'à nos jours. II était à
Rome, où il la composa. Je ne m'aviserai pas ici d'en faire l'extrait.
Il suffit de dire qu'il voulut plaire à Rome et aux siens, et qu'il
employa la plus belle latinité, et tout l'art dans lequel les jésuites
sont si grands maîtres, à flatter et à établir les prétentions les plus
ultra-montaines, et à canoniser la doctrine la plus décriée des
théologiens et des casuistes de son ordre. Il fit plus\,: il fit par ses
éloges des saints du premier ordre, et des martyrs qui méritent un culte
public, des jésuites les plus abhorrés pour les fureurs de la Ligue,
pour la conspiration des poudres en Angleterre, et pour celles qui ont
été tramées contre la vie d'Henri IV\,: tout cela prouvé par la
supériorité du pape sur le temporel des rois, son droit d'absoudre leurs
sujets du serment de fidélité, de les déposer et de disposer de leur
couronne, enfin par le principe passé chez eux en dogme qu'il est permis
de tuer les tyrans, c'est-à-dire les rois qui incommodent. Le public
frémit à cette lecture, et le parlement voulut faire son devoir.

Le P. Tellier soutint fort et ferme un ouvrage qui portait le nom de son
auteur, qui était muni de l'approbation de ses supérieurs, et qui était
si conforme à l'esprit, aux maximes, à la doctrine et à la constante
conduite de la société. Il m'en vint parler plusieurs fois. Je ne lui
cachai rien de ce que je pensais des énormités de ce livre, et de
l'audace de le publier. J'admirai les cavillations de ses réponses et la
pertinacité de son attachement à introduire ces horreurs. Je ne fus pas
moins surpris de sa constance à vouloir me persuader, et de sa patience
à supporter mes réponses. Quoique depuis la perte du Dauphin il n'eût
plus les mêmes raisons de me cultiver, il ne s'en relâcha pourtant pas
le moins du monde. Il ne pouvait ignorer en quelle situation j'étais
avec M. le duc de Berry, et surtout avec M. le duc d'Orléans. Il voyait
le roi vieillir, et un Dauphin dans la première enfance\,: un jésuite a
tous les temps présents. Il eut meilleur marché du roi, quoique ce livre
attaquât si directement la puissance, la couronne et la vie même des
rois. Il se souvenait apparemment du testament de mort du P. de La
Chaise\,; je veux dire de l'avis si prodigieux qu'il lui donna et qui
est rapporté, t. VII, p.~49. Il aima mieux tout passer aux jésuites que
de les irriter au hasard des poignards.

Il manda plusieurs fois le premier président et le parquet pour imposer
à leur zèle, qui n'allait à rien moins qu'à flétrir la personne de
Jouvency et de ses approbateurs, à faire lacérer et brûler son livre par
la main du bourreau, à mander et admonester les supérieurs et les gros
bonnets du ressort, et leur faire abjurer à la barre du parlement en
public ces détestables maximes. Le premier président voulait faire sa
cour, et se concilier les jésuites\,; il ne voulait pas aussi s'aliéner
le parlement\,; toute sa considération à la cour et dans le monde
dépendait de la sienne dans sa compagnie. Il nageait donc avec art entre
deux eaux, et c'est ce qui tira tant la chose en longueur. L'affaire
aboutit enfin à la suppression du livre par arrêt du parlement sans
lacération ni brûlure, et à mander les supérieurs des trois maisons de
Paris au parlement, à qui le premier président fit une admonition légère
et honnête, et qui déclarèrent à peu près ce qu'on voulut, mais en
termes si généraux, et si éloignés de rien de particulier sur les
maximes et sur leur P. Jouveney, que ce fut plutôt une dérision qu'autre
chose, et qu'ils se ménagèrent en quantité force portes de derrière, à
l'indignation du public, et au frémissement du parlement, à qui le roi
mit un bâillon à la bouche. Le P. Tellier parut fort mécontent, ravi en
secret d'avoir si bien fasciné le roi, et qu'il ne leur en eût pas coûté
davantage.

L'abbé de Castries, frère du chevalier d'honneur de
M\textsuperscript{me} la duchesse d'Orléans, fut en ce temps-ci premier
aumônier de M\textsuperscript{me} la duchesse de Berry\,; il l'était
ordinaire de M\textsuperscript{me} la Dauphine, pour avoir un titre
d'habiter la cour avec son frère, où il était dans la meilleure
compagnie. Il avait été jeune et bien fait\,; il était de ces abbés que
le roi s'était promis de ne faire jamais évêques. C'était un homme doux,
mais salé, avec de l'esprit, et fait pour la société. Il vit encore dans
un grand âge, confiné dans son archevêché d'Alby, où il est fort aimé,
commandeur de l'ordre, et ayant refusé Toulouse et Narbonne.
M\textsuperscript{me} la duchesse de Berry prit en même temps
Longepierre pour secrétaire de ses commandements, manière de bel esprit
de travers, et de fripon d'intrigue, dont on a déjà parlé et dont on
pourra parler encore.

Frédéric III, électeur de Brandebourg, né en 1657, mourut le 25 février
de cette année. Celui d'aujourd'hui est son petit-fils. Il suivit les
traces de l'électeur son père dans son opposition à la France et dans
son attachement à la maison d'Autriche. Il servit puissamment l'empereur
en toutes occasions, et aux guerres de Hongrie et du Rhin. Il se trouva
le plus puissant des électeurs et celui que l'empereur avait le plus à
ménager. Cela lui fit imaginer de se déclarer lui-même roi de Prusse,
comme on l'a dit en son temps, après s'être assuré de l'esprit et de la
reconnaissance de l'empereur en cette qualité, et de plusieurs princes
de l'empire, et se déclara roi lui-même le 18 janvier à Kœnigsberg,
capitale de la Prusse ducale, en un festin qu'il y donna à ses premiers
généraux et à ses ministres, et aux principaux seigneurs de cette Prusse
et de ses autres États. De trois femmes qu'il épousa, il eut son
successeur, père de celui d'aujourd'hui, d'une Nassau, tante paternelle
du prince d'Orange devenu depuis roi d'Angleterre, à la succession
duquel les électeurs de Brandebourg ont prétendu par là. Frédéric n'eut
pas la joie d'être reconnu roi de Prusse par la France et l'Espagne\,;
il mourut avant la paix de ces deux couronnes avec l'empereur et
l'empire, qui ne fut conclue qu'un an après, et par laquelle son fils
fut reconnu partout roi de Prusse.

Les électeurs de Cologne et de Bavière arrivèrent\,: le premier à Paris,
dans une maison du quartier de Richelieu que son envoyé lui avait
meublée\,; l'autre, dans une petite maison à Suresne, dans leur
incognito ordinaire. Peu de jours après, l'électeur de Cologne vit le
roi fort courtement, mené dans son cabinet par le petit escalier de
derrière, après le sermon, par Torcy\,; deux jours après, le roi reçut
l'électeur de Bavière en même lieu et à même heure et de la même
façon\,; mais l'électeur demeura longtemps avec lui. Ils ne couchèrent
ni l'un ni l'autre à Versailles.

Il est temps maintenant de parler d'un règlement que j'obtins en ce
temps-ci, pour le gouvernement de Blaye, et qui serait peu intéressant
ici sans les suites étrangères qu'il causa. On a vu ailleurs que les
usurpations du maréchal de Montrevel et ses procédés là-dessus n'avaient
pu être arrêtés par tout ce que j'y mis du mien, et comment il ne voulut
plus de l'arbitrage de Chamillart dès qu'il fut tombé, et refusa ensuite
au maréchal de Boufflers de s'en mêler. On a vu aussi que cela m'avait
empêché d'aller en Guyenne, quand, après l'étrange effet du parti de
Lille, je voulus me retirer tout à fait de la cour. Lassé des
impertinences continuelles d'un fou, qui l'était au point de dire dans
Bordeaux qu'il ne m'y donnerait pas la main, et de se faire moquer de
lui là-dessus par l'archevêque, le premier président, l'intendant et par
tout le monde, je songeai, à la mort du duc de Chevreuse, à rendre mon
gouvernement indépendant de celui de Guyenne. La Vrillière se chargea de
le proposer au roi, qui reçut si bien la chose, que j'eus tout lieu de
l'espérer. Mais lorsque bientôt après je vis le gouvernement de Guyenne
donné au second fils de M. du Maine, je compris qu'il ne pouvait plus
s'en parler\,; mais je voulais sortir d'affaires et savoir à quoi m'en
tenir. Je pris donc le parti d'aller à M. du Maine, de lui parler en
deux mots des entreprises continuelles du maréchal de Montrevel, de lui
dire à quoi pour cela j'avais pensé et fait parler au roi à la mort de
M. de Chevreuse, que je cessais d'y penser dès que M. d'Eu avait la
Guyenne, mais que je le priais de trouver bon que je lui apportasse un
mémoire de l'état des questions de mon droit, raisons et usages\,; qu'il
voulût bien en demander autant au maréchal de Montrevel des siennes, que
je savais qui allait arriver à Paris, de juger lui-même les questions et
les prétentions entre M. son fils et moi, puisque Montrevel n'en tenait
que la place, de demander après au roi de tourner au règlement perpétuel
ce qu'il aurait jugé, afin que je m'ôtasse de la tête ce qui me serait
ôté, et qu'une fois pour toutes aussi je demeurasse certain et paisible
dans ce qui me serait laissé.

M. du Maine qui, de sa vie, quoi que j'eusse fait, n'avait cessé de me
rechercher, me combla de politesse et de remercîments d'un tel procédé,
et accepta ce que je lui proposais. Montrevel arriva\,; il n'osa éviter
le règlement, et d'en passer par où M. du Maine jugerait à propos\,;
mais il fut si fâché de se voir au pied du mur sur des usurpations sans
fondement, que je m'aperçus qu'il me saluait fort négligemment avec une
affectation marquée lorsque je le rencontrais, et à Marly où il vint
cela était continuel, tellement que je me mis à le regarder entre deux
yeux, et à lui refuser le salut tout net. Au bout de quelques jours de
cette affectation de ma part, voilà un homme hors des gonds, qui va
trouver M. du Maine, qui dit que je l'insulte, et qui se met aux
plaintes les plus vives. J'allai peu après chez M. du Maine pour mon
affaire. À la fin de la conversation, il me parla de celle que le
maréchal avait eue avec lui, et me demanda ce que c'était que cela. Je
le lui dis et j'ajoutai que je ne craignais pas, depuis que je vivais
dans le monde, d'être accusé de manquer de politesse avec qui que ce
fût, mais que je n'étais pas accoutumé aussi que qui que ce fût s'avisât
de prendre des airs avec moi\,; que ceux de Montrevel m'avaient engagé à
lui marquer que je méprisais les fats et les matamores, et que je ne le
faisais que pour qu'il le sentît. M. du Maine me voulut arraisonner sur
le lieu où nous étions, sur ce qui pouvait résulter d'être ainsi sur le
pied gauche avec un homme qu'on rencontrait à tous moments, et qu'il y
avait des sottises dont il ne fallait pas s'apercevoir ou en rire. Je
répondis que j'en riais aussi, mais que de laisser faire des sottises à
mon égard, je n'y étais pas accoutumé, et que le maréchal m'y
accoutumerait moins qu'homme du monde\,; que je comprenais fort bien, le
connaissant aussi fou qu'il était, qu'il était capable d'une incartade,
mais que je me croyais bon aussi pour la lui faire rentrer au corps, et
le roi trop juste pour ne s'en pas prendre à qui la ferait, non à qui
l'essuierait et la repousserait, et qu'en deux paroles Montrevel pouvait
compter que je ne changerais pas de manières avec lui qu'il n'en
changeât et totalement le premier avec moi\,; qu'au demeurant s'il
n'était pas content il n'avait qu'à prendre des cartes. Je me séparai
là-dessus d'avec M. du Maine, qui ne trouva point mauvais ce que je lui
dis, mais qui aurait désiré autre chose.

Je n'ai point su ce qu'il dit à Montrevel, mais à deux jours de là, je
fus surpris de voir Montrevel qui m'évitait souvent, et qui pouvait
alors le faire aisément, m'attendre à sa portée, et me faire devant
beaucoup de monde dans le salon la révérence du monde la plus profonde,
la plus marquée, la plus polie. Je la lui rendis honnête, et depuis ce
moment là la politesse qu'on se doit les uns aux autres demeura rétablie
entre nous. Je pressais M. du Maine, le maréchal tirait de longue. Il se
fiait pourtant à ce goût bizarre et constamment soutenu que le roi avait
eu pour lui toute sa vie, en la protection secrète du maréchal de
Villeroy, qui était son ami de fatuité et de vieille galanterie, mais
qui ne voulait pas se montrer contre moi, enfin dans l'intérêt du comte
d'Eu qu'il soutenait devant son père, parce qu'il faisait toutes les
fonctions de gouverneur de Guyenne. Nous étions, lui et moi, fort
éloignés de compte\,; il prétendait beaucoup plus qu'aucun gouverneur de
province sur aucun gouverneur particulier dont le gouvernement était
entièrement assujetti au gouvernement général de la province. Moi, au
contraire, je ne lui voulais passer aucune autorité sur moi, ni de se
mêler en aucune sorte de quoi que ce pût être de civil ni de militaire
dans toute l'étendue de mon petit gouvernement, qui était beaucoup moins
que les gouverneurs de province n'en avaient eu sur les gouverneurs et
les gouvernements de leur dépendance, laquelle toutefois je
reconnaissois, mais en gros. Les choses s'étaient toujours passées ainsi
entre M. le prince de Conti, M. d'Épernon, et tous les gouverneurs et
commandants de Guyenne et mon père, et j'avais preuves écrites et par
lettres de ces gouverneurs ou commandants de la province et par des
décisions et des ordres du roi, de tout ce que je prétendais.

Montrevel, au contraire, n'en pouvait fournir aucune, mais il comptait
que ses cris, la musique de son discours, dont la singulière harmonie
suppléait à son avis au sens commun qu'il n'avait guère, son mérite, ses
dignités militaires, l'usage de tous les autres gouverneurs ou
commandants généraux des provinces, sa faveur, son importance, la
considération de l'engagement où il s'était mis, lui ferait emporter le
tout, sinon la plus grande partie de ses usurpations. La chose m'était
encore plus importante qu'à tout autre gouverneur dépendant\,; il n'y a
que les princes du sang qui, sans être dans leurs gouvernements, y
donnent leurs ordres sans lesquels il ne s'y fait rien, à qui ceux qui
ont le commandement en leur absence rendent compte de tout, et qui y
commandent absents comme présents. Mon père était dans ce même usage, le
roi l'y avait mis et maintenu dans le souvenir de l'important service
qu'il lui avait rendu par ce gouvernement pendant les troubles, dont
j'ai parlé au commencement de ces Mémoires. Après lui je m'y étais
maintenu contre diverses attaques, où le roi avait imposé en ma faveur,
et par des ordres écrits par le secrétaire d'État, tellement que j'avais
toute la raison, le droit et l'intérêt de ne pas subir le joug audacieux
et nouveau de ce vieux bellâtre. M. du Maine eut avec lui des
conversations fréquentes, La Vrillière, secrétaire d'État de la
province, pareillement, et l'un et l'autre tant qu'il voulut\,; mais
après tout il fallut finir.

La Vrillière dressa donc un projet de règlement avec M. du Maine pour le
rapporter au roi en vingt-cinq articles, parce que j'avais demandé que
tout fût bien distinct et expliqué pour ne m'exposer pas à des queues et
à de nouvelles contestations. Outre que mon droit était clair et prouvé,
et l'usage constant et constaté jusqu'aux entreprises de Montrevel
contre lesquelles, dès la première, j'avais toujours réclamé, La
Vrillière était mon ami, et de père en fils intime, et M. du Maine avait
grand désir de m'obliger en chose qu'il me voyait fort sensible, et dont
il jugeait que son fils n'userait jamais que par procureur, et il
n'était pas fâché d'une occasion à se montrer équitable contre son
propre fils, et de ne négliger rien pour émousser l'envie que ce nouveau
présent avait ranimée. Enfin le dimanche 19 mars, après le sermon, le
règlement fut décidé par le roi dans son cabinet avec M. du Maine et La
Vrillière seuls, et des vingt-cinq articles j'en gagnai vingt-quatre à
pur et à plein. L'unique que je perdis fut que le gouverneur ou le
commandant général de Guyenne, venant dans Blaye même, ville et
citadelle, en absence et en la présence du gouverneur de Blaye, y serait
accompagné de ses gardes en bandoulières et en casaques. J'avais voulu
pourvoir à la folie de la main que Montrevel avait débitée qu'il ne me
donnerait pas chez lui, mais je n'avais pas cru devoir permettre que
cette impertinence parût dans le règlement avoir été imaginée. Cet
article porta donc que les gouverneurs ou commandants généraux de
Guyenne et le gouverneur de Blaye, se trouvant ensemble dans la
province, et étant tous deux officiers de la couronne, vivraient
ensemble suivant le rang de leurs offices de la couronne.

Par cette décision, non-seulement le maréchal de Montrevel ne put plus
me contester la main dans sa maison, mais il fut mis hors d'état d'oser
me contester la préséance sur lui partout, hors dans la mienne, comme je
le prétendais bien aussi. Il fut enragé, outré, et ne put se tenir les
deux premiers jours. Je ne sais qui lui fit sentir sa folie, et combien
il déplairait au roi et à M. du Maine, et me donnerait lieu de me moquer
de lui\,: cela le fit passer d'une extrémité à l'autre. Il débita qu'il
avait obtenu tout ce qu'il désirait, fit la meilleure mine qu'il put,
mais il ne sut durer vis-à-vis de moi, et au bout de huit jours il s'en
retourna brusquement en Guyenne. Ce règlement portait qu'il serait
enregistré dans l'hôtel de ville de Blaye\,; je n'y perdis pas de temps,
et le maréchal en arrivant à Bordeaux en trouva partout des copies
répandues qui le comblèrent de rage et de fureur. Ce fut pourtant une
rage mue\footnote{On appelle \emph{rage mue} celle où l'animal atteint
  de cette maladie écume sans mordre.}, car je fis diverses punitions,
et même emprisonner des bourgeois de Blaye, et longtemps, pour lui avoir
porté des plaintes, leur faisant dire publiquement que c'était
précisément pour cela, et je le fis publier. Le maréchal avala la pilule
et n'osa ni branler ni se plaindre. Oncques depuis il ne se mêla de quoi
que ce pût être du gouvernement de Blaye, et nous n'avons pas ouï parler
l'un de l'autre.

J'aurais été infiniment content sans l'incroyable noirceur de
Pontchartrain. On a vu qu'ayant les plus fortes raisons de contribuer à
sa perte, et ayant tout à fait rompu avec lui, bien loin de lui nuire je
l'avais sauvé\,; que de là j'avais fait le raccommodement et la réunion
sincère de son père avec le duc de Beauvilliers malgré ce dernier lors
tout-puissant, et que de là j'étais rentré dans les termes ordinaires
avec Pontchartrain, qui, à l'exemple de son père, n'avait pu se
dispenser de me combler de remercîments et de protestations de
reconnaissance éternelle. Cette reconnaissance néanmoins n'avait pas
encore été jusqu'alors à ôter ce qui avait été entre nous la pierre de
scandale. Il ne me parlait point des milices de Blaye, ni de ses
officiers gardes-côtes, et moi je ne lui en voulais rien dire, et
j'attendais toujours, C'était à Marly que j'avais vu assez souvent M. du
Maine\,; je n'avais pas accoutumé d'aller chez lui qu'aux occasions de
compliments de tout le monde. Marly est fait de façon que chacun voit où
on va, surtout aux pavillons et à la Perspective où M. du Maine avait
son appartement fixe. Pontchartrain était grand fureteur, même des
choses les plus indifférentes\,: il sut ces visites redoublées\,; il en
fut d'autant plus surpris que j'avais trop vécu avec lui pour qu'il
ignorât mon sentiment sur les bâtards. Il m'en parla, je répondis
simplement que j'allais quelquefois voir M. du Maine. La réponse excita
encore sa curiosité. Il sut, je n'ai jamais su comment, de quoi il
s'agissait. Il prévint le roi sur ses gardes-côtes, tellement que le
règlement fait et décidé, et les milices de Blaye décidées de tous
points appartenir à la nomination et à l'administration du gouverneur de
Blaye, le roi de lui-même ajouta\,: «\,sans préjudice à l'entier effet
de l'édit de création des capitaines gardes-côtes,\,» moyennant quoi
ayant gagné tout ce que je prétendais sur les milices de Blaye contre
les gouverneurs et commandants généraux de Guyenne, je le perdais en
plein contre Pontchartrain et ses capitaines gardes-côtes. C'était à
Versailles où le règlement fut fait, et où j'appris en même temps ce
tour de Pontchartrain. Il est aisé de comprendre à qui a vu ce qui
s'était passé là-dessus, et depuis, à quel point j'en fus indigné.

J'allai trouver La Chapelle, un des premiers commis de Pontchartrain et
son affidé, et à son père qui s'était en dernier lieu mêlé de cette
affaire entre nous, et qui savait ce que j'avais fait pour Pontchartrain
avec M. de Beauvilliers, et le raccommodement de ce duc avec son père.
Je contai à La Chapelle ce qui venait de m'arriver, et tout de suite
j'ajoutai que je savais parfaitement toute la disproportion de crédit et
de puissance qu'il y avait entre un secrétaire d'État et moi, mais que
je savais aussi qu'on réussissait quelquefois dans un objet quand on y
postposait toutes choses, et que bien fermement je sacrifierais tout et
ma propre fortune, grandeur, faveur, biens et tout ce qui pourrait me
flatter en ma vie, à la ruine et à la perte radicale de Pontchartrain,
sans que rien me pût jamais détourner d'y travailler sans cesse, et d'y
mettre tout ce qui serait en moi, sans qu'il y eût considération
quelconque qui m'en pût détourner un seul instant, et qu'avec cette
suite et ce travail infatigable, quelquefois on parvenait à réussir dans
un temps ou dans un autre. La Chapelle eut beau chercher à m'apaiser et
des expédients sur la chose, je lui dis que je n'en voulais ouïr parler
de ma vie\,; que Pontchartrain jouirait de mes milices en pleine
tranquillité, et moi de l'espérance et du plaisir de travailler de tout
mon esprit et de tout ce qui serait en moi et sans relâche à le perdre
et à le culbuter\,; et je sortis de sa chambre, qui était tout en haut
chez Pontchartrain au château. La Chapelle, dans l'effroi de la fureur
avec laquelle je lui avais fait une déclaration si nette, descendit
sur-le-champ chez le chancelier, à qui il conta tout. Il n'y avait pas
une demi-heure que je m'étais renfermé dans ma chambre qu'un valet de
chambre du chancelier vint me prier instamment de sa part de vouloir
bien aller sur-le-champ chez lui. Je m'y rendis.

Je le trouvai qui se promenait seul dans son cabinet fort triste, et
l'air fort en peine. Dès qu'il me vit\,: «\,Monsieur, me dit-il,
qu'est-ce que La Chapelle vient de me conter\,? cela peut-il être
possible\,? --- Et de quoi s'est-il avisé, monsieur, répondis-je, de
vous l'aller conter\,?» Le chancelier me redit mot pour mot ce que
j'avais dit à La Chapelle\,; je convins qu'il n'y avait pas un mot de
changé, et j'ajoutai que c'était ma résolution bien ferme et bien
arrêtée dont rien dans le monde ne m'ébranlerait\,; que j'étais fâché
que La Chapelle eût été indiscret\,; mais que, puisqu'il l'avait été
jusqu'à la lui dire, j'étais trop vrai pour la lui dissimuler. Il n'y
eut rien que le chancelier ne me dît et n'employât pour me toucher. Je
lui remis le fait de Marly, et celui de Fontainebleau, et ce qui s'était
passé auparavant entre son fils et moi qui m'avait publiquement brouillé
avec lui et fait cesser de le voir, et je lui paraphrasai l'ingratitude
dont il payait de l'avoir empêché d'être chassé et remis en selle.

Le chancelier convint de l'infamie, mais toujours cherchant à me toucher
sur lui-même, sur la chancelière, sur la mémoire de sa belle-fille, sur
ses petits-fils\,; moi à lui répondre que tout cela n'empêchait pas que
son fils ne fût un monstre également détestable et détesté, et qui
m'avait mis au point de tenter tout pour en avoir justice, et pour le
perdre si radicalement qu'il n'en pût jamais revenir\,; que je
connaissois en plein l'inégalité infinie des forces, mais que je savais
aussi que, quand on était bien déterminé à ne rien craindre et à tout
tenter, à ne se rebuter ni de la longueur ni des obstacles, quelquefois
les cirons parvenaient à renverser des colosses, et que c'était à quoi
je sacrifierais biens, repos, fortune, sans que nulle considération
quelconque m'en pût ralentir un instant. Je ne voulus tâter d'aucun
expédient dont il me rendit le maître sur l'affaire qui m'irritait. Je
lui dis que je me confessais vaincu, et son fils, avec ses gardes-côtes,
maître de mes milices\,; qu'il pouvait jouir en plein de sa victoire,
que je n'y mettrais pas le plus léger obstacle\,; mais de les recevoir
de sa bonté, de sa grâce, de l'honneur de sa protection, après me les
avoir arrachées en dol et en scélératesse, que j'aimerais mieux perdre
mon gouvernement avec elles, que de lui devoir quoi que ce fût, parce
que tout ce que je lui voulais devoir, et l'en payer comptant autant
qu'il me serait jamais et dans tous les temps possible, c'était haine
mortelle et complète éradication.

Jamais je ne vis homme si profondément touché, ni si totalement
confondu. Ce qu'avait fait son fils, ce que, malgré son forfait, j'avais
fait pour lui, et la scélératesse dont il payait cet extrême service,
accablait le père, qui ne trouvait rien à y opposer. Il me connaissoit
jusque dans les moelles. Il sentait que je tiendrais exactement parole,
et que, quel que fût le puissant établissement de son fils, un ennemi
nerveux, implacable, qui se donne pour tel, qui met le tout pour le
tout, et qui est incapable de lâcher prise, est toujours fort dangereux
contre un homme aussi haïssable et aussi universellement haï qu'il
savait qu'était son fils. Il était de tout temps mon ami le plus intime
après le duc de Beauvilliers\,; il voyait le roi vieillir\,; il
n'ignorait pas à quoi j'en étais avec M. le duc de Berry et ce que je
pouvais auprès de M. le duc d'Orléans par l'amitié d'enfance et les
services que je lui avais rendus en tous genres de la plus extrême
importance, et le seul homme qui, vis-à-vis du roi, de Monseigneur, de
M\textsuperscript{me} de Maintenon et de la plus affreuse cabale,
n'avait jamais rougi de lui. Le chancelier en tremblait pour son fils,
et ne savait que dire ni que faire. Un silence assez long succéda à une
conversation si forte. De temps en temps ses yeux tournés sur moi me
parlaient avec honte et tendresse, et nous nous promenions par ce
cabinet. Je lui dis que je le croyais trop juste pour cesser de m'aimer
pour avoir été poignardé par son traître de fils, et d'une façon bien
pire que gratuite\,; que je le plaignais bien de l'avoir engendré\,;
mais que je redoublerais pour lui d'attachement et de respect, de
tendresse, pour lui faire oublier, s'il était possible, les justes et
invariables dispositions qu'il venait de me forcer de lui montrer. Il
m'embrassa\,; il me dit que, quand il voudrait ne me plus aimer, cela ne
lui serait pas possible\,; que j'étais trop en colère pour me parler
davantage, mais qu'il ne voulait point cesser d'espérer de mon amitié
pour lui, de mes réflexions, du bénéfice du temps. Nous nous embrassâmes
encore, moi sans rien répondre, et nous nous séparâmes ainsi.

J'eus le lendemain la même scène avec la chancelière. Je ne fus avec
elle ni moins franc, ni moins ferme, ni plus mesuré. Le père et la mère
connaissoient également leur fils\,; mais la mère, quoique traitée par
lui avec moins d'égards encore que le père, avait pour lui un faible et
une tendresse que le père n'avait pas. Elle ne put néanmoins ne pas
convenir du guet-apens, et des précédents torts de son fils avec moi, et
de l'excès de son ingratitude\,; mais elle revenait toujours au pardon
et aux expédients. Je me tirai d'avec elle par tous les respects et les
amitiés personnelles, mais sans faiblir le moins du monde.
M\textsuperscript{me} de Saint-Simon eut incontinent son tour\,; sa
piété, sa douceur, sa sagesse la rendirent modérée en expressions, mais
n'altérèrent point ce qu'elle se devait à elle-même, et elle ne fit que
s'affliger avec eux. Ils me firent parler par le premier écuyer, qui n'y
gagna pas plus qu'eux. Je cessai de voir Pontchartrain, même de
l'approcher et de lui parler en lieux publics, comme chez le roi et à
Marly, et à peine le saluai-je\,; lui, d'un embarras le plus grand du
monde sitôt qu'il m'apercevait, et force révérences.

Je redoublai de voir le chancelier et la chancelière\,; je demeurai avec
eux tout comme j'y étais devant, ils espéraient par là m'apaiser peu à
peu à la longue\,; et les choses en demeurèrent ainsi. Je ne fis pas
semblant dans le monde de cette restriction du règlement\,; je remerciai
le roi de la justice qu'il m'avait faite, mais je dis mon avis sur
Pontchartrain à M. du Maine, en le remerciant, qui se montra à moi fort
choqué de la réserve sur les gardes-côtes, et ne connaître pas moins et
n'aimer pas mieux Pontchartrain que moi. La Vrillière, qui savait
l'affaire dès son origine, et tout ce qui s'y était passé, et comment
j'avais sauvé Pontchartrain dans le temps même que j'avais le plus lieu
de m'en plaindre, fut indigné de ce dernier trait, et ne me cacha rien
de ce qu'il pensait de son perfide cousin, que d'ailleurs il n'aimait
pas, et dont il était traité avec la hauteur de grand et important
ministre, quoique secrétaire d'État comme lui. La vérité était que les
deux charges étaient fort inégales. On verra dans la suite ce que ce
forfait de Pontchartrain lui coûta.

\hypertarget{chapitre-xv.}{%
\chapter{CHAPITRE XV.}\label{chapitre-xv.}}

1713

~

{\textsc{Extraction abrégée de Tallard.}} {\textsc{- Mariage de son fils
avec une fille du prince de Rohan.}} {\textsc{- Fiançailles du duc de
Tallard et de la fille du prince de Rohan dans le cabinet du roi, et la
cause de cet honneur.}} {\textsc{- Signature du roi par lui déclarée de
nul poids aux contrats de mariage hors sa famille.}} {\textsc{- Adresse,
puis hardiesse des secrétaires d'État pour se décrasser de leur qualité
essentielle de notaires publics et de secrétaires du roi.}} {\textsc{-
Maréchal de Tallard signe partout au-dessus du prince de Rohan, et le
duc de Tallard au-dessus de sa future.}} {\textsc{- Abus faux d'une
galanterie du roi dont les Rohan tâchent d'abuser le monde.}} {\textsc{-
Renonciations.}} {\textsc{- Réflexions sommaires.}} {\textsc{- Pairs
conviés de la part du roi, chacun par le premier maître des cérémonies,
de se trouver au parlement.}} {\textsc{- Embarras de M. le duc de Berry
pour répondre au compliment du premier président\,; comment levé.}}
{\textsc{- Ducs de Berry et d'Orléans vont de Versailles au parlement.}}
{\textsc{- Messe à la Sainte-Chapelle.}} {\textsc{- Marche de la
Sainte-Chapelle à la grand'chambre.}} {\textsc{- Séance en bas.}}
{\textsc{- Pairs séants et absents\,; nombre de pairs et de pairies.}}
{\textsc{- M. le duc de Berry demeure court.}} {\textsc{- Entre-deux de
séance.}} {\textsc{- M. le duc de Berry et tous pairs en séance en
haut.}} {\textsc{- Orgueilleuse lenteur des présidents à revenir en
place, pour lesquels nul ne se lève.}} {\textsc{- Séance en haut.}}
{\textsc{- Deux petites aventures risibles.}} {\textsc{- Levée de la
séance et sortie.}} {\textsc{- Dîner au Palais-Royal.}} {\textsc{-
Retour à Versailles.}} {\textsc{- Indiscret compliment de
M\textsuperscript{me} de Montauban à M. le duc de Berry.}} {\textsc{-
Désespoir et réflexions de M. le duc de Berry.}}

~

Le maréchal de Tallard avait deux fils, dont l'aîné, qui promettait,
avait, comme on l'a dit en son lieu, été tué à la bataille d'Hochstedt.
Il ne lui en restait plus qu'un qui avait quitté le petit collet à la
mort de son frère, et qui avait un régiment d'infanterie, à
l'établissement duquel son père n'avait pu pourvoir pendant sa longue
prison. Quoique d'assez bonne noblesse, elle n'était pas illustrée, et
par conséquent peu connue. Point de grands fiefs, peu d'emplois et dans
le plus médiocre, des mères comme eux au plus, excepté une Montchenu,
une Beauffremont, une Gadagne, et tout cela en diverses branches et
moderne\,; la Tournon et la d'Albon toutes récentes. Le père du maréchal
était puîné de la Tournon et fit sa branche. Il épousa, en 1646,
Catherine de Bonne, fille d'Alexandre, seigneur d'Auriac et vicomte de
Tallard, qui venait d'un frère puîné du trisaïeul du connétable de
Lesdiguières et de Marie de Neuville, fille du marquis d'Alincourt,
gouverneur de Lyon, Lyonnais, etc., et de sa seconde femme Harlay-Sancy,
sœur de père et de mère du premier maréchal de Villeroy, laquelle se
remaria à Louis-Charles de Champlais, sieur de Courcelles, lieutenant
d'artillerie, sous le nom duquel elle a tant fait parler d'elle, et est
morte fort vieille en 1688. Par ce mariage il eut la terre de Tallard
dont il porta le nom, et par le premier maréchal de Villeroy, frère de
sa femme, il fut sénéchal de Lyon, et commanda dans le gouvernement du
maréchal de Villeroy en son absence. De ce mariage est venu le maréchal
de Tallard, qui était ainsi cousin germain du second maréchal de
Villeroy, dont il tira toute sa protection toute sa vie. Il avait donc
grand besoin d'alliance\,; et comme il était riche et grandement établi,
surtout esclave de toute faveur, et aboyant toujours après elle, tout
lui fut bon pour faire nager son fils, par conséquent lui-même, en toute
sorte d'éclat. Celui des Rohan était lors en tout son brillant, et il
crut, en s'amalgamant à eux, arriver au plus haut de la fortune.

Le prince de Rohan avait un fils unique et trois filles, toutes trois
belles. Ce fut où Tallard adressa ses vœux. Le maréchal de Villeroy
était de tous les temps plus que l'ami intime de la duchesse de
Ventadour. Son grand état, ses grands biens, la perspective de sa place
dans le lointain, une grande amitié, l'unissaient avec grand poids aux
Rohan. Il s'agissait d'une de ses petites-filles. Tallard s'accommodait
de tout, pourvu qu'il en pût obtenir une\,; par cette voie et à ces
conditions cela lui fut bientôt accordé. Le prince de Rohan voulait
marier ses filles pour l'honneur et le crédit de leur alliance, réserver
tout à son fils, substituer tout à son défaut et de ses fils, aux
Guéméné, leur marier une de ses filles convenable en âge, et de donner
gros à celle-là aux dépens des deux autres. Les biens, la dignité, le
gouvernement de Tallard, qu'ils espérèrent faire tomber à son fils, un
fils unique, l'esprit accort du père qu'ils comptaient mettre dans leur
dépendance, toujours actif, occupé et plein de vues dont ils espéraient
bien profiter, tout cela leur plut et le mariage fut bientôt conclu, et
le maréchal se démit de son duché en faveur de son fils.

Le roi, lassé de faire dans son cabinet des fiançailles d'autres que des
princes du sang, qui s'étaient hasardés quelquefois à lui en faire
sentir l'indécence, ne put en refuser une encore plus marquée à la
petite-fille de celle qu'il avait tant aimée, et pour l'amour de
laquelle il avait princisé les Rohan. Cet honneur des fiançailles dans
le cabinet du roi, qui est une des distinctions que les princes
étrangers ont emblée, ne s'accorde régulièrement que lorsque l'époux et
l'épouse sont l'un et l'autre de ce rang. Le roi passa outre en faveur
de la fille du fils de M\textsuperscript{me} de Soubise, quoiqu'elle ne
fût plus, mais dont la constante faveur porta sans cesse sur sa famille.
Ainsi le mardi 14 mars, les fiançailles se firent dans le cabinet du roi
par l'évêque de Metz, premier aumônier, avec tout l'apparat possible,
sur les six heures du soir\,; le prince de Rohan prit pour soi, et pour
sa fille, toutes les qualités de prince qu'il lui plut, que le maréchal
de Tallard ne lui contesta pas dans le contrat de mariage, et il n'y eut
point de difficulté pour la signature du roi, qui avait déclaré depuis
très-longtemps que sa signature aux contrats de mariage hors de sa
famille, n'était que pour l'honneur, et qu'elle n'approuve, ne donne et
ne confirme quoi que ce soit dans ces actes, et ne donne aucun poids à
rien de ce qui s'y met.

C'est, pour le dire en passant, ce qu'ont saisi les secrétaires d'État
pour décrasser leur existence. Elle était tout en leur qualité de
notaires du roi. C'est par cette qualité que leur signature est devenue
nécessaire à tous les actes que le roi signe et qui la rend valide par
la force que lui donne l'attestation de la leur, que cette signature du
roi est de lui-même, et n'est pas fausse et supposée, ce qui opère
qu'elle ne vaudrait pas seule sans celle du secrétaire d'État. Deux
secrétaires d'État signaient donc toujours tous les contrats de mariage
que le roi signait, en qualité de ses notaires, et ils sont si bien
notaires, que, s'ils voulaient passer des actes entre particuliers comme
font les notaires et les signer d'eux, il n'y serait pas besoin d'autres
notaires. Depuis que l'avilissement et la confusion a prévalu par maxime
de gouvernement, que par là les secrétaires d'État ont commencé à
devenir des métis, puis des singes, des fantômes, des espèces de gens de
la cour et de condition, enfin admis et associés en toute parité aux
gens de qualité, et que le roi a signé les contrats de mariage de
quiconque a voulu lui en présenter, jusque des personnes les plus viles,
les secrétaires d'État se sont abstenus d'y signer, et ont laissé la
fonction aux notaires. Restaient ceux qui étaient signés en cérémonies
aux fiançailles qui se faisaient dans le cabinet du roi, où les
secrétaires d'État n'avaient osé secouer leur fonction de notaires.

Les qualités des parties prétendues dans les contrats ne firent point de
difficulté tant que cet honneur des fiançailles dans le cabinet du roi
fut réservé aux princes qui étaient de maison souveraine ou de celle de
Longueville, dont la grandeur des services, des emplois et des alliances
continuelles était parvenue à la même égalité, même avec des avantages
sur les véritables princes des maisons de Lorraine et de Savoie. Mais
lorsque les Bouillon, à force de félonies et d'épouvanter le cardinal
Mazarin, furent devenus princes\,; que les Rohan, à force de fronde, de
troubles, de manéges et d'art, eurent commencé à pointer, et que la
beauté de M\textsuperscript{me} de Soubise eut achevé ce que la faveur
et les intrigues de la fameuse duchesse de Chevreuse et de la princesse
de Guéméné, sa belle-sœur, avaient commencé, les titres pris dans les
contrats de mariage de ces princes factices, que les véritables ne leur
passaient point avec eux, firent difficulté et furent longtemps sans
pouvoir être admis. D'autres particuliers, excités par la facilité de
prétendre et d'entreprendre, se mirent à en hasarder aussi.

Ces discussions, quoique si faciles à trancher court, fatiguèrent le
roi, qui ne voulait ni les confirmer ni les admettre, mais à qui, dans
l'esprit qu'il avait pris, les prétentions et les confusions plaisaient.
C'est ce qui produisit cette déclaration qu'il fit, que sa signature
n'autorisait et ne confirmait rien dans les contrats de mariage hors de
sa famille, et qu'elle n'était simplement que d'honneur\,; de là peu à
peu les secrétaires d'État lui représentèrent l'effet confirmatif de
leur signature apposée aux actes qu'il signait. Ils se gardèrent bien de
lui expliquer qu'elle n'était confirmative que parce qu'elle attestait
que c'était celle du roi, et que, par conséquent, elle ne pouvait pas
opérer plus que celle du roi. Ils lui firent peur pour la confirmation
et l'autorisation de titres qu'il ne voulait ni donner ni passer, d'un
acte qui les porterait passé devant eux et signé du roi et d'eux, et par
cette industrie ils lui firent trouver bon qu'ils se dispensassent
désormais de passer et de signer aucun de ces contrats de mariage comme
secrétaires d'État, même ceux des vrais princes, où il n'y aurait point
de difficulté pour les titres, afin de ne point marquer de différence,
et de les laisser tous aux notaires dans l'ordre ordinaire, excepté ceux
de sa famille. C'est ainsi que les secrétaires d'État se sont peu à peu
défaits de la crasse de leur origine, et sont parvenus où on les voit.
Mais ce dépouillement ne leur a pas suffi encore\,: ils ne pouvaient
signer le nom du roi dans tout ce que leurs bureaux expédient, que par
la qualité de secrétaires du roi.

Ce reste de bourgeoisie, quoique moins fâcheux que le notariat, leur a
déplu. Mais de pygmées ils étaient devenus géants, et s'étaient enfin
débarbouillés de l'étude de notaires\,; c'en était assez pour un règne,
quelque prodigieux qu'il eût été. Ils en attendirent un autre\,: tout y
fut pour eux à souhait. Un roi qui ne pouvait ni voir ni savoir, un
homme de leur espèce, maître absolu et sans contradiction du roi et de
l'État, et qui soufflait et protégeait la confusion par son intérêt
propre, qui monta au comble avec l'anéantissement de tout\,; un
chancelier à qui les exils n'avaient laissé que la terreur et une
flexibilité de girouette, la conjoncture ne pouvait pas être plus
favorable pour secouer leur état essentiel de secrétaires du roi, sans
que ceux-là osassent branler, ni le chancelier, leur protecteur né,
ouvrir la bouche. Ils se dressèrent donc à eux-mêmes des lettres qui les
autorisèrent à signer le nom du roi sans être secrétaires du roi, les
présentèrent hardiment au sceau, et le chancelier les scella sans oser
dire une seule parole. Dès que cela fut fait, ils vendirent leurs
charges de secrétaires du roi, et ceux qui sont parvenus depuis aux
charges de secrétaires d'État, et qui n'en avaient point de secrétaires
du roi, se sont bien gardés d'en prendre, quoique cela fût indispensable
auparavant. De cette façon, ceux qui n'étaient rien sont enfin devenus
tout, jusqu'à dépouiller leur origine essentielle qui leur faisait
honte, et comme les bassins de la balance, ceux qui étaient tout et
d'origine et d'essence sont tombés au néant.

Pour revenir aux fiançailles, le roi, toujours galant et touché des
figures aimables, plus encore du tendre souvenir de la grand'mère de la
fiancée, dit au duc de Tallard qu'il le croyait trop galant pour signer
le premier et fit signer sa future\,; mais il lui marqua lui-même
l'endroit pour y signer, mettant le bout du doigt sur le papier, puis
fit signer le duc de Tallard au-dessus d'elle, dont il lui avait fait
laisser la place. Le maréchal de Tallard alla signer immédiatement
ensuite, et aussitôt après lui le prince de Rohan. Ce détail, ils n'en
parlèrent pas. Ils espérèrent apparemment que la nombreuse assistance ou
l'oublierait ou pourrait ne l'avoir pas remarqué, et débitèrent la
galanterie du roi comme un avantage de princerie qu'il avait décidé pour
eux. Ils firent courir partout ce mensonge qui persuada les provinces et
ceux qui sont ignorants de ces sortes de choses. Les autres se moquèrent
d'eux, et les Tallard, contents de la réalité et d'en avoir la preuve
par le contrat de mariage même, où l'ordre des signatures démentait la
fausse vanterie, et les articles aussi où le maréchal de Tallard avait
encore signé devant le prince de Rohan, et le registre encore du curé,
ne firent semblant de rien. À minuit le mariage fut célébré par le
cardinal de Rohan dans la chapelle, où le roi ni aucun prince ni
princesse n'allèrent. Le curé de Versailles dit la messe. Il y avait
force conviés partagés à souper en quatre lieux différents, qui furent
chez M\textsuperscript{me} de Ventadour où furent les mariés, chez le
maréchal de Tallard, chez le prince de Rohan et chez le cardinal de
Rohan. Le lendemain elle reçut, sur le lit de la duchesse de Ventadour,
les visites de toute la cour et celles que les duchesses ont accoutumé
de recevoir des personnes royales.

L'affaire des renonciations était mûre. La paix était arrêtée. Le roi
était pressé de la voir signée par son plus instant intérêt\,; et la
cour d'Angleterre, à qui nous la devions toute, n'en avait pas moins de
consommer ce grand ouvrage, pour jouir, avec la gloire de l'avoir
imposée à toutes les puissances, du repos domestique qu'agitait sans
cesse le parti qui lui était opposé, et qui, excité par les ennemis de
la paix du dehors, ne pouvait cesser de donner de l'inquiétude au
ministère de la reine, tant que par le délai de la signature, les vaines
espérances de la troubler et de l'empêcher, subsisteraient dans les
esprits. Le roi d'Espagne avait satisfait sur ce grand point des
renonciations avec toute la solidité et la solennité qui se pouvaient
désirer des lois, coutumes et usages d'Espagne\,: il n'y avait plus que
la France à l'imiter.

On a dit sur cette matière tout ce dont à peu près elle se trouve
susceptible, et la matière est encore plus éclaircie parmi les
Pièces\footnote{Voir les Pièces. (\emph{Note de Saint-Simon}.)}. Ce
serait donc répéter inutilement que vouloir représenter de nouveau ce
que peuvent être des renonciations à la couronne de France d'un prince
et d'une branche aînée en faveur de ses cadets, contre l'ordre constant,
et jamais interrompu depuis Hugues Capet, sans que la France l'accepte
par une loi nouvelle dérogeant à celle de tous les siècles et par une
loi revêtue des formes et de la liberté qui puissent lui acquérir la
force et la solidité nécessaire à un acte si important\,; et la
renonciation à leur droit à la couronne d'Espagne, uniquement fondée sur
celle au droit à la France et sur l'accession plus prochaine par le
retranchement de toute une branche en faveur de deux princes et de la
leur, et des autres des princes du sang après, suivant leur aînesse, qui
soumis au roi le plus absolu et le plus jaloux de l'être qui ait jamais
régné, grand-père de l'un, oncle et beau-père de l'autre, grand-père
encore d'une autre façon des deux princes du sang, sont forcés
d'assister avec les pairs à la lecture et à l'enregistrement de ces
actes, sans, qu'avec leur lecture, on ait auparavant exposé, moins
encore traité la matière, ni après, que personne ait été interpellé
d'opiner, ni que, si on l'avait été, personne eût osé dire un seul mot
que de simple approbation. C'est néanmoins tout ce qui fut fait, comme
on le va voir, pour opérer ce grand acte destiné à régler, d'une manière
jusqu'alors inouïe en France, un ordre nouveau d'y succéder à la
couronne, d'en consolider un autre guère moins étrange de succéder à la
monarchie d'Espagne, et assurer par là le repos à toute l'Europe, qui ne
l'avait pu trouver à l'égard de l'Espagne seule dans la solennité des
renonciations du traité des Pyrénées et des contrats de mariage de Louis
XIII et de Louis XIV, tous enregistrés au parlement, et le traité des
Pyrénées et le contrat de mariage de Louis XIV avec ses plus expresses
renonciations, faits et signés aux frontières par les deux premiers
ministres de France et d'Espagne en personne, et jurés solennellement
par les deux rois en présence l'un de l'autre, au milieu des deux cours.

On ne sent que trop l'extrême différence de ce qui se passa alors avec
ce qui vient d'être présenté et qui va être raconté, et si lors de la
paix des Pyrénées et du mariage du roi, il ne s'agissait pas
d'intervertir l'ordre de la succession à la couronne de France, et d'y
en établir une dont tous les siècles n'avaient jamais ouï parler.

Ce culte suprême dont le roi était si jaloux pour son autorité, parce
que son établissement solide avait été le soin le plus cher et le plus
suivi de toute sa longue vie, ne put donc recevoir la moindre atteinte,
ni par la nouveauté du fait, ni par l'excès de son importance pour le
dedans, pour le dehors, pour sa propre maison, ni par la considération
de sa plus intime famille, ni par celle que cette idole à qui il
sacrifiait tout allait bientôt lui échapper à son âge, et le laisser
paraître nu devant Dieu comme le dernier de ses sujets. Tout ce qu'on
put obtenir pour rendre la chose plus solennelle fut l'assistance des
pairs. Encore sa délicatesse fut-elle si grande, qu'il se voulait
contenter de dire en général qu'il désirait que les pairs se trouvassent
au parlement pour les renonciations. Je le sus quatre jours auparavant.
Je parlai à plusieurs, et je dis à M. le duc d'Orléans que, si le roi se
contentait de s'expliquer de la sorte, il pouvait compter qu'aucun pair
n'irait au parlement, et que c'était à lui à voir ce qui lui convenait
là-dessus pour tirer d'une méchante paye ce qu'il serait possible\,;
mais que, si les pairs n'étaient pas invités de sa part, chacun par le
grand maître des cérémonies, ainsi qu'il s'est trouvé pratiqué, pas un
seul ne se trouverait au parlement. Cet avis ferme, et qui eût été suivi
de l'effet, comme on a vu qu'il était arrivé sur le service de
Monseigneur à Saint-Denis, réussit. M. le duc d'Orléans et M. le duc de
Berry en parlèrent au roi, et insistèrent, de manière que Dreux alla
lui-même chez tous les pairs qui logeaient au château à Versailles, et à
ceux qu'il ne trouva point leur laissa le billet qui se trouvera dans
les Pièces, portant que M. le duc tel est averti de la part du roi qu'il
se traitera tel jour au parlement de matières très-importantes,
auxquelles Sa Majesté désire qu'il assiste. Signé, Dreux, et daté. À
ceux qui, étaient à Paris, il se contenta de leur envoyer le billet\,;
pour les princes du sang et légitimés, il fallut qu'il les trouvât,
ainsi ils n'eurent point de billet. Les Anglais enfin n'ayant pu obtenir
mieux, et pressés au dernier point, comme on l'a dit, de finir,
voulurent bien se persuader que c'était tout ce qui se pouvait faire.
Voici donc enfin ce qui se fit.

La séance devait commencer par un compliment du premier président de
Mesmes à M. le duc de Berry, qui devait lui répondre. Il en fut fort en
peine. M\textsuperscript{me} de Saint-Simon à qui il s'en ouvrit, trouva
moyen par un subalterne d'avoir le discours du premier président, et le
donna à M. le duc de Berry pour y régler sa réponse. Cet ouvrage lui
sembla trop fort\,: il l'avoua à M\textsuperscript{me} de Saint-Simon,
et qu'il ne savait comment faire. Elle lui proposa de m'en charger, et
il fut ravi de l'expédient. Je fis donc une réponse d'une page et demie
de papier à lettre commun et d'écriture ordinaire. M. le duc de Berry la
trouva fort bien, mais trop longue pour l'apprendre\,; je l'abrégeai\,;
il la voulut encore plus courte, tellement qu'elle n'avait au plus que
les trois quarts d'une page. Le voilà donc à l'apprendre par cœur\,; il
en vint à bout, et la récita dans son cabinet seul à
M\textsuperscript{me} de Saint-Simon la veille de la séance, qui
l'encouragea du mieux qu'elle put.

Le mercredi 15 mars, je me rendis à six heures du matin chez M. le duc
de Berry en habit de parlement, et peu après M. le duc d'Orléans y vint
aussi en même équipage avec une grande suite. Vers six heures et demie
ces deux princes montèrent dans le carrosse de M. le duc de Berry\,; le
duc de Saint-Aignan et moi nous mîmes au devant. Il était aussi en habit
de parlement, et il était premier gentilhomme de la chambre de M. le duc
de Berry\,; à la portière, de son côté, son capitaine des gardes avec le
bâton\,; à l'autre, le premier gentilhomme de la chambre de M. le duc
d'Orléans. Plusieurs carrosses des deux princes suivirent remplis de
leur suite, et force gardes de M. le duc de Berry avec leurs officiers
autour de son carrosse. Il fut fort silencieux en chemin. J'étais
vis-à-vis de lui, et il me parut fort occupé de tout ce qu'il allait
trouver et dire. M. le duc d'Orléans, au contraire, fut fort gai, et fit
des contes de sa jeunesse et de ses courses nocturnes à pied dans Paris
qui lui en avaient appris les rues, auxquels M. le duc de Berry ne prit
aucune part. On arriva assez légèrement à la porte de la Conférence,
c'est-à-dire, aujourd'hui qu'elle est abattue, au bout de la terrasse et
du quai du jardin des Tuileries.

On trouva là les trompettes et les timbales des gardes de M. le duc de
Berry qui firent grand bruit tout le reste de la marche, qui ne fut plus
qu'au pas jusqu'au palais, où on alla droit à l'escalier de la
Sainte-Chapelle, à l'entrée de laquelle l'abbé de Champigny, trésorier,
les reçut comme ils ont accoutumé de recevoir les fils de France.
L'appui des deux stalles du choeur les plus proches de l'autel, du côté
de l'épître, était couvert d'un drap de pied avec des carreaux où les
deux princes se placèrent. Je laissai la troisième stalle vide, et je
retirai le carreau qu'on y avait mis à la quatrième. M. de Saint-Aignan
se mit sur le sien à la cinquième. Il n'y eut point d'autres carreaux,
et personne que nous ne monta dans les hautes stalles, d'un côté ni
d'autre. Les officiers principaux des deux princes se mirent dans les
stalles basses des deux côtés vers l'autel, laissant vides les deux
stalles qui étaient au-dessous de celles où étaient les deux princes. La
Sainte-Chapelle était assez remplie de monde, parmi lequel il y avait
des gens de qualité venus pour les accompagner, mais non dans leurs
carrosses, de Versailles, où il n'y eut que leur suite.

La messe basse étant finie au grand autel, on sortit de la chapelle, à
la porte de laquelle se trouvèrent deux présidents à mortier et deux
conseillers de la grand'chambre députés du parlement pour venir recevoir
M. le duc de Berry. Le court compliment reçu et rendu, on se mit en
marche, les deux présidents aux deux côtés de M. le duc de Berry,
derrière lequel était le capitaine de ses gardes avec le bâton. Il était
précédé de M. le duc d'Orléans entre les deux conseillers\,; je marchais
immédiatement seul devant ce prince, et le duc de Saint-Aignan seul
aussi immédiatement devant moi. Les officiers principaux des deux
princes et beaucoup de gens de qualité marchaient confusément devant et
derrière, et les gardes de M. le duc de Berry, le mousquet sur l'épaule
avec leurs officiers, côtoyaient la marche des deux côtés et avaient
grand'peine à faire faire place.

La foule du peuple, depuis la Sainte-Chapelle jusqu'à la grand'chambre,
était telle, qu'une épingle ne serait pas tombée à terre, et des gens
grimpés de tous les côtés où ils purent. La séance était entière lorsque
M. le duc de Berry y arriva, c'est-à-dire les princes du sang et
légitimés, tous les autres pairs, tout le parlement. Tournelle, enquêtes
et requêtes étaient en place avec la grand'chambre, les conseillers
d'honneur, les honoraires et quatre anciens maîtres des requêtes\,;
toute la séance était en bas, et en haut et derrière la séance sur des
bancs fleurdelisés pour tout ce qui avait séance, mais qui ne pouvait
tenir dans le carré ordinaire, où il n'y eut presque de place que pour
les pairs. On était en bas parce que ce qu'on allait faire était supposé
à huis clos, mais toute la grand'chambre était pleine en confusion de
toutes sortes de personnes debout en foule. On fit asseoir sur les
derniers bancs de derrière tout ce qu'on put de gens de la cour et de
personnes de qualité. Les deux princes, suivis des deux présidents à
mortier, traversèrent le parquet pour aller prendre leurs places\,; le
duc de Saint-Aignan et moi prîmes les nôtres, et entrâmes en séance
immédiatement avant eux\,; les deux conseillers, qui à l'entrée de la
séance étaient demeurés en arrière, gagnèrent les leurs comme ils
purent. Toute la séance se leva et se découvrit à l'approche des princes
dès l'entrée de la séance, avant que nous y entrassions, et ne se rassit
et se couvrit que lorsqu'ils s'assirent et se couvrirent. Le duc de
Shrewsbury, accompagné de l'introducteur des ambassadeurs et de quelques
Anglais de sa suite, était en haut dans la lanterne, du côté de la
cheminée, qu'on avait préparée pour lui, comme témoin nécessaire de cet
acte de la part de l'Angleterre. Je marquerai ici les pairs qui étaient
en séance, et à côté ceux qui ne s'y trouvèrent pas, parmi lesquels la
plupart n'avaient pas l'âge porté par l'édit de 1711 pour être reçus au
parlement. On verra ainsi tout ce qui existait alors de ducs et pairs en
France.

PAIRS EN SÉANCE.

M. le duc de Berry. M. le duc d'Orléans. MM. les Duc de Bourbon. Prince
de Conti. Duc du Maine. Comte de Toulouse. Archevêque-duc de Reims,
Mailly, depuis cardinal. Évêque-duc de Laon, Clermont-Chattes.
Évêque-duc de Langres, Clermont-Tonnerre. Évêque-comte de Châlons,
Noailles. Évêque-comte de Noyon, Châteauneuf-Rochebonne. Duc de La
Trémoille. Duc de Sully. Duc de Richelieu. Duc de Saint-Simon. Duc de La
Force. Duc de Rohan-Chabot. Duc d'Estrées. Duc de La Meilleraye et
Mazarin. A.\footnote{Les lettres marquent les pères {[}démis{]} et les
  fils qui ont des démissions. (\emph{Note de Saint-Simon}.)} Duc de
Villeroy. C. Duc de Saint-Aignan. Duc de Foix. Duc de Tresmes. Duc de
Coislin, êvêque de Metz. D. Duc de Charost. Duc de Villars, maréchal de
France. Duc de Berwick, maréchal de France. Duc d'Antin. Duc de
Chaulnes.

PAIRS ABSENTS.

MM\hspace{0pt}. les Cardinal de Janson, évêque-comte de Beauvais. Il se
mourait, et de plus, les cardinaux-pairs ne vont point au parlement,
parce qu'ils n'y seyent qu'au rang de leur pairie. Duc d'Uzès, était en
Languedoc. Duc d'Elbœuf. Duc de Ventadour. Tous deux n'avaient jamais
voulu prendre la peine de se faire recevoir au parlement. Duc de
Montbazon, malade. Duc de Luynes. Duc de Brissac. Duc de Fronsac. Tous
trois n'avaient pas l'âge d'être reçus. Duc de La Rochefoucauld,
aveugle. Duc de Valentinois, à Monaco. Duc de Bouillon, malade. Duc
d'Albret, non reçu. Duc de Luxembourg, en son gouvernement de Normandie.
A. Duc de Villeroy, maréchal de France, démis. B. Duc de Grammont. B.
Duc de Guiche. Démis l'un et l'autre. B. Duc de Louvigny, non reçu. Duc
de Mortemart, non reçu. C. Duc de Beauvilliers, démis. Duc de Noailles,
en quartier de capitaine des gardes. Duc d'Aumont, ambassadeur en
Angleterre. D. Duc de Béthune, démis. Cardinal de Noailles, archevêque
de Paris. Duc de Boufllers, non reçu. Duc d'Harcourt, maréchal de
France, était chez lui incommodé en Normandie.

La séance était ainsi d'un fils de France, d'un petit-fils de France, de
deux princes du sang, de deux bâtards, de cinq pairs ecclésiastiques et
de dix-huit pairs laïques\,: les absents étaient deux princes du sang
enfants, deux pairs ecclésiastiques cardinaux, dix pairs absents ou
malades, neuf non reçus, la plupart trop jeunes, et six qui, ayant donné
leur démission à leur fils ou frère, n'entraient plus au parlement. Cela
faisait alors sept pairies ecclésiastiques, et sept archevêques ou
évêques-pairs, trente-sept duchés-pairies laïques, et par les démissions
quarante-deux ducs et pairs, sans compter les bâtards. Ils étaient donc
vingt-cinq absents par diverses causes, et M. le duc de Berry compris,
nous étions vingt-neuf en séance. Elle aurait bien valu la peine que le
chancelier fût venu la tenir\,: il n'aimait pas les cérémonies\,; il
n'était jamais venu au parlement depuis qu'il était chancelier\,: ce qui
se devait passer semblait peu dans les règles. Le roi, qui n'avait
consenti qu'à peine à tout ce qui passait la solennité d'un
enregistrement ordinaire, ne lui proposa point d'y aller, et lui était
encore plus éloigné de se le faire dire, et d'avoir envie de s'y
trouver.

M. le duc de Berry en place, on eut assez de peine à faire faire
silence. Sitôt qu'on put s'entendre, le premier président fit son
compliment à M. le duc de Berry. Lorsqu'il fut achevé, ce fut à ce
prince à répondre. Il ôta à demi son chapeau, le remit tout de suite,
regarda le premier président, et dit : «\,Monsieur\ldots\,» Après un
moment de pause, il répéta\,: «\,Monsieur\ldots\,» II regarda la
compagnie, et puis dit encore\,: «\,Monsieur\ldots\,» Il se tourna à M.
le duc d'Orléans, plus rouges tous deux que le feu, puis au premier
président, et finalement demeura court sans qu'autre chose que
\emph{«\,}Monsieur\,» lui pût sortir de la bouche. J'étais vis-à-vis du
quatrième président à mortier, et je voyais en plein le désarroi de ce
prince\,: j'en suais, mais il n'y avait plus de remède. Il se tourna
encore à M. le duc d'Orléans qui baissait la tête. Tous deux étaient
éperdus. Enfin, le président, voyant qu'il n'y avait plus de ressource,
finit cette cruelle scène en ôtant son bonnet à M. le duc de Berry, et
s'inclinant fort bas comme si la réponse était finie, et tout de suite
dit aux gens du roi de parler. On peut juger quel fut l'embarras de tout
ce qui était là de la cour, et la surprise de toute la magistrature. Les
gens du roi exposèrent donc de quoi il s'agissait, et en firent après
une longue pièce d'éloquence\,: c'était de retirer des registres du
parlement des lettres patentes qui conservaient le droit à la couronne
de France au roi d'Espagne et à sa branche, quoique absents et non
regnicoles, quand il s'en alla en Espagne, et de faire la lecture de sa
renonciation pour lui et pour toute sa branche à la couronne de France,
et celles de M. le duc de Berry et de M. le duc d'Orléans à la couronne
d'Espagne, pour eux et pour leur postérité, et d'enregistrer toutes ces
trois renonciations. Le premier président expliqua les intentions du
roi. L'avocat Joly de Fleury porta la parole et fit la réquisition\,;
les conclusions du procureur général furent lues\,; on opina du
bonnet\,: tout cela fut fort long.

L'arrêt d'enregistrement prononcé, les présidents se levèrent avec toute
la magistrature\,; ils firent une révérence profonde à M. le duc de
Berry, qui se découvrit sans se lever\,; les présidents s'en allèrent à
la buvette, et toute la magistrature les y suivit. M. le duc d'Orléans
ne se leva point du tout non plus, ni au salut, ni lorsqu'ils se
retirèrent. Sur cet exemple, les deux princes du sang et les deux
bâtards, qui se lèvent toujours pour les présidents à mortier, parce
qu'ils se lèvent pour eux, ne se levèrent point du tout\,; et les pairs,
qui jamais ne se lèvent pour les présidents à mortier ni pour le premier
président, parce qu'ils ne se lèvent pas pour eux, demeurèrent
pareillement assis. On se tint donc en place pendant que la robe vidait
tous ses bancs, puis chacun s'approcha des princes et les uns des
autres, et les personnes de qualité et de la cour quittèrent leurs
places, et entrèrent dans le parquet, où les princes et tout le monde
étaient debout, pêle-mêle, à causer les uns avec les autres. Au bout
d'un quart d'heure, M. le duc d'Orléans me fit appeler parmi tout ce
monde, et me demanda s'il ne fallait pas se mettre en place avant
l'arrivée des présidents et de la magistrature. Je lui dis que cela se
pouvait, mais qu'il suffisait aussi d'être avertis à temps pour se
placer un instant auparavant, ou même arriver tous en place en même
temps qu'eux. Il jugea qu'ils allaient revenir, parce qu'il ne
s'agissait que de prendre leurs grandes robes rouges, avec leurs
épitoges, et leur mortier à la main, et qu'ils ne voudraient pas faire
attendre M. le duc de Berry. Ainsi il me dit de faire avertir les pairs
que M. le duc de Berry et lui allaient monter aux hauts siéges, et s'y
mettre en place. Cela s'exécuta un moment après, et le parquet se vida.
Chacun alla rechercher à s'asseoir en lieu de voir et d'entendre. Les
gens du parlement avaient cependant redoublé un banc aux hauts siéges, à
droite, couvert d'un tapis fleurdelisé, pour les pairs qui ne pourraient
avoir place sur le banc fixe ordinaire, adossé à la muraille, moyennant
quoi il y eut place pour tous.

Je ne sais ce qui se passa entre les princes après qu'ils furent en
place, car, bien que je fusse sur le banc adossé à la muraille, j'étais
loin d'eux et le quinzième, parce que les pairs ecclésiastiques, qui
joignent le coin du roi aux hauts siéges, à gauche, aux lits de justice,
se mettent à droite quand ce n'est que parlement comme ce jour-là. Peu
de temps après que nous fûmes tous en séance, attendant le parlement à
revenir, je m'entendis appeler de main en main par les pairs d'au-dessus
de moi, qui me dirent d'aller parler à M. le duc de Berry et à M. le duc
d'Orléans, qui me demandaient. Je ne sais si M. le Duc, qui s'était
peut-être trouvé embarrassé de se lever à son ordinaire, ou de ne se
point lever, à l'exemple des deux premiers princes, à la sortie des
présidents, ne les avait point tentés de se lever à leur rentrée.
J'allai donc les trouver joignant le coin du roi, et comme il n'y avait
personne que nous en place, ni eux, ni les pairs, devant qui je passai
et repassai, ne se levèrent point\,; car autrement, lorsqu'on est en
véritable séance, les fils de France, princes du sang et autres pairs,
se lèvent tout debout pour un pair qui arrive, et ne se rassoient qu'en
même temps que lui. M. le duc d'Orléans me mit donc debout entre lui et
M. le duc de Berry, assis et tourné à eux, et là ils me demandèrent
s'ils se lèveraient lorsque le premier président, suivi des autres,
rentrerait par la lanterne de la buvette, et coulerait le long de leur
banc jusque près d'eux. Je leur dis que non\,; qu'ils devaient demeurer
découverts, pour l'être lorsque les présidents paraîtroient\,; les
laisser arriver tons à leurs places, et leur rendre une légère
inclination de corps, sans bouger d'ailleurs, lorsque, avant de
s'asseoir, ils leur feraient la révérence, et cette inclination unique
pour tous, en passant leurs yeux sur eux le long de leur banc. Ils s'en
tinrent là sans ajouter rien davantage. M. le Duc, qui en entendit
quelque chose, m'arrêta comme je passais devant lui pour me retirer à ma
place, et me demanda s'il se lèverait. Je souris, et je lui dis que
j'ignorais ce qu'il voulait bien accorder à ces messieurs-là\,; mais que
M. le duc de Berry ni M. le duc d'Orléans ne se lèveraient, ni n'en
feraient pas le moindre semblant, parce qu'ils ne le devaient pas, ni
les pairs ne s'en remueraient pas, et je regagnai ma place.

La morgue présidentale n'avait garde de manquer une si belle occasion de
s'exercer sur des fils de France. Ils prolongèrent leur toilette plus de
trois gros quarts d'heure, et ils excitèrent les murmures tout haut, que
nous entendions de nos places. Enfin ils arrivèrent, et je remarquai que
la rougeur monta bien forte au visage du premier président, et des deux
ou trois premiers qui le suivaient, lorsqu'ils virent M. le duc de Berry
et M. le duc d'Orléans ne branler pas à leur arrivée, les deux princes
du sang et les deux bâtards ne remuer pas davantage, et qu'ils n'eurent
de tous, ainsi que des pairs, qu'ils saluèrent aussi tournés vers eux,
et regardant le long de leurs bancs, que la légère inclination que
j'avais proposée. En même temps, les siéges bas et les bancs
fleurdelisés qu'on avait ajoutés derrière se garnirent de toute la
magistrature. Elle fut quelque temps à se placer, et les huissiers après
à faire faire silence.

Comme c'était jouer à la \emph{Madame en haut}, comme on avait fait en
bas, ou, en présence de tout ce que la grand'chambre avait pu contenir
de spectateurs, on avait fait semblant d'être seuls à huis clos, et
comme s'il ne s'agissait, en cette nouvelle séance, que de la
promulgation de ce qui s'était fait en la précédente, le premier
président cria qu'on ouvrît les portes et qu'on fit entrer. C'était pour
la forme\,; elles n'avaient pas été fermées un moment de toute cette
longue matinée, et tout était tellement rempli qu'il n'y put entrer
personne au delà de ce qui y était et y avait toujours été. Quand ce
premier vacarme des huissiers fut passé, qu'ils eurent après crié
silence, et que le bruit fut un peu apaisé, on recommença à lire et à
débiter, mais en autres termes, pour varier l'éloquence des gens du roi,
les mêmes choses qui s'étaient lues et plaidées en la séance d'en bas,
en sorte que la longueur en fut excessive.

Les choses les plus sérieuses, quelquefois même les plus tristes, sont
assez souvent mêlées d'aventures plaisantes, dont le contraste surprend
le rire des plus graves. Je ne puis m'empêcher d'en rapporter deux dont
je fus le témoin bien près en cette cérémonie, et fort en peine de ce
qui m'en arriverait à la première. Mon rang à la séance des bas siéges
me plaça entre les ducs de Richelieu et de La Force. Il y avait déjà
assez longtemps qu'ils étaient en séance en attendant M. le duc de
Berry. Peu après son arrivée, je sentis frétiller le bonhomme Richelieu,
qui bientôt après me demanda si cela serait long. Je lui dis que je le
croyais, par les lectures et par la parade de discours des gens du roi.
Le voilà à grommeler et à trouver cela fort mauvais. Il ne fut pas
longtemps en repos sans en revenir aux questions et aux frétillages, et
à me dire enfin qu'il se mourait d'envie d'aller à la garde-robe, et
qu'il fallait donc qu'il sortît. Je lui représentai l'indécence de
sortir d'une séance où il était vu de tout ce qui y était depuis les
pieds jusqu'à la tête, et où il n'y avait devant lui que le vide du
carré du parquet de la séance. Cela ne le contenta point, et j'eus
bientôt une nouvelle recharge. Je connaissois l'homme par expérience,
que, pour sa rareté, je n'ai pas omise ci-dessus (t. I\^{}er, p.~162).
Je savais qu'il prenait presque tous les soirs de la casse, souvent un
lavement le matin, avec lequel il sortait, et le promenait trois ou
quatre heures, et le rendait chez qui il se trouvait. La frayeur me
saisit pour ses chausses, et par conséquent pour mon nez. Je me mis donc
à regarder comment je pourrais me défaire d'un si dangereux voisin, et
je vis avec douleur que la chose était impossible, par l'excès de
l'entassement de la foule. Pour le faire court, les bouffées de sortir,
les menaces de ne pouvoir plus se retenir continuèrent toute la séance,
et redoublèrent tellement sur la fin, que je me crus perdu plus d'une
fois. Lorsqu'elle finit, je priai l'abbé Robert, conseiller-clerc de la
grand'chambre, qui se trouva assis précisément derrière nous, et qui
avait entendu tout ce colloque, de tâcher à faire sortir M. de
Richelieu. On y eut toutes les peines du monde, à force de soins de
l'abbé Robert et d'huissiers qu'il appela à son secours. Il ne revint
point pour la séance des hauts siéges.

La scène qui m'y amusa n'eut rien de menaçant. M. de Metz s'y trouva
placé le dos à mes genoux sur ce banc redoublé dans la largeur en long
des hauts siéges, au bas de la banquette qui règne au bas du banc fixe
ordinaire qui est adossé à la muraille, sur lequel j'étais. Bientôt
après qu'on eut commencé, voilà M. de Metz à s'impatienter, à gloser sur
l'inutilité de ce qui se débitait, à demander si ces gens-là avaient
résolu de nous faire coucher au palais, à frétiller, et finalement à
dire qu'il crevait d'envie de pisser. Il était plaisant, même avec un
naturel comique qui perçait jusque dans les choses les plus sérieuses.
Je lui proposai de pisser devant lui sur les oreilles des conseillers
qui se trouvaient au-dessous de lui aux bas siéges. Il secouait la tête,
parlait tout haut, apostrophait l'avocat général entre ses dents, et se
trémoussait de manière que les ducs de Tresmes et de Charost, entre qui
il était, lui disaient à tous moments de se tenir, comme ils auraient
fait à un enfant, et que nous mourions de rire. Il voulait sortir tout
de bon, il voyait la chose impossible, il jurait qu'on ne le
rattraperait jamais à pareille fête\,; quelquefois il protestait qu'il
allait se soulager aux dépens de lui et de qui il appartiendrait\,;
enfin il nous divertit toute la séance. Je ne vis jamais homme si aise
que lui quand elle finit.

Il était fort tard quand tout fut achevé. La séance se leva\,; les
princes descendirent par le petit degré du coin du roi. Les deux
présidents et les deux conseillers qui avaient reçu M. le duc de Berry à
la Sainte-Chapelle se trouvèrent dans le débouché du parquet, marchèrent
comme ils avaient fait en venant, et le conduisirent au même degré de la
Sainte-Chapelle. Pendant que les princes descendaient des siéges hauts
par ce petit degré du coin du roi, les pairs et les présidents qui
étaient debout se saluèrent, et reployèrent en même temps chacun le long
du banc où il était assis, les plus anciens les premiers\,; les
présidents sortirent par la lanterne de la buvette, les pairs par celle
de la cheminée, comme on était entré, et les pairs sortirent ensemble,
deux à deux, précédés d'un huissier à l'ordinaire. M. de Saint-Aignan et
moi les quittâmes au sortir de la grand'chambre, pour rejoindre M. le
duc de Berry et M. le duc d'Orléans, et monter en carrosse avec eux. Ils
allèrent droit au Palais-Royal, au pas, avec la même pompe qu'ils
étaient arrivés au palais. La conversation en chemin fut fort sobre\,;
M. le duc de Berry paraissait consterné, embarrassé, mais aussi dépité.
En arrivant au Palais-Royal, ils reprirent tous deux leur habit
ordinaire, et M. de Saint-Aignan et moi les nôtres.

M. le duc d'Orléans avait convié entre les deux séances beaucoup de
pairs et de gens de qualité à dîner au Palais-Royal avec M. le duc de
Berry. Il m'avait chargé aussi de prier des pairs et ceux des personnes
de qualité qu'il me nomma que je trouverais sous ma main entre les deux
séances, qu'il ne trouverait peut-être pas sous la sienne, et ses
principaux officiers d'en prier beaucoup de sa part, ce qui leur était
plus aisé, parce qu'ils étaient répandus avec eux hors la séance. On
pirouetta quelque peu de temps dans ce grand appartement du Palais-Royal
que M. le duc d'Orléans avait magnifiquement accommodé et augmenté,
jusqu'à ce que les conviés pussent être arrivés du palais. On servit une
table de prodigieuse grandeur, qui fut également splendide et délicate,
sans aucun plat gras. M. le duc de Berry se mit au milieu dans un
fauteuil, reçut la serviette que lui présenta M. le duc d'Orléans, et
eut seul une soucoupe pour boire et une serviette sous son couvert, mais
point de cadenas\footnote{Voy., t. I\^{}er, p.~32, note.}. M. le duc
d'Orléans se mit sans intervalle à sa droite, sur un siége tout pareil à
ceux de toute la compagnie, MM. de Reims et de Laon se mirent auprès
d'eux à droite et à gauche, et les autres ducs ensuite. M. de Poix se
mit vis-à-vis d'eux au milieu. Leurs principaux officiers étaient à
table et beaucoup des gens de qualité. Ceux de M. le duc d'Orléans s'y
dispersèrent pour en faire les honneurs\,; M. le duc d'Orléans les fit
aussi lui-même avec beaucoup de grâce et de liberté, mais avec dignité
et mesure. On y fut longtemps, parce que le repas fut grand et bon, et
que chacun mourait de faim. La multitude des voyeurs, le nombre de ceux
qui étaient à table, ni la quantité des plats et des services,
n'empêchèrent pas la promptitude de les relever quand il était temps
avec tout l'ordre possible, et que chacun ne fût servi comme à une table
de cinq ou six couverts. L'extrême sérieux de M. le duc de Berry, et son
silence devant et pendant le repas, en ôta la gaieté. Chacun causait
avec ses voisins, et la faim et la bonne chère empêchèrent qu'on ne
s'ennuyât. Avant, pendant et après, M. le duc d'Orléans fut d'une
politesse infinie et très-attentif pour tout le monde. Les deux princes
du sang et les deux légitimés qui s'étaient trouvés au parlement ne
furent point invités au Palais-Royal, ni l'ambassadeur d'Angleterre.

Les deux princes partirent bientôt après qu'ils furent sortis de table,
et furent au pas jusqu'à la porte Saint-Honoré, avec la pompe qu'ils
étaient entrés le matin dans Paris. Ils parurent l'un et l'autre fort
scandalisés de plusieurs choses qu'ils avaient remarquées au parlement,
les unes à l'égard des pairs seulement, les autres qu'ils avaient
partagées avec eux. Je les supprime ici, parce qu'il y aura lieu d'en
parler dans la suite. Du reste, M. le duc de Berry, qui ne se rasséréna
point pendant tout le chemin, tint le carrosse dans le sérieux et la
réserve. Ils mirent pied à terre à Versailles, dans la cour des Princes,
apparemment parce que les gardes de M. le duc de Berry ne l'auraient pu
suivre dans la grande cour. Ils trouvèrent à leur portière un message
qui les attendait. La duchesse de Tallard avait, comme on l'a dit, été
fiancée la veille, mariée la nuit, et recevait ce jour-là ses visites
sur le lit de la duchesse de Ventadour. Elle envoya donc attendre les
deux princes, et les prier de vouloir bien venir chez sa petite-fille
avant d'entrer chez eux, s'ils voulaient lui faire l'honneur de l'aller
voir, parce que les visites étaient finies, et qu'elle n'attendait plus
qu'eux pour sortir de dessus ce lit. Ils y allèrent tout droit.

Ils furent reçus, entre autres, par la princesse de Montauban, qui, avec
sa flatterie ordinaire, et sans savoir un mot de ce qui s'était passé,
se mit à crier, dès qu'elle aperçut M. le duc de Berry, qu'elle était
charmée de la grâce et de la digne éloquence avec laquelle il avait
parlé au parlement, et paraphrasa ce thème de toutes les louanges dont
il était susceptible. M. le duc de Berry rougit de dépit, sans dire une
parole, et marchant toujours pour gagner le lit\,; elle de redoubler,
d'admirer sa modestie, qui le faisait rougir et ne point répondre, et ne
cessa point qu'ils ne fussent arrivés auprès de la mariée. M. le duc de
Berry n'y demeura que quelques moments debout et s'en alla. Il fut
reconduit comme il avait été reçu, et toujours poursuivi par cette
vieille sur les merveilles qu'il avait faites, et les applaudissements
qu'il s'était attirés du parlement et de tout Paris. Délivré d'elle à la
fin par le terme de la conduite, il s'en alla chez M\textsuperscript{me}
la duchesse de Berry, où il trouva du monde, n'y dit mot à personne, à
peine à M\textsuperscript{me} la duchesse de Berry, prit
M\textsuperscript{me} de Saint-Simon, et s'en alla chez lui seul avec
elle, où il s'enferma dans son cabinet.

Il s'y jeta dans un fauteuil, s'écria qu'il était déshonoré, et le voilà
aux hauts cris et à pleurer à chaudes larmes. Il raconta à
M\textsuperscript{me} de Saint-Simon, à travers les sanglots, comment il
était demeuré court au parlement sans pouvoir proférer une parole\,; à
appuyer sur l'affront que cela lui faisait devant une telle assistance,
qui se saurait partout, et qui le ferait passer pour un sot et pour un
imbécile\,; puis tomba sur les compliments qu'il avait reçus de
M\textsuperscript{me} de Montauban, qui, dit-il, s'était moquée de lui
et l'avait insulté, et qui savait bien sûrement ce qui lui était
arrivé\,; et de là à l'appeler par toutes sortes de noms dans la
dernière fureur contre elle. M\textsuperscript{me} de Saint-Simon
n'oublia rien pour l'adoucir et sur son aventure et sur celle de
M\textsuperscript{me} de Montauban, en l'assurant qu'elle ne pouvait pas
savoir ce qui s'était passé au parlement, dont personne encore n'était
informé à Versailles, et que la flatterie lui avait fait dire tout ce
qu'elle ne faisait que se figurer. Rien ne prit\,: les plaintes et le
silence se succédèrent toujours parmi les larmes. Puis tout à coup se
prenant au duc de Beauvilliers et au roi, et accusant son éducation\,:
«\,Ils n'ont songé, s'écria-t-il, qu'à m'abêtir et à étouffer tout ce
que je pouvais être. J'étais cadet, je tenais tête à mon frère, ils ont
eu peur des suites, ils m'ont anéanti\,; on ne m'a rien appris qu'à
jouer et à chasser, et ils ont réussi à faire de moi un sot et une bête,
incapable de tout, et qui ne sera jamais propre à rien, et qui sera le
mépris et la risée du monde.\,» M\textsuperscript{me} de Saint-Simon en
mourait de compassion, et n'oublia rien pour lui remettre l'esprit. Cet
étrange tête-à-tête dura près de deux heures qu'il était à peu près
temps d'aller au souper du roi. Il recommença le lendemain avec moins de
violence. Peu à peu M\textsuperscript{me} de Saint-Simon le consola
quoique imparfaitement. M\textsuperscript{me} la duchesse de Berry
n'osait guère lui en rien dire, M. le duc d'Orléans beaucoup moins\,;
mais personne n'a osé depuis parler, non-seulement à lui, mais devant
lui de cette séance du parlement, ni de rien de tout ce voyage à Paris.
Le même jour, au sortir du parlement, le duc de Shrewsbury dépêcha des
courriers en Angleterre et à Utrecht qui hâtèrent très-promptement la
signature de la paix entre toutes les puissances, excepté l'empereur.

\hypertarget{chapitre-xvi.}{%
\chapter{CHAPITRE XVI.}\label{chapitre-xvi.}}

1713

~

{\textsc{L'impératrice va de Barcelone à Vienne par l'Italie fort
incognito.}} {\textsc{- Plénipotentiaires d'Espagne reçus à Utrecht.}}
{\textsc{- Orry rappelé en Espagne.}} {\textsc{- Bassesse, caractère et
fortune du duc de Bournonville.}} {\textsc{- La paix signée, publiée\,;
fêtes à Paris.}} {\textsc{- Hardie politique de M. et de
M\textsuperscript{me} du Maine.}} {\textsc{- Bailliage d'Haguenau assuré
à M. de Châtillon.}} {\textsc{- Quarante-huit mille livres
d'augmentation de pension à Madame.}} {\textsc{- Douze mille livres de
pension au duc de Charost.}} {\textsc{- Vingt mille livres de pension
assurées à M\textsuperscript{me} de Monasterol.}} {\textsc{- Fiefmarcon
lieutenant général de Roussillon.}} {\textsc{- Lueurs trompeuses sur
l'archevêque de Cambrai.}} {\textsc{- Mort de Montgaillard, évêque de
Saint-Pons.}} {\textsc{- Mort de L'Aigle\,; son caractère.}} {\textsc{-
Mort et caractère de Sévigné.}} {\textsc{- Mort, caractère et fortune du
vieux Clérembault.}} {\textsc{- Mort de la marquise de Mirepoix.}}
{\textsc{- Mort de la comtesse d'Uzes.}} {\textsc{- Mort, fortune et
caractère du cardinal de Janson.}} {\textsc{- Beauvais donné à l'abbé de
Saint-Aignan, malgré le duc de Beauvilliers.}} {\textsc{- Adresse hardie
de Rome sur ses bulles.}} {\textsc{- Naissance et mort du duc
d'Alençon.}} {\textsc{- Électeurs de Cologne et de Bavière voient le roi
plusieurs fois.}} {\textsc{- Princesse de Conti, fille du roi, achète
l'hôtel de Lorges, à Paris.}} {\textsc{- Mariage d'Aubigny avec
M\textsuperscript{lle} de Villandry.}} {\textsc{- Villars s'excuse de
servir, puis va sur le Rhin\,; Besons sur la Moselle}} {\textsc{-
Harcourt, destiné au Rhin, hors d'état de servir.}} {\textsc{- Cent
mille livres à Villars.}} {\textsc{- Départ des généraux.}} {\textsc{-
Steinbok et ses troupes prisonniers des Danois.}} {\textsc{- Châteauneuf
ambassadeur en Hollande\,; Bonac à Constantinople\,; du Luc en Suisse.}}
{\textsc{- Abbé de Mornay\,; quel, et pourquoi en Portugal.}} {\textsc{-
Lassai fils envoyé en Prusse, où il ne fut point.}} {\textsc{-
Lœwenstein évêque de Tournai.}}

~

Jennings, un des amiraux d'Angleterre, avait déjà porté l'impératrice de
Barcelone à Gênes, et on vit le moment que les Catalans s'opposeraient à
son départ à main armée. Elle traversa l'Italie avec peu de suite et
fort incognito, et gagna le plus tôt qu'elle put Inspruck, puis Vienne.
Jennings revint après faire le transport des troupes anglaises qui
depuis longtemps ne sortaient plus de leurs quartiers. Le duc d'Ossone,
sûr d'être admis à Utrecht, y était allé de Paris, et Monteléon
d'Angleterre. Orry, qui était resté à Paris depuis que le roi l'avait
fait chasser d'Espagne et avait été fort près de le faire pendre, y fut
rappelé par le crédit de M\textsuperscript{me} des Ursins. Le roi
d'Espagne en désira le consentement du roi, qui ne le voulut jamais
donner, mais qui permit qu'il partît sans son aveu, et il y retourna de
la sorte. Cette souveraineté de M\textsuperscript{me} des Ursins
accrochait la paix d'Espagne. On en verra le détail dans les
Pièces\footnote{Voir les Pièces sur la souveraineté avortée de la
  princesse des Ursins. (\emph{Note de Saint-Simon}.)} et combien le roi
le trouva mauvais. C'est ce qui fit la fortune du baron de Capres,
qu'elle envoya de sa part à Utrecht.

D'Aubigny y était déjà, qui n'y passait point les antichambres, et que
son petit état faisait mépriser. Elle crut donc qu'un cadet de
Bournonville qui avait de l'esprit, de l'entregent, de l'intrigue, qui
portait un nom distingué dans les Pays-Bas, qui y avait force parents,
et qui était un homme à tout faire pour arriver à plaire et à parvenir,
percerait et viendrait à bout de la chose du monde qu'elle passionnait
le plus démesurément. Elle y fut trompée. Capres se déshonora par une
commission si ridicule et si fort au-dessous de lui, ne put être reçu à
rien traiter à Utrecht, et y essuya tous les dégoûts possibles que sa
mission attira à sa personne, Mais pour lui, il réussit à ce qu'il
voulait, qui était de plaire à la distributrice des grâces de toutes les
sortes. M\textsuperscript{me} des Ursins lui sut si bon gré d'avoir fait
ce voyage de sa part, et de tout ce qu'il y avait essuyé pour l'amour
d'elle qu'elle ne tarda pas à l'en récompenser. Il n'avait ni grâces ni
aucun bien vaillant\,; elle le mit à son aise et lui fit donner la
Toison, bientôt après la grandesse, enfin la compagnie wallone des
gardes du corps du roi d'Espagne. J'ai pressé ces petits événements afin
de n'avoir pas à y revenir. Les Pièces, où tout ce qui regarde la paix
se trouve si bien expliqué, me dispensent d'en rien dire ici en détail.

Le vendredi saint, 14 avril, Torcy entra sur les huit heures du soir
chez M\textsuperscript{me} de Maintenon, menant au roi le chevalier de
Beringhen, aujourd'hui premier écuyer et chevalier de l'ordre, chargé
par le maréchal d'Huxelles d'apporter la nouvelle tant désirée de la
signature de la paix, faite enfin le lundi précédent 10, fort avant dans
la nuit, avec l'Angleterre, la Hollande, le Portugal, et les deux
nouveaux rois de Sicile et de Prusse\,; et, pour le dire tout de suite,
on eut les ratifications le 14 mai, et le 22 la publication de la paix
se fit dans Paris avec grande solennité\footnote{La paix et la guerre se
  publiaient dans l'ancienne monarchie avec des formes solennelles, le
  prévôt des marchands et autres officiers municipaux ou royaux allaient
  avec des archers et des hérauts d'armes en faire la proclamation dans
  les divers quartiers de Paris. On trouve la description d'une de ces
  solennités dans le \emph{Journal de l'avocat Barbier}, à la date du 12
  février 1749.}.

M. et M\textsuperscript{me} du Maine, qui songeaient fort dès lors à se
rendre populaires, vinrent de Sceaux chez le duc de Rohan voir passer la
cérémonie, dans la place Royale, s'y montrer sur {[}un{]} balcon, et y
jeter de l'argent au peuple\,; libéralité qui n'aurait pas réussi auprès
du roi à d'autres. Il y eut, le soir, beaucoup de feux devant les
maisons, et plusieurs furent illuminées. Le 25 mai, on chanta le
\emph{Te Deum} à Notre-Dame avec l'assistance ordinaire\,; le soir,
grand feu d'artifice à la Grève, qui fut suivi d'un superbe festin que
le duc de Tresmes, gouverneur de Paris, donna à ses dépens à l'hôtel de
ville aux ambassadeurs, et à grand nombre de personnes distinguées de la
cour et de la ville, des deux sexes, et les vingt-quatre violons pendant
le repas.

Ce temps sembla celui des grâces\,; on ne le négligea pas. Je me suis
trompé sur la mort du duc Mazarin. Son extrémité à son âge l'avait fait
croire\,; il n'est mort que vers la fin de cette année-ci. Ainsi, après
cette correction, je n'en parlerai plus. Il avait donné le bailliage
d'Haguenau de vingt mille livres de rente à son fils en mariage. Le peu
de cas qu'on était accoutumé depuis longues années à faire de lui, et
l'extrême mépris où la vie honteuse, scandaleuse, obscure de son fils
l'avait fait tomber, avisèrent Voisin de demander au roi ce bailliage
pour Châtillon son gendre, qui a fait depuis une si grande et si
inespérée fortune. Voysin l'obtint pour que Châtillon en jouît après la
mort du duc Mazarin, et qu'il passât après Châtillon à sa postérité
masculine. Le duc de La Meilleraye eut beau crier, la partie n'était pas
égale, mais le public fut étrangement indigné de l'audace et de
l'avidité de ce ministre, qui donna le premier exemple de la violence
d'enlever le bien par pure faveur à des personnes vivantes, en droit et
en possession de tout temps, c'est-à-dire depuis que le roi en avait pu
disposer, et cela sans ombre de droit, de dette ni de prétention
quelconque que le pouvoir et le vouloir de ravir. Il ne fut pas
longtemps sans faire passer sur la tête de M\textsuperscript{me} de La
Rochepot sa fille une pension de six mille livres qui lui avaient valu
les voyages du roi en Flandre lorsqu'il y était intendant.

Madame, qui avait peine à fournir à la dépense de son grand état avec
quatre cent mille livres de rente, demanda du secours au roi, qui, avec
excuses du peu, lui donna quarante mille livres d'augmentation.

Le duc de Charost, qui n'avait rien vaillant, et qui était entre son
père et sa mère et ses deux fils, eut en même temps douze mille livres
de pension.

Monasterol, ministre depuis fort longtemps de l'électeur de Bavière en
France, où il faisait une dépense en tout prodigieuse, avait une pension
du roi de trente mille livres. Il avait épousé par amour une des plus
belles femmes de Paris, au scandale de tout le monde, qui était veuve
d'un vieux La Chétardie, gouverneur de Thionville, frère du curé de
Saint-Sulpice, directeur de M\textsuperscript{me} de Maintenon après M.
de Chartres. Elle n'avait rien, et avait épousé ce vieillard dont elle
eut un fils, bien longtemps depuis ambassadeur en Russie où il a tant
fait parler de lui, et dont il a tant tiré d'honneurs et de biens de la
czarine. Monasterol obtint que, s'il venait à mourir, il demeurerait de
sa pension vingt mille livres de rente à sa femme.

Fiefmarcon, longtemps depuis chevalier de l'ordre en 1724, eut la
lieutenance générale du Roussillon par la mort du vieux Quinçon et la
protection des Noailles.

Il y avait eu depuis quelque temps des lueurs que les amis de
l'archevêque de Cambrai avaient avidement saisies pour se flatter.
Personne ne s'était hasardé de prononcer son nom devant le roi, même
lorsque du vivant du Dauphin les gens de la cour qui servaient en
Flandre s'empressaient le plus de lui faire la leur en passant et
repassant, et se détournaient même exprès. Il en avait si magnifiquement
usé pour les troupes et pour leurs officiers de toutes conditions
pendant toute la guerre, et encore à la dernière campagne, que Maréchal
en avait parlé devant le roi plus d'une fois, et presque toutes les fois
le roi y avait pris courtement, mais assez bien. J'en avais averti le
duc de Chevreuse, qui vivait encore, et le duc de Beauvilliers, qui en
furent touchés d'une joie d'autant plus sensible, qu'ils étaient depuis
bien longtemps hors de toute espérance à son égard. Ratabon, évêque
d'Ypres, ne bougeait guère de Paris, et prétendait qu'il y avait une
vapeur dans sa cathédrale qui le faisait évanouir chaque fois qu'il y
entrait. C'était un homme d'esprit, du monde, et qui était si bien avec
les jésuites que ce pouvaient être les cendres de Jansénius, son célèbre
prédécesseur, qui opéraient cet effet sur lui. On lui donna l'évêché de
Viviers, et le P. Tellier, qui était tout à M. de Cambrai, sans oser le
montrer, et dont le crédit croissait sans cesse, fit un tour de force et
bombarda cet évéché d'Ypres pour l'abbé de Laval, grand vicaire de M. de
Cambrai, qui l'avait élevé tout jeune, et l'avait toujours nourri et
entretenu généreusement chez lui, parce qu'il était un peu son parent,
et que cette branche très-cadette de Laval-Montigny avait à peine du
pain. Cet abbé de Laval avait extrêmement profité d'une générosité si
bien placée\,; il était savant, fort homme de bien, s'était beaucoup
fait aimer. Il n'avait jamais quitté l'archevêque, qu'il aimait et
respectait comme son père, et dont il était chéri de même. Cet
attachement était l'exclusion la plus formelle\,: aussi personne ne
pensait à rien pour lui lorsque le P. Tellier fit de lui-même ce grand
coup qui releva tout à fait les espérances sur l'archevêque même, et qui
ravit M. de Beauvilliers. On verra que les suites en furent trompeuses.
Le pauvre abbé de Laval mourut à Ypres peu de mois après avoir été
sacré. L'école d'où il sortait était fort opposée à celle de Jansénius,
sûrement au moins pour ce monde\,; cette mort précipitée fut-elle un
coup de Jansénius\,? L'abbé de Laval fut le dernier évêque d'Ypres de la
nomination du roi qui la perdit avec cette place par l'exécution de la
paix.

Un saint et grand évêque mourut en ce temps-ci, Montgaillard, évêque de
Saint-Pons, que ses vertus épiscopales, son grand savoir, une constante
résidence de plus de quarante années, une vie tout apostolique, une
patience humble, courageuse, prudente, invincible avaient singulièrement
illustré sous la persécution des jésuites qui y engagèrent le roi
pendant presque tout son épiscopat.

Je regrettai un de mes voisins de la Ferté, le mari de
M\textsuperscript{me} de L'Aigle, dame d'honneur de
M\textsuperscript{me} la Duchesse, tous deux fort des amis de mon père
et des miens. Je n'ai guère connu un couple d'autant d'esprit, de
politesse, mieux instruit de tout et plus capable d'amitié. M. de
L'Aigle, accablé d'infirmités, s'était retiré depuis plusieurs années
chez lui à l'Aigle, d'où il ne sortait plus. C'est un des plus beaux et
des plus complets marquisats qu'il y ait en France, à six lieues de chez
moi. Il y mourut à soixante-quinze ans, tout à lui, n'ayant jamais rien
perdu de sa tête ni des agréments de sa conversation.

Sévigné mourut aussi et sans enfants, retiré depuis quelque temps avec
sa femme dans le faubourg Saint-Jacques, dans une grande piété. Il était
fils de M\textsuperscript{me} de Sévigné, si connue encore par ses
lettres. Elle l'avait fort mis dans le monde et dans la meilleure
compagnie. C'était un bon et honnête homme, mais moins un homme d'esprit
que d'après un esprit, qui avait eu des aventures bizarres, peu mais
bien servi, et qui du naturel charmant et abondant de sa mère et du
précieux guindé et pointu de sa sœur, avait fait un mélange un peu
gauche.

M. de Luxembourg perdit sans aucun regret son beau-père Clérembault,
qu'on n'appelait que Clérembault la Perruque, parce qu'il était accusé
d'acheter les siennes sur les quais\,; au moins en avaient-elles toute
la mine. Il s'appelait Gillier, était peu de chose, et beaucoup moins
encore par son personnel. Il avait été bien fait et parfaitement beau.
On le voyait encore à plus de cent ans qu'il avait bien comptés, un
vieux bellâtre qui, jusqu'à cet âge, et au delà, venait toutes les
semaines ennuyer la cour, où jamais il n'avait été de rien. Il avait été
maître d'hôtel de M\textsuperscript{me} Henriette d'Angleterre,
lorsqu'elle épousa Monsieur. Le maréchal du Plessis n'avait pu refuser à
la reine mère d'être gouverneur de Monsieur. Il était demeuré
surintendant de sa maison et premier gentilhomme de sa chambre. Il
mourut, duc et pair de 1665, à la fin de 1675. Le comte du Plessis, son
fils, était premier gentilhomme de la chambre de Monsieur, en
survivance. II avait épousé en 1659 Marie-Louise Le Loup de Bellenave,
qui fut dame d'honneur de Madame en survivance de la maréchale du
Plessis, dont un fils unique tué devant Luxembourg à vingt ans, sans
alliance, en mai 1684, par quoi le chevalier du Plessis, frère puîné de
son père, devint duc et pair de Choiseul, en qui cette dignité s'est
éteinte. Le comte du Plessis, son frère aîné, fut tué à la prise
d'Arnheim en Hollande, à trente-huit ans, en 1672, et mourut ainsi
devant son père. Sa veuve s'amouracha de Clérembault qu'elle voyait tous
les jours chez Madame, et l'épousa. C'était un second mariage bien
infime en comparaison du premier, et de la dame d'honneur de Madame avec
un de ses maîtres d'hôtel. Cette Madame n'était plus Henriette
d'Angleterre. Elle était morte le 30 juin 1670\,; et Monsieur était
remarié, dès la fin de 1672, à la fille de l'électeur palatin, à qui la
coutume constante de l'Allemagne rendait la mésalliance plus étrange,
car la comtesse du Plessis avait passé de la première Madame à elle. On
trouva donc moyen de faire Clérembault son premier écuyer pour rendre ce
mariage moins insupportable, et on lui fit acheter encore le petit
gouvernement de Toul. Il était riche, sa femme encore plus\,; la mort du
duc de Choiseul, fils unique de son premier lit, la mit encore dans une
plus grande abondance. L'un et l'autre avaient quitté Madame. Ils
étaient extrêmement avares, et amassèrent de grands biens, dont la
duchesse de Luxembourg leur fille unique, morte devant sa mère, a fait
passer à son fils, le duc de Luxembourg d'aujourd'hui.
M\textsuperscript{me} de Clérembault est morte en 1724 à
quatre-vingt-quatre ans. Elle avait beaucoup d'esprit, et un reste de
considération. Elle et son mari étaient plus avares l'un que l'autre.

La marquise de Mirepoix mourut en même temps assez jeune. Elle était
fille aînée du duc et de la duchesse de La Ferté, et veuve de Mirepoix,
sous-lieutenant des mousquetaires, sans enfants, qui était frère aîné du
père du marquis de Mirepoix, aujourd'hui chevalier de l'ordre, aîné de
la maison de Lévi. M\textsuperscript{me} de Mirepoix tenait assez de
choses de sa mère. Elle s'était ruinée, et vivait assez esseulée dans le
couvent de la Conception, à Paris.

La comtesse d'Uzès mourut aussi en couches. Elle était fille du
lieutenant de roi de Condé, qui était brigadier, et veuve d'un financier
appelé Hamelin. C'était une grande femme qui avait été belle et bien
faite, qui n'avait pas quarante ans, à qui M. Chamillart avait voulu du
bien, que j'ai fort vue à l'Étang, où elle se faisait aimer de tout le
monde. Elle a laissé trois fils du comte d'Uzès, frère du duc d'Uzès,
qui n'avait rien.

L'État et la religion firent une grande perte en la personne du cardinal
de Janson, évêque, comte de Beauvais, et grand aumônier de France, qui
mourut à Paris, 24 mars de cette année, à quatre-vingt-trois ans, ayant
toujours la tête parfaitement entière. Le roi le regretta beaucoup, le
public aussi, et son diocèse et les pauvres amèrement. Ce sont de ces
hommes rares et illustres qui méritent de s'y arrêter\,; et je le ferai
d'autant plus volontiers qu'entre beaucoup d'amis qu'il eut toute sa
vie, il l'était très-particulier de mon père, et fort des miens. Il fut
un moment coadjuteur de Digne, puis évêque de Marseille, où il fut
chargé de toutes les affaires de Provence, au grand regret du comte de
Grignan, lieutenant général de la province, comme on le voit par les
lettres de M\textsuperscript{me} de Sévigné. Ces affaires firent
connaître sa capacité aux ministres.

Forbin, son parent éloigné, mais de même nom, mort capitaine des
mousquetaires gris, était dès lors bien avec le roi, et fort ami de
Bontems qui le devint de l'évéque de Marseille, et qui le servit
très-bien auprès du roi toute sa vie. Il y avait déjà sept ou huit ans
qu'il gouvernait toutes les affaires de Provence, lorsqu'il fut envoyé
ambassadeur en Pologne en 1674, à l'occasion de l'élection d'un roi. Son
habileté y réunit tous les partis lorsqu'on s'y attendait le moins. Le
fameux Jean Sobieski, grand maréchal et gouverneur général de la
couronne, fut unanimement proclamé. La reconnaissance lui fit offrir sa
nomination au cardinalat à l'évêque de Marseille, qui ne voulut
l'accepter qu'après en avoir obtenu la permission du roi. Peu après son
retour, il fut en 1679 transféré à Beauvais, et renvoyé un an après
ambassadeur en Pologne, et vers divers princes d'Allemagne. En 1630, il
eut l'ordre du Saint-Esprit, et le 13 février 1690, Alexandre VIII,
Ottobon, le fit cardinal. Ce pape, que le duc de Chaulnes avait mis sur
le saint-siége, avait trompé la France. À sa mort nos cardinaux allèrent
à Rome. Janson y contribua beaucoup à l'élection d'Innocent XII,
Pignatelli, l'un des plus sages, des meilleurs et des plus saints papes
qui eussent occupé le saint-siége depuis bien longtemps. Janson demeura
à Rome, chargé des affaires de France, et y termina tous les démêlés
qu'elle avait eus sous les deux derniers pontificats. Après sept années
de résidence à Rome, il revint en France. Deux ans après, la mort
d'Innocent XII l'y fit retourner pour le conclave, avec les autres
cardinaux français. Clément XI, Albane, y fut élu, et Janson demeura
encore auprès de lui, chargé des affaires de France, jusqu'en 1706,
qu'il apprit par le même courrier du roi la mort du cardinal de Coislin,
et qu'il était grand aumônier en sa place, avec la permission de revenir
l'exercer. II partit bientôt après de Rome, qu'il ne revit plus.

Le cardinal de Janson était un fort grand homme, bien fait, d'un visage
qui, sans rien de choquant ou de singulier, n'était pourtant pas
agréable, et avait quelque chose de pensif sans beaucoup promettre. Il
était plein d'honneur et de vertu, il avait un grand amour de ses
devoirs et de la piété. C'était une sage et excellente tête, se
possédant toujours parfaitement, et qui par là a réussi en perfection
dans toutes ses négociations, et a mieux servi le roi à Rome qu'aucun
autre qui y ait été chargé de ses affaires. Il y était plus craint et
plus considéré que pas un d'eux, parce que, avec une parole lente et
désagréable par l'organe, qui avait un son étranglé, il avait une
sagacité qui ajoutait beaucoup à la finesse de son esprit et à sa
justesse, qui était grande, en sorte qu'il n'a jamais pu être trompé,
même à Rome. Il était consommé dans les affaires par une longue
habitude, magnifique en tout et partout avec beaucoup d'ordre, fort
désintéressé, affable aux plus petits, naturellement obligeant, fort
poli, mais avec choix et dignité, quoiqu'il le fût à tout le monde, et
l'homme du monde le plus capable d'amitié, de fidélité à ses amis et de
les bien servir. Il était né pauvre. Son frère aîné et le père du
marquis de L'Aigle, de la mort duquel je viens de parler, avaient épousé
les deux filles du bonhomme La Saladie, qui avait été autrefois fort
estimé et fort avancé à la guerre. La chapelle du château de l'Aigle
vaut huit cents livres de rente fondée au chapelain. Ce fut le premier
bénéfice qu'il eut, et que par reconnaissance il a voulu garder toute sa
vie. Il y payait un chapelain, et faisait donner le reste aux pauvres du
lieu depuis qu'il fut devenu grand seigneur. Étant cardinal et grand
aumônier, il se plaisait à dire, devant tout le monde, à M. et à
M\textsuperscript{me} de L'Aigle, qu'il était le grand aumônier du roi
et le leur, et qu'il se faisait honneur de demeurer le leur, parce
qu'alors qu'il n'avait rien il s'était trouvé bien heureux que leur père
lui eût donné de quoi vivre par cette chapelle.

Il avait l'âme et toutes les manières d'un grand seigneur, doux et
modeste, l'esprit d'un grand ministre né pour les affaires, le cœur d'un
excellent évêque, point cardinal, au-dessus de sa dignité, tout français
sur nos libertés et nos maximes du royaume, sur les entreprises de Rome,
avec netteté, inébranlable là-dessus jusqu'à l'éclat, et parfaitement
instruit de ces matières jusqu'à avoir dit plus d'une fois aux ministres
romains, et au pape même, que, quelque flatté qu'il fût de sa pourpre,
il se tenait plus honoré de l'épiscopat que du cardinalat, et que son
chapeau ne lui tenait à rien. Cette fermeté constante et vraie a souvent
eu de grands effets. Tout bon courtisan qu'il était, il fut aussi peu
timide au dedans qu'au dehors, et aussi impénétrable au crédit et aux
artifices des jésuites, dont il ne s'émut jamais et qu'il contint
toujours en crainte et en respect, comme on l'a vu. On a vu aussi
combien le roi regretta de ne pouvoir le mettre dans son conseil, et les
excellentes raisons qui l'en détournèrent, et que la France pleurera
longtemps avec des larmes de sang n'avoir pas été suivies après lui.

Quelque accoutumé qu'il fût aux affaires, quelques agréments qu'il
trouvât dans le monde, où il était universellement honoré et où il avait
beaucoup d'amis, parce qu'il en méritait, quelques faveurs, quelques
distinctions qu'il trouvât toujours à la cour, il ne se plaisait nulle
part tant que dans son diocèse, où il était singulièrement respecté, et
il se peut dire adoré, surtout des pauvres de tous les états à qui il
faisait de grandes aumônes. Il aidait et soutenait fort la noblesse\,;
et tant qu'il a été en France il a toujours passé plus de sept ou huit
mois tous les ans à Beauvais à y visiter son diocèse, et à y remplir
toutes ses fonctions avec beaucoup d'application et de vigilance. Le roi
donna l'archevêché d'Arles à son neveu, l'abbé de Janson, lors de la
translation de M. de Mailly, longtemps depuis cardinal, d'Arles à Reims.
Le cardinal de Janson s'y opposa tant qu'il put. Il dit au roi qu'il
connaissoit son neveu, que c'était un petit génie, fort homme de bien,
mais à qui il ne voudrait pas confier une place de vicaire de village,
et absolument incapable de l'épiscopat\,; que, si le roi voulait lui
faire du bien, il lui serait très-obligé et très-aise s'il lui voulait
donner une abbaye de dix-huit ou vingt mille livres de rente, que ce
serait de quoi vivre et prier Dieu en repos, et beaucoup plus qu'il n'en
fallait à son neveu. Il eut beau insister, le roi tint bon. On a
longuement vu depuis combien le cardinal pensait juste. Sa mort arriva
dans une funeste époque. Avec la liberté et la fermeté qu'il avait, et
la confiance du roi telle qu'il la possédait, il eût pu empêcher ce
torrent de maux qui la suivirent dans l'Église, et qui n'épargnèrent pas
l'État\,; et son funeste successeur n'aurait pas acheté sa charge, comme
il fit enfin du P. Tellier, et par elle n'eût pas eu les accès dont il
fit pour la payer un si pernicieux usage, comme on l'éprouva bientôt
après.

Au bout de quinze jours, le roi donna les deux belles abbayes qu'il
avait\,: Marchiennes, en Flandre, au cardinal Ottobon\,; Corbie, de
cinquante mille livres de rente, au cardinal de Polignac. Il nomma en
même temps à Beauvais l'abbé de Saint-Aignan, qui était encore à Orléans
au séminaire. Le duc de Beauvilliers représenta au roi que, encore qu'il
parût que son frère eût de la piété et de l'application aux choses de
son état, il était encore trop jeune pour être aussi assuré de lui qu'il
convenait de l'être pour le faire évêque. Il n'y eut rien qu'il
n'employât pour faire changer le roi là-dessus, avant que la nomination
fût sue. Le roi fut inflexible, loua la délicatesse de M. de
Beauvilliers, s'appuya sur tout le bien qui lui était revenu de son
frère, ajouta que Beauvais ne vaquait pas toujours, et à point, et qu'il
voulait bien lui dire que, s'il était encore d'usage, comme dans les
anciens temps, que des fils de France fussent évêques, il n'aurait rien
de mieux à donner à son second fils que Beauvais. Le pape lui refusa des
bulles, parce que l'abbé de Saint-Aignan avait, par ordre du roi,
soutenu dans ses thèses les propositions de l'assemblée du clergé de
1682.

Ce n'était pas que Rome fût en droit ni même en volonté de ce refus,
mais pour montrer, par cette difficulté faite au frère d'un ministre de
cette distinction, à quoi devaient s'attendre tous les autres, effrayer
la cour et faire perdre ainsi l'habitude de soutenir ces maximes, qui
était déjà fort tombée en désuétude, et qui y tomba après de plus en
plus. Il avait été réglé qu'elles le seraient par tous ceux qui auraient
à prendre des degrés, et que le parlement y tiendrait la main. Cela se
fit pendant quelque temps, puis on s'en relâcha à la française, et sous
Alexandre VIII, Ottobon, le clergé sembla les abandonner, par la lettre
honteuse que le roi l'engagea d'écrire à ce pape pour obtenir des bulles
qu'Innocent XI avait refusées, et qu'on sollicitait depuis quatorze ans.
Depuis cette époque ces propositions ne furent plus soutenues qu'à la
dérobée, et par des bouffées de mécontentement de la cour de Rome, qui
sut profiter de tous les avantages qu'on lui laissait prendre pour les
anéantir, et qui a su depuis se saisir de bien d'autres, et se mettre en
beau chemin de réduire la France au point d'ignorance, d'adoration et de
dépendance où elle a réduit l'Italie et les Espagnes. Le refus dura six
mois entiers. Contente alors d'avoir fait un exemple si humiliant et si
instructif, et n'osant aussi trop se commettre, les bulles furent
accordées par bonté, avec le \emph{gratis} ordinaire aux fils et aux
frères des ministres. L'abbé de Saint-Aignan parut en parfait
séminariste. Jamais rien de si gauche, de si plat, de si béat. Je
proposai au duc de Beauvilliers de lui donner un maître à danser, pour
lui apprendre au moins à faire la révérence et à entrer dans une
chambre. Il afficha la régularité la plus exacte, et il remit
Saint-Germer près Beauvais, la seule abbaye qu'il eût, pour n'être pas
en pluralité de bénéfices. On la donna à l'abbé Begon, depuis évêque de
Toul, parent proche des Colbert, qui fut choisi pour être le conducteur
du jeune prélat, sous le nom de grand vicaire. M. de Beauvilliers ni le
roi ne vécurent pas assez pour voir combien il y avait eu de sagesse et
de raison dans les craintes et les refus du duc de Beauvilliers de faire
son frère évêque si promptement, que ses désordres éclatants et
persévérants firent enfin renfermer dans un monastère pour le reste de
ses jours, presque gardé à vue, et forcément démis de son évêché pour
éviter la dégradation et la déposition juridique.

M\textsuperscript{me} la duchesse de Berry accoucha, sur les quatre
heures du matin du dimanche 26 mars, d'un prince qui fut appelé duc
d'Alençon. Il vint à sept mois, et la flatterie fut telle que presque
toute la cour se trouva née ou avoir des enfants à ce terme. La joie en
fut courte\,; il donna plusieurs alarmes par sa délicatesse, et il
mourut le samedi 25 avril à minuit. Le roi nomma le duc de Saint-Aignan
et le marquis de Pompadour pour accompagner la corps à Saint-Denis. Il
partit de Versailles le lundi 27 avril après dîner, avec les gardes, les
pages, et les carrosses de M. le duc de Berry\,; l'évêque de Séez
portant le cœur, eut pour cette raison la première place, et M. de
Saint-Aignan la seconde, au derrière du carrosse, comme duc\,; M. et
M\textsuperscript{me} de Pompadour au devant, elle comme gouvernante\,;
et le petit corps posé entre eux. Lorsqu'ils eurent passé les cours, et
un peu avancé dans l'avenue, M. de Saint-Aignan força par politesse
M\textsuperscript{me} de Pompadour de changer de place avec lui. De
Saint-Denis ils furent porter le cœur au Val-de-Grâce. M. {[}le duc{]}
et M\textsuperscript{me} la duchesse de Berry furent extrêmement
touchés.

L'électeur de Bavière qui était toujours à Suresne, et qui s'y amusait à
chasser dans la forêt de Saint-Germain et ailleurs, à des retours de
chasse chez lui, à un gros jeu, et à donner des fêtes champêtres à
l'occasion de la paix, qui n'était pourtant pas encore bien agréable
pour lui, dîna le 21 avril chez d'Antin, à Versailles, vit le roi après
dans son cabinet par les derrières, y fut peu, le suivit à la volerie,
et s'en retourna le soir à Suresne. L'électeur de Cologne vit le roi le
lendemain de la même façon, et fut longtemps avec lui. Huit ou dix jours
après, le roi étant à Marly et courant le cerf, l'électeur de Bavière se
trouva à la chasse, et descendit après à Marly, chez d'Antin. Il fut
jouer au salon où M. le duc de Berry l'attendit\,; il revint souper chez
d'Antin, puis jouer au salon jusqu'à quatre heures du matin, et s'en
alla à Suresne. Deux jours après, l'électeur de Cologne vint
l'après-dînée à Marly, vit le roi dans son cabinet, et prit congé de
lui. Le lendemain, l'électeur de Bavière se trouva comme l'autre fois à
la chasse du roi, joua au retour dans le salon avec Madame et
M\textsuperscript{me} la duchesse de Berry et force dames, soupa chez
d'Antin, et retourna au salon après. Le roi fit pour lui une chose
singulière\,; il vint voir jouer, et jeta de l'argent à l'électeur pour
être des réjouissances. Il n'y fut pas longtemps, mais cela fut fort
marqué. Le jeu se poussa assez loin, après lequel l'électeur regagna
Suresne. Quelques jours après il revint encore à la chasse, soupa chez
d'Antin, et joua dans le salon avant et après souper. Il se trouva
bientôt après à une autre chasse. Le roi se promena après dans un
jardin, où l'électeur le vint joindre aussitôt au mail\,; ils y virent
jouer, et la promenade continua ensuite, l'électeur à pied avec les
courtisans, et le roi dans son petit chariot qui lui en fit une
civilité. Après la promenade, l'électeur joua dans le salon à
l'ordinaire avant et après le souper que d'Antin lui donna. Il revint
encore après faire une autre chasse et jouer dans le salon, et revint
aussitôt après voir aller les dames à la roulette, qui est un
divertissement qu'il ne connaissoit point\,; mais ces dernières fois il
ne vit le roi qu'à la chasse. Il ne parut plus que pour prendre congé du
roi à Versailles, qu'il vit peu de temps dans son cabinet, pour s'en
aller à Compiègne. Ce fut en ce temps-ci que M\textsuperscript{me} la
princesse de Conti, fille du roi, acheta à vie l'hôtel de Lorges du duc
de Lorges, qui vendait tout d'un côté, et bâtissait et dépensait tant
qu'il pouvait de l'autre. Cette acquisition, à la suite de celle du
comte de Toulouse et de d'Antin, augmenta la surprise. Le roi en aurait
été si choqué dans d'autres temps qu'ils n'auraient osé le hasarder\,;
mais il commençait à être si dégoûté de tout, par les malheurs de sa
famille, qu'il ne prenait presque plus de part à rien que celle qu'on
l'engageait à prendre. Ces précautions d'établissements à Paris de gens
qui ne pouvaient découcher de la cour, excepté d'Antin, et encore
celui-là avec mesure, permission et prétexte, donnèrent fort à penser
sur la santé du roi, de la décadence de laquelle on ne s'apercevait
pourtant pas encore au dehors de son plus secret intérieur. Quelque
temps après M\textsuperscript{me} la princesse de Conti acheva
d'acquérir cette maison en propriété.

L'ombre de M\textsuperscript{me} de Maintenon qui couvrait et avait été
si utile à d'Aubigny, son prétendu cousin, et à l'archevêque de Rouen,
son oncle, fit son mariage avec M\textsuperscript{lle} de Villandry,
riche héritière, et dans son voisinage.

L'opiniâtreté de l'empereur, qui retint l'empire dans ses intérêts, fit
porter toutes nos forces sur le Rhin et sur la Moselle. Villars fut
destiné à la Moselle, et Harcourt pour le Rhin. Bientôt après Villars
s'excusa sur sa blessure, et voulut aller à Baréges\,; Besons lui fut
substitué, et le 12 et le 15 mai furent fixés pour le départ des
généraux en chef des deux armées\,; mais une nouvelle attaque
d'apoplexie mit le maréchal d'Harcourt hors d'état de servir, et il
abdiqua de lui-même. Cela changea le voyage de Baréges\,; le maréchal de
Villars accepta l'armée du Rhin. Le roi lui donna cent mille francs pour
refaire son équipage dont il s'était défait, comptant ne point servir.
Il partit aussitôt après, Besons aussi.

On apprit que Steinbok n'avait pu se soutenir davantage au milieu de
tant d'ennemis, dans des pays contraires, éloignés de la Suède, où il
n'avait pu repasser. Son armée était réduite à huit ou dix mille hommes,
enfermée et affamée de toutes parts, en sorte qu'il fut réduit à se
rendre prisonnier de guerre avec elle, moyennant passage en sûreté dans
le pays de Schonen, en payant leur rançon, que le roi de Danemark
promit, et eux de ne point porter les armes d'un an.

Le roi choisit pour l'ambassade d'Hollande Châteauneuf-Castaignières,
conseiller au parlement, qui s'était fort bien acquitté du même emploi
en Portugal et à Constantinople, et dont on s'était servi dans un
intervalle en Espagne sans caractère. Bonac, qui y était avec caractère
d'envoyé, et qui en revenait parce que M. de Brancas y allait
ambassadeur, fut nommé à l'ambassade de Constantinople\,; le comte du
Luc à celle de Suisse\,; et l'abbé de Mornay à celle de Portugal. Il
était fils de M. et M\textsuperscript{me} de Montchevreuil, et néanmoins
il n'avait jamais pu être évêque. Il était fort bien fait, et avait du
mérite, de l'esprit, du monde, du savoir\,; mais le roi, qui s'était
persuadé qu'il avait fait plus d'usage de ses talents corporels que des
autres, n'avait jamais pu en revenir. Il n'était plus fort jeune\,; le
roi crut le désembourber par les emplois étrangers, où en effet il
réussit fort bien. Lassai fils fut destiné pour la Prusse. Il n'y alla
point\,; on verra qu'il fit mieux.

Le comte de Lœwenstein, avec un fort beau visage et bien fait, fut plus
heureux avec moins de contrainte\,; mais il était Allemand et frère et
de M\textsuperscript{me} de Dangeau, le même qu'on a vu naguère député
du chapitre de Strasbourg au roi, pour l'adoucissement des preuves. Il
n'avait aucuns ordres. Il reçut en ce temps-ci les bulles de l'évêché de
Tournai, que M. de Beauvau venait de quitter pour n'être point sous une
domination étrangère\,; et, avec Tournai, il eut permission du pape de
retenir le grand doyenné de Strasbourg, et ses canonicats de Strasbourg
et de Cologne, outre les deux abbayes qu'il avait en France.

\hypertarget{chapitre-xvii.}{%
\chapter{CHAPITRE XVII.}\label{chapitre-xvii.}}

1713

~

{\textsc{Menées sourdes et profondes du P. Tellier et de Bissy, évêque
de Meaux.}} {\textsc{- Voysin substitué à Torcy pour les .affaires du
cardinal de Noailles.}} {\textsc{- Bissy nommé au cardinalat.}}
{\textsc{- Projet énorme du P. Tellier.}} {\textsc{- L'affaire du
cardinal de Noailles portée à Rome.}} {\textsc{- P. Daubenton et
Fabroni\,; quels.}} {\textsc{- Ils dressent seuls, et en secret, la
constitution \emph{Unigenitus}.}} {\textsc{- Le pape engagé de parole
positive à ne donner sa constitution que de concert et approuvée du
cardinal de La Trémoille en particulier, et du sacré collége en
général.}} {\textsc{- Audacieuse visite du P. Tellier au cardinal de
Rohan.}} {\textsc{- Caractère du cardinal de Rohan\,; son éducation.}}
{\textsc{- Il doit tout au cardinal de Noailles.}} {\textsc{- Priviléges
de la vie des cardinaux.}} {\textsc{- Combat intérieur du cardinal de
Rohan.}} {\textsc{- Tallard entraîne le cardinal de Rohan au P.
Tellier.}} {\textsc{- Cardinal de Rohan grand aumônier.}} {\textsc{-
Cardinal de Polignac maître de la chapelle du roi.}} {\textsc{- Orgueil
de son serment.}} {\textsc{- Il reçoit le bonnet de la main du roi\,; il
le harangue à la tête de l'Académie française sur la paix.}} {\textsc{-
Vittement recteur de l'Université\,; sa belle harangue et son
très-singulier effet.}}

~

Le P. Tellier avançait à grands pas vers le but qu'il s'était proposé
toute sa vie, pour lequel il avait travaillé sans cesse dans l'obscurité
du cabinet, et sa place et le crédit prodigieux qu'il y avait acquis le
mettaient en état de tout oser pour y arriver. On a vu le caractère
terrible de ce jésuite\,; les conjonctures lui étaient les plus
favorables pour le grand projet qu'il avait formé. Il avait affaire à un
prince qui, de son aveu même, était de la plus profonde ignorance, élevé
par la reine sa mère dans l'opinion que ce qu'on appelait jansénistes
était un parti républicain dans l'Église et dans l'État, ennemis de son
autorité qui était son idole, inaccessible toute sa vie à tout ce qui
n'était pas entièrement dévoué au parti opposé, accoutumé par les idées
ultramontaines de la reine sa mère, et du cardinal Mazarin, à tout céder
à la cour de Rome, et à déployer son autorité sur les parlements pour
les y faire fléchir\,; à exiler, même à emprisonner les particuliers qui
par de savants écrits blessaient Rome en s'élevant contre ses
usurpations sur l'Église et sur les couronnes\,; soigneusement entretenu
dans cet esprit par ses confesseurs toujours jésuites, et par
M\textsuperscript{me} de Maintenon, gouvernée depuis si longtemps par le
même esprit, qui était celui de M. de Chartres, son ancien directeur de
toute confiance et de tout Saint-Sulpice, à qui M. de Chartres l'avait
comme léguée en mourant, entre les mains du curé La Chétardie, et de
Bissy, évêque de Toul, puis de Meaux, qui, par le voisinage si proche de
ce dernier diocèse, ne la perdait presque pas de vue.

Bissy, dont l'âme était forcenée d'ambition, sous le pharisaïque
extérieur d'un plat séminariste de Saint-Sulpice, était de tout temps
abandonné aux jésuites comme à ceux dont il attendait tout pour sa
fortune, et sans lesquels il sentait qu'il ne pouvait rien se promettre
par lui-même, sans famille, sans amis, sans accès, et relégué à Toul, où
il n'était pas même du clergé de France. On a vu en son temps combien il
y exerça la patience de M. de Lorraine, pour se faire transférer
ailleurs par ses cris\,; l'usage qu'il en sut faire à Rome, où il
entretint un agent exprès pour se débrouiller un chemin au cardinalat,
appuyé des jésuites\,; et comme il ne voulut point de Bordeaux, trop
éloigné de la cour, quand il s'y vit si bien produit par M. de Chartres,
et que ses affaires à Rome par rapport à la Lorraine et à ses espérances
prenaient un tour à ne lui plus faire regarder Toul comme un cul-de-sac,
et à ne lui plus permettre de quitter cet évêché que pour quelque autre
qui favorisât encore mieux ses espérances, tel que fut Meaux.

Il était trop initié pour ignorer l'aversion de M\textsuperscript{me} de
Maintenon et même de Saint-Sulpice pour les jésuites\,; il était aussi
trop habile pour se refroidir avec des amis immortels, et d'une
puissance permanente, pour épouser la fantaisie d'une femme qui, à son
âge, pouvait manquer à tous moments, et d'une troupe de barbes sales,
qui sans elle n'avait point de consistance, et que les jésuites tôt ou
tard crosseraient avec le pied.

Il cacha donc à M\textsuperscript{me} de Maintenon, qui, par la
mécanique de ses journées, ne voyait le jour que par le trou d'une
bouteille, et qui était la plus grande dupe du monde de ceux pour qui
elle se prévenait, il lui cacha, dis-je, son union ancienne et la plus
intime avec les jésuites comme tels, et ne lui laissa voir de liaison
entre lui et le P. Tellier, que par la nécessité du concert pour la
bonne cause, pour l'Église, pour la pureté de la doctrine, c'était à
dire contre le cardinal de Noailles\,; et il lui en faisait d'autant
mieux sa cour, que M\textsuperscript{me} de Maintenon, peu à peu tombée
dans le dernier emportement sur cette affaire, était bien aise d'être
informée des démarches du P. Tellier auprès du roi, pour agir de concert
et en conséquence, de croire même les diriger sans toutefois vouloir ni
voir ni ouïr parler du P. Tellier, ni qu'il sût rien qu'en gros, et pour
la nécessité seulement par rapport à elle et sans elle\,; et c'est ce
qu'elle croyait faire par Bissy, sans s'être jamais doutée qu'ils ne
fussent tous deux qu'un cœur et qu'une âme, ni qu'il fût livré aux
jésuites.

D'autre part, le P. Tellier faisait faire tout ce qu'il voulait par
M\textsuperscript{me} de Maintenon auprès du roi sur cette affaire, par
le même Bissy, sans y paraître. Par ces manéges obscurs ils conduisirent
où ils voulurent un roi enfermé à cet égard sous leur clef, et qui pour
ministre de tout ce qui regardait cette affaire, n'avait plus Torcy
qu'ils avaient rendu suspect par son alliance avec les Arnauld, et par
l'évêque de Montpellier son frère. Ils lui avaient substitué Voysin,
créature et âme damnée de M\textsuperscript{me} de Maintenon et de sa
fortune, et aussi ignorant d'ailleurs et aussi vendu qu'il le leur
fallait. De cet antre de ténébreuse intrigue, sortit la nomination de
Bissy au cardinalat, que sans concert, mais avec une ardeur égale,
M\textsuperscript{me} de Maintenon et le P. Tellier procurèrent
également, et que Rome reçut avidement, comme de celui dont elle ferait
le plus grand usage, et qui pour elle foulerait tout aux pieds. Ce fut
un grand pas pour le P. Tellier, dont il se promit toutes choses, mais
il en voulait tant opérer à la fois, qu'il crut avoir besoin d'un
renfort de secours.

Le premier plan sur lequel il avait travaillé n'avait été, comme on l'a
dit, que pour donner des morailles au pape, et lui donner des affaires
en France qui le forçassent de ménager les jésuites et d'abandonner
leurs affaires des cérémonies chinoises, dès lors réduites pour eux à un
état désespéré. La double vue était de se venger du cardinal de
Noailles, monté sans eux sur le siége de la capitale, et dont la faveur
et l'estime balançait leur pouvoir sur la distribution des bénéfices.
Parvenus à lui soustraire grand nombre d'adhérents pour avoir reconnu sa
faiblesse, et l'avoir manifestée au monde, par le consentement que le
roi lui arracha pour la radicale destruction de Port-Royal des Champs,
et bientôt après à le brouiller avec M\textsuperscript{me} de Maintenon,
jusqu'à la rendre sa plus ardente ennemie, et de là avec le roi, sur les
\emph{Reflexions morales} du P. Quesnel, Tellier se promit toutes choses
de l'affadissement du sel de la terre, qu'il reconnut en plein dans les
assemblées des évêques sur cette affaire. L'interdiction générale de la
chaire et du confessionnal de tous les jésuites du diocèse de Paris,
excepté du confesseur unique du roi, et pour le roi tout seul, combla la
mesure du désir de la plus éclatante vengeance dans les jésuites et dans
le P. Tellier, et la déplorable conduite du cardinal de Noailles qui,
dans la suite, se sépara de ses évêques, de son chapitre, des écoles, et
des corps des curés et des congrégations régulières qui étaient toute sa
force au dedans et tout son appui au dehors, porta les vues du P.
Tellier au plus haut point de ses désirs. Tout ce qu'il voulait était de
mettre un tel trouble et une telle division dans cette affaire, qu'on
fût obligé de la porter à Rome contre toutes les lois de l'Église, tout
usage et toute raison, qui veulent que les contestations soient
nettement jugées, et juridiquement, dans les lieux où elles naissent,
sauf l'appel au pape qui, par ses légats envoyés sur les lieux, revoit
et réforme le premier jugement, ou le confirme d'une manière aussi
juridique. Or cette forme juridique ne peut être autre qu'un concile, où
l'auteur d'un livre qui excite la contestation soit appelé et pleinement
entendu, pour rendre raison lui-même de sa foi, et des termes et du sens
des propositions qui sont examinées, comme le P. Quesnel vivant lors ne
cessait de le demander de vive voix, et de le requérir expressément par
écrit, au pape et aux évêques, ou quand l'auteur est mort, d'entendre en
sa place ceux qui en veulent prendre la défense. Ce n'était pas là le
jeu du P. Tellier. Il ne savait que trop penser du succès de cette
affaire traitée de la sorte. Il la voulait étrangler par autorité, et
s'en faire après une matière de persécution à longues années, pour
établir en dogme de foi leur école, à grand'peine jusqu'alors tolérée
dans l'Église.

Son dessein, en faisant renvoyer l'affaire au pape, fut donc de le faire
prononcer par une constitution qui, en condamnant un grand nombre de
propositions tirées de ce livre, les condamnât d'une façon atroce, mît
par leurs contraires l'école de Molina en honneur, et en dogme
implicite, en ruinant toutes les écoles catholiques uniquement écoutées
et suivies dans l'Église, et comme cela ne se pouvait espérer en termes
clairs, qui auraient porté leur propre anathème sur le front, il voulut
une condamnation \emph{in globo }qui, en n'épargnant rien et tombant sur
tout, se pût sauver par un vague qui se pouvait appliquer ou détourner
suivant le besoin, et par là même hasarder de condamner dans ce livre
des propositions purement extraites de saint Paul et d'autres endroits
de l'Écriture, et d'autres de saint Augustin et d'autres Pères en termes
formels, qui est la première fois qu'on l'ait osé, pour tirer de là des
conséquences nécessaires en faveur de Molina contre saint Augustin,
saint Thomas et toutes les autres écoles, et à la longue parvenir par
degrés à faire ériger les propositions de l'école de Molina, les plus
opposées à toutes les autres écoles, en dogmes, et flétrir par
conséquent tout ce qui au contraire a servi de règle jusqu'à présent
dans l'Église.

Pour atteindre à ce but, il fallait autant d'adresse et de ténèbres que
d'audace dans la manière de dresser la bulle ou constitution, la dérober
aux cardinaux et aux théologiens de Rome, surtout aux partisans sans
nombre de saint Augustin et de saint Thomas, y flatter Rome et le pape
sur les plus énormes prétentions ultramontaines, assez solidement pour
attacher leur plus vif intérêt au maintien de cette pièce sans toutefois
que cela fût assez grossier pour choquer le roi, ou se mettre en danger
que les parlements le pussent vaincre à cet égard, et pourtant la
fabriquer de manière que le pape se trouvât engagé en des condamnations
tellement insoutenables, qu'il se sentît hors de moyens d'en pouvoir
donner aucune explication si les évêques de France s'avisaient de lui en
demander, et que la superbe de sa prétendue infaillibilité l'empêchât
toujours de souffrir que d'autres attentassent à interpréter eux-mêmes,
que par là il se roidît à la faire recevoir purement et simplement, et
que les jésuites, ayant pour eux le pape et Rome également intéressés
pour leur pouvoir, et pour leur embarras, le roi en France engagé dès en
la demandant à la faire recevoir, et trop entêté de son autorité pour
n'y pas employer toute sa puissance, ils eussent par là une préférence
de leur école sur les ruines de toutes les autres, qui portée par les
deux puissances également, éblouirait l'ignorance ou la faiblesse des
évêques, attirerait les autres par l'ambition, forcerait tout théologien
d'être publiquement pour ou contre, grossirait infiniment leur parti, et
leur donnerait lieu d'anéantir l'autre une fois pour toutes par une
inquisition et une perquisition ouverte contre des gens également en
butte à l'autorité de Rome et à celle du roi\,; par là accoutumer toute
tête à ployer sous ce joug, et de degré en degré l'ériger en dogme de
foi, et c'est là malheureusement où aujourd'hui nous en sommes.

La division habilement semée dans les divers partis parmi les évêques
assemblés en diverses façons sur cette affaire, tous ne crurent plus en
pouvoir sortir que par Rome. Le roi écrivit donc au pape de la façon la
plus pressante pour lui demander une décision, mais de la manière la
plus partiale contre le livre du P. Quesnel. Le pape s'en crut quitte
par la condamnation qu'il en fit à laquelle le cardinal de Noailles
adhéra en retirant l'approbation qu'il y avait autrefois donnée. Mais ce
qui suffisait en soi n'était pas le compte du P. Tellier. Il voulut une
constitution qui condamnât une foule de propositions extraites de ce
livre\,; en la manière et par les raisons qui viennent d'être
expliquées. Le roi redoubla d'instances auprès du pape, et le P. Telier,
pour les mettre l'un et l'autre hors d'état de pouvoir reculer dans les
suites, fit en sorte que le roi répondît au pape sur son autorité dans
son royaume, que sa constitution y serait reçue sans difficulté de
quelque part que ce fût.

Le P. Tellier n'eut pas à Rome des conjonctures moins favorables qu'en
France. Le P. Daubenton dont j'aurai occasion de parler ailleurs, plus
savant, plus accort, plus rompu au monde et aux cours, mais au fond non
moins déterminé jésuite que le P. Tellier, congédié de confesseur du roi
d'Espagne par les intrigues de M\textsuperscript{me} des Ursins à qui
son crédit et ses manéges firent ombrage, était passé en Italie où il
restait assistant français du général des jésuites, qui est pour chaque
grande nation la première place après la sienne. Il était donc à Rome,
et il y vivait comme les plus importants de ses confrères et les plus
initiés dans les mystères les plus secrets de leur compagnie, dans la
plus étroite liaison et la plus réciproque confiance avec le cardinal
Fabroni. J'ignore s'il était de ceux que les jésuites savent
s'approprier à Rome, depuis les plus éminents personnages jusqu'aux plus
obscurs par leurs présents, et les pensions proportionnées à l'état et
au service qu'ils en tirent. Cette politique ne leur est pas nouvelle,
et les a de tout temps bien utilement servis, elle n'est pas même
ignorée\,; mais ni ceux qu'ils soudoient, ni ceux qui sont soudoyés,
n'ont garde de s'en vanter. À l'égard de Fabroni, la mince fortune où il
est né, celle qu'il a faite, l'appui déclaré qu'il a trouvé chez les
jésuites dans tous les temps de sa vie, celui qu'il leur a rendu à
découvert aussitôt qu'il s'est vu en état de le faire, l'application, la
suite et souvent la fureur qu'il a montrée à soutenir toutes leurs
causes, tous leurs intérêts, ceux même des personnes en qui ils en ont
pris, ont pu faire croire qu'il ne leur était pas vendu pour rien, parce
qu'il est vrai et public, et lui-même ne s'en cachait pas, qu'il était
plus ardent jésuite que les plus forcenés de l'espèce même du P.
Tellier, et plus occupé qu'eux-mêmes de leurs affaires.

C'était un bourgeois de Pistoie, venu à Rome avec de l'esprit, de la
scolastique, du feu, de l'application au travail le plus ingrat, et la
résolution de percer à quelque prix que ce pût être. Porté constamment
par les jésuites, il parvint à quarante ans à être, en 1691, secrétaire
des mémoriaux, et quatre ans après secrétaire de la congrégation de la
propagation de la foi, où il eut moyen de déployer son savoir-faire en
faveur de ses patrons. On ne connaît plus à Rome que le droit canon, et
à leur mode, et la scolastique. Le cardinal Albane, qui était jeune et
peu foncé, se livra à Fabroni pour le conduire dans sa fonction de
secrétaire des brefs\,; il s'en trouva bien. Il s'accoutuma si fort à le
consulter dans la suite, et peu à peu il se laissa tellement subjuguer à
cet esprit haut et violent, qu'il devint son maître. Devenu pape, il le
fit cardinal, et augmenta ainsi sa servitude. Fabroni et Daubenton
firent donc le projet de la constitution par ordre du pape.

Le roi avait demandé qu'elle fût concertée avec le cardinal de La
Trémoille, tant à l'égard du fond même que pour éviter ce qui y pourrait
causer de l'embarras par rapport aux maximes de France. L'affaire
faisait du bruit. Une décision dogmatique, et en première instance pour
la France, réveilla la cour de Rome\,; le sacré collége prétendit la
chose assez importante, et même précisément de nature à être
consultée\,; plusieurs des plus anciens et des plus considérables en
parlèrent au pape qui trouva juste d'en avoir leur avis, et qui leur
promit à tous de la manière la plus positive que le projet de cette
constitution leur serait présenté, qu'ils le pourraient examiner chacun
en particulier à leur gré, puis s'assembler plusieurs en congrégations
différentes, et qu'elle ne serait dressée que conformément à l'avis du
plus grand nombre des cardinaux. Le pape donna la même parole au
cardinal de La Trémoille pour ce qui le regardait, comme chargé des
affaires du roi. Les choses en étaient là lors de la mort du cardinal de
Janson et de la nomination de Bissy au cardinalat.

Quelque puissant renfort que le P. Tellier comptât bien de trouver dans
l'élévation de Bissy à la pourpre, la grandeur et l'étendue de ce qu'il
se proposait lui parut mériter de ne pas négliger de se rassembler
toutes les forces qu'il pourrait. L'éclat où se trouvait le nouveau
cardinal de Rohan par les établissements de sa maison, de ses alliances,
de ses liaisons, plus encore le parti qu'il se proposait de tirer en se
l'acquérant, du goût personnel du roi pour le fils de
M\textsuperscript{me} de Soubise, et de prendre ainsi le roi de toutes
parts, engagea ce hardi jésuite à n'en pas faire à deux fois, et de
faire montre de toute sa puissance au cardinal de Rohan, pour le mettre
de son côté par la crainte, et par la récompense toute présente. Il
l'alla voir et lui exposa tout net ses intentions avec une audace et une
autorité qui ne craignait rien. Il lui dit donc qu'il ne pouvait douter
qu'instruit comme il l'était, il ne pensât comme il devait sur l'affaire
de l'Église qui était portée à Rome, mais qu'il ne suffisait pas à un
homme établi comme il l'était de bien penser, comme il supposait et
voulait se persuader qu'il pensait bien, mais qu'il fallait encore bien
faire, non-seulement bien faire, mais tout faire, tout entreprendre,
tout exécuter pour mettre la bonne doctrine à couvert, et pour écraser
une fois pour toutes ce parti séditieux qui troublait l'Église depuis si
longtemps\,; que le roi y était entièrement disposé, que le succès en
était assuré, que c'était à lui de voir quel parti il voulait y prendre,
se perdre auprès du roi à qui il devait tout, et de qui il se pouvait,
en se conduisant bien, se promettre encore bien davantage, ou demeurer
dans une neutralité qui ne pourrait pas se soutenir longtemps, et qui le
déshonorerait et lui ôterait en attendant toute considération\,; ou
enfin, s'attacher au devoir de son état, de sa reconnaissance pour le
roi, en se déclarant pour l'Église et pour la bonne cause, et pour ne
lui rien celer, en n'y ménageant rien et en marchant dans un concert
intime, entier, inaltérable, avec ceux qui en faisaient leur affaire, et
qui lui répondaient en prenant ce parti, mais en s'y engageant de la
sorte, qu'il pouvait compter sur la charge de grand aumônier, et sur
tous les agréments, les grâces, les privances et toute la confiance du
roi. Rohan fut étrangement étourdi d'un compliment si net, et qui lui
présentait si à découvert la paix ou la guerre. Il balbutia, et dans son
trouble il ne put rien tirer de lui-même que des compliments, et tout ce
que l'incertitude et l'étonnement put couvrir sous les plus grandes
politesses. Ce n'était pas la monnaie dont Le Tellier se payait\,; il se
leva froidement, dit au cardinal qu'il s'aviserait\,; que, comme il
désirait d'être son serviteur, il souhaitait et il espérait que ce
serait bien, et que, lorsque ses réflexions seraient faites, il comptait
qu'il lui en ferait part, mais qu'il devait l'avertir de ne les pas
faire longues, parce que la charge de grand aumônier ne pouvait vaquer
longtemps. Il se retira en même temps, et laissa le cardinal épouvanté
d'une déclaration si audacieuse.

Le cardinal de Rohan était net avec de l'esprit naturel, qui paraissait
au triple par les grâces de sa personne, de son expression, du monde le
plus choisi dont le commerce l'avait formé, par les intrigues et les
liaisons où M\textsuperscript{me} de Soubise l'avait mis de fort bonne
heure. Son naturel était bon, doux, facile, et sans l'ambition et la
nécessité qu'elle impose, il était né honnête homme et homme
d'honneur\,; d'ailleurs d'un accès charmant, obligeant\,; d'une
politesse générale et parfaite, mais avec mesure et distinction\,; d'une
conversation aisée, douce, agréable. Il était assez grand, un peu trop
gros, le visage du fils de l'Amour, et outre la beauté singulière, son
visage avait toutes les grâces possibles, mais les plus naturelles, avec
quelque chose d'imposant et encore plus d'intéressant, une facilité de
parler admirable et un désinvolte merveilleux pour conserver tous les
avantages qu'il pouvait tirer de sa princerie et de sa pourpre, sans
montrer ni affectation ni orgueil, et n'embarrasser ni lui-même ni les
autres\,; attentif surtout à se mettre bien avec les évêques, à se les
attirer et à se conserver l'attachement de toute la gente doctrinale,
qu'il s'était fait un capital de s'acquérir sur les bancs, et à quoi il
avait parfaitement su réussir.

Il était de juin 1674. Le cardinal de Noailles était dans l'apogée de sa
faveur lorsqu'il fut question de séminaire et de théologie pour
l'heureux fils de la belle Soubise. Elle avait su toute sa vie ménager
tout, et sa faveur extrême et déclarée et toujours soutenue, lui avait
tout facilité. Elle était donc bien de tout temps avec les Noailles,
trop clairvoyants pour ne pas désirer encore plus d'être de ses amis.
Par eux et par M\textsuperscript{me} de Maintenon même, à qui elle en
fit sa cour, elle donna son fils au cardinal de Noailles dès son entrée
dans l'archevêché de Paris, et le lui remit pour se reposer entièrement
sur lui de toute son éducation ecclésiastique. Ces considérations
engagèrent ce prélat d'en faire comme de son neveu\,; et cet intrus
neveu, déjà fait aux manéges de sa mère, n'oublia rien pour faire du
prélat comme d'un véritable oncle en toutes choses, parce qu'il sentit
que sa fortune en dépendait et qu'elle ne pouvait être que grande et
prompte, s'il engageait par sa conduite cet oncle adoptant à la vouloir.
Il le mit à Saint-Magloire dont il fit son séminaire de confiance,
choisit des gens pour former et veiller sur ses mœurs et ses études, et
pour lui en rendre un compte particulier. Les charmes de la personne de
l'élève furent secondés par tout l'art d'une conduite qui répondit en
tout aux vastes desseins de sa mère sur lui, et la facilité de son
esprit à tout ce qu'on lui voulut apprendre. Son application, ses
progrès, sa modestie, sa politesse, son attention à plaire, lui
gagnèrent ses maîtres et tout Saint-Magloire, et prêtres de l'Oratoire
et séminaristes. Il se fit une réputation. Il ne fut pas moins adroit,
ni moins attentif en Sorbonne, ni avec moins de succès. Il travailla de
bonne foi à apprendre\,; et en effet il acquit de la science qu'il sut
tripler par la grâce et la facilité de son débit, et tellement gagner ce
peuple lettré, que, tout grossier, pédant et farouche qu'il soit de sa
nature, il ne voulut que l'admirer et le vanter. Tant de bons
témoignages ne demeurèrent point oisifs. Noailles se faisait un plaisir
de les porter au roi et à M\textsuperscript{me} de Maintenon, charmé
lui-même de son élève, et le roi plus content encore d'avoir tant où
s'appuyer pour travestir en justice les inclinations et les penchants de
son cœur.

M\textsuperscript{me} de Soubise était morte dans l'attachement et la
reconnaissance pour le cardinal de Noailles, sans lequel elle sentait
que toute sa faveur et toute la volonté du roi aurait été peu
fructueuse, et elle avait inculqué ces sentiments à son fils, dont l'âge
et le chemin ne semblaient pas pouvoir entrer jamais en opposition avec
un bienfaiteur à qui il devait tant, et à qui il se ferait toujours tant
d'honneur de rendre.

De si fortes raisons s'appuyaient dans le cardinal de Rohan par d'autres
plus touchantes. Prince avec sa maison par la grâce du roi et la beauté
de sa mère, des biens immenses et de grands établissements y étaient
entrés. Il avait passé sa première jeunesse sous la férule, dans le
travail, dans toutes sortes de contraintes pour arriver à une grande
fortune. Il y était parvenu avec rapidité, que ses mœurs, délivrées
d'Argus, ne lui avaient pas procurée. Il se voyait avant quarante ans
évêque de Strasbourg et cardinal, avec plus de quatre cent mille livres
de rente, le goût des plaisirs, de la magnificence, du repos, après tant
de travaux si contraires à sa paresse naturelle. Il lui semblait qu'il
n'avait plus rien à désirer qu'à jouir d'un état où tout est devenu
permis, et où on n'a plus à compter avec personne. Un cardinal est en
droit de passer sa vie au jeu, à la bonne chère et avec les dames les
plus jeunes et les plus jolies\,; d'avoir sa maison pleine de monde pour
le rendez-vous et la commodité des autres, de leurs amusements, de leurs
plaisirs et pour le centre des siens\,; d'y donner des bals et des
fêtes, et d'y étaler tout le luxe et la splendeur en tout genre qui peut
flatter, surtout de n'entendre plus parler de livres, d'étude, de rien
d'ecclésiastique\,; d'aller régner dans son diocèse sans s'en mêler\,;
de n'en être pas seulement importuné par ses grands vicaires, ni par le
valet sacré et mitré payé pour imposer les mains\,; et d'y vivre sans
inquiétude dans un palais à la campagne, au milieu d'une cour, comme un
souverain, parmi le jeu, les dames et les plaisirs, pleinement affranchi
là comme à Paris et à la cour de toute bienséance. Ce n'est pas que nos
cardinaux vécussent tous de la sorte, mais ils en avaient toute liberté.
Le cardinal de Bouillon en avait usé dans toute son étendue, et celui-ci
en jouissait aussi pleinement\,; il était fait pour être et vivre en
grand seigneur, et ne se refuser aucune chose\,: il avait de quoi y
fournir parfaitement, et le roi, si volontiers austère pour les autres,
était accoutumé, non-seulement à passer, mais à trouver tout bon des
cardinaux. Il était bien doux à celui-ci de vivre de la sorte\,; c'était
son penchant et son goût\,; c'était avec la haute fortune, cet état
d'entier affranchissement qui le flattait le plus, et dont la
perspective l'avait le plus soutenu dans le fâcheux chemin qui l'y avait
fait atteindre. Que pouvoir se proposer de préférable à la jouissance
d'un état si heureux qui ne voit rien au-dessus de soi, ni de plus
libre, et quel prétexte d'en profiter en plein qui fût plus naturel et
plus honnête que l'attachement et la reconnaissance pour un homme à qui
il devait tout, du su de tout le monde, dont les mœurs et la conduite
était en vénération la mieux établie\,; qui était son ancien d'âge de
vingt-quatre ans, d'épiscopat de vingt-deux, de cardinalat de treize,
archevêque de la capitale\,; uni et à la tête des plus saints et les
plus savants corps et particuliers de Paris, auxquels tant d'autres des
provinces se jaignaient, vers qui les premiers inclinaient, qui avait
pour lui une famille puissante, et tout ce qui n'était pas esclave des
jésuites, c'est-à-dire tous les honnêtes gens de tous états\,? Le
cardinal de Rohan, entraîné par des raisons si homogènes à lui-même,
trouva dans sa famille un homme qui n'y était pas nouvellement entré
pour n'en pas profiter. Tallard, qui sut par le cardinal même et par le
prince de Rohan l'insolence de la proposition du P. Tellier, trouva
cette ouverture admirable, et le comble du bonheur des Rohan.

Plus le discours du confesseur avait eu la hauteur de celui d'un favori
premier ministre, plus il en tira parti, pour montrer aux Rohan, d'un
côté les enfers ouverts sous leurs pas, de l'autre les cieux qui les
appelaient dans leur gloire. Il leur représenta l'intérêt et le naturel
terrible du jésuite et des siens, M\textsuperscript{me} de Maintenon,
que ce parti avait arrachée de l'estime, de l'amitié, de l'alliance et
des liaisons de confiance les plus intimes du cardinal de Noailles, qui
s'étaient changées en elle en fureur et en poursuite la plus à découvert
et la plus violente, le roi qui avait hautement épousé ce parti, qui
était exactement fermé à n'écouter que ceux qui y étaient les plus
ardents, qui y avait mis son autorité et sa conscience, qui n'était
occupé ni entretenu d'autre chose, qui regardait le parti opposé comme
ennemi de l'Église et de l'État, comme républicain, comme ennemi de son
autorité et de sa personne, et qui depuis son enfance était nourri dans
ce préjugé contre tout ce que les jésuites voulaient traiter de
jansénistes. Il leur fit peur par l'exemple du cardinal de Bouillon,
qu'une semblable affaire, et toutefois sans ombre de jansénisme, et avec
le confesseur pour lui, avait perdu pour l'archevêque de Cambrai, et
dont eux-mêmes par l'affaire de Strasbourg avaient comblé la disgrâce,
qui avait été au moment d'ôter le rang à sa maison. Il leur fit
considérer que les neutres, surtout d'une considération en ce genre
aussi rare qu'était la sienne, ne seraient regardés qu'avec dépit et
mépris des deux côtés, outre que les occasions qui surviendraient chaque
jour dans le cours de cette affaire lui rendraient la neutralité bien
difficile à soutenir\,; que c'était à lui à se tâter lui-même pour voir
s'il se croyait capable de soutenir tous les dégoûts, et de toute
espèce, que le roi se plairait à faire tomber sur lui, et tous ceux
encore qu'à l'abri de l'entier discrédit les jésuites sauraient lui
susciter de toutes les façons, et par toutes sortes de canailles, qui
aujourd'hui se croient honorés de le voir passer dans son antichambre.

Après l'avoir ébranlé de la sorte, Tallard lui fit honte de voir un
autre que lui grand aumônier, et Bissy en sa place à la tête du parti
favori, et en avoir toute l'autorité, le ralliement, la faveur, la
confiance, les privances du roi, et de lui devenir nécessaire toute sa
vie\,; tandis que lui-même serait au rebut, et aurait peut-être
l'affront de voir Bissy entrer au conseil, lui qui se tiendrait heureux
de lui porter partout son portefeuille, et disposer de toutes les
grandes places de l'Église que le besoin continuel que le confesseur
aurait de lui l'empêcherait de lui contester. De là, venant à toute la
disproportion de Bissy à lui, il étala tous les avantages qu'il tirerait
sans cesse pour les siens, s'il se mettait à la tête de ce parti, avec
le goût que le roi avait pour lui et pour sa famille\,; qu'il serait en
état de tout prétendre et de tout obtenir, et même avec apparence d'être
porté jusque dans le conseil. Il ignorait sans doute, ou voulut ignorer,
ce qui était échappé là-dessus au roi à l'égard du cardinal de Janson,
rapporté ci-dessus.

Après avoir flatté le cardinal de Rohan de pouvoir mettre ainsi tout à
ses pieds, il se moqua de sa délicatesse sur le cardinal de Noailles,
qui n'en serait pas moins perdu quand il se perdrait avec lui, dont il
ne serait et ne passerait jamais que pour le disciple, en se rangeant de
son côté, ni pouvait jamais atteindre à aucun des avantages et de la
considération qui se tirait de la qualité de chef de parti, qui
demeureraient tous au cardinal de Noailles, par qui seul il végéterait,
et au fond lui serait compté pour rien\,; au lieu que prenant le parti
contraire, et dans ce parti se trouvant de bien loin sans égal en
naissance, établissements, considération et dignité, il se verrait tout
à coup vis-à-vis du cardinal de Noailles avec la supériorité que lui
donnerait la faveur si déclarée du parti dont il serait le chef, et le
chef sans collègue, parce que Bissy, devenu cardinal, ne pourrait en
aucun genre approcher de sa distinction par tout, et par cette
disproportion inhérente serait, malgré son âge, à son égard, moins que
lui à celui du cardinal de Noailles, s'il avait la folie d'en préférer
le parti.

Ce qui rendait Tallard si éloquent était son intérêt propre. Il ne
s'était allié aux Rohan que pour en profiter. Il regardait leur faveur
comme un chemin à lui ouvert pour tout. Il comprenait qu'aucun des deux
frères n'entrerait dans le conseil, et la chose était visible. Mais lui
qui avait passé par tous les genres d'affaires considérables, qui
n'avait ni rang ni attachement étranger, qui avait vu Harcourt si
souvent près d'y entrer et que sa santé mettait hors de toute portée, il
se flatta que les jésuites feraient pour lui ce qu'ils ne pourraient
pour le cardinal de Rohan, par leur intérêt propre. Il voulait la
pairie, il voulait la survivance de son gouvernement, il voulait une
grande charge\,; en un mot que ne voulait-il point, et que n'espérait-il
point en mettant le cardinal de Rohan à la tête d'un parti qui pouvait
et pourrait tout, et dont par là il espérait bien de se mêler\,! Enfin
il acheva de déterminer le cardinal de Rohan, en lui persuadant qu'il
n'aurait que l'honneur de la conduite de l'affaire et des assemblées,
d'être à la tête du clergé de France, à la place du cardinal de
Noailles, lui, à son âge, et qui par son siége n'était point de ce
clergé\,; qu'il en deviendrait le modérateur et l'arbitre\,; et que pour
le travail il en chargerait des commissaires et des bureaux qui lui
présenteraient la besogne toute faite, dont il n'aurait que l'honneur.
Ce point de paresse tenait fort le cardinal, et ce fut aussi celui que
Tallard vainquit le dernier\,; mais son ambitieux bien-dire sut aussi en
triompher, et jeter le cardinal de Rohan dans une fondrière, dont sa
paresse et la flétrissure de son honneur lui ont coûté de sourds et de
cuisants repentirs, et où sa vanité a eu fort à souffrir de l'égalité
qu'à force de souplesse le cardinal de Bissy usurpa enfin pour le moins
avec lui, dans la réalité de vrai chef de confiance de tout ce parti.

Le cardinal de Rohan, agité, battu plusieurs jours, ne put résister à
son frère et à Tallard, que ce maréchal avait gagné. Son marché fut
grossièrement conclu au mot du P. Tellier, dont il devint l'esclave en
même temps qu'il prêta le serment de grand aumônier de France. Moins je
prétends m'étendre sur l'histoire de la constitution même, qui remplit
seule des in-folio, et plus je crois devoir en montrer les ténébreuses
trames, auxquelles seules je crois devoir me restreindre. Quelque peu de
cas que les jésuites fissent de l'esprit léger et du cœur encore plus
volage du cardinal de Polignac, il était cardinal, et ils ne voulurent
pas le mécontenter. La rage de courtisan, sous laquelle il gémit toute
sa vie, lui avait fait passionnément désirer la charge de maître de la
chapelle du roi, c'est-à-dire uniquement des musiciens de la chapelle,
depuis qu'elle vaquait par la mort de l'archevêque de Reims. Devenu
cardinal, il ne la souhaita pas moins, et, bien que d'autres cardinaux
l'eussent possédée, il crut que sa pourpre y flatterait le roi,
contribuerait à la lui faire donner, et ferait encore plus sa cour\,; il
ne se trompa pas, surtout avec le concours des jésuites\,; mais sa
nouvelle dignité fit un embarras.

Cette charge, qui n'est pas des premières, ni même des secondes, ne
prête serment qu'entre les mains du grand maître de la maison du roi, et
ce grand maître était un prince du sang. Comment donc oser lui souffler
un droit acquis, mais comment aussi ployer la pourpre romaine à cette
sorte d'humiliation\,? Le respect du roi, légué par le Mazarin, pour
cette sacrée pourpre l'emporta cette fois sur celui dont il se montrait
si jaloux pour les princes de son sang. M. le Duc était son petit-fils,
et dans la première jeunesse. Il donna la chapelle à Polignac, et régla
que, pour cette fois et sans conséquence, sous prétexte d'être pressé
d'entrer en fonctions, il profiterait du voyage que M. le Duc allait
faire pour la première fois en Bourgogne et y tenir les états, pour de
son consentement prêter, en son absence, serment entre les mains du roi,
et cela se fit tout de suite avec la charge de grand aumônier.

En même temps, le cardinal de Polignac reçut le bonnet des mains du roi,
présenté par l'abbé Howard, camérier du pape. C'était raison qu'un
camérier anglais apportât une barrette de la nomination du roi
d'Angleterre, mais ce ne l'était pas que le nommé fût le négociateur à
Utrecht de tout ce qui fut convenu contre le prince à qui il devait sa
fortune. Malgré l'orgueil de la pourpre, la vanité du bien-dire perça.
Le cardinal de Polignac ne dédaigna pas de paraître devant le roi à la
tête de l'Académie française, à la suite de tous les corps qui le
haranguèrent sur la paix. Ses grâces, ses charmes et son bien-dire, si
odoriférant et si flatteur, céda toutefois à la justesse et à
l'éloquence mâle et naturelle du recteur de l'Université, qui enleva
tous les suffrages avec tant de violence, qu'il fut interrompu par les
applaudissements, et que le roi lui fit une réponse pleine de
l'admiration de son discours. Vittement, c'était son nom, ne s'en éleva
pas davantage, n'en demeura pas moins renfermé dans la poussière des
colléges, et ne cultiva personne\,; mais, ce qui ne s'est peut-être
jamais vu, et dans une cour comme elle était alors, sa harangue ne
sortit point de la mémoire du roi. Elle y surnagea, chose encore plus
extraordinaire, à tout ce qui le pouvait rendre suspect sur la doctrine,
et des mœurs trop pures et trop austères pour le goût d'alors\,; cette
harangue seule et qu'on crut oubliée avec tant et {[}tant{]} d'autres,
prévalut à tout, et le fit deux ans après sous-précepteur du roi
d'aujourd'hui, par le souvenir toujours présent qu'en avait conservé
Louis XIV. On verra en son temps que ce fut le seul bon choix qu'il fit
pour l'éducation de ce jeune prince, qui eut aussi le sort ordinaire de
ce qu'il y a de meilleur dans les cours.

\hypertarget{chapitre-xviii.}{%
\chapter{CHAPITRE XVIII.}\label{chapitre-xviii.}}

1713

~

{\textsc{\emph{Histoire de France} du P. Daniel\,; son succès\,; son
objet\,; sa prompte chute\,; récompense.}} {\textsc{- Cardinal Gualterio
à la cour.}} {\textsc{- Cause de sa disgrâce à Rome, et de ce que les
nonces en France n'y reçoivent plus la nouvelle de leur promotion à. la
pourpre.}} {\textsc{- Grâces faites au cardinal Gualterio, qui retourne
à Rome.}} {\textsc{- Retour du maréchal d'HuxelIes et de Ménager}}
{\textsc{- Mérite de Ménager, à qui le roi donne une pension de dix
mille livres.}} {\textsc{- Mort, caractère, friponnerie, état et famille
de Sainctot.}} {\textsc{- Branche très-effective de La Tour non reconnue
par les La Tour-Bouillon.}} {\textsc{- Plaisant tour là-dessus de
Wartigny au cardinal de Bouillon,}} {\textsc{- Querelle du duc d'Estrées
et du comte d'Harcourt.}} {\textsc{- Prétentions des maréchaux de France
et leurs tentatives de juridiction sur les ducs, avortées.}} {\textsc{-
Court abrégé de la nouveauté, de l'absurdité et du peu de succès des
prétentions d'autorité des maréchaux de France sur les ducs, et de la
manière d'accommoder leurs querelles.}} {\textsc{- Maréchal d'Estrées
commissaire du roi sur l'insulte de Mademoiselle à Madame.}}

~

Les libéralités si populaires et si surprenantes, par rapport au génie
du roi, de M. et de M\textsuperscript{me} du Maine que nous avons
rapportées à l'occasion de la publication de la paix à Paris, ne
tardèrent pas à se développer. Les jésuites, si adroits à reconnaître
les faibles des monarques, et si habiles à saisir tout ce qui peut
eux-mêmes les protéger et les conduire à leurs fins, montrèrent à quel
point ils y étaient maîtres. On vit paraître une nouvelle, et assurément
très-nouvelle, \emph{Histoire de France}, en trois volumes in-folio fort
gros, portant le nom du P. Daniel pour auteur, qui demeurait à Paris en
leur maison professe, dont le papier et l'impression était du plus grand
choix, et le style admirable. Jamais un français si net, si pur, si
coulant, les transitions heureuses, en un mot tout ce qui peut attacher
et charmer un lecteur\,: préface admirable, promesses magnifiques,
courtes dissertations savantes, une pompe, une autorité la plus
séductrice. Pour l'histoire, beaucoup de roman dans la première race,
beaucoup encore dans la seconde, et force nuages dans les premiers temps
de la troisième. Tout l'art, tout le ménagement des ombres et du
clair-obscur, ainsi que dans le plus beau tableau, y parurent sous le
masque d'une apparente simplicité, et tout le secours, aux endroits les
plus scabreux, que l'esprit put fournir à une audace qui se sent
appuyée. En un mot, tout l'ouvrage parut très-évidemment composé pour
persuader, sous l'air naïf d'un homme qui écarte les préjugés avec
discernement, et qui ne cherche que la vérité, que la plupart des rois
de la première race, plusieurs de la seconde, quelques-uns même de la
troisième, ont constamment été bâtards, très-souvent adultérins et
doublement adultérins, que ce défaut n'avait pas exclus du trône, et n'y
avait jamais été considéré comme ayant rien qui en dût ni pût éloigner.
Je dis ici crûment ce que la plus fine délicatesse couvre, mais en
l'exprimant pourtant très-manifestement dans tout le tissu de l'ouvrage,
avec une négligence qui détourne tant qu'elle peut les yeux du dessein
principal, et ne laisse que l'agréable surprise de ces découvertes
historiques dont la vérité, égarée dans les ténèbres de plusieurs
siècles, est due aux persévérantes veilles d'un savant qui les consacre
toutes à chercher, à puiser, à comparer, à remonter aux sources les plus
cachées, et aux travaux duquel la postérité demeure redevable des
lumières qui éclaircissent ce qui avait été ignoré jusqu'alors.
L'éblouissement fut d'abord extrême, et la vogue du livre telle, que
tout y courut jusqu'aux femmes. Le même intérêt qui l'avait fait
composer était aussi de le répandre. On a vu sur la campagne de Lille,
et on verra dans la suite, combien ceux que cet intérêt regardait et
conduisait étaient prodigieux en ténébreuses intrigues et à disposer, en
magiciens, de la fureur de la mode. Les louanges de ce livre
transpirèrent de chez M\textsuperscript{me} de Maintenon\,; le roi en
parla, et demanda à quelques-uns de sa cour s'ils le lisaient\,; les
plus éveillés sentirent de bonne heure combien il était protégé\,:
c'était bien sûrement l'unique livre historique dont le roi et
M\textsuperscript{me} de Maintenon eussent jamais parlé. Aussi parut-il
bientôt à Versailles sur toutes les tables des gens de la cour\,; et
hommes et femmes, on ne parla d'autre chose, avec des éloges merveilleux
qui étaient quelquefois plaisants dans la bouche de personnes, ou fort
ignorantes, ou qui, incapables de lecture, se donnaient pour faire et
goûter celle-là. Mais cette surprenante vogue eut un inconvénient\,: on
s'aperçut que toute cette vaste histoire, qui semblait éplucher de si
près les temps ténébreux, ne s'attachait dans les autres qu'à la partie
purement militaire, aux camps, aux marches, à tout exploit de guerre
jusqu'à un détail d'un parti de quarante ou de cinquante chevaux, ou
d'autant de gens de pied, qui en rencontrait un autre, et qui, dans un
long récit, n'oubliait pas la plus légère circonstance. En s'étendant de
la sorte, on se donne un vaste champ, et c'est aussi ce qui remplit les
trois volumes. Mais de négociations, de cabales et d'intrigues de cour,
de portraits de personnages, de fortunes, de chutes, de ressorts des
événements, pas un mot en tout l'ouvrage que sèchement, courtement et
précisément comme les gazettes, souvent encore plus superficiellement.
De choses de lois de cérémonies publiques, de fêtes des divers temps,
même silence, tout au plus même laconisme\,; et sur les matières de
Rome, puis de la Ligue, c'est un plaisir de le voir courir sur ces
glaces avec ses patins de jésuite.

À la fin les connaisseurs le méprisèrent, et il résulta de tant
d'applaudissements une très-méchante histoire, qui n'avait pu être autre
de la plume dont elle sortait, par la politique de la compagnie, mais
qui avait très-industrieusement et très-frauduleusement rempli le but
unique qui l'avait fait faire. L'ouvrage tomba donc\,; il y eut des
savants qui écrivirent des dissertations contre\,; mais le point délicat
principal, le point qui l'avait fait naître et couronner en naissant, ne
fut presque pas touché en France avec la plume, tant on y en sentit le
danger.

Le P. Daniel en tira du roi deux mille francs de pension, ce qui est
prodigieux pour un régulier, même jésuite, avec le titre
d'historiographe de France. Il jouit en plein de ses mensonges qu'il
n'ignorait pas, et peut-être moins que bien d'autres\,; et avec sa
faveur et sa pension il se moqua de tout ce qu'on écrivit contre son
Histoire, sans y répondre un mot, parce que lui-même savait bien qu'en
penser.

Les pays étrangers ne furent pas si sobres que les François sur ces rois
en si grand nombre prétendus bâtards, et cette bâtardise si capable du
trône\,; mais on eut grand soin de ne pas laisser infecter la France de
ces fâcheuses vérités. Il n'y avait que seize mois qu'on avait perdu le
Dauphin, la Dauphine et le prince leur fils aîné\,; il faut du temps
pour écrire une pareille Histoire de France.

J'eus le plaisir de revoir mon ami le cardinal Gualterio. Nous nous
écrivions toutes les semaines et fort ordinairement en chiffre, pour
nous entretenir plus librement, et ce commerce a duré régulièrement
jusqu'à sa mort. Étant nonce, il avait reçu la nouvelle de sa promotion
à Paris, et sa calotte, puis son bonnet des mains du roi. Il avait
extrêmement réussi. Le roi l'aimait et le considérait\,; les ministres y
avaient pris confiance. Il s'était fait beaucoup d'amis.

Il avait eu la complaisance de visiter, en partant, M. du Maine et le
comte de Toulouse avec le même cérémonial que les princes du sang, mais
ce qui lui fit auprès du roi le plus sensible mérite le perdit à Rome.
Il y fut mal reçu du pape, de ses ministres, du sacré collége, y fut
longtemps fort retiré par être abandonné, et en proie à la plus fâcheuse
disgrâce.

C'est aussi le dernier nonce qui ait reçu en France l'avis de sa
promotion. Ils ont eu si peur à Rome d'une récidive, car les bâtards
n'avaient jamais reçu cet honneur avant Gualterio, que toutes les fois
que les nonces de France ont été promus depuis, ils ont eu ordre de
prendre congé et de partir, de façon qu'aucun d'eux n'en a reçu la
nouvelle et la calotte qu'à l'entrée de l'Italie Jamais ils ne l'ont
bien pardonné à Gualterio à Rome, de manière que non-seulement ne se
voyant plus papable, mais hors d'espérance de tout emploi, hors du plus
commun parmi des cardinaux, il se donna publiquement à la France, et mit
les armes du roi sur sa porte comme un cardinal national. Il se chargea
aussi, à faute de mieux, des affaires du roi d'Angleterre. Il eut une
pension du roi, et les abbayes de Saint-Remi de Reims, et de
Saint-Victor à Paris.

Assez oisif à Rome, il voulut venir voir le roi et ses amis encore une
fois en sa vie, et il arriva à la mi-juin à Paris, et tout de suite à la
cour. Le roi fut véritablement touché de ce voyage, et le lui témoigna
par toutes sortes d'amitiés et de distinctions\,: il fut de tous les
Marlys. Le cardinal de Rohan le logea et le fournit d'équipages.

Je ris fort avec lui de la peur qu'il avait faite aux ministres. Les
maximes du roi, dont j'ai parlé plus d'une fois, et dont il s'était
expliqué à l'occasion du cardinal de Janson, ne les purent rassurer. Les
princes changent quelquefois, la face de la cour l'était totalement
depuis le départ de ce cardinal\,; l'exemple du Mazarin les intimida,
ils ne purent comprendre qu'un homme de cet âge et de cette dignité
entreprît, de gaieté de cœur, un si grand voyage sans objet que celui
qui, en effet, l'amenait. Ils furent du temps à tâter le pavé avec
lui\,; mais à la fin, ne voyant rien eclore, ils reprirent leurs esprits
et leurs anciennes manières avec lui.

Il fut extrêmement fêté de tout le monde, et avec empressement du plus
distingué. Il ne quitta la cour que pour aller voir le roi d'Angleterre
en Lorraine, et passer deux jours, chemin faisant, dans son abbaye de
Reims avec l'archevêque son ami. Il vit peu le roi en particulier, qui
lui promit l'ordre\,; il fut du voyage de Fontainebleau, très-bien logé,
et il y prit congé du roi et de ses amis au commencement d'octobre, avec
le serrement d'un bon cœur qui compte bien ne les revoir plus, et le roi
en parut peiné lui-même et le combla de bontés. Il était venu par mer à
Marseille, il s'en retourna par Turin, d'où il s'alla embarquer à Gênes.

Le maréchal d'Huxelles, accompagné de Ménager, salua le roi, le 21 juin,
arrivant d'Utrecht à Versailles. Il y avait été aussi peu d'accord avec
Polignac qu'à Gertruydemberg, et l'avait traité avec une humeur et une
hauteur qui ne convenait pas à l'égalité de leur caractère, et moins
encore à l'inégalité de leur naissance. Polignac, qui voyait la pourpre
s'approcher de lui de plus en plus, glissa sur tout avec accortise sans
céder sur les affaires\,; il évita sagement l'éclat et la brouillerie
ouverte, mais ils ne se sont guère vus depuis, et n'ont pas montré faire
grand cas l'un de l'autre. Ménager n'oublia point avee eux ce qu'il
était, et ne se laissa point gâter par son égalité monstrueuse de
caractère\,; il les satisfit également l'un et l'autre avec beaucoup
d'art, de douceur et de déférence\,; et, bien que plus penché par
Polignac par la douceur de ses mœurs, et aussi sur le fond des affaires
et la manière de les conduire, qui venait toute mâchée de Torcy, mais où
le maréchal voulait toujours mettre du sien, Ménager ne fut pas inutile
entre eux, et servit très-bien pour les choses du commerce qui étaient
peu connues des deux autres, et dont il était particulièrement chargé.
Il fut donc fort bien reçu, et eut en arrivant une pension de dix mille
livres.

Sainctot mourut subitement à quatre-vingt-cinq ou six ans. C'était une
famille plébéienne. Il avait eu un frère conseiller au parlement. Il
avait été longtemps maître des cérémonies. On a pu voir (t. II, p.~80)
quelle avait été sa probité dans cette charge, et la friponnerie avérée
de ses registres qu'il fut forcé d'avouer et de réparer. C'était un
homme tout doucereux, et avec cela tout avantageux, tout esclave de la
faveur aux dépens de vérité et de justice, et qui se croyait en droit de
favoriser qui il lui plaisait en passe-droits. Il eut tant de
discussions avec Blainville du temps qu'il était grand maître des
cérémonies, auquel il tâchait toujours de s'égaler, qu'il fut contraint
de vendre sa charge de maître des cérémonies. Il acheta en même temps
une des deux d'introducteur des ambassadeurs, où il fit maintes
sottises, comme on a vu (t. II, p.~78 et suiv.), entre plusieurs autres
qui n'ont pas valu la peine d'être rapportées. Il avait un fils aîné qui
se tourna au plus mal\,; et il avait cédé sa charge à son second fils
depuis quelques années, qui s'y est conduit bien plus sagement que lui.
Il laissa une grande et assez vilaine fille qui épousa, deux ans après,
le comte de La Tour, sur lequel il n'est peut-être pas inutile de
s'arrêter un moment.

Ces La Tour étaient une branche de la maison de La Tour-Bouillon, que
MM. de Bouillon devenus princes ne voulaient point reconnaître, parce
qu'ils ne l'étaient pas devenus avec eux et qu'ils étaient demeurés
pauvres et peu connus, jusqu'à réputer à injure qu'on leur en parlât et
qu'on les crût de même maison qu'eux, sans toutefois aucune autre
raison, ni avoir osé leur disputer leurs armes et leur nom, comme
M\textsuperscript{me} de Soubise avait fait pour les noms et armes à la
branche de Rohan Gué de L'Isle ou du Poulduc, qui malgré tout son crédit
y fut contradictoirement maintenue par un arrêt du parlement de
Bretagne. Ce comte de La Tour, gendre de Sainctot, avait un frère aîné
fort peu accommodé qui ne laissa que des filles, pendant la vie duquel
il servait en Italie subalterne, puis capitaine d'infanterie. Le
cardinal de Bouillon, passant en un de ses voyages de Rome, dîna chez M.
de Vaudemont. Wartigny, brigadier alors de dragons, duquel il a été
parlé quelquefois, était une manière d'effronté fort plaisant, d'un
commerce ordinairement fort doux, mais qui se choquait volontiers des
impertinences. Il le fut apparemment en ce repas de celles du cardinal
de Bouillon qui y était un grand maître. Sortant de table, Wartigny
trouva sous sa main le comte, lors appelé le chevalier de La Tour, parmi
une foule d'officiers qui étaient venus bayer là, et faire leur cour à
M. de Vaudemont. Il le prit par le bras, et au milieu de tout ce grand
monde, le mène au cardinal et lui dit qu'il le supplie de lui permettre
de lui présenter un gentilhomme de sa maison, qui par sa valeur et sa
conduite méritait ses bontés et ses secours, et que tous ceux qui le
connaissoient lui rendraient le témoignage qu'il n'était pas indigne de
l'honneur qu'il avait de porter son nom et ses armes. Le cardinal de
Bouillon, qui ne s'attendait à rien moins qu'à ce compliment, pour lui
si étrange et si publiquement fait, rougit jusqu'au blanc des yeux,
regarda Wartigny avec des yeux de fureur, tourna le dos sans répondre,
et se hâta de gagner la pièce où on allait en sortant de table,
grommelant de colère entre ses dents. L'assistance se mit fort à rire et
à se moquer de l'orgueil si déplacé du cardinal, et à remercier Wartigny
de lui avoir donné cette scène. Passons maintenant à l'origine de cette
branche.

Anne IV de La Tour, seigneur d'Oliergues et vicomte de Turenne, l'un des
chambellans de Louis XI, eut d'Anne de Beaufort, sa cousine germaine,
qu'il avait épousée par dispense en 1444, plusieurs enfants dont un
continua la postérité, et un seul puîné qui fit la branche de ces La
Tour dont on parle ici. Ce puîné fut Antoine-Raymond de La Tour, et sa
branche porta le nom de La Tour-Murat. Il était frère d'Antoine de La
Tour, vicomte de Turenne, l'un des chambellans de Charles VIII, père de
François II de La Tour, vicomte de Turenne, qui commença beaucoup à
figurer, dont le fils François II de La Tour, vicomte de Turenne, épousa
une fille du célèbre Anne, duc de Montmorency, connétable de France,
lequel fut père du maréchal de Bouillon à qui Henri IV fit épouser
l'héritière de Bouillon et Sedan, père de MM. de Bouillon et de Turenne,
et grand-père du cardinal de Bouillon, etc. C'en est assez pour faire
voir d'où et quand la branche de La Tour-Murat s'est formée.

Il est vrai qu'elle ne fut pas heureuse en richesses ni en honneurs. Les
alliances n'en furent pas plus flatteuses, excepté une La Fayette
qu'épousa ce chef de la branche, et une Apchier qu'ils eurent dans la
suite. Ce chef de branche, qui lui-même commença l'obscurité dans
laquelle toute sa postérité est demeurée, fut bisaïeul de Jacques de La
Tour, Seigneur de Murat, qui sur la présentation de ses titres fut
maintenu dans sa soblesse par Fortia, intendant d'Auvergne, le 18 juin
1677. Ce Jacques de La Tour était au quatrième degré avec le maréchal de
Bouillon, c'est-à-dire enfants des issus de germains\,; et ce même
Jacques de La Tour était le propre grand-père du gendre de Sainetot,
c'est-à-dire que ce gendre de Sainctot et le cardinal de Bouillon
étaient au sixième degré. Les autres Bouillon ne les reniaient pas avec
moins d'indignation que le cardinal, tant la princerie affole les
cervelles. Ce gendre de Sainctot a laissé des fils, outre lesquels il y
a encore la branche de La Tour, seigneurs de Blanchas et de
Saint-Exupéry, sortie d'un puîné du fils aîné du chef de la branche de
Murat, et dans le même néant qu'elle. Longtemps depuis la mort de Louis
XIV, les Bouillon réduits à quatre têtes\,: le duc de Bouillon, le
prince de Turenne son fils unique, le comte d'Evreux apoplectique et
hors d'état de se remarier, et le cardinal d'Auvergne, ils ont été
tentés de faire justice et de reconnaître enfin ces La Tour. Tantôt ils
le voulurent, tantôt ils ne le voulurent plus. Après ils se partagèrent
sur le oui et le non. Le point était ce dieu de princerie. Ils
courtisèrent le cardinal Fleury qui avait tant fait d'énormités pour
eux, et ils en espérèrent celle de princiser aussi ces pauvres
petits-cousins, sans quoi il eût été bien fâcheux de les reconnaître. Le
cardinal est mort sans le leur accorder, et ils sont encore à les
reconnaître.

Une querelle, arrivée dans la fin de juin, à un souper chez la duchesse
d'Albret, entre le duc d'Estrées et le comte d'Harcourt, fit grand bruit
dans le monde. On a vu ailleurs le peu qu'était et que valait ce petit
duc d'Estrées. Le comte d'Harcourt, qui longtemps depuis la mort du roi
obtint une terre du duc de Lorraine en Lorraine, lui fit donner le nom
de Guise et se fit appeler le comte de Guise, était une manière d'escroc
et de bandit qui ne valait guère mieux. Il était fils du prince et de la
princesse d'Harcourt desquels j'ai parlé ailleurs. Le maréchal
d'Huxelles, qui se trouva par hasard le plus ancien des maréchaux de
France qui fussent à Paris, leur envoya à Chacun un exempt de la
connétablie pour demeurer auprès d'eux. Ils ne voulurent pas les
recevoir ni l'un ni l'autre, parce que les ducs ni les princes étrangers
ne reconnaissent point l'autorité ni la juridiction des maréchaux de
France, et n'y ont jamais été soumis, encore que ce tribunal ait saisi
toutes les occasions de l'entreprendre et de l'usurper. Le rare est que
les ducs-maréchaux de France se sont d'ordinaire plus souciés d'une
autorité passagère, et trouvés plus touchés des prétentions d'un office
de la couronne, que leur amour-propre leur persuadait acquis par leur
mérite, que des prérogatives d'une dignité héréditaire et inhérente à
leur maison.

Le maréchal de Villeroy, malgré tant de raisons personnelles de se
défendre de cette fatuité, en était plus enivré qu'aucun autre. Il parla
au roi\,; et, comme ce fut sans contradicteur, il obtint une lettre de
cachet sur-le-champ, qui enjoignit à ces messieurs de se rendre à la
Bastille ou de recevoir ces mêmes exempts. Ils les reçurent donc, mais
par cet ordre du roi et non par celui des maréchaux de France, et s'en
expliquèrent ainsi en les recevant.

Quelques jours après, les maréchaux de France assemblés leur mandèrent
de venir à leur tribunal\,; le comte d'Harcourt ne se trouva point chez
lui, le duc d'Estrées, qui n'était point sorti alors, refusa de
comparaître. Le maréchal de Villeroy vint crier au roi sur le danger
qu'il n'arrivât quelque chose entre ces messieurs dans la difficulté de
terminer leur affaire, et n'osa jamais parler de leur prétendue
désobéissance. Là-dessus le roi, qui craignit en effet qu'ils ne se
rencontrassent en se dérobant aux exempts, qu'il avait mis auprès d'eux
par lettre de cachet et non de l'autorité des maréchaux de France,
ordonna une nouvelle lettre de cachet à chacun d'eux, portant ordre de
s'aller remettre à la Bastille, sans nulle mention dans ces lettres de
cachet de leur désobéissance ni de l'autorité des maréchaux de France,
et une troisième au gouverneur de la Bastille pour les y recevoir, parce
qu'il n'y peut recevoir personne sans lettre de cachet du roi. Au bout
d'un mois de cette querelle, le roi nomma les maréchaux de Villeroy,
d'Huxelles et de Tessé pour, en qualité non de maréchaux de France mais
de commissaires choisis par lui, terminer l'affaire de ces messieurs.
Ces trois commissaires s'assemblèrent donc à Paris chez le maréchal de
Villeroy, qui envoya une lettre de cachet du roi au gouverneur de la
Bastille pour faire sortir le duc d'Estrées et le comte d'Harcourt, et
les envoyer chez lui tout droit après leur dîner. Comme il ne s'agissait
plus de tribunal ni de la prétendue autorité des maréchaux de France,
mais de celle du roi par ses commissaires nommés pour ce, ces messieurs
obéirent sans difficulté. Aussi n'y parut-il rien de maréchaux de
France. Les commissaires se levèrent et les reçurent avec toute la
civilité possible, ne leur dirent pas un seul mot sur leur prétendue
désobéissance, ni sur la prétendue autorité de l'office de maréchaux de
France, ni de la leur. Le duc et le comte ne leur firent pas aussi la
moindre excuse de ce qu'ils avaient toujours refusé de la reconnaître,
et ne leur dirent pas un seul mot sur tout ce qui s'était passé. Le
maréchal de Villeroy, dès qu'il les eut salués, leur dit tout court
qu'ayant appris, par les informations qu'ils avaient tous trois faites,
que les bruits qui avaient couru dans le monde n'étaient pas véritables,
et les voyant contents l'un de l'autre (sans toutefois leur avoir rien
demandé, ni dit un mot de plus que ce que je rapporte, ni ouï le son de
leur voix), ils n'avaient qu'à les prier, et non ordonner, de
s'embrasser et de vivre en amitié. Ils s'embrassèrent à l'instant, et
toujours en parfait silence. Aussitôt après le maréchal de Villeroy
ajouta que les bruits de leur querelle avaient été grands\,; que si dans
la suite ils venaient à se brouiller, on ne pourrait s'empêcher de
regarder cette brouillerie comme une suite de la première, et que le roi
leur défendait toute voie de fait, sans parler d'eux-mêmes. Il les pria
tout de suite (pria et non ordonna) de s'embrasser encore\,; ils le
firent et se retirèrent aussitôt avec le même silence et force civilités
des trois maréchaux commissaires, auxquelles ils ne répondirent qu'en
les saluant. Ils allèrent de là où bon leur sembla en pleine liberté, et
on n'a pas ouï parler d'eux depuis.

On ne se jettera pas ici dans une longue parenthèse pour montrer combien
la prétention des maréchaux de France est destituée de raison, qu'elle
n'a jamais eu lieu avec tous leurs efforts, et qu'elle n'était tombée
dans l'esprit de pas un d'eux avant plus du milieu du règne de Louis
XIV. Ce serait aussi perdre le temps que de vouloir montrer la
différence entière de la dignité de pair, de celle même de duc, d'avec
l'office de maréchal de France. L'évidence en saute aux yeux\,; elle se
voit en tout et partout\,; les maréchaux de France eux-mêmes n'ont
jamais imaginé de s'y comparer\,; et si à la guerre les maréchaux de
France effacent en tout les ducs, l'argument est trop fort pour, avoir
jamais été proposé, puisque les princes du sang eux-mêmes n'y sont pas
exceptés. Personne ne leur conteste tout avantage purement militaire,
mais pour la juridiction attachée à leur office, ils ne sauraient
montrer qu'ils aient seulement pensé d'y soumettre les ducs avant le
milieu du règne de Louis XIV, et la confusion que les ministres de ce
prince lui inspirèrent de jeter pour abaisser toute hauteur, et sous
prétexte de son autorité, pour établir la leur et se tirer de leur néant
pour arriver ainsi par degrés où on les voit aujourd'hui parvenus, en
quoi le nombre de ces quatorze ducs et pairs, puis des quatre autres
ajoutés après à la fin de 1663 et 1665, contribua beaucoup.

Depuis la nouvelle naissance de cette prétention, il s'est trouvé peu
d'exemples d'occasion de vouloir l'exercer. La querelle des ducs
d'Aumont et de La Ferté fut la première\,; les maréchaux de France
n'oublièrent rien pour en profiter. C'était un temps de guerre vive et
heureuse, par conséquent de crédit et de brillant pour eux\,; néanmoins
ils ne purent parvenir à soumettre ces deux ducs à leurs ordres, en
tirer la moindre excuse, ni oser leur faire la plus légère réprimande de
ce qu'ils avaient fait sauter leur degré aux exempts de la
connétablie\footnote{La connétablie était primitivement le tribunal du
  connétable de France. Lorsque Louis XIII eut supprimé cette dignité,
  on donna le nom de connétablie au tribunal où les maréchaux de France
  jugeaient les questions relatives au point d'honneur. --- Il y avait
  une autre connétablie qui connaissait de tous les crimes ou délits
  commis par des gens de guerre dans les camps, pendant les marches ou
  dans les garnisons. Saint-Simon parle ici de la première connnétablie
  ou tribunal des maréchaux.} qu'ils leur avaient envoyés, et qui furent
de plus menacés d'être jetés par les fenêtres, avec des paroles fort peu
décentes pour le tribunal qui les envoyait\,; et l'affaire finit par la
qualité de commissaires du roi, en vertu de laquelle et point du tout de
l'autorité de leur office, les maréchaux de France les accommodèrent
avec force civilités et compliments, les firent embrasser, les
conduisirent, et en toute cette action, dans toute laquelle il ne fut
aucune mention de tout ce qui s'était passé contre leur prétendue
autorité, il n'y eut rien qui sentît la forme de tribunal, ni aucune
autre chose que l'autorité du roi très-modestement exercée en qualité de
ces commissaires.

On a vu dans ces Mémoires une querelle du duc de Lesdiguières avec
Lambert, depuis lieutenant général, dont les maréchaux de France
n'osèrent prendre la moindre connaissance, quoique arrivée en lieu
public à Paris, et qui fut accommodée par le maréchal de Duras seul,
beau-père du duc de Lesdiguières, non comme maréchal de France, mais en
qualité de commissaire du roi.

C'est donc encore ce qui est arrivé ici. Le duc d'Estrées et le comte
d'Harcourt ont si peu été mis à la Bastille pour avoir refusé de
reconnaître la juridiction des maréchaux de France, et de recevoir leurs
exempts, et tellement pour qu'en attendant leur accommodement il
n'arrivât rien entre eux, que, s'il en eût été autrement, le tribunal
n'eût pas manqué d'user de son droit\,; comme il est arrivé tant de fois
quand des personnes soumises à leurs ordres par état y ont été
réfractaires, et de les envoyer arrêter avec main-forte, et conduire au
For-l'Évêque\footnote{Le For-l'Évêque (\emph{forum episcopi}) était
  primitivement le siége de la juridiction de l'évêque de Paris. Ce
  bâtiment fut transformé plus tard en prison, et enfin détruit en 1780.}
qui est la prison de leur tribunal. Ici il fallut avoir recours à
l'autorité du roi, qui, bien loin de livrer ces messieurs à celle des
maréchaux de France, fit expédier une lettre de cachet à chacun des deux
querellants et une troisième au gouverneur de la Bastille\,: aux uns
pour se rendre, à l'autre pour les recevoir à la Bastille, qui est la
prison particulière où il n'entre et ne sort personne sans un ordre du
roi immédiat, qui en fit expédier de pareils pour les en faire sortir,
sans la moindre mention par conséquent des maréchaux de France\,; et si
les exempts leur furent envoyés avant d'aller à la Bastille, les y
conduisirent, et les en accompagnèrent immédiatement depuis la Bastille
jusque chez le maréchal de Villeroy, le premier des trois commissaires
du roi, ce fut uniquement pour qu'il n'arrivât rien entre eux pendant
ces intervalles. D'ailleurs, de sept ou huit maréchaux de France qui
étaient lors dans Paris, où même le maréchal de Montesquiou était revenu
de Flandre pour n'y plus retourner, et M. de Tingry allé en sa place
pour y commander comme lieutenant général du pays, il n'y eut que trois
maréchaux de France nommés par le roi pour être ses commissaires\,; et
par conséquent leur prétendue juridiction de maréchaux de France n'y fut
pour rien, puisque les autres maréchaux de France furent exclus, et que
ces trois-là même n'agirent en rien dans cette affaire par l'autorité de
leurs offices, mais uniquement par celle du roi comme ses commissaires
nommés pour cela. Aussi nulle forme de tribunal ordinaire chez le
maréchal de Villeroy\,: ni le maître des requêtes rapporteur devant eux,
ni le secrétaire du tribunal ne s'y trouvèrent, ni l'arrangement et
l'ordre accoutumé, ni même le jour ordinaire\,: on affecta de choisir le
dimanche. Aussi pas la moindre mention de l'autorité des maréchaux de
France, pas la plus imperceptible réprimande de l'avoir méprisée, et de
ne l'avoir pas voulu reconnaître, pas la moindre idée d'excuse à cet
égard, et quand le maréchal de Villeroy leur défendit les voies de fait
et les fit embrasser, il leur dit que le roi leur défendait les voies de
fait, et non pas le prononcé ordinaire, qui est\,: «\,Nous vous
défendons,\,» et de même «\,Nous vous ordonnons de vous embrasser,\,»
etc.\,; mais\,: «\,Nous vous prions,\,» parce qu'alors ils n'y mettaient
pas l'autorité du roi comme à la défense des voies de fait, et ils
parlaient d'eux-mêmes comme commissaires du roi\,: toutes différences
entières qui effacent leur autorité et ne laissent que celle du roi. Ils
leur firent après force civilités\,; le maréchal d'Huxelles, qui le
premier avait pris connaissance de la querelle, et envoyé les exempts,
ne fut pas des commissaires\,; en un mot, {[}il n'y eut{]} quoi que ce
soit en cet accommodement qui ait senti le maréchal de France.

Bien est vrai que les fils de France ou les princes du sang ont souvent
accommodé ces sortes de querelles, quand, par la qualité de l'une des
personnes, elles passaient le pouvoir des maréchaux de France. Monsieur,
M. le duc d'Orléans, M. le Prince père et fils, et d'autres princes du
sang l'ont fait plus d'une fois, et d'ordinaire à la chaude. Mais en
cette occasion M. le duc d'Orléans n'était à aucune portée du roi de se
mêler de rien\,; tous les princes du sang étaient d'un âge à ne le
pouvoir faire\,; et les bâtards n'en étaient pas encore là, quelque
proches qu'ils s'en vissent. Il fallut donc bien recourir à la voie des
commissaires\,; et, dès que c'étaient des commissaires du roi nommés par
lui, et qui n'agirent qu'en cette qualité unique, il n'importait plus
qu'ils fussent pris d'entre les maréchaux de France, puisque cet office
demeurait muet et impuissant en eux, et qu'il y disparaissait en entier
sous le nom et par l'autorité de la commission personnelle, qui ne leur
permit plus d'agir que par celle de leur commission.

Personnes de plus haut parage sans comparaison que le duc d'Estrées et
le comte d'Harcourt avaient bien eu des maréchaux de France pour
commissaires du roi, et en chose où une satisfaction ne se pouvait
éviter plus ou moins grande. On voit par les Mémoires de Mademoiselle ce
qui lui arriva avec Madame, qui était sa belle-mère, et qui partageait
avec elle le palais de Luxembourg, où elles logeaient ensemble, et se
haïssaient parfaitement. La querelle fut poussée au point que
Mademoiselle arracha le bâton des mains d'un officier des gardes de
Madame, le cassa contre son genou à deux mains, et lui en jeta les
morceaux au visage, devant un grand monde, à la vue et dans
l'appartement de Madame, et avec des paroles d'un grand mépris pour
Madame. Il était tout naturel que le roi lui-même réglât une affaire si
éclatante et si grave entre sa cousine germaine et la veuve du frère du
roi son père, d'autant plus qu'il n'y avait personne en autorité de s'en
mêler, ni qui de plus osât le prétendre. Je n'ai point su ce qui en
empêcha le roi, si ce n'est d'éviter les importunités qu'il aurait eues
de ces princesses\,; mais il les renvoya au vieux maréchal d'Estrées,
père du cardinal, qu'il nomma son commissaire pour juger et accommoder
cette affaire, et Mademoiselle raconte elle-même dans ses Mémoires tout
ce qu'il s'y passa, les peines que cela lui donna, et la satisfaction
que le maréchal d'Estrées ordonna, et que Mademoiselle fit à Madame,
telle que le maréchal la prescrivit, à son grand dépit, et dont Madame,
aussi au sien, fut obligée de se contenter, qui la prétendait plus
grande, avec défenses à l'une et à l'autre, et à leurs officiers, etc.
On ne pensera pas sans doute que les maréchaux de France aient ni
prétendent avec autorité et juridiction sur les fils et filles de
France, parce {[}que{]} ce que le roi devait et pouvait naturellement
décider lui-même entre elles, il le renvoya à juger à un maréchal de
France, en qualité de son commissaire. Qu'il y en ait un ou plusieurs,
ce sont toujours des commissaires qui agissent comme tels, et non comme
maréchaux de France, et on a vu que le maréchal de Duras fut nommé seul
commissaire pour accommoder la querelle du duc de Lesdiguières, duquel
même il était beau-père, et le logeait chez lui.

En voilà bien assez sur une chose aussi évidente que le peu de fondement
de la prétention des maréchaux de France, sa très-récente nouveauté, et
la nullité entière de son exercice. J'ajouterai seulement qu'outre les
Mémoires de Mademoiselle, je l'ai ouï conter à mon père, qui était fort
son serviteur, et à bien des contemporains, dans ma jeunesse, avec des
circonstances peu agréables, qu'il m'a paru qu'elle avait supprimées. Ce
qui est certain, c'est que le maréchal d'Estrées manda chez lui les
principaux officiers de Madame, et que Mademoiselle alla chez lui
plusieurs fois là-dessus\,: et le tout sans que le roi ait en tout cela
parlé lui-même.

Venons maintenant à une autre sorte de querelle, ou plutôt à ce qui la
produisit, et qui oblige à reprendre les choses de plus haut.

\hypertarget{chapitre-xix.}{%
\chapter{CHAPITRE XIX.}\label{chapitre-xix.}}

1713

~

{\textsc{Proposition de mariage conduite par M\textsuperscript{lle} de
Conti entre une fille de M. le duc d'Orléans et M. le prince de Conti.}}
{\textsc{- M\textsuperscript{lle} de Conti, accusée de faire manquer le
mariage pour son intérêt, en est irréconciliablement brouillée avec
M\textsuperscript{me} la duchesse de Berry.}} {\textsc{-
M\textsuperscript{me} la Princesse fait ordonner par le roi le double
mariage de M. le Duc avec M\textsuperscript{lle} de Conti, et de M. le
prince de Conti avec M\textsuperscript{lle} de Bourbon.}} {\textsc{-
Présent ordinaire du roi aux princes et princesses du sang qui se
marient.}} {\textsc{- Fiançailles, mariage, festin, chemises et visites
du double mariage de M. le Duc et de M. le prince de Conti.}} {\textsc{-
Mauvais ménage du prince et de la princesse de Monaco.}} {\textsc{-
Grâces très-insolites accordées à M. de Monaco pour la transmission de
son duché-pairie.}} {\textsc{- Mariage du fils du comte de Roucy proposé
avec M\textsuperscript{lle} de Monaco, que M\textsuperscript{me} de
Monaco rompt avec éclat\,; {[}elle{]} vient à Paris et à la cour, où
elle trouve peu d'agréments.}} {\textsc{- Mariage du duc d'Olonne avec
M\textsuperscript{lle} de Barbezieux.}} {\textsc{- Mariage de
Pontchartrain avec M\textsuperscript{lle} de Verderonne, où le
chancelier me force d'assister.}} {\textsc{- Mort de la comtesse de
Prado.}} {\textsc{- Extraction et fortune des Prado.}} {\textsc{- Mort
de la duchesse d'Angoulême, veuve du bâtard de Charles IX.}} {\textsc{-
Mort de l'évêque de Rosalie\,; sa famille\,; sa vie.}} {\textsc{- Mort
de l'abbé Régnier.}} {\textsc{- Changement de charges chez Madame.}}
{\textsc{- Beauvau archevêque de Toulouse.}} {\textsc{- Amusements du
roi chez M\textsuperscript{me} de Maintenon.}} {\textsc{- Audience de
congé du duc et de la duchesse de Shrewsbury, à Marly, tout à fait
inusitée.}}

~

M\textsuperscript{lle} de Conti était amie de M\textsuperscript{me} la
duchesse de Berry dès leur jeunesse, quoique la première eût six ans
plus que l'autre. Elles se voyaient souvent. Leur séjour de Paris y
contribuait. Les filles de M\textsuperscript{me} la Duchesse étaient
élevées à Versailles, et il n'y avait jamais eu d'amitié entre
M\textsuperscript{me} la Duchesse et M\textsuperscript{me} la princesse
de Conti sa belle-sœur. Il y avait bien longtemps aussi qu'elle était
éteinte entre M\textsuperscript{me} la duchesse d'Orléans et
M\textsuperscript{me} la Duchesse, tellement que, outre l'éloignement
des lieux, leurs enfants n'étaient pas pour vivre ensemble.
M\textsuperscript{lle} de Conti menait une vie fort contrainte\,;
M\textsuperscript{me} sa mère avait de l'humeur et tenait quelque chose
de M. le Prince son père. M\textsuperscript{me} la Princesse, à qui feu
M. le prince de Conti était attaché d'un tendre respect, l'avait fort
aimé, et elle chérissait M\textsuperscript{lle} de Conti avec d'autant
plus de tendresse que M. le prince de Conti l'avait toujours aimée avec
passion, et lui en avait laissé de grandes marques par son testament.
C'était donc M\textsuperscript{me} la Princesse qui était l'appui et la
consolation de M\textsuperscript{lle} de Conti, qui avait en elle toute
confiance, qui versait dans son sein toutes ses peines, mais chez qui,
par son âge, sa dévotion et son genre de vie, elle ne pouvait pas
trouver d'amusement. La connaissance faite avec Mademoiselle lui en
procura par de petites parties à Paris et à Saint-Cloud, et l'amitié se
lia tellement entre elles qu'elle subsista depuis le mariage de
M\textsuperscript{me} la duchesse de Berry, qui lui sut un gré infini de
la joie qu'elle en eut, et qu'elle ne cacha point malgré le dépit public
de M\textsuperscript{me} la Duchesse et de ses filles, de
M\textsuperscript{me} la princesse de Conti sa tante, et de celui même
que M\textsuperscript{me} la Princesse en voulut bien prendre, en quoi
elle fut autorisée par M\textsuperscript{me} sa mère\,; la seule
princesse du sang qui en fut bien aise. Cela serra encore les liens de
leur amitié, tellement que M\textsuperscript{lle} de Conti, qui ne
paraissait presque jamais à Versailles, y venait quelquefois pour
M\textsuperscript{me} la duchesse de Berry, laquelle aussi lui donnait
souvent des rendez-vous et des collations à Saint-Cloud.

Ces dispositions de la mère et de la fille firent naître la pensée à
M\textsuperscript{me} la duchesse d'Orléans de faire sonder
M\textsuperscript{lle} de Conti, par M\textsuperscript{me} la duchesse
de Berry, sur le mariage d'une de M\textsuperscript{lle}s ses sœurs avec
M. le prince de Conti son frère, et si cela prenait, de se servir d'elle
auprès de M\textsuperscript{me} sa mère pour le faire réussir. M. le duc
d'Orléans approuva ce dessein. Pour moi je le trouvai hasardeux, parce
qu'il me semblait difficile d'obvier à tous les hasards qui pouvaient
instruire le roi de ces démarches, et que, jaloux au point où il l'était
de disposer seul de tout dans sa famille, et parmi les princes du sang,
non-seulement il romprait le mariage, mais disposé aussi mal qu'il
l'était alors à l'égard de M. le duc d'Orléans et de
M\textsuperscript{me} la duchesse de Berry, ils s'exposeraient tous aux
suites de son mécontentement et du déplaisir qu'il aurait, et où il
serait poussé de reste à leur faire sentir qu'il ne faisait pas bon
traiter des mariages à son insu. M\textsuperscript{lle} de Chartres,
belle et bien faite, avait alors quinze ans, mais elle était extrêmement
bègue, et montrait déjà quelque goût pour se faire religieuse.
M\textsuperscript{lle} de Valois, parfaitement belle, mais plus grasse,
en avait treize, et on aurait laissé choisir entre les deux. Mes
réflexions n'arrêtèrent ni M. {[}le duc{]} ni M\textsuperscript{me} la
duchesse d'Orléans, à qui ces princesses commençaient à peser, et qui
étaient suivies de trois autres. M\textsuperscript{me} la duchesse de
Berry parla à Saint-Cloud à M\textsuperscript{lle} de Conti, qui parut
ravie de la proposition et de ce qu'on s'adressait à elle. Elle en
rendit compte à M\textsuperscript{me} sa mère, qui goûta fort la chose.
M\textsuperscript{lle} de Conti, qui avait promis un secret sans
réserve, en fit confidence à M\textsuperscript{me} la Princesse. Elle
avait vingt-cinq ans. Elle se lassait cruellement d'être tenue comme une
petite fille dans l'ennui et les humeurs de l'hôtel de Conti, et elle
n'y voyait par son âge d'autre débouché que d'épouser M. le Duc, à quoi
l'aigreur extrême du procès de la succession de M. le Prince ne
disposait pas M\textsuperscript{me} la Duchesse ni M\textsuperscript{me}
la princesse de Conti. Elle avait beaucoup d'esprit et de douceur,
d'agrément et d'insinuation dans l'esprit. Elle avait un beau visage\,;
mais sa taille, quoique assez grande, n'y répondait pas.

De cette confidence, il résulta que M\textsuperscript{me} la Princesse,
qui avait jusqu'alors fait des efforts inutiles pour porter ses enfants
à s'accommoder sur la succession de M. le Prince et à se raccommoder
ensemble, ouvrit tout à coup les yeux à un moyen fort naturel auquel
elle n'avait point pensé jusque-là, qui fut un double mariage entre ses
petits-enfants. De les y porter par elle-même, elle n'en pouvait
attendre aucun succès\,; mais pensa que le roi, qui avait tâché aussi de
les empêcher de plaider et de les raccommoder, et qui s'en était bien
voulu entremettre plus d'une fois, pourrait être susceptible d'un
expédient si convenable en lui-même, et qui partait naturellement à
éteindre les aigreurs et à engager un accommodement sur le testament de
M. le Prince, et que le roi serait d'autant plus porté à leur imposer
pour faire faire le double mariage, qu'il serait sûrement blessé
d'apprendre, par une voie étrangère, qu'il y avait des pourparlers
très-avancés d'un mariage de M. le prince de Conti avec une fille de M.
le duc d'Orléans. Je n'entreprendrai point de percer un mystère qui se
passa tête à tête entre M\textsuperscript{lle} de Conti et
M\textsuperscript{me} la princesse sa grand'mère. Ce qui est certain,
c'est que les apparences ne parurent pas pour M\textsuperscript{lle} de
Conti, qui trahit le secret qu'elle avait promis. M\textsuperscript{me}
la Princesse n'avait jamais passé pour avoir de l'esprit ni de la
résolution. Son état et sa vertu la faisait respecter extérieurement
dans sa famille\,; son peu de lumière et de force l'y faisait mépriser
en effet\,; en sorte qu'avec des millions dont elle était maîtresse
absolue de disposer comme elle eût voulu par la nature des biens, et par
les lois et les coutumes, elle ne laissa pas d'être toujours comptée
pour rien, et de n'influer pas le moins du monde sur quoi que ce soit
dans sa famille. Sa timidité était extrême avec le roi\,; elle en avait
à l'égard de tout le monde, et de tous ses enfants. M. le Prince l'avait
matée jusqu'à l'avoir abrutie, et la disposition naturelle y était
entière. Il est donc très-difficile d'imaginer qu'elle ait pris
d'elle-même, et subitement, la vue d'un double mariage sûrement à faire
malgré les mères veuves, et dans la plus vive aigreur l'une contre
l'autre, qui de plus ne s'étaient jamais aimées\,; de rompre pour cela
avec la même violence un mariage goûté et comme arrêté\,; et d'opérer
tout cela par l'autorité absolue du roi sans nul autre instrument auprès
de lui qu'elle-même\,; tandis que M\textsuperscript{lle} de Conti
faisait par là le plus grand mariage qu'elle pût espérer, et l'unique
auquel son âge et sa naissance lui pussent permettre d'arriver, et
d'espérer de ne passer pas le reste de sa jeunesse dans l'ennui et dans
l'esclavage sous lequel elle se désespérait.

La résolution prise par M\textsuperscript{me} la Princesse d'aller
parler au roi, M\textsuperscript{lle} de Conti se trouva bien
embarrassée pour se tirer d'affaires avec M\textsuperscript{me} sa mère
et avec M\textsuperscript{me} la duchesse de Berry. Entre la résolution
et l'exécution il n'y eut qu'un point, parce qu'il était à craindre que,
les choses avancées autant qu'elles l'étaient entre M. {[}le duc{]} et
M\textsuperscript{me} la duchesse d'Orléans et M\textsuperscript{me} la
princesse de Conti, ils n'en parlassent au roi, et que, le mariage une
fois agréé, il n'y eût plus de remède. M\textsuperscript{lle} de Conti
demanda donc un rendez-vous à M\textsuperscript{me} la duchesse de Berry
à Saint-Cloud, pour chose fort pressée, pour le lendemain de son
message, qu'elle n'envoya que tard. Toutes deux partirent de Versailles,
et de Paris pour Saint-Cloud, en même temps que M\textsuperscript{me} la
Princesse pour Versailles, afin que celle-ci ne pût être gagnée de la
main auprès du roi par M. le duc d'Orléans averti.

Je ne sais comment M\textsuperscript{lle} de Conti tourna son discours à
Saint-Cloud\,; mais il fallut bien avouer au moins qu'elle n'avait pas
gardé le secret qu'elle avait promis, et par là tout au moins elle était
cause de la résolution que M\textsuperscript{me} la Princesse avait
prise, et de la promptitude avec laquelle elle l'exécutait. Il n'en
fallut pas davantage pour persuader à M\textsuperscript{me} la duchesse
de Berry que M\textsuperscript{lle} de Conti ne s'était servie de la
confiance qu'elle avait eue en elle que pour en profiter pour elle-même,
en violant son secret et en poussant M\textsuperscript{me} la Princesse
à une démarche dont la force et la promptitude lui ressemblaient si peu,
et dont tout le fruit était pour M\textsuperscript{lle} de Conti. Elle
ne lui cacha pas ce qu'elle en pensait, et la traita avec toute
l'indignité et toute la hauteur qu'elle crut qu'elle méritait. Les
larmes de colère et de dépit allongèrent la visite plus que les
discours. Jamais M\textsuperscript{me} la duchesse de Berry ne lui a
pardonné, et s'est piquée jusqu'à la mort de lui faire sentir en toute
occasion publique, car de particulières il n'y en eut plus entre elles,
tout le poids de sa haine, de son mépris et de son rang. Elle rendit à
M. {[}le duc{]} et à M\textsuperscript{me} la duchesse d'Orléans ce
qu'elle venait d'apprendre. Tous trois comprirent aussitôt qu'il n'y
avait plus à compter sur leur mariage, et furent bien en peine du
silence qu'ils en avaient gardé au roi.

M\textsuperscript{me} la Princesse, tout en arrivant à Versailles, fit
dire au roi qu'elle le suppliait de lui marquer un moment où elle pût
avoir l'honneur de lui rendre compte en particulier de quelque chose qui
pressait fort, et qui était très-important à sa famille. Le roi ne la
fit pas attendre, et la manda dans son cabinet. L'audience fut longue\,;
je n'en dirai rien\,; mais, si on en ignora le détail, on sut bientôt
que le roi s'était fort offensé d'avoir appris un mariage arrêté dans sa
famille, sans qu'aucune des parties lui en eût dit un mot, qu'il trouva
que M\textsuperscript{me} la Princesse avait raison d'être piquée de son
côté du secret que lui en faisait M\textsuperscript{me} sa fille, et que
sur-le-champ le double mariage fut décidé. Le roi désirait d'autant plus
ardemment de pouvoir remettre la paix dans cette famille, que l'aigreur
y était parvenue au plus haut degré, parce qu'il prévoyait sagement que
M. du Maine y serait toujours la partie faible, et que cette paix lui
était d'une plus grande importance que ne pouvaient être les biens qu'il
tirerait par des arrêts.

Dans cette résolution bien arrêtée, il lava la tête rudement dès le soir
même à M. {[}le duc{]} et à M\textsuperscript{me} la duchesse d'Orléans,
et à M\textsuperscript{me} la duchesse de Berry, et leur défendit de
penser davantage à un mariage qu'ils avaient osé non-seulement penser,
mais fort avancer sans lui en avoir parlé, et su s'il l'aurait agréable.
Ce même soir, il parla à M\textsuperscript{me} la Duchesse en père, mais
en maître qui veut être obéi sans réplique, sur le mariage de son fils
avec M\textsuperscript{lle} de Conti, et de sa fille aînée avec M. le
prince de Conti, dont M\textsuperscript{me} la Duchesse fut d'autant
plus étourdie qu'elle ignorait parfaitement l'autre mariage si prêt à
faire, et ce que M\textsuperscript{me} la Princesse était venue faire à
Versailles. M\textsuperscript{me} la princesse de Conti fut mandée à
Paris. Le roi la vit dans son cabinet, et trouva en elle la plus ferme
résistance. Elle dit au roi qu'il fallait que les procès fussent jugés
avant qu'elle pût entendre à rien\,; que de plus on lui avait fait
d'autres propositions très-convenables pour M\textsuperscript{lle} sa
fille, dans lesquelles elle était entrée\,; qu'enfin
M\textsuperscript{lle} de Bourbon n'avait point de bien. Le roi discuta
avec elle, il prit toutes sortes de tons\,; puis, voyant qu'il
n'avançait pas davantage, il parla en roi et en maître, et déclara à
M\textsuperscript{me} la princesse de Conti qu'il voulait le double
mariage, qu'il le voulait présentement, et qu'il les ferait tous deux
malgré elle, si elle ne se rendait pas à sa volonté, à la raison et à
tous les ménagements qu'il voulait bien avoir pour elle. Elle sortit en
furie du cabinet du roi, et s'en alla tout de suite à Paris, où elle se
retrancha sur les difficultés, et où M\textsuperscript{lle} de Conti
passa cruellement son temps jusqu'à son mariage.

M. le prince de Conti n'eut aucun tort dans le cours de cette affaire.
Il était élevé dans la haine des Condé\,; il fut fâché de la rupture de
son mariage avec une fille de M. le duc d'Orléans, et fâché aussi
d'épouser celle de M\textsuperscript{me} la Duchesse, que cet
établissement ne consola pas d'avoir, comme on l'a vu, manqué M. le duc
de Berry, après tant de soins, de menées et de cabales, quoique la mère
et la fille ne fussent pas insensibles au dépit de M. {[}le duc{]} et de
M\textsuperscript{me} la duchesse d'Orléans, et à celui de
M\textsuperscript{me} la duchesse de Berry, de se voir enlever avec
hauteur pour elles le parti dont ils se tenaient assurés.

M\textsuperscript{me} la Princesse, ravie d'un si prompt et si entier
succès, se tint à Versailles à tout événement, et vit le roi plusieurs
fois tête à tête, pour rompre les difficultés dont M\textsuperscript{me}
sa fille se hérissait, et pour presser la conclusion. Le roi lui envoya
plusieurs fois Pontchartrain, qui par son ordre employa à la fin les
menaces. Elles eurent leur effet, et on envoya à Rome pour les
dispenses, tandis qu'on se mit à travailler aux contrats de mariage. La
négociation fut fort courte. Le roi voulut que ces mariages fussent
faits et consommés avant que M. le Duc et M. le prince de Conti
partissent pour l'armée d'Allemagne. Il en coûta cinq cent mille livres
au roi, qui donne toujours cent cinquante mille livres à chaque prince
du sang qui se marie, et à chaque princesse du sang qui se marie cent
mille livres.

Enfin les deux fiançailles se firent le samedi 8 juillet, sur le soir,
dans le cabinet du roi, par le cardinal de Rohan, revenu exprès de
Strasbourg, où il ne faisait que d'arriver. M\textsuperscript{me} la
Duchesse et M\textsuperscript{me} la princesse de Conti n'y firent prier
que les parents, mais jusqu'à un degré assez étendu. La foule ne laissa
pas d'y être grande de tout ce qui ne l'avait pas été.
M\textsuperscript{lle} de Charolais et M\textsuperscript{lle} de La
Roche-sur-Yon portèrent la queue de la mante des deux fiancées. Le
lendemain dimanche 9, le cardinal de Rohan dit la messe à midi dans la
chapelle, en présence du roi et de toute la cour, et il y maria les deux
princes et les deux princesses, qui furent mis tous quatre sous le même
poêle. Il n'y eut point de dîner ni de plaisirs. Le soir, toute la
maison royale, tous les princes et princesses du sang, M. et
M\textsuperscript{me} du Maine et leurs deux fils, et M. le comte de
Toulouse, soupèrent avec le roi chez lui. Il passa avec eux tous dans
son cabinet, au sortir de table\,; et un quart d'heure après il
descendit dans l'appartement de feu M. le Prince, que
M\textsuperscript{me} la Princesse avait conservé entier, et qui était
double. Les deux noces y couchèrent\,; le roi donna la chemise aux deux
mariés, et M\textsuperscript{me} la duchesse de Berry aux deux mariées.
Ce ne fut pas sans prodiguer à l'une des deux ses plus perçants dédains.
Le lendemain lundi, après dîner, le roi retourna au même appartement
voir les deux mariées chacune sur son lit, où toute la cour abonda le
reste de la journée. Dès le soir M. le prince de Conti entra après le
souper dans le cabinet du roi, jusqu'à son coucher, comme mari de sa
petite-fille, privilége attaché uniquement à cette qualité. M. le Duc
avait près de quatre ans moins que sa nouvelle épouse, et M. le prince
de Conti deux moins que la sienne. De cette affaire
M\textsuperscript{me} la princesse de Conti demeura indignée contre sa
fille, outrée contre M\textsuperscript{me} la Princesse, plus aigrie que
jamais contre M\textsuperscript{me} la Duchesse, de plus en plus
attachée à suivre les procès et à ne vouloir pour rien ouïr parler
d'aucun accommodement, et en amitié liée et publique avec M. {[}le
duc{]} et M\textsuperscript{me} la duchesse d'Orléans et avec
M\textsuperscript{me} la duchesse de Berry. Un mariage moins important
fit aussi bien du désordre et de l'éclat. Ce fut celui de la fille aînée
de M. de Monaco avec le fils aîné du comte de Roucy. M. de Monaco avait,
comme on l'a vu en son lieu, épousé autrefois une fille de M. le Grand,
pour obtenir le rang de prince étranger. Il l'avait eu\,; mais, dès
l'instant du mariage, son père et M. le Grand s'étaient fort brouillés,
comme on l'a vu aussi en même temps, et peu après le mari et la femme
avaient fort mal vécu ensemble. À la fin elle avait été emmenée à Monaco
une première fois, d'où on a vu aussi qu'elle s'était tirée par la plus
abominable calomnie contre son beau-père. Celui-ci étant mort quelques
années après ambassadeur à Rome, son fils, qui prit le nom de prince de
Monaco, y remena sa femme, et l'y tint avec lui bien des années. Le
ménage n'en fut pas plus concordant\,; la vie de Monaco, avec un mari
qu'on n'aima jamais, était bien différente de la vivacité de la vie et
des plaisirs de la cour, et de la maison ouverte et magnifique de M. le
Grand. Elle demeura même quelquefois seule pendant quelques courts
voyages que M. de Monaco faisait à Paris et à la cour.

Il n'avait que des filles\,; il n'espérait plus avoir d'enfants, et son
unique frère était prêtre. Sa branche finissait en eux, et le
duché-pairie de Valentinois s'y éteignait. Il chercha donc à faire un
mariage pour sa fille aînée, qui plût au roi, dont il se proposa
d'obtenir la continuation de sa dignité pour sa fille, et le roi ne s'y
rendit pas difficile. Il lui promit une nouvelle érection avec le rang
d'ancienneté de cette nouvelle date pour celui qui épouserait sa fille
aînée, et la permission de se démettre de son duché en sa faveur dès le
moment du mariage pour que sa fille, qui depuis ce rang de prince était
assise, ne se trouvât pas debout. Dès que cela fut enfilé de la sorte,
M. de Monaco représenta qu'encore qu'il ne pût espérer d'autres enfants,
et que son âge et bien plus sa santé ne lui dût pas faire envisager de
survivre à sa femme, ce cas néanmoins pouvait arriver\,; qu'alors la
grâce extraordinaire que le roi lui accordait lui deviendrait bien
amère, parce qu'elle lui ôterait le moyen de continuer sa dignité dans
sa postérité en se remariant, et ayant un fils, cas même qui au fond
serait embarrassant pour son gendre par les règles du droit. Le roi, qui
avait commencé à le favoriser dans ses dispositions domestiques, voulut
bien encore ajouter une grâce bien plus singulière. Il lui promit un
clause dans l'érection nouvelle qui se ferait en faveur du gendre qu'il
chaisirait qu'advenant la mort de M\textsuperscript{me} de Monaco, un
second mariage de M. de Monaco, et qu'il en eût un fils depuis le
mariage de sa fille, ce fils lui succéderait en la dignité et en
l'ancienneté de son duché-pairie de Valentinois, et pour sa postérité,
auquel cas son gendre demeurerait sa vie durant duc et pair, mais que sa
dignité demeurerait éteinte en sa personne, et ne passerait pas aux fils
de son mariage avec sa fille. M. de Monaco, plus comblé qu'il n'avait
osé l'espérer, se mit à chercher pour sa fille un parti qui fût agréable
au roi, et qui lui convînt à lui-même, et en fut d'autant plus pressé
que ces grandes et insolites grâces ne pouvaient s'exécuter, ni même
s'expédier, qu'en faisant actuellement le mariage de sa fille, et qu'il
lui était important de les faire consommer par celui qui les lui
accordait.

Le monde en fut bientôt informé, et ce fut à qui pourrait se faire duc
et pair par ce mariage, Le comte de Roucy y pensa des premiers pour son
fils. Le chancelier, à qui la mémoire de sa belle-fille était toujours
infiniment chère, l'y servit de tout son pouvoir, MM. de La
Rochefoucauld et de La Rocheguyon de même, il fit agir tous ses amis, et
il gagna M. de Monaco, Le roi ne voulut pas s'en mêler, mais témoigna
approuver et avoir ce mariage très-agréable. Pour venir au contrat, il
fallut venir à M\textsuperscript{me} de Monaco, parce qu'il fallait
qu'elle y parlât, et que, par la disposition des affaires de M. de
Monaco, on ne s'y pouvait passer d'elle. Enragée comme elle était contre
lui, c'en fut assez qu'il voulût ce mariage pour qu'elle refusât d'y
consentir. Le besoin qu'on eut d'elle dressa vers elle toutes les
batteries, et rendit M. de Monaco complaisant. Elle eut peur d'être
forcée par l'autorité de M. le Grand. Elle sembla donc se radoucir et
entrer en examen, tandis qu'elle travailla à le gagner. L'examen lui en
fournit les moyens. On ne marie point ses enfants sans mettre papiers
sur table. Le comte de Roucy avait été toute sa vie un panier percé, la
comtesse de Roucy noyée de dettes et de procès de sa maison. On vit donc
de grandes terres, de grandes dettes, nul ordre, de grands embarras, et
des gens qui avaient toujours vécu d'industrie, de crédit, et de faire
ce qu'on appelle des affaires. D'un autre côté M. de Monaco avait des
terres d'une grande étendue. Valentinois est immense, c'était son duché.
Ni ce morceau ni Monaco ne pouvaient aller qu'à l'aînée\,; il y avait
beaucoup de dettes, quatre filles à pourvoir, et l'abbé de Monaco à
partager qui ne l'était pas encore. M\textsuperscript{me} de Monaco fit
démontrer cela à sa famille, s'assura de son appui, et déclara après que
jamais elle ne consentirait à un mariage qui par l'état et la nature des
biens et des affaires de part et d'autre, se trouvait impossible sans
folie. L'argument était pressant et souffrait peu de réplique. M. le
Grand, avec sa hauteur et sa brutalité ordinaire, s'emporta à la cour\,;
ses enfants, le maréchal de Villeroy, le secondèrent\,; le vacarme fut
très-grand. M. de Monaco de dépit mit sa fille dans un couvent à Aix,
avec défense de la laisser voir à sa mère, qui assurée de sa famille
prit le temps que son mari s'en était allé se dissiper à Gênes, et
arriva à Paris chez M. le Grand.

Elle crut y régner comme du temps de sa mère, et nager comme autrefois
dans les plaisirs de la cour. Elle y fut trompée. M\textsuperscript{lle}
d'Armagnac était devenue la maîtresse de la maison\,; elle se souvenait
des préférences continuelles que sa sœur lui avait fait essuyer du temps
de M\textsuperscript{me} d'Armagnac. M. le Grand reçut
M\textsuperscript{me} de Monaco froidement, et tout d'abord lui déclara
qu'une femme brouillée avec son mari, et qui pour cela venait chez son
père, ne devait pas en sortir un instant, ne faire sa cour au roi que
par devoir et rarement, ne faire aucune visite et n'en recevoir point,
se contenter du grand monde qui abondait chez lui, mais ne point jouer,
ne point se parer, être très-uniment vêtue, et négligemment coiffée, et
s'éloigner régulièrement de toutes parties et de tous plaisirs. Cette
harangue fut moins une remontrance qu'un ordre très-positif, et d'un
père devant lequel tout tremblait dans sa famille. M\textsuperscript{me}
de Monaco n'avait ni équipage, ni domestique, ni un sou pour s'en
donner. Son mari n'était pas pour lui laisser toucher quoi que ce fût,
et M. le Grand aussi peu d'humeur à lui donner plus que le couvert et la
nourriture à sa table. Onze ans de séjour de suite à Monaco l'avaient
changée à n'être pas connaissable\,; elle ne put se le dissimuler à
l'accueil qu'elle reçut à la cour, où elle ne sortit pas de
l'appartement de son père, à y voir régner sa soeur, et y jouer le plus
gros jeu du monde. Elle fit rompre le mariage avec éclat, mais
d'ailleurs elle ne fit que changer d'ennuis et de peines. Nous verrons
bientôt que Matignon en profita.

Un autre mariage se fit avec moins de bruit. Le duc de Châtillon, plus
qu'estropié d'une blessure au pied qui peu à peu lui avait engourdi les
nerfs et l'avait rendu comme paralytique, se démit de son duché à son
fils unique, qu'il fit appeler duc d'Olonne, et le maria à la fille
unique et fort riche que Barbezieux avait laissée de son premier mariage
avec la sœur du duc d'Uzès, dont M\textsuperscript{me} de Louvois fit
magnifiquement la noce.

Il y avait cinq ans au plus que Pontchartrain avait perdu une femme de
tous points adorable, l'unique peut-être qui eût pu avoir la vertu, la
raison, la conduite et l'incomparable patience de l'être de lui, et dont
la considération, comme on l'a vu en son lieu, l'avait soutenu et lui
avait sauvé sa place. Il s'était bientôt lassé de la comédie forcée de
sa douleur, et quoiqu'il eût deux fils, il voulut absolument se
remarier. Sa figure, hideuse et dégoûtante à l'excès, mais agréable, et
même charmante en comparaison de tout le reste, n'empêcha pas la
séduction de l'éblouissement de sa place. M\textsuperscript{lle} de
Verderonne, qui était riche, et qui était L'Aubépine comme ma mère, mais
parente éloignée, en voulut bien.

Le chancelier, qui voyait avec la dernière peine la façon dont je me
conduisais à l'égard de son fils, se mit dans la tête un replâtrage pour
le public, et d'exiger que j'allasse à la noce. Je m'écriai à la
proposition. Il ne se rebuta point. Je m'adressai à la chancelière qui,
là-dessus plus raisonnable que lui, essaya de le persuader\,: tout fut
inutile. Il pria, pressa, conjura, se fâcha, prit le ton d'autorité
qu'il avait sur moi. Finalement nous capitulâmes. Je lui déclarai donc
que la violence qu'il exerçait sur moi par cette complaisance était une
tyrannie\,; que je ne changerais pour son fils ni de disposition, ni de
volonté, ni de projet\,; que je les lui réitérais même, moyennant quoi
je ne voyais pas ce qu'il y avait à gagner ni pour les uns ni pour les
autres, à me traîner à une noce où, par le souvenir de sa première
belle-fille, je ne pourrais être qu'affligé, et où, par ce qui s'était
passé, il était bien difficile que son fils ne se trouvât fort
embarrassé de ma présence, et moi au désespoir de la sienne. Je ne sais
ce que le chancelier imagina, mais il me passa tout, pourvu que
j'allasse à cette noce, que je visse par-ci par-là M. de Pontchartrain,
c'est-à-dire que je ne fisse plus profession de ne point voir son fils,
et de lui tourner le dos partout où je le rencontrais. Il voulut
peut-être lui ôter un dégoût public fort nouveau à sa place, détourner
par là les remarques journalières du monde, et ses raisonnements sur une
conduite à laquelle le chancelier semblait bien consentir, puisqu'elle
n'avait rien changé dans l'intimité, ni dans la continuité de notre
commerce, et par conséquent aggraver les torts de son fils.
{[}J'ignore{]} s'il espéra, en ôtant cette rudesse extérieure, que le
temps nous rapprocherait, émousserait ma haine, mes résolutions, mes
projets\,; quoi qu'il en fût, je ne pus résister au chancelier.

Il n'osa exiger de M\textsuperscript{me} de Saint-Simon la même
complaisance. La mémoire de sa chère cousine était trop avant dans son
cœur pour lui permettre de voir une cérémonie qui la lui rappellerait
d'une manière si touchante. Elle ne put même répondre à tout ce que la
nouvelle femme lui prodigua d'avances\,; la place qu'elle tenait lui fut
insupportable. Elle le lui avoua, et ne la vit presque point.

Pour moi, je fus donc à la noce comme on va à la potence. Elle fut faite
à Pontchartrain avec un très-petit nombre de personnes. L'évêque de
Chartres diocésain les maria. Le chancelier et la chancelière ne
cessèrent d'y pleurer leur première belle-fille\,; ils ne s'en cachèrent
pas même. Les amis et les proches s'en contraignirent peu. Tout le
domestique ne discontinua d'être en larmes. Ce qui s'y trouva du côté de
M\textsuperscript{lle} de Verderonne demeura dans un sombre que les
maussaderies du bel époux ne rassérénèrent pas. Jamais je ne trouvai
deux jours si longs en ma vie.

De si tristes noces font souvenir de la mort, et pénètrent de
réflexions. Aussi apprit-on la mort d'une fille du maréchal de Villeroy,
mariée à Lisbonne au comte de Prado en 1688, dont nous avons vu
longtemps le fils logé, nourri et entretenu de tout très-noblement par
le maréchal de Villeroy, avec lequel il fit quelques campagnes, et
longtemps depuis la paix à Paris. Il s'appelait J. de Souza, et il était
troisième marquis Das Minas, sixième comte de Prado, huitième seigneur
de Beriguel, gentilhomme de la chambre du roi de Portugal, conseiller de
guerre, mestre de camp général dans ses troupes, général de sa
cavalerie, tous grands titres qui s'acquièrent promptement et ne sont
pas grand'-chose. L'entêtement du roi de Portugal pour la grandeur de la
dignité de patriarche de Lisbonne qu'il avait obtenue du pape pour le
siége de cet archevêché dont il fit un colosse, causa l'exil du comte de
Prado et la confiscation du peu qu'il avait, et le réduisit, de peur de
pis pour sa personne, à se sauver de Portugal pour n'avoir pas voulu
arrêter son carrosse devant celui du patriarche dans les rues de
Lisbonne. C'est ce qui le fit venir à Paris. Sa paix faite enfin avec le
roi de Portugal, il retourna à Lisbonne, où peu après il fut assassiné
sortant d'une église, en septembre 1622, par don Juan de La Cueva et
Mendoza. Il n'avait qu'un seul fils qu'il avait perdu depuis quelques
mois sans alliance, et il ne faisait que de commencer à jouir de son
bien. Il n'y avait pas un an que son père était mort.

Ce père, qui s'appelait le marquis Das Minas et avait près de
quatre-vingts ans, est celui qui a toujours commandé l'armée portugaise
contre Philippe V, qui prit force places en Espagne, qu'il garda peu,
entra même dans Madrid, qu'il ne put conserver, et qui commandait une
aile de l'armée de l'archiduc avec dix-huit bataillons portugais à la
bataille d'Almanza, que le duc de Berwick gagna complètement le 25 avril
1707, et qui eut de si grandes suites. Das Minas continua de servir en
chef jusqu'à la paix. Il avait été vice-roi du Brésil, président du
conseil des Indes à son retour, et successivement gouverneur de
plusieurs provinces de Portugal. Son père avait eu un gouvernement de
province, la présidence du conseil des Indes, l'ambassade de Rome. Il
avait été grand écuyer et grand maître des rois Jean IV et Alphonse VI.
Il était la sixième génération directe et masculine de Roderic de Souza,
bâtard de Martin-Alphonse de Souza, fils de Pierre-Alphonse de Souza,
dont le père Alphonse-Denis était bâtard d'Alphonse III, roi de
Portugal, mort en 1279. Ce fut une chose très-rare de voir encore une
belle-fille de Charles IX bâtarde vivre jusqu'en cette année, dans
laquelle elle mourut en ce temps-ci, de vieillesse et de misère. Elle
s'appelait Fr.~de Nargonne. Elle était fille du baron de Mareuil, et
avait eu un frère page du duc d'Angoulême, bâtard de Charles IX. Il
avait épousé, en 1591, la fille aînée du dernier connétable de
Montmorency à Pézénas, dont il ne lui resta qu'un fils qui ne le
survécut que de trois ans, qui a été le dernier duc d'Angoulême. Le
père, veuf de la Montmorency en 1636, devint amoureux de la sœur de son
page, et l'épousa en février 1644. C'était une grande femme parfaitement
belle et bien faite encore quand je l'ai vue, qui avait quelque chose de
doux, mais de majestueux. Elle représentait la dignité et la vertu, qui
fut chez elle sans tache et sans ride en tout genre toute sa vie. M.
d'Angoulême la laissa veuve sans enfants et fort mal pourvue, en 1650.
Il avait près de soixante-dix-huit ans. Son fils ne s'en mit pas fort en
peine, qui mourut à la fin de 1653, à cinquante-sept ans\,; sa veuve
encore moins, qui était La Guiche, fille du grand maître de
l'artillerie, la même dont j'ai parlé au commencement de ces Mémoires,
chez qui ma mère fut élevée et mariée, et qui mourut, en 1682, à
quatre-vingt-quatre ans. Elle ne pouvait supporter une belle-mère, et si
inférieure, après laquelle il fallait passer.

Cette belle-mère était donc fort pauvre et fort abandonnée dans un
appartement d'un couvent de Sainte-Élisabeth à Paris, où elle vivait
d'une pension du roi de vingt mille livres et de fort peu d'autre chose.
Elle venait une fois ou deux l'année à la cour, où sa vertu et sa
conduite la faisait bien recevoir de tout le monde et du roi avec
distinction, mais sans avoir jamais participé à aucun des nouveaux
honneurs comme la duchesse de Verneuil, sous prétexte que la bâtardise
de son mari n'était pas des rois Bourbons. Les malheurs de la guerre,
qui avaient porté tout à l'extrémité, suspendirent le payement des
pensions. M\textsuperscript{me} d'Angoulême eut beau représenter qu'elle
n'avait au monde de subsistance que la sienne, le roi ne fut point
touché de la laisser mourir de faim, dont elle serait très-certainement
morte sans une vieille demoiselle qui lui était attachée depuis
longtemps, et à elle, qui avait un petit bien à douze ou quinze lieues
de Paris. Elle l'y mena, ne pouvant plus payer son couvent ni sa
nourriture, et elle a vécu plusieurs années chez cette demoiselle à ses
dépens, et y est morte sans que le roi, ni ses bâtards, ni les riches
héritiers des deux ducs d'Angoulême, aient pu l'ignorer, et sans qu'ils
en aient eu la moindre honte.

Un autre personnage singulier mourut en ce même temps à Paris, dans le
séminaire des Missions-Étrangères. Il était troisième fils du célèbre
Lyonne, ministre et secrétaire d'État, et il était né à Rome en 1655,
pendant l'ambassade de son père vers les princes d'Italie. Il n'avait
que seize ans quand il le perdit. Son frère, qui avait la survivance du
père, n'en put soutenir seul le poids. Il culbuta presque aussitôt, et
cette famille tomba en désarroi malgré l'alliance du duc d'Estrées qui
ne put la soutenir. La dévotion et le désastre firent prendre à l'abbé
de Lyonne le parti des missions d'Orient. Il fut sacré évêque \emph{in
partibus} de Rosalie. Il travailla plus de vingt ans avec un grand zèle
dans ces pays éloignés, et il acquit une grande connaissance des lettres
et des sciences chinoises. Il revint en France avec les ambassadeurs de
Siam en 1686, et s'en retourna avec eux l'année suivante. De Siam il
passa à la Chine, où il se brouilla fort avec les jésuites sur les
cérémonies chinoises, ainsi que tous les autres missionnaires. Ces
affaires-là le firent revenir à Rome en 1703, pour y soutenir la cause
contre les jésuites. Il y demeura plusieurs années. Il revint de Rome à
Paris, dans le séminaire des Missions-Étrangères, y travailler avec eux
pour la même affaire, et il y mourut dans une vie fort retirée et fort
appliquée, sans avoir quitté le dessein de retourner aux missions, qui
lui avait toujours fait conserver sa grande barbe.

L'abbé Regnier, secrétaire perpétuel de l'Académie française, mourut
aussi à plus de quatre-vingts ans. Il avait un talent particulier pour
les langues et la poésie, et il avait fait quantité de vers français,
latins, espagnols et italiens. Il avait passé presque toute sa vie dans
l'hôtel de Créqui, et il était fort répandu et bien reçu dans les
meilleures compagnies.

Souliers, chevalier d'honneur de Madame, mourut aussi. C'est un Janson,
fort bon homme, et que M\textsuperscript{me} de Maintenon envoyait
quelquefois chercher les après-dînées à Marly, pour venir jouer au
trictrac avec elle. Je ne sais comment cela s'était fait. Il était
l'unique qui eût cette privance, mais il n'en tira aucun parti.
Mortagne, qui était premier écuyer de Madame, passa à la charge de
chevalier d'honneur, et il vendit celle de premier écuyer à un
arrière-Simiane, mais ce ne fut que quelque temps après, parce que le
frère de Souliers, qui était en Provence, eut d'abord la charge de
chevalier d'honneur.

Le roi fut si content de la conduite de Beauvau, évêque de Tournai,
pendant et après le siége de cette place, surtout de ce qu'il n'avait
pas voulu en demeurer évêque depuis la prise, qu'il lui donna
l'archevêché de Toulouse, vaquant par la mort du frère de Villacerf et
de Saint-Pouange. Il passa depuis à Narbonne, et fut avec le marquis de
Beauvau, son frère, de la promotion de l'ordre de 1724.

Les amusements étaient de plus en plus fréquents les soirs chez
M\textsuperscript{me} de Maintenon, où rien ne pouvait remplir le vide
de la pauvre Dauphine. Le duc de Noailles qui, comme on l'a vu, y était
devenu fort étranger, chercha à s'y raccrocher par une idylle dont il
fit faire les paroles par Longepierre, sur la paix, et la musique par La
Lande, maître de la musique de la chapelle. Le roi la fit chanter
plusieurs fois. C'était à Marly, où le voyage fut fort long.

Le duc de Shrewsbury, pressé de retourner en Angleterre, obtint ce qui
ne s'était point fait encore pour aucun autre ambassadeur, ni autre
ministre étranger, et il le regarda comme une grâce. Il vint seul sans
cortége et sans introducteur des ambassadeurs à Marly, comme un
courtisan, dîner chez Torcy, qui lui donna de la part du roi son
portrait enrichi de soixante mille livres de diamants. Il vit le roi le
matin en arrivant, et, seul avec lui dans son cabinet, prit congé. Sa
femme était venue le même jour dîner chez M\textsuperscript{me} la
princesse de Conti, et l'après-dînée elle fut prendre aussi congé du roi
dans son cabinet, et tous deux s'en retournèrent le soir à Paris, d'où
ils partirent, sans avoir pris d'autres congés.

\hypertarget{chapitre-xx.}{%
\chapter{CHAPITRE XX.}\label{chapitre-xx.}}

1713

~

{\textsc{Siége de Landau.}} {\textsc{- La garnison et celle de
Kayserslautern se rendent prisonnières.}} {\textsc{- Biron perd un bras
à Landau et en a le gouvernement.}} {\textsc{- Villars, chevalier de la
Toison d'or, passe le Rhin\,; investit Fribourg.}} {\textsc{- Cardinal
de Bouillon s'achemine des Pays-Bas à Rome.}} {\textsc{- Électeur de
Bavière voit le roi à Marly.}} {\textsc{- Voyage de Fontainebleau par
Petit-Bourg.}} {\textsc{- L'électeur de Bavière y vient passer quinze
jours et retourne à Compiègne.}} {\textsc{- Mariage du prince de
Robecque et de la fille du comte de Solre.}} {\textsc{- Branche de
Robecque de la maison de Montmorency.}} {\textsc{- Fortune du prince de
Robecque en Espagne\,; sa mort, et son frère.}} {\textsc{- Branche de
Solre de la maison de Croï.}} {\textsc{- Origine de cette maison.}}
{\textsc{- MM. de Solre sortis de la branche de Chimay.}} {\textsc{-
Évêque de Cambrai fait duc.}} {\textsc{- Chimère du fils aîné du dernier
comte de Solre.}} {\textsc{- Branche d'Havré de la maison de Croï sortie
de la branche de Solre.}} {\textsc{- Éclat près d'arriver entre le duc
de La Rochefoucauld et moi, arrêté par le duc de Noailles.}} {\textsc{-
Trois mille livres d'augmentation de pension à Saint Hérem.}} {\textsc{-
Douze mille livres d'appointements à Bloin sur la Normandie pour le
gouvernement de Coutances.}} {\textsc{- Comte de La Mothe, rappelé, voit
le roi dans son cabinet.}} {\textsc{- Sage politique du roi sur les
emplois dans les provinces.}} {\textsc{- Naissance de l'infant don
Ferdinand.}}

~

Besons fit le siége de Landau, où Villars vint une fois ou deux se
promener et faire le général. Il commandait l'armée qui couvrait le
siége. La tranchée y fut ouverte la nuit du 24 au 25 juin. Pendant ce
temps-là Dillon alla attaquer Kayserslautern. Six cents hommes et
trente-sept officiers qui le défendaient sous un colonel, se rendirent
prisonniers de guerre. Biron, lieutenant général, aujourd'hui duc et
pair, et doyen des maréchaux de France, y perdit un bras à une grande
sortie, et n'a pas servi depuis. Villars fit cependant force
détachements au long et au large, et à son ordinaire ne s'oublia pas
pour les contributions. Le 19 août on battit la chamade à Landau. On ne
put convenir que le 20. Le prince Alexandre de Wurtemberg, gouverneur,
se rendit avec sa garnison prisonnière de guerre. Il en sortit quatre
mille huit cents hommes, qui furent distribués en la haute Alsace, et le
prince de Wurtemberg eut un congé de trois mois. Il resta douze cents
blessés dans la place, où il ne se trouva plus que vingt milliers de
poudre et soixante pièces de canon, la plupart hors de service. Lutteau,
frère de la maréchale de Besons, apporta la prise au roi, et Valory,
frère de l'ingénieur qui avait conduit les travaux du siége, en apporta
le détail et trente-neuf drapeaux.

Villars eut en même temps la Toison, sans qu'on ait jamais su par où, et
sans avoir eu aucun rapport de guerre ni d'affaires avec l'Espagne.
C'était un homme qui voulait tout, et le plus impudent qu'il fût
possible à se vanter et à demander. La surprise de cette Toison fut
universelle. Il passa le Rhin le 12 septembre, partie au Port-Louis,
partie sur le pont de Strasbourg. Il prit fort aisément les
retranchements que les ennemis avaient faits près de Fribourg, et
incontinent après il investit cette place.

Le cardinal de Bouillon, méprisé au dernier point dans tous les
Pays-Bas, depuis l'étrange mariage qu'il avait fait de sa nièce, et le
procès perdu en conséquence contre la duchesse d'Aremberg, ne savait
plus où se tenir dans ces provinces, après avoir essayé et changé de
divers séjours. Il s'était encore fait moquer de lui par l'air important
qu'il avait pris d'affecter de se tenir à portée d'Utrecht, comme si les
affaires d'un aussi petit particulier que lui eussent pu y être
traitées. Ce prétexte finit à sa confusion, il se retira chez l'évêque
de Ruremonde, d'où, ne sachant plus que devenir, il s'achemina enfin à
Rome par l'Allemagne et le Tyrol, à quatre ou cinq lieues par jour, et
force séjours pour tuer le temps et allonger son voyage.

L'électeur de Bavière arriva de Compiègne en cette petite maison qu'il
avait empruntée à Suresne dans le même temps que le roi apprit la prise
de Landau qu'il lui manda par d'Antin. Il vint quelques jours après, sur
le soir, à Marly, ayant passé la journée à voir jouer les eaux à
Versailles. Il fut quelque temps seul avec le roi dans son cabinet,
soupa chez d'Antin, joua au salon avant et après souper, avec M. {[}le
duc{]} et M\textsuperscript{me} la duchesse de Berry, et s'en retourna à
Suresne.

Le mercredi 30 août, le roi tint le conseil d'État à Marly, dîna à son
petit couvert, puis alla tout droit coucher à Petit-Bourg, chez d'Antin,
et le lendemain à Fontainebleau. Il avait dans son carrosse
M\textsuperscript{me} la duchesse de Berry auprès de lui,
M\textsuperscript{me} la Duchesse, sa nouvelle belle-fille, et
M\textsuperscript{lle} de Charolais au devant\,; M. le duc de Berry et
la nouvelle princesse de Conti aux portières\,; Madame, qui était un peu
incommodée, aima mieux aller dans son carrosse. L'électeur de Bavière y
arriva le samedi 9 septembre, dans le logement d'un concierge du jardin
de Diane, qu'on lui avait meublé tout auprès de celui de d'Antin, qui
lui avait fait accommoder une petite loge pour être incognito à la
comédie, et y entrer et en sortir commodément quand il voudrait. D'Antin
se chargea de lui donner à dîner et à souper, et de lui fournir force
joueurs chez lui dès le matin, et toute la journée. Il fut à plusieurs
chasses à cheval, et à plusieurs promenades du roi autour du canal, où
d'Antin le mena toujours dans son carrosse. Il avait les soirs force
dames à jouer chez lui, et allait toujours chez M\textsuperscript{me} la
duchesse de Berry les jours qu'il y avait jeu chez elle. Il vit le roi
un quart d'heure seul dans son cabinet le mardi 26 septembre, après son
lever, y prit congé de lui, et partit pour aller passer un jour dans une
maison qu'il venait d'acheter à Saint-Cloud, et de là retourner à
Compiègne. Il ne vit le roi dans son cabinet que cette seule fois à
Fontainebleau.

La comtesse de Solre vint avec sa fille à Fontainebleau prendre congé du
roi pour mener sa fille en Espagne épouser le prince de Robecque et être
dame du palais de la reine d'Espagne. Il ne sera pas inutile de
s'arrêter un peu ici.

M. de Robecque était de la maison de Montmorency, d'une branche sortie
du second fils de Louis de Montmorency, chef de la branche de Fosseux
devenue depuis l'aînée de la maison de Montmorency, et de Marguerite de
Wastines qui s'établit aux Pays-Bas. Ogier, ce puîné de Fossieux qui fit
la branche de MM. de Robecque, ni son fils ne figurèrent point\,; son
petit-fils figura fort peu, Louis, fils de ce dernier, encore moins\,;
mais il eut par son mariage avec J. de Saint-Omer, les terres de
Morbecque et de Robecque, et quelques autres, et par sa mère, dame
d'honneur de la reine de Hongrie, gouvernante des Pays-Bas, fille
d'Adrien III Villain, et de Marguerite Stavèle, dame d'Isenghien, la
terre d'Esterres et quelques autres. Esterres fut érigé en comté en
1611. Jean, son fils, servit beaucoup en Hongrie, eut la Toison d'or et
le gouvernement d'Aire\,; il fut créé par Philippe IV prince de
Robecque, ce qui ne donne que la dénomination et nul rang ni privilége,
et marquis de Morbecque. Il avait épousé Madeleine de Lens, et il mourut
en 1631. Eugène, son fils, prince de Robecque, fut gendre du duc
d'Arschot-Ligne-Aremberg, et beau-père du comte de Brouay-Spinola. Ce
prince de Robecque eut la Toison d'or, et il commandait dans Saint-Omer
lorsque le roi prit cette place en 1677. Il mourut en 1683. Son fils,
Philippe-Marie, prince de Robecque, passa en 1678 au service de France,
et mourut de maladie à Briançon en 1691, ayant un régiment. Il avait
épousé une fille du comte de Solre, chevalier de la Toison d'or, père du
chevalier de l'ordre du Saint-Esprit, et d'Isabelle Claire Villain, sœur
du prince d'Isenghien, gendre du maréchal d'Humières, et père du
maréchal d'Isenghien. L'autre sœur du prince d'Isenghien, gendre du
maréchal d'Humières, fut mariée en Espagne au duc de Montellano. Elle
fut choisie par la princesse des Ursins dans sa première disgrâce pour
être camarera-mayor de la reine, en sa place, qu'elle reprit à son
retour, et {[}celle-ci{]} l'aima et la protégea toujours depuis. Elle
fut depuis camarera-mayor de la princesse des Asturies, fille de M. le
duc d'Orléans, morte à Paris reine d'Espagne et veuve.

Ce prince de Robecque mort à Briançon laissa une fille religieuse et
deux fils. L'aîné, à l'occasion duquel cette descendance est traitée,
porta le nom, sans rang ni distinction nulle part, comme ses pères, de
prince de Robecque\,; le cadet celui de comte d'Esterres. Tous deux
servirent en France\,: l'aîné fut maréchal de camp. À la fin de 1709, il
passa, avec l'agrément du roi, en Espagne, pour s'y attacher. La
duchesse de Montellano était, comme on l'a vu, sœur de sa grand'mère, et
le comte de Solre, chevalier du Saint-Esprit, lieutenant général au
service de France, était frère de sa mère. Ce comte de Solre avait
épousé une Bournonville, cousine germaine de la maréchale de Noailles,
filles des deux frères, et fort liée avec elle. Le crédit de la
maréchale de Noailles et celui de la duchesse de Montellano sur
M\textsuperscript{me} des Ursins, qui avait fort connu et aimé aussi la
comtesse de Solre dans les anciens temps qu'elle avait passés à Paris,
firent la fortune du prince de Robecque en Espagne. Il fut fait
lieutenant général en arrivant, fort approché du roi d'Espagne,
gentilhomme de la chambre bientôt après, grand d'Espagne de la première
classe en avril de cette année, pour épouser M\textsuperscript{lle} de
Solre, sa cousine germaine, car le mariage en fut réglé dès lors, et on
le verra en 1716 colonel du régiment des gardes wallones. Il eut aussi
la Toison d'or, mais il mourut sans enfants, un mois après avoir eu les
gardes wallones.

Son frère, le comte d'Esterres, eut le régiment de Normandie, et est
devenu lieutenant général en France avec grande distinction. Le duc de
Noailles l'envoya porter la nouvelle de la réduction de Girone, où il
s'était signalé, au roi d'Espagne à Saragosse, en 1711, qui lui donna la
Toison d'or. Il a depuis succédé aux biens et à la grandesse de son
frère, mais sans quitter la France. Il n'est pas temps d'en dire
davantage sur lui. Venons maintenant au comte de Solre, qui est une
branche de la maison {[}de{]} Croï. On verra bientôt pourquoi je
m'arrête à quelques remarques.

La plupart des grandes maisons ont des chimères, et ces chimères leur
font plus de mal que de bien. Celle-ci a poussé la folie jusqu'à une
généalogie qui la conduit depuis Adam jusqu'à André II, roi de
Hongrie\,; et cette généalogie, bien écrite et bien enluminée, est
étalée dans le château d'Havré. Les armes de Hongrie et les leurs sont
les mêmes\,; de cela seul vient leur prétention de sortir des rois de
Hongrie, sans pouvoir en montrer d'autres titres. Le maréchal de Besons
portait celles de Suède. Les Bazin\footnote{Le manuscrit porte
  \emph{Bazin} et non \emph{Besons}, comme on l'a imprimé dans les
  précédentes éditions. Bazin, ou Basin, était le nom de famille des
  Besons.} sont encore trop nouveaux pour en rien conclure. S'ils
s'élèvent, ils auront dans quelques siècles le même titre pour sortir
des premiers rois de Suède que la maison de Croï pour venir de ceux de
Hongrie. Les ducs de Sully et de Montausier portaient les mêmes armes\,;
jamais les Béthune ni les Sainte-Maure n'ont imaginé sortir de la même
souche. MM. de Hennin, comte de Bossu, et depuis prince de Chimay, et
MM. de Noailles, portent aussi les mêmes armes, sans avoir imaginé
d'être parents\,: les uns des Pays-Bas, les autres de Limousin\,; et
toutes ces mêmes armes se portent par tous en plein et sans alliance.
Ces exemples ne sont pas rares, et ne sont rien moins que concluants. De
l'extrémité d'Adam et des rois de Hongrie, on a passé à celle de vouloir
fixer au fameux Chièvres, gouverneur de Charles-Quint, l'époque de
l'élévation de la maison de Croï, qui est une autre absurdité, puisque
son grand-père paternel fut grand maître de France en 1462, chevalier de
la Toison d'or en 1475, et gendre d'Antoine de Lorraine, comte de
Vaudemont\,; et son grand-père maternel était Louis de Luxembourg, comte
de Saint-Paul, de Brienne et de Ligny, connétable de France. En voilà
assez pour montrer le ridicule de cette calomnie. Voyons maintenant
quelle est la vérité sur cette maison.

La terre de Croüy ou Croï a donné l'origine, l'être et le nom à cette
maison. Cette terre, qui se trouve nommée et écrite en ces deux façons,
dont la dernière a prévalu, est située sous Pecquigny, près la rivière
de Somme, et l'abbaye du Gard est bâtie dans les marais de Croï.
Eustache, seigneur de Pecquigny ou Picquigny, car ce nom s'écrit aussi
en ces deux manières, avait la terre de Croï en 1066, et la fondation du
chapitre de Pecquigny le prouve. Il était aussi vidame d'Amiens. Son
petit-fils Gérard, sire de Pecquiquy et vidame d'Amiens, possédait
encore la terre de Croï et tous ses environs. Cela se prouve par la
fondation qu'il fit de l'abbaye du Gard. Il la bâtit sur le terroir de
Croï\,; lui donna la moitié de ce village et des fermes voisines, et
cela est de 1115. Enfin Gilles, seigneur de Croï, qui est le premier de
cette maison que l'on connaisse, est nommé homme lige d'Enguerrand,
vidame d'Amiens, dans un titre de l'abbaye du Gard de 1215. Cela fait un
gentilhomme le premier connu de sa race, et dans une antiquité fort
ordinaire, qui a un très-médiocre fief dont il porte le nom, qui devient
celui de sa postérité, et qui relève en plein d'un seigneur dont la
grande seigneurie rend ce fief fort petit, ainsi que le gentilhomme dont
il est tout l'avoir, sans qu'on sache par où il lui est venu. Mais il
est vrai que la postérité de ce gentilhomme ne tarda pas à s'illustrer,
et qu'elle eut le bonheur de s'élever en tous genres à pas de géant.
Tout y est petit et obscur jusqu'à Jacques I\^{}er, sire de Croï, qui
vivait sans figure en 1287, qui épousa Marguerite d'Araisnes, dont le
fils, qu'on ne voit pourtant point figurer, et qui fut Jacques II, sire
de Croï et d'Araisnes, épousa en 1313 Marie de Pecquigny, fille du
vidame d'Amiens. Cette alliance fut le premier grand pas. Guillaume
1\^{}cr, seigneur de Croï et d'Araisnes, épousa, en 1354, Isabeau, fille
et héritière d'André, seigneur de Renti, et de Marie de Brimeu. Ce fut
encore une autre illustration, jointe à une grande fortune de biens, qui
fut estimée telle que toute la maison de Croï, qui en est sortie, a
toujours constamment, et dans toutes ses branches jusqu'à aujourd'hui,
écartelé ses armes, au deuxième et troisième de Renti. Jean, premier
sire de Croï, de Renti, etc., fils de ce mariage, épousa Marguerite de
Craon, et fut tué en 1415 à la bataille d'Azincourt. Ce fut lui qui
commença la grandeur de sa maison. Il fut chambellan du roi et des deux
derniers ducs de Bourgogne, et grand bouteiller de France. Ses sœurs
furent bien mariées. Un de ses fils fit la branche de Chimay\,; et son
fils aîné Antoine, dont il a été parlé d'avance, fut gendre d'Antoine de
Lorraine, comte de Vaudemont. Il fut sire de Croï, de Renti, de
Beaurain, de Rosay, de Bar-sur-Aube, comte de Beaumont, de Porcan et de
Guines. Il fut grand maître de France en 1462, puis chevalier de la
Toison, fut surnommé le Grand, et mourut en 1475. Arschot lui vint par
sa femme avec d'autres terres. Son second fils fit la branche de Rœux.
Son aîné ne fut pas si heureux que lui\,; il épousa la fille du
connétable de Saint-Pol, comme on l'a déjà dit, et fut père de deux fils
qui ne figurèrent point, et d'un troisième qui fut le célèbre seigneur
de Chièvres, gouverneur de Charles-Quint. En voilà assez pour montrer
quelle est la maison de Croï, qui a eu le bonheur d'être illustre en
tout genre, en toutes ses branches. Il est temps de nous ramener à celle
de Solre. Jean de Croï, second fils de Jean, sire de Croï, et de
Marguerite de Craon, et frère du grand maître de France, figura fort
dans les Pays-Bas, où il eut toute sa vie de grands emplois de guerre et
de paix. Il fut chevalier de la Toison d'or. Charles, dernier duc de
Bourgogne, érigea en sa faveur en comté la terre de Chimay, qu'il avait
acquise du dernier seigneur de Morœil\footnote{Il y a Morœil dans le
  manuscrit et non Mareuil, comme on l'a imprimé dans les précédentes
  éditions.}. Il en porta le nom, qui devint celui de sa branche. Il
épousa une héritière de Lalain\,; il eut beaucoup d'enfants\,; il n'y
eut que les trois premiers qui figurèrent et beaucoup. L'aîné seul de
tous continua la postérité. Le second fut évêque de Cambrai, et ce fut
lui qui le premier fut évêque et duc de Cambrai, par lettres de
l'empereur Maximilien I\^{}er, de 1510, titre sans nul rang et de pure
décoration, dès lors et toujours depuis. Philippe de Croï, comte de
Chimay, l'aîné de tant d'enfants, figura grandement toute sa vie, maria
de même ses filles et ses fils, qu'il eut de Walpurge de Meurs, et
mourut en 1482. De ses trois fils le second n'eut point de postérité\,;
le troisième fit la branche de Solre, où on va revenir. L'aîné, qui
figura presque autant que son père, fit un très-grand mariage\,; il
épousa en 1495 Louise d'Albret, vicomtesse de Limoges, dame d'Avesnes et
de Landrecies\,; sœur de Jean d'Albret, roi de Navarre\,; fille d'Alain
dit le Grand, sire d'Albret comte de Gavre, de Dreux, de Penthièvre et
de Périgord, et\^{}de Françoise de Bretagne. Il mourut en 1527, et ne
laissa que deux filles, dont l'aînée reporta ce grand héritage dans sa
maison par son mariage avec Philippe II, sire de Croï, premier duc
d'Arschot\,; et l'autre, qui ne laissa pas d'être fort riche, épousa
Charles, comte de Lalain. Leur père avait été créé prince de Chimay en
1486, par l'empereur Maximilien I\^{}er, titre d'honneur sans aucun
rang.

Antoine de Croï, frère puîné de ce premier prince de Chimay, fit la
branche de Solre. Il porta le nom de seigneur de Sempy, servit
Maximilien I\^{}er, eut la Toison d'or et le gouvernement du Quesnoy, et
fut gendre de Jacques de Luxembourg, marquis de Richebourg. Jacques, son
fils, ne figura point, quoique chevalier de la Toison d'or. Il épousa
Yolande, fille aînée de Philippe de Lannoy, chevalier de la Toison d'or,
dont il eut les terres de Molembais, et de Solre qui donna le nom à sa
branche. Philippe son fils alla en Espagne, où il fut créé comte de
Solre en 1591. Il fut aussi chevalier de la Toison d'or, capitaine de la
garde du roi d'Espagne à Bruxelles, grand écuyer et conseiller d'État
des archiducs, et grand bailli de Tournai\,; il mourut au commencement
de 1612. Il fut marié trois fois\,: d'Anne de Beaufort en Artois, il eut
J. de Croï, comte de Solre, son fils aîné qui continua la branche\,;
d'Anne de Croï, dame de Renti, un fils qui fut chef des finances des
Pays-Bas, gouverneur de Tournai, en faveur duquel Philippe IV érigea la
terre d'Havré en duché en 1627, dont il avait épousé l'héritière qui
était aussi Croï, mourut en 1640 et ne laissa que des filles. De
l'héritière de Coucy, veuve d'un Mailly, que le premier comte de Solre
épousa en troisièmes noces, il ne laissa qu'un fils qui fit la branche
des ducs d'Havré.

J. de Croï, quatrième de cette branche, et second comte de Solre, oncle
paternel du premier duc d'Havré qui n'eut point de suite, et frère aîné
de celui qui fit la branche des ducs d'Havré, fut chevalier de la Toison
d'or, capitaine de la garde espagnole, du conseil de Flandre,
gentilhomme de la chambre du roi d'Espagne, et mourut à Madrid en 1640.
J, de Lalain, sa femme, lui apporta Renti qu'elle eut de sa mère qui
était Croï, et de son père la terre et ville de Condé, qui est devenue
une des bonnes places du roi, mais dont la seigneurie est demeurée aux
comtes de Solre. Son fils, troisième comte de Solre, fut chevalier de la
Toison d'or comme son père, son grand-père, son aïeul, et son trisaïeul
chef de cette branche, figura peu ou point, se tint aux Pays-Bas. C'est
celui dont on a parlé par avance, qui épousa la Villain-Isenghien, dont
il a eu le comte de Solre qui épousa la Bournonville, prit le service de
France, fut chevalier du Saint-Esprit en 1688, le cinquante-neuvième de
la promotion, c'est-à-dire après vingt-sept gentilshommes, et en ayant
onze après lui. Il est mort à Paris en 1718, lieutenant général et
gouverneur de Roye, Péronne et Montdidier, à soixante-dix-sept ans.
C'est lui dont la femme vint prendre congé à Fontainebleau avec leur
fille pour la mener épouser le prince de Robecque en Espagne, comme on
l'a vu d'abord, à l'occasion de quoi cette digression a été faite.
M\textsuperscript{lle} de Solre était cousine germaine du prince de
Robecque, dont la mère était soeur du comte de Solre. Outre cette fille
il eut deux fils\,: l'aîné porta le nom de comte de Croï\,; le cadet de
comte de Beaufort, qui succéda au régiment du chevalier de Solre son
frère, tué à la bataille de Malplaquet, et qui, lassé longtemps après de
n'avancer pas assez dans le service de France, est passé en Espagne. Or
voici pourquoi la digression.

Le comte de Croï, fils aîné du comte de Solre, chevalier du
Saint-Esprit, était un homme fort singulier. Il voulut profiter de la
simplicité et du peu d'esprit de son père pour devenir le maître dans la
famille. Sa mère, qui était une femme d'esprit, et volontiers
d'intrigue, ne s'accommoda pas de ce projet\,; ils luttèrent longtemps
l'un contre l'autre, jusqu'à ce que le fils sut si bien gagner et
gouverner son père qu'il le brouilla avec sa mère. Les éclats
domestiques percèrent, les parents et les amis s'en mêlèrent et y
échouèrent. La comtesse de Solre maltraitée au dernier point voulut se
séparer\,; la conjoncture du mariage de sa fille se présenta. Elle
n'était plus jeune, avait toujours été laide, elle avait perdu
l'espérance de s'établir. Sa mère l'avait toujours aimée avec passion\,;
et réciproquement. Elle saisit une occasion si naturelle de séparation
sans éclat, et mena sa fille en Espagne, dans la résolution, qu'elle a
tenue, d'y vivre avec elle et de n'en revenir jamais. Après son départ
son fils demeura le maître absolu. Il fut lieutenant général en 1718, un
mois avant la mort de son père, après laquelle il se fit appeler le
prince de Croï\,; et il épousa une fille du comte de Milandon, du côté
de Liége, vers l'Allemagne, qu'il infatua de sa nouvelle chimère.

On n'est prince que par être de maison actuellement souveraine. On vient
de voir l'origine de la maison de Croï fort éloignée de cette
extraction. Aucun de cette maison n'a prétendu l'être\,; et s'il y a eu
un ou deux princes de l'empire, ce n'a pas été d'origine, ç'a été par
érection des empereurs\,; ce n'a pas été même dans la branche de
Solre\,; et ces princes des empereurs n'ont aucun rang en France, ni
ailleurs que chez l'empereur, et encore fort court, et en Allemagne.
J'ai vu sans cesse la comtesse de Solre et sa fille debout au souper, à
la toilette et dans tous les lieux où les duchesses et les princesses
sont assises. Le comte de Solre n'imagina pas de faire la moindre
difficulté de prendre l'ordre parmi et fort au-dessous du milieu des
gentilshommes, et de se trouver toute sa vie parmi eux à toutes les
cérémonies de l'ordre du Saint-Esprit. Rien de tout cela ne put balancer
la fantaisie de ce premier prince de sa race. Il se retira dans ses
terres\,; sa femme avec ses nouvelles prétentions n'en sortit point. Ils
s'y amusèrent à épargner et à plaider, à faire les princes dans leur
maison sans y voir personne\,; et ce fondateur de princerie mourut chez
lui à Condé à la fin de 1723, à quarante-sept ans, fort mal avec son
frère qui voulait son bien, et point du tout être prince. La femme, avec
un fils presque en nourrice, demeura veuve chez elle, fit appeler cet
enfant le prince de Croï, et vint enfin avec lui à Paris quand il fut
d'âge à l'établir. Elle ne mit pas en doute d'être assise\,; il est vrai
aussi qu'on ne mit pas en doute que cela ne se devait pas. Elle jeta feu
et flammes, elle intrigua, elle n'alla point à la cour, mais elle fit
tant de bruit que le cardinal Fleury donna d'emblée un régiment à son
fils. Elle l'a depuis marié à une fille du duc d'Harcourt, et leur
tabouret est encore à venir\,; mais il viendra tôt ou tard, dans un pays
de confusion, et où, comme que ce soit, il n'y a qu'à prétendre, être
audacieux, impudent, et ne quitter point prise. Puisque j'en ai tant dit
sur la maison de Croï, voyons la branche d'Havré qui vient d'achever de
s'établir en France.

Philippe-François de Croï, qui a fait la branche des ducs d'Havré, fut
fils unique du troisième mariage du premier comte de Solre avec la veuve
de Louis de Mailly, seigneur de Rumesnil, fille aînée et héritière de
Jacques II de Coucy, seigneur de Vervins. Il épousa Marie-Claire de
Croï, unique héritière de la branche des marquis d'Havré qui était veuve
de son frère, que Philippe IV, comme on l'a dit, fit duc d'Havré en
1627, et qui ne laissa que trois filles mariées, et un fils unique qui
se fit carme, et mourut nommé à l'évêché de Gand. Philippe-François de
Croï devint donc duc d'Havré par ce mariage, et fut chef de la branche
des ducs d'Havré. Il fut fait grand d'Espagne, chevalier de la Toison
d'or, gouverneur du duché de Luxembourg et comté de Chiny, et chef des
finances des Pays-Bas. Il mourut à Bruxelles en 1650. Il ne laissa qu'un
fils qui eut la Toison, et fut fait prince et maréchal de l'empire je ne
sais par où, et mourut à Bruxelles en 1694. Il avait épousé en 1668 la
fille et héritière d'Alexis d'Halluyn, seigneur de Wailly près d'Amiens,
et de plusieurs autres terres. Elle a vécu fort vieille, et est demeurée
seule et la dernière de la maison d'Halluyn. Je l'ai vue plusieurs fois
à Paris venir voir ma mère. Elle n'allait point à la cour parce qu'elle
n'avait point de rang\,; les princes de l'empire n'en ont aucun en
France, et les grands d'Espagne n'y en avaient point encore. Elle n'eut
que deux fils qui vécurent, et des filles. L'aîné des fils s'avança au
service de Philippe V\,; il fut lieutenant général et colonel du
régiment des gardes wallones, à la tête duquel il fut tué en héros à la
bataille de Saragosse que les ennemis gagnèrent en septembre 1710\,; il
n'était point marié. Son frère lui succéda au titre de duc d'Havré, à la
grandesse, et au régiment des gardes wallones. La princesse des Ursins
lui fit épouser la fille de sa sœur, la duchesse Lanti, qu'elle fît
venir en Espagne, et qu'elle fit dame du palais. Quelque temps après la
disgrâce de M\textsuperscript{me} des Ursins, on voulut faire quelques
changements considérables dans les gardes wallones, fort désagréables à
ce régiment\,; le duc d'Havré s'y opposa avec tant d'opiniâtreté que le
régiment lui fut ôté, et donné au prince de Robecque, comme on a vu
ci-devant. Comme il était adoré dans ce régiment, le marquis de Lavère,
frère du prince de Chimay qui en était lieutenant-colonel, et lieutenant
général dans les troupes d'Espagne, quitta avec toute la tête et dans le
reste tout ce qu'il y avait de meilleur. Le duc d'Havré revint en France
avec sa femme, qui perdit sa place de dame du palais. Ils se retirèrent
dans leurs terres de Picardie, où le duc d'Havré mourut sans avoir paru
à la cour ni dans le monde. Sa veuve s'appliqua fort à raccommoder les
affaires de cette famille qui étaient fort délabrées. Elle est sœur du
prince de Lanti que M\textsuperscript{me} des Ursins avait fait grand
d'Espagne par un mariage à Madrid, et du cardinal Lanti qui vient d'être
promu fort jeune, et qui vit à Rome. Elle a marié ses deux fils\,:
l'aîné à une fille du maréchal de Montmorency\,; l'autre en Espagne à la
fille unique de son frère, qui le fait grand d'Espagne, et où il s'est
allé établir. Le duc d'Havré a un régiment, jouit ici de son rang de
grand d'Espagne, et n'a jamais eu non plus que son père ni sa mère, les
chimères de princerie de son cousin le prétendu prince de Croï.

Peu de temps après que le roi fut à Fontainebleau, j'appris qu'il
paraissait sous le manteau un mémoire de M. de La Rochefoucauld sur sa
prétention d'ancienneté contre moi, où l'avocat s'était, faute de
meilleures raisons, laissé aller à quelques impertinences\,; et j'en fus
assuré par une copie qui me tomba entre les mains. J'y fis sur-le-champ
une réponse, où je ne ménageai rien de tout ce que jusqu'alors j'avais
couvert avec tant de peine, et où d'ailleurs je n'épargnai pas qui
m'attaquait. Le duc de Noailles, que je voyais fort familièrement alors,
me surprit avec cette pièce entre les mains. Il fut effrayé de son
tissu. Il me conjura de ne la pas répandre, et d'attendre qu'il eût
parlé au duc de La Rocheguyon. Il revint promptement m'assurer que M. de
La Rocheguyon désavouait la pièce dont j'avais lieu de me plaindre,
qu'il retirerait tout ce qui en avait paru, et qu'il la supprimerait de
façon qu'on ne la verrait jamais, pourvu que je voulusse bien aussi
supprimer ma réponse. Je dis au duc de Noailles que je ne cherchais
point querelle dans cette affaire, comme il n'y avait que trop paru dans
toute ma conduite\,; mais qu'il ne fallait pas croire aussi que ce fût
par manque de moyens, de hauteur et de courage\,; qu'il paraîtroit
quelques copies de ma réponse, comme il en avait paru du mémoire auquel
elle répondait\,; et que, si le mémoire disparaissait, comme il m'en
portait parole, je ne répandrais pas davantage de réponses, et prendrais
pour bons tous les compliments et les protestations dont il était
chargé\,; sinon, que je ne m'entendais point aux subterfuges\,; et que,
de bouche et par écrit, je ne ménagerais rien, et tâcherais, en procédés
et en choses, de faire durement repentir qui m'attaquait lorsque j'avais
le moins lieu de m'y attendre. En effet, je parlai, et je distribuai
quelques exemplaires de ma réponse. Tout aussitôt le mémoire désavoué
disparut à Paris et à la cour, où presque personne ne l'avait vu. Le duc
de Noailles, et après lui le duc de Villeroy, et le duc de La Rocheguyon
ensuite, m'accablèrent de civilités et de protestations, moi de réponses
un peu froides, et il ne fut plus question d'écrits. Cela ne laissa pas
de faire du bruit que le roi voulut ignorer, qui même ne songea pas
alors à décider cette question de préséance jugée par l'édit de 1611,
mais que les cris de M. de La Rochefoucauld l'avaient forcé à lui
accorder de se la faire rapporter de nouveau, et à la juger comme si
elle n'eût pas été décidée.

Le roi donna trois milles livres d'augmentation de pension à
Saint-Herem, gouverneur et capitaine de Fontainebleau, qui en avait déjà
une pareille, pour qu'il eût six mille livres de pension, comme avait
son père. En même temps il chargea la province de Normandie de douze
mille livres d'appointements pour le gouvernement de Coutances, en
faveur de Bloin, un de ses premiers valets de chambre, à qui il avait
donné le haras de Normandie qu'avait Monseigneur. Il est vrai que, pour
un valet qui avait d'autres pensions, et avec elles la pécunieuse
intendance de Versailles et de Marly, c'était peu que le double d'un
seigneur fort mal dans ses affaires.

Le comte de La Mothe était demeuré exilé depuis sa reddition de Gand. Il
fit tant agir auprès du roi qu'il eut permission de venir le saluer à
Fontainebleau, et d'entrer même dans son cabinet, où il voulut entrer en
quelque justification. Le roi lui dit assez froidement qu'il la tenait
pour faite et qu'il était content de lui. Avec cela il sortit du
cabinet, et son affaire fut finie. Il parut après à la cour et dans le
monde en liberté, mais sans aucune marque de bienveillance tant que le
roi vécut.

Je ferai mention ici d'une bagatelle pour montrer combien le roi, qui
avait été élevé parmi les troubles, et qui y avait pris quelques bonnes
maximes de gouvernement, s'en départait difficilement. Le petit
gouvernement d'Alais, en Languedoc, vaqua\,; il le donna à Baudoin qu'il
estimait, et qui avait été lieutenant-colonel du régiment de Vendôme. On
peut juger que M. du Maine, gouverneur de Languedoc, y avait influé, et
pour un officier qui avait été attaché à M. de Vendôme. Peu de temps
après, je ne sais comment il arriva que le roi sut que Baudoin était de
Languedoc\,; aussitôt il lui fit dire de rendre le brevet de ce petit
gouvernement, avec promesse d'avoir soin de lui en donner un autre\,; et
donna le gouvernement d'Alais à d'Iverny, brigadier d'infanterie, qui
n'était point de ce pays-là.

La reine d'Espagne accoucha pour la dernière fois d'un quatrième prince.
Il eut pour parrain et marraine le roi et la reine de Sicile, ses aïeuls
maternels, et fut nommé Ferdinand. Il est devenu prince des Asturies par
la mort de tous les princes ses aînés. Il a épousé la fille du roi de
Portugal et de la sœur des empereurs Joseph et Charles, derniers de la
maison d'Autriche, dont il n'a point d'enfants. Il naquit à Madrid le 23
septembre de cette année, et y fut proclamé et juré aux cortès de 1724
successeur de la monarchie des Espagnes.

\hypertarget{note-i.-des-chanceliers-et-gardes-des-sceaux-pendant-la-premiere-moitiuxe9-du-xviie-siuxe8cle.}{%
\chapter{NOTE I. DES CHANCELIERS ET GARDES DES SCEAUX PENDANT LA
PREMIERE MOITIÉ DU XVII\^{}e
SIÈCLE.}\label{note-i.-des-chanceliers-et-gardes-des-sceaux-pendant-la-premiere-moitiuxe9-du-xviie-siuxe8cle.}}

Les chanceliers et gardes des sceaux de la première moitié du xvii\^{}e
siècle ont été fort nombreux. Saint-Simon n'en parle qu'en passant et
sans entrer dans les détails (p.~70 du présent volume). Un écrivain, qui
avait connu presque tous ces magistrats, comme il le dit lui-même, a
donné sur eux les détails les plus précis. Voici ce passage des Mémoires
inédits d'André d'Ormesson\footnote{Ms.~fol.~11 et suiv.}\,: «\,Philippe
Hubault, comte de Chiverni, fut fait garde des sceaux en l'an 1577 et
chancelier en l'an 1583 par le décès du chancelier de Birague, et tint
les sceaux jusques en octobre 1588, qu'il fut disgracié. Le roi Henri
III donna les sceaux à François de Montholon, fils du garde des sceaux
de Montholon, ancien avocat de la cour et avocat de Ludovic, duc de
Nevers, lequel (Montholon) n'avoit jamais vu le roi ni la cour. Après la
mort de Henri III, en août 1589, il fut démis de sa charge, et les
sceaux baillés en garde à Charles, cardinal DE BOURBON, puis au MARÉCHAL
DE BIRON (ARMAND DE GONTAUT), qui les garda jusques en juillet 1590, que
le roi les rendit audit comte et chancelier de Chiverni, qui demeura
dans sa charge jusques à sa mort, qui fut au mois d'août 1599, en sa
maison de Chiverni, près de Blois.

«\,Messire Pomponne de Bellièvre, fils de Claude de Bellièvre, premier
président au parlement de Grenoble, ayant été président au parlement de
Paris, surintendant des finances, employé en diverses ambassades, à la
conférence de Suresne\footnote{La conférence de Suresne, commencée le 29
  avril 1593 entre Henri IV et les catholiques modérés, eut pour
  résultat l'abjuration de ce roi.}, au traité de Vervins, où fut
conclue la paix entre la France et l'Espagne, en l'an 1598, à l'avantage
de la France (cinq ou sis places de Picardie ayant été rendues par les
Espagnols aux François), fut fait chevalier de France en août 1599, par
le décès de M. le chancelier de Chiverni, et exerça cette charge avec
grande intégrité jusques à sa mort. Il rendit les sceaux en 1605, qui
furent baillés à M. Nicolas Bruslart de Sillery, et mourut au mois de
septembre 1607 et fut enterré dans sa chapelle en l'église de
Saint-Germain l'Auxerrois.

\emph{«\,}Messire Nicolas Bruslart, seigneur de Sillery, fils de Pierre
Bruslart, président de la troisième chambre des enquêtes, après avoir
été conseiller de la cour, président aux enquêtes, ambassadeur en
Suisse, ambassadeur à Rome, président de la cour, conseiller d'État fort
employé, fut fait garde des sceaux en l'année 1605 et chancelier en
septembre 1607 (au mois de janvier), par le décès de M. de Bellièvre. Il
exerça cette charge paisiblement jusqu'au mois de mai 1616 qu'il fut
renvoyé en sa maison et les sceaux, baillés à M. du Vair, premier
président du parlement de Provence, En avril 1617, après la mort du
maréchal d'Ancre, et la disgrâce de la reine mère (Marie de Médicis) et
de toute sa bande, Nicolas Bruslart fut rétabli en la première place du
conseil, les sceaux étant tenus par MM. du Vair, Mangot, du Vair, de
Luynes, de Vic et de Caumartin, après la mort duquel les sceaux lui
furent rendus en janvier 1623. Il fut derechef disgracié en février
1624.

«\,Messire Guillaume du Vair, conseiller d'Église au parlement de Paris,
puis maître des requêtes de création nouvelle en 1614, puis premier
président du parlement de Provence, fut appelé au mois de mai 1616 pour
être garde des sceaux. En novembre suivant, les sceaux lui furent ôtés
et baillés à M. Claude Mangot. Après la mort du maréchal d'Ancre, au
mois d'avril 1617, les sceaux lui furent rendus et les tint jusqu'à sa
mort au siége de Tonneins, le troisième août 1621. Son corps fut apporté
à Paris, et enterré dans une chapelle des Bernardins.

«\,Messire Claude Mangot, après avoir été conseiller de la cour et
commissaire en la seconde chambre des requêtes du palais, maître des
requêtes dix-huit ans, nommé premier président de Bordeaux et
{[}avoir{]} exercé par commission la charge de secrétaire d'État, fut
élu garde des sceaux en novembre 1616 par la disgrâce de M. du Vair, et
les rendit le 14 avril 1617, le jour que le maréchal d'Ancre fut tué. Il
mourut en 1624, sans avoir été rétabli en sa charge.

«\,Messire Charles d'Albert, duc de Luynes, connétable de France en
avril 1621, tint les sceaux après la mort de M. du Vair, en août 1621,
et scelloit en présence du roi et des officiers du sceau, recevoit les
serments des officiers et en faisoit toutes les fonctions jusqu'au jour
de sa mort, qui fut le 14 décembre 1621, au siége de Monchenu. Son corps
fut porté et enterré à Maillé en Touraine, qu'il avoit fait ériger en
duché et fait porter le nom de Luynes.

«\,Messire Mery de Vic, frère de M. de Vic, grand capitaine, gouverneur
de Calais, après avoir été conseiller de la cour, maître des requêtes,
ambassadeur en Suisse, ancien conseiller d'État, fut fait garde des
sceaux le 20 décembre 1621, après le décès du duc de Luynes, le roi
étant lors à Bordeaux, où ledit sieur de Vic avoit été envoyé vers MM.
du clergé. Ledit sieur de Vic mourut à Pignas le 12 septembre 1622. Son
corps fut rapporté et enterré en sa terre d'Armenonville près de Senlis.

«\,En attendant que le roi eût choisi un garde des sceaux furent commis
pour sceller six conseillers d'État qui étaient à sa suite au siége de
Montpellier. MM\hspace{0pt}. de Caumartin, de Bullion, de Léon, Viguier,
Préaux et Halligre scelloient.

«\,Messire Louis Le Fevre, seigneur de Caumartin, après avoir été
conseiller à la cour, maître des requêtes, président au grand conseil,
ambassadeur en Suisse, ancien conseiller d'État, fut fait garde des
sceaux au camp de Montpellier, le 24 septembre 1622, et mourut en sa
maison de Paris le samedi 21 janvier 1623, et fut enterré en sa chapelle
de l'église Saint-Nicolas des Champs, où j'assistai.

«\,Le lundi 23 janvier 1623, le roi rendit les sceaux à M. le chancelier
de Siliery, à l'instance de M. de Pisieux son fils. Ainsi, après sept
ans et six gardes des sceaux, il rentra dans la pleine et entière
fonction de la charge de chancelier, jusqu'au second jour de janvier que
le roi lui ôta les sceaux, qu'il bailla à M. Halligre le samedi 6
janvier 1624, et au mois de février ensuivant, ledit chancelier de
Sillery fut renvoyé en sa maison de Sillery avec M. de Pisieux,
secrétaire d'État, son fils, disgracié comme son père, où il mourut
d'une dyssenlerie le 1\^{}er jour d'octobre 1624. Son corps fut apporté
et enterré en sa terre de Marines près de Pontoise.

«\,Messire Étienne Halligre, natif de Chartres, après avoir été
conseiller au grand conseil en l'an 1588, fut fait intendant de la
maison de Charles de Bourbon, comte de Soissons, entra dans le conseil
du roi en l'an 1610, et après plusieurs emplois dans les provinces de
Languedoc et de Bretagne, il fut fait garde des sceaux le 6 janvier
1624, et chancelier et surintendant de la maison de la reine audit an
par le décès du chancelier de Sillery. Il fut renvoyé en sa maison de la
Rivière près de Chartres, le 1\^{}er jour de juin 1626, où il mourut le
mardi 11 décembre 1635, et y est enterré.

«\,Messire Michel de Mabiliac ayant été conseiller de la cour en 1588,
maître des requêtes, conseiller d'État, surintendant des finances avec
M. de Champigny en août 1624, puis seul en janvier 1626, fut fait garde
des sceaux le 1\^{}er juin 1626 par la disgrâce de M. le chancelier
Halligre. Les sceaux lui furent ôtés à Glatigny, le roi étant à
Versailles, le mardi 12 novembre 1630. Il finit ses jours dans le
château de Châteaudun, où il mourut au mois d'août 1632, et est enterré
aux Carmélites du faubourg Saint-Jacques, dans sa chapelle.

«\,Messire Chaules de L'Aubépine de Chateauneuf, fils de M. de
Châteauneuf, doyen du conseil, après avoir été conseiller d'Église,
conseiller d'État, ambassadeur en Flandre et en Angleterre, chancelier
de l'ordre du Saint-Esprit, conseiller ordinaire du roi en ses conseils
par le règlement de Coinpiègne \footnote{Règlement relatif à
  l'organisation du conseil d'État, en date du 1\^{}e'juin 1624.}, fut
fait garde des sceaux par la disgrâce de M. de Marillac dans Versailles,
le 12 novembre 1630, fut aussi fait intendant de la maison de la reine,
comme étoit M. de Marillac. Il fut arrêté prisonnier dans Saint-Germain
en Laye le vendredi 25 février 1633, et mené prisonnier dans le château
d'Angoulême, dont il sortit en juillet 1643.

«\,Messire Pierre Séguier, sieur d'Autry, fils de M. Séguier lieutenant
civil, et petit-fils de Pierre Séguier président à la cour, après avoir
été conseiller à la cour, maître des requêtes, intendant de la justice
en Guyenne près le duc d'Épernon, président de la cour par la
résignation d'Antoine Séguier son oncle et bienfaiteur, fut fait garde
des sceaux par la disgrâce de M. de Châteauneuf et la faveur du cardinal
de Richelieu, le lundi 28 février 1633, et fut fait chancelier le 19
décembre 1635 par le décès de M. le chancelier Halligre, le cardinal de
Richelieu l'ayant fait attendre huit jours, avant qu'en prêter le
serment au roi.

«\,Au mois de juin 1643, M. de Châteauneuf, sorti de la prison du
château d'Angoulême, vint demeurer à Montrouge. La tapisserie étoit de
fleurs de lis\,; le cordon bleu et le Saint-Esprit sur sa robe de satin,
et ne pouvant rentrer dans sa charge, comme il s'y attendait, après la
mort du cardinal de Richelieu, il se résolut d'y faire sa demeure et de
ne point rentrer dans Paris en cet état, la charge étant toujours
exercée par M. le chancelier Séguier, qui l'exerce encore en ce mois
d'avril que j'écris cette page.

«\,J'ai écrit cette liste de chanceliers et gardes des sceaux à Ormesson
le lundi 30 et dernier jour d'avril 1646, afin de m'en mieux
ressouvenir, les ayant presque tous connus familièrement depuis M. le
chancelier de Bellièvre, qui me fit faire le serment de maître des
requêtes au mois de janvier 1605, et le chancelier de Chiverni qui me
scella les lettres de conseiller de la cour en 1598, en vertu desquelles
je fus reçu au parlement en 1600, que j'ai aussi vu plusieurs fois
accompagnant H. le président d'Ormesson mon père\footnote{Le père
  d'André d'Ormesson était président à la chambre des comptes.}.\,»

André d'Ormesson a ajouté postérieurement quelques renseignements sur
les chanceliers et gardes des sceaux pendant la Fronde\,: «\,Le mardi
1\^{}er mars 1650, M. de La Vrillière (Phélypeaux), secrétaire d'État,
alla reprendre les sceaux de M. Séguier, chancelier de France, lequel se
retira à Pontoise près de la mère Jeanne sa sœur, religieuse carmélite,
et puis à Rosny chez son gendre\,; et le mercredi, second de mars, jour
des Cendres, la reine régente remit lesdits sceaux entre les mains du
sieur de Châteauneuf, qui prit la qualité de garde des sceaux et ne fit
point de nouveau serment, étant rentré dans son ancienne charge et
n'ayant point été interdit ni condamné, mais seulement emprisonné.

«\,Le 3 avril 1651, M. de Châteauneuf rendit les sceaux qui furent à
l'instant baillés à M. le premier président, duquel on les retira le 13
avril pour les rendre à M. le chancelier.

«\,Le 7 septembre 1651, le roi retira les sceaux du chancelier et les
rendit à Mathieu Mole, premier président. Le jeudi 8 septembre 1651,
jour de la nativité de Notre-Dame, M. le chancelier fut renvoyé en sa
maison. M. de Châteauneuf fut fait chef du conseil du roi, et messire
Mathieu Molé, premier président du parlement de Paris, fut fait garde
des sceaux de France, et tint le premier conseil des parties le mardi 19
septembre 1651.

«\,Messire Mathieu Molé, ci-devant premier président du parlement de
Paris, et garde des sceaux de France, décéda à Paris en la maison du
président (\emph{sic}) Séguier le 3 janvier 1656, jour de sainte
Geneviève, à six heures du matin, et les sceaux furent rendus à messire
Pierre Séguier, chancelier de France, le lendemain mardi 5 janvier 1656,
à onze heures du matin par le roi, la reine et le cardinal Mazarin.
Voilà la troisième fois que l'on lui donne les sceaux de France.\,»

\hypertarget{note-ii.-ruxe8glement-fait-par-louis-xiv-uxe0-la-mort-du-chancelier-suxe9guier-pour-la-tenue-du-sceau.}{%
\chapter{NOTE II. RÈGLEMENT FAIT PAR LOUIS XIV, À LA MORT DU CHANCELIER
SÉGUIER, POUR LA TENUE DU
SCEAU.}\label{note-ii.-ruxe8glement-fait-par-louis-xiv-uxe0-la-mort-du-chancelier-suxe9guier-pour-la-tenue-du-sceau.}}

Il y eut à la mort du chancelier Séguier, arrivée en 1672, une lutte
entre les deux principaux minisires de Louis XIV, Colbert et Louvois,
pour faire donner la charge vacante à un de leurs parents ou du moins à
une de leurs créatures. Saint-Simon rappelle brièvement cette rivalité
(p.~72 de ce volume). Les Mémoires du temps n'en disent rien, et le
règlement que fit alors le roi, et auquel renvoie Saint-Simon, ne se
trouve pas dans le recueil des \emph{Anciennes lois françaises}. Pour
suppléer à ce silence, nous citerons un pasbsge du \emph{Journal
d'Olivier d'Ormesson}, qui donne l'analyse du règlement et l'exposé des
circonstances qui le rendirent nécessaire. Ce passage contient de
curieux détails sur l'organisation de l'ancienne chancellerie et sur la
manière dont on y scellait les actes royaux. Les maîtres des requêtes et
d'autres officiers en faisaient le rapport. Le chancelier ou le garde
des sceaux, assisté de conseillers d'État, prononçait sur la validité
des actes. En certains cas, il les rejetait comme contraires aux lois ou
obtenus par des moyens frauduleux.

«\,Le jeudi 28 janvier 1672, dit Olivier d'Ormesson\footnote{\emph{Journal},
  fol.~188 recto.} mourut à Saint-Germain, à sept heures du soir, M.
Pierre Séguier, chancelier de France, après trente-neuf ans de services
dans cette charge, depuis le 10 février 1633 qu'il reçut les sceaux
vacants par la disgrâce de M. de Châteauneuf\footnote{Charles de
  L'Aubépine, marquis de Châteauneuf, avait été nommé garde des sceaux
  en 1630\,: il fut disgracié et emprisonné en 1633. Il mourut en 1653.
  Voy. l'article sur les chanceliers et gardes des sceaux.}, et en 1635
la dignité de chancelier de France par la mort de M. Haligre\footnote{Étienne
  d'Aligre, nommé chancelier en 1624, mourut le 11 décembre 1635.},
décédé en sa terre de la Rivière. Depuis quelques années ledit sieur
chancelier (Séguier) étoit fort déchu de la vigueur de son esprit, et
sur la fin il ne connoissoit plus ceux qui l'abordoient, et avoit perdu
sa mémoire\,; mais dans ses derniers jours l'esprit lui étoit revenu
entier, et il est mort avec beaucoup de piété et de connoissance. Sa
famille avoit reporté au roi les sceaux quelques jours auparavant, et le
roi les avoit reçus avec bien de l'honnêteté, et dit qu'il ne les
vouloit garder qu'en dépôt et pour les rendre à M. le chancelier
lorsqu'il seroit revenu en sa santé.

«\,La vacance de la charge de chancelier fait beaucoup raisonner sur le
choix que le roi fera pour remplir cette place. D'abord l'on a dit que
c'était pour M. Le Tellier\footnote{Michel Le Tellier était secrétaire
  d'État depuis 1643\,; il devint chancelier en 1677, et mourut en 1685.},
depuis pour M. le premier président\footnote{Le premier président était
  alors Guillaume de Lamoignon, né en 1617, premier président en 1658,
  mort en 1677.}, et chacun nomme celui qui lui plaît\,; mais le roi ne
se découvre point, sinon qu'à son dîner ayant été dit qu'il y avoit eu
des chanceliers gens d'épée, l'on a dit qu'il vouloit choisir un homme
d'épée.

\emph{«\,}Le jeudi 5 février, étant chez M. Boulanger d'Hacqueville, il
me montra un paquet, qu'il venoit de recevoir de la part de M.
Haligre\footnote{Étienne d'Aligre, fils du précédent, fut successivement
  conseiller au grand conseil, conseiller d'État et chancelier en
  1674\,; il mourut à quatre-vingt-cinq ans, le 25 octobre 1677. Olivier
  d'Ormesson écrit \emph{Aligre} tantôt avec H, tantôt sans H. Comme il
  écrivait, en 1672, au moment même des événements qu'il raconte, il
  faut reconnaître que l'orthographe de ce nom était alors incertaine.
  Nous en faisons la remarque, parce que Saint-Simon insiste sur ce
  point (p.~73).}, qui étoit un règlement fait par le roi, par lequel il
dit que Sa Majesté ayant résolu de retenir les sceaux, elle fait savoir
ses intentions sur ce qu'elle entend être observé jusqu'à ce qu'elle en
ait autrement disposé\,: qu'elle donnera sceau un jour chaque semaine\,;
qu'elle a fait choix des sieurs Aligre, de Sève, Poncet, Boucherat,
Pussort et Voysin, conseillers d'État, pour y avoir séance et voix
délibérative, avec six maîtres des requêtes, dont elle fera choix au
commencement de chacun quartier\footnote{\emph{Journal}, fol.~188 recto.},
et le conseiller du grand conseil grand rapporteur en semestre\,; et
choisit pour le présent quartier les sieurs Barentin, Boulanger
d'Hacqueville, Le Pelletier, de Faucon, de Lamoignon, Pellisson.

«\,Les conseillers d'État {[}seront{]} assis selon leur rang, et les
maîtres des requêtes debout autour de la chaise du roi. Le grand
audiencier\footnote{Officier de la grande chancellerie chargé de faire
  rapport des lettres de grâce, de noblesse, etc.} et garde des
rôles\footnote{Le garde des rôles ou garde-rôle conservait le rôle des
  officiers royaux, en tenait registre et faisait sceller leurs
  provisions.} seront debout après le dernier conseiller d'État, et le
chauffe-cire\footnote{Officier de chancellerie qui préparait la cire
  pour sceller les actes.} ensuite, et le contrôleur au bout, les
garde-quittances et autres officiers derrière les chaires des
conseillers d'État. Les lettres de justice seront rapportées les
premières, remplies du nom de celui qui en aura fait le rapport et par
lui signées en queue. Le grand audiencier présentera ensuite les lettres
dont il sera chargé\,; le garde des rôles, les provisions des offices,
et les secrétaires du roi feront lecture des lettres de grâce qui seront
délibérées par les conseillers d'État et les maîtres des requêtes
présents et résolus par Sa Majesté. Les procureurs et les syndics des
cinq colléges des secrétaires du roi\footnote{Il y avait, d'après l'édit
  de mars 1704, trois cent quarante secrétaires du roi, qui étaient
  chargés d'expédier les actes royaux que l'on présentait au sceau.}
auront entrée, et en sera choisi dans chacun collége, savoir huit de
l'ancien, quatre des cinquante-quatre, autant des soixante-six, deux des
trente-six et un des vingt de Navarre. Le procureur du roi des requêtes
de l'hôtel\footnote{Les requêtes de l'hôtel formaient un tribunal chargé
  de connaître des causes des officiers de la maison du roi et de
  plusieurs autres privilégiés.}, et {[}procureur{]} général des grande
et petites chancelleries\footnote{La grande chancellerie était celle où
  s'expédiaient les actes émanés du roi et scellés du grand sceau par le
  chancelier ou le garde des sceaux. Les petites chancelleries étaient
  annexées aux parlements et aux tribunaux pour sceller les actes
  d'émancipation et autres qui étaient revêtus du petit sceau.}, aura
entrée et place derrière les maîtres des requêtes. Voilà ce que contient
ce règlement en neuf articles dont j'ai copie, fait à Saint-Germain en
Laye le 4 février 1672, signé LOUIS, et plus bas Colbert\footnote{Les
  maîtres des requêtes servaient à tour de rôle pendant trois mois ou un
  quartier.}.

«\,Ce règlement fait raisonner\,; on ne l'approuve pas ne pouvant pas
durer longtemps ni les affaires s'expédier. L'on dit que la raison de ce
règlement est pour avoir le temps de réformer tous les abus que l'on
prétend être dans la chancellerie, et diminuer l'autorité et la fonction
de cette charge de chancelier. Car, comme on a pris pour maxime de
supprimer les grandes charges, celles de connétable,
d'amiral\footnote{Les charges de connétable et d'amiral avaient été
  supprimées sous le règne de Louis XIII, en 1626.}, l'on veut aussi
sinon supprimer, au moins anéantir celle de chancelier, et donner toute
l'autorité aux ministres\,; et sur cela l'on m'a dit que M. le
Prince\footnote{Il s'agit ici de Louis de Bourbon (le grand Condé).}
avoit observé que l'on n'avoit supprimé ces deux grandes charges que
pour faire M. Colbert amiral et M. de Louvois connétable, et comme M.
Colbert fait depuis dix ans la principale partie de la charge de
chancelier en distribuant tous les emplois aux maîtres des requêtes, en
proposant seul au roi les personnes propres pour remplir les charges qui
viennent à vaquer, les donnant toutes à ses parents\footnote{Nous avons
  déjà fait remarquer qu'Oliv. d'Ormesson, disgracié pour sa noble et
  courageuse conduite dans le procès de Fouquet, n'était pas disposé à
  juger Colbert avec impartialité.}, comme celle de premier président de
la cour des aides et de lieutenant civil à M. Le Camus, et celle de
procureur général de la cour des aides à M. Dubois, fils du premier
commis de l'épargne, son parent\,; de premier président à Rouen à M.
Pellot qui a épousé une Camus. Étant le maître de l'agrément pour toutes
les charges de la robe, dont on ne peut être pourvu d'une seule que par
son ministère à cause de la consignation du prix, M. Colbert qui a
usurpé tout cet emploi sur la charge de chancelier, par la foiblesse du
défunt, ne veut pas le perdre par l'établissement d'un nouveau
chancelier qui voudra faire sa charge, «\,Le lundi 8 février, le roi
tint le premier sceau où le règlement fut observé exactement\,: les
maîtres des requêtes rapportèrent, et le roi écouta toutes choses avec
une attention et une connoissance surprenante.

«\,M. Haligre tint le lendemain le conseil dans le château, et fit les
mêmes fonctions que le chancelier, ayant pris sa place et signant les
arrêts comme lui. Il y a un règlement pour cela qui ne dit {[}rien autre
chose{]} sinon qu'en attendant que le roi ait pourvu à la charge de
chancelier, M. Haligre comme doyen fera les fonctions pour l'expédition
des affaires de justice et des finances.\,»

\hypertarget{note-iii.-madame-la-comtesse-et-vardes.}{%
\chapter{NOTE III. MADAME LA COMTESSE ET
VARDES.}\label{note-iii.-madame-la-comtesse-et-vardes.}}

L'aventure de Vardes et de M\textsuperscript{me} la Comtesse a été
racontée par M\textsuperscript{me} de La Fayette\footnote{\emph{Histoire
  de M\textsuperscript{me} Henriette}, coll. Petitot, t. LXIV, p.~410.}
et par M\textsuperscript{me} de Motteville\footnote{\emph{Mémoires de
  M\textsuperscript{me} de Motteville}, coll. Petitot, t. XLI, p. 180,
  228.}. M. Amédée Renée, dans ses \emph{Nièces de Mazarin, }ouvrage où
il a su rendre la science agréable et piquante, a rappelé ces intrigues
qui causèrent une véritable révolution a la cour de Louis XIV, en
faisant bannir deux personnages renommés par leur élégance, leur esprit
et leurs brillantes aventures. Le comte de Guiche\footnote{Armand de
  Grammont, comte de Guiche, né en 1637, mort en 1673.} et
Vardes\footnote{François-René du Bec-Crespin, marquis de Vardes, mort en
  1688.} ne se relevèrent pas de cette disgrâce. On peut ajouter aux
documents relatifs à ces intrigues le récit qu'en a tracé Olivier
d'Ormesson\footnote{\emph{Journal}, II\^{}e partie, fol.~97.}\,: «\,M.
de Bar nous dit une intrigue découverte à la cour, et comme je l'ai sue
aussi d'autres personnes et qu'elle peut avoir des suites, je la veux
écrire tout entière, comme je l'ai apprise. Il y a quelques années que
l'intelligence de Madame avec M. le comte de Guiche fit un grand
éclat\footnote{2. Ces événements sont de la fin de l'année 1662, d'après
  les \emph{Mémoires de M\textsuperscript{me} de Motteville}.}. M. le
comte de Guiche fut envoyé en Lorraine, après l'accommodement de
Lorraine, et il fit ensuite le voyage de Pologne. M. de Vardes fut
commis pour retirer les lettres des mains de M\textsuperscript{lle} de
Montalais, et étoit le confident entre les deux\,; mais il ne rendit pas
toutes les lettres, et il en retint deux qu'il mit entre les mains de
M\textsuperscript{me} la Comtesse pour s'en servir contre Madame en cas
de besoin.

«\,Dans ce même temps les amours de M\textsuperscript{lle} de La
Vallière et du roi commençoient, et M\textsuperscript{me} la Comtesse
vouloit les rompre. Elle prit une enveloppe d'un paquet du roi d'Espagne
à la reine, et concerta une lettre avec Vardes comme du roi d'Espagne à
la reine, qui lui donnoit avis des amours de M\textsuperscript{lle} de
La Valliere et du roi, et ils la firent traduire en espagnol par le
comte de Guiche, la firent écrire \footnote{2. Ces événements sont de la
  fin de l'année 1662, d'après les \emph{Mémoires de
  M\textsuperscript{me} de Motteville}.}par le beau-frère de Gourville,
et l'envoyèrent à Gourville en Flandre afin qu'il l'envoyât par un
courrier.

«\,Cette lettre fut adressée à la señora Molina, Espagnole, pour la
rendre à la reine\footnote{2. Ces événements sont de la fin de l'année
  1662, d'après les \emph{Mémoires de M\textsuperscript{me} de
  Motteville}.}. Elle la donna au roi qui jugea que c'était une lettre
supposée, mais ne put découvrir d'où elle venoit, et l'on prétend qu'il
soupçonna M\textsuperscript{me} de Navailles\footnote{Gouvernante des
  filles d'honneur de la reine. Voy. las \emph{Mémoires de
  M\textsuperscript{me} de Motteville}.}, et que c'est la véritable
cause de sa disgrâce. Depuis, M. de Vardes s'étant brouillé avec Madame
pour avoir dit au fils de M. le comte d'Harcourt qu'il devoit s'adresser
à Madame sans s'amuser aux suivantes, le roi l'a envoyé, à la prière de
Madame, à Aigues-Mortes\footnote{Vardes était gouverneur d'Aigues-Mortes
  depuis 1660.}, sans lui vouloir cependant de mal, disant qu'il serait
son solliciteur d'affaires.

«\,M\textsuperscript{me} la Comtesse, ennuyée de ce long exil, a fait
prier Madame de s'adoucir, et pour l'y obliger lui a fait dire qu'elle
avoit des lettres et de quoi lui donner de la peine. Madame s'en étant
irritée, et sachant par le comte de Guiche l'histoire de la lettre, elle
l'a dite au roi. Ce fut dans la tribune le jour du ballet qu'elle en fit
sortir M\textsuperscript{me} la Comtesse\,; et le roi l'ayant pressée de
faire quelque civilité à M\textsuperscript{me} la Comtesse et lui disant
qu'elle la devoit ménager ayant des lettres, sur cela Madame lui dit la
lettre espagnole\footnote{Ce dénoûment d'une intrigue qui remontait à
  l'année 1661 se place au mois de mars 1665.}.

«\,Le comte de Guiche mandé aussitôt par le roi, après avoir obtenu son
pardon, lui a dit toute l'intrigue et a fort chargé Vardes, et le roi a
pris par écrit sa déclaration et la lui a fait signer. L'on dit que le
comte de Guiche a découvert encore d'autres intrigues sur l'affaire de
Dunkerque, et qu'il avoit conseillé à Madame de s'y retirer avec
Monsieur, et que, soutenue du roi d'Angleterre, elle se feroit
considérer, et l'on parle que ces lettres ont été rendues au roi, par
lesquelles il mandoit à Madame\,: \emph{Votre timide beau-frère n'est
qu'un fanfaron et un avare. Quand une fois vous serez dans Dunkerque,
nous lui ferons faire, le bâton haut}, \emph{tout ce que nous voudrons}.
Le roi a envoyé un exempt à Vardes avec des gardes pour l'arrêter
prisonnier et le conduire dans la citadelle de Montpellier et lui
ordonner de se défaire de sa charge. M. le maréchal de Grammont a eu de
longues conférences avec le roi, et l'on dit qu'il a obtenu le pardon
pour son fils\,; mais néanmoins que c'est un homme dont la fortune est
perdue.\,»

\hypertarget{note-iv.-le-duc-de-mazarin.}{%
\chapter{NOTE IV. LE DUC DE
MAZARIN.}\label{note-iv.-le-duc-de-mazarin.}}

Le duc de Mazarin, dont Saint-Simon retrace le caractère (p.~277, 278,
279 de ce volume), a été représenté par tous les contemporains comme un
maniaque, auquel la jalousie et une dévotion ridicule avaient troublé
l'esprit. Hortense Mancini, qu'il avait épousée\footnote{Voy. les
  \emph{Nièces de Mazarin}, par Amédée Renée.}, donne une idée de sa
jalousie dans le passage suivant de ses Mémoires : «\,Je ne pouvois,
dit-elle, parler à un domestique, qu'il ne fût chassé le lendemain. Je
ne recevois pas deux visites de suite d'un même homme, qu'on ne lui fît
défendre la maison. Si je témoignois quelque inclination pour une de mes
filles, on me l'ôtoit aussitôt. Si je demandois mon carrosse, il
défendoit en riant qu'on y mit les chevaux et plaisantoit avec moi sur
cette défense\ldots{} Il aurait voulu que je n'eusse vu que lui seul au
monde.\,» Le duc de Mazarin ne se borna pas à exercer sur sa femme cette
ridicule et tyrannique surveillance, il fit mutiler les statues ou
barbouiller les tableaux du palais Mazarin qui lui paraissaient blesser
la décence\footnote{Voy. les \emph{Mémoires de l'abbé de Choisy}, coll.
  Petitot, 2\^{}e série, t. LXIII, p.~207.}. Il poussa la manie des
réformes jusqu'à vouloir intervenir dans les amours de Louis XIV et de
M\textsuperscript{lle} de La Vallière. Un grave contemporain, Olivier
d'Ormesson, raconte dans son \emph{Journal }inédit\footnote{\emph{Journal
  d'Olivier d'Ormesson}, fol.~80 verso et 81 recto\,; à la date du 16
  décembre 1665.} cette aventure qui peint le duc de Mazarin\,: «\,Je
veux écrire une histoire véritable de M. le duc Mazarin, lequel, ayant
formé le dessein d'avertir le roi du scandale que sa conduite avec
M\textsuperscript{lle} de La Vallière cause dans son royaume, communia,
il y eut dimanche huit jours, et alla au Louvre au lever du roi, et lui
ayant dit qu'il souhaitoit parler à Sa Majesté en son particulier, le
roi le fit entrer dans son cabinet. Là il dit au roi, après bien des
excuses de la liberté qu'il prenoit, qu'il avoit senti un mouvement dans
sa conscience depuis quelque temps\,; qu'il venoit de communier et qu'il
se sentoit plus pressé qu'auparavant de dire à Sa Majesté le scandale
qu'il donnoit à toute la France par sa conduite avec
M\textsuperscript{lle} de La Vallière, etc. Le roi lui ayant laissé dire
tout ce qu'il avoit à dire, lui dit\,: \emph{Avez-vous tout dit\,? Il y
a longtemps que je sais que vous êtes blessé là, }mettant la main sur
son front.\,»

\end{document}
